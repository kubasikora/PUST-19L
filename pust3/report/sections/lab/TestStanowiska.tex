\chapter{Wyznaczenie punktu pracy}
Po sprawdzeniu możliwości sterowania i pomiaru w komunikacji ze stanowiskiem
przystąpiliśmy do wyznaczenia punktu pracy. Od prowadzącego otrzymaliśmy
informacje o wartości sygnału sterującego w punkcie pracy równej
$Upp = 26\%$. W celu wyznaczenia wartości sygnału wyjściowego w punkcie pracy
Ypp na wejście obiektu podaliśmy stałe wejście o wartości Upp i zaczekaliśmy aż
wyjście ustabilizuje się. Po odczekaniu około 10 minut, wyjście obiektu ustabilizowało
się na wartości $ Ypp = 32^{\circ} C$.

\begin{figure}[b]
    \centering
    \begin{tikzpicture}
    \begin{axis}[
    width=0.98\textwidth,
    xmin=0.0,xmax=600,ymin=24,ymax=36,
    xlabel={$k$},
    ylabel={$y[k]$},
    legend pos=south east,
    y tick label style={/pgf/number format/1000 sep=},
    ] 
    \addlegendentry{$y[k]$},
    \addlegendimage{no markers,green}
    \addplot[green, semithick, thick] file{../data/lab/zad1/punkt_pracy.csv};
    
    \end{axis}
    \end{tikzpicture}
    \caption{Odpowiedzi procesu na sterowanie w punkcie pracy}
    \label{lab1_odpowiedzi_rys}
\end{figure}