\chapter{Weryfikacja poprawności działania i dobór parametrów
algorytmów regulacji jednowymiarowego procesu laboratoryjnego o
istotnie nieliniowych właściwościach.}

\section{Wyznaczenie punktu pracy}
Po sprawdzeniu możliwości sterowania i pomiaru w komunikacji ze stanowiskiem
przystąpiliśmy do wyznaczenia punktu pracy. Od prowadzącego otrzymaliśmy
informacje o wartości sygnału sterującego w punkcie pracy równej
$Upp = 26\%$. W celu wyznaczenia wartości sygnału wyjściowego w punkcie pracy
Ypp na wejście obiektu podaliśmy stałe wejście o wartości Upp i zaczekaliśmy aż
wyjście ustabilizuje się. Po odczekaniu około 5 minut, wyjście obiektu ustabilizowało
się na wartości $ Ypp = 32^{\circ} C$.

\section{Określenia wzmocnienia funkcji sterowania}
