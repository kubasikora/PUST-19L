\chapter{Wyznaczenie charakterystyki statycznej obiektu}
\label{lab2}

Aby wyznaczyć charakterystykę statyczną wykonaliśmy eksperyment mający na celu zbadanie wzmocnienia w zależności od sterowania. Dla kolejnych wartości sygnału sterującego: $20, 30, 40...$ oczekiwaliśmy stabilizacji sygnału wyjściowego obiektu. Wynik eksperymentu przedstawia wykres \ref{eksperyment}. Na wykresie możemy zauważyć, że przyrost wartości sygnału wyjściowego zmniejsza się wraz ze zwiększaniem wartości sterowania. Punkty stabilizacji sygnału posłużyły do wyznacznia charakterystyki statycznej obiektu. Wynik aproksymacji przedstawia wykrest \ref{char_stat}.\\
\indent{} Przegięcie wykresu wskazuje na to, że mamy do czynienia z obiektem nieliniowym, dlatego niemożliwe jest wyznaczenie jednego wzmocnienia statycznego 
opisującego cały proces.

\begin{figure}[b]
    \centering
    \begin{tikzpicture}
    \begin{axis}[
    width=0.98\textwidth,
    xmin=0.0,xmax=2040,ymin=30,ymax=55,
    xlabel={$k$},
    ylabel={$y[k]$},
    legend pos=south east,
    y tick label style={/pgf/number format/1000 sep=},
    ] 
    \addlegendentry{$y[k]$},
    \addlegendimage{no markers,red}

    \addplot[red, semithick] file{../data/lab/zad2/eksperyment.csv};  
    \end{axis}
    \end{tikzpicture}
    \caption{Przebieg eksperymentu przy wartościach sterowania: $20, 30, 40, 50, 60, 70, 80$}
    \label{eksperyment}
\end{figure}

\begin{figure}[t]
    \centering
    \begin{tikzpicture}
    \begin{axis}[
    width=0.98\textwidth,
    xmin=20.0,xmax=80,ymin=32,ymax=52,
    xlabel={$u$},
    ylabel={$y[u]$},
    legend pos=south east,
    y tick label style={/pgf/number format/1000 sep=},
    ] 
    \addlegendentry{$y[u]$},
    \addlegendimage{no markers,red}

    \addplot[red, semithick] file{../data/lab/zad2/char_stat.csv};  
    \end{axis}
    \end{tikzpicture}
    \caption{Charakterystyka statyczna obiektu}
    \label{char_stat}
\end{figure}