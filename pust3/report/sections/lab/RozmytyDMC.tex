\chapter{Regulacja z wykorzystaniem rozmytego regulatora DMC}
\label{lab5i6}

W ramach laboratorium przeprowadziliśmy próbę regulacji badanym 
nieliniowym obiektem za pomocą algorytmu predykcyjnego DMC. Podobnie jak w przypadku 
regulatora PID zdecydowaliśmy się na trapezową funkcję przynależności, jednak jako zmienną 
decyzjną wybraliśmy wartość zadaną a nie aktualną wartość wyjściową. 

Zgodnie z poleceniem, założyliśmy $D = N = N_{\mathrm{u}} = 300$ i $\lambda = 1$.
Wyniki eksperymentu zostały przedstawione na rysunkach \ref{dr_1_y} i \ref{dr_1_u}.
Zgodnie z naszymi oczekiwaniami,

\begin{figure}[t]
    \centering
    \begin{tikzpicture}
    \begin{axis}[
    width=0.98\textwidth,
    xmin=1,xmax=1000,ymin=30,ymax=55,
    xlabel={$k$},
    ylabel={$y[k]$},
    legend pos=south east,
    y tick label style={/pgf/number format/1000 sep=},
    ] 
    \addlegendentry{$y[k]$},
    \addlegendentry{$y^{\mathrm{zad}}$},
    \addlegendimage{no markers,red},
    \addlegendimage{no markers,blue}

    \addplot[red, semithick] file{../data/lab/zad5/proba2_output.csv};
    \addplot[blue, semithick] file{../data/lab/zad5/proba2_stpt.csv};  
    \end{axis}
    \end{tikzpicture}
    \caption{Przebieg procesu regulacji obiektu nieliniowego rozmytym regulatorem DMC}
    \label{dr_1_y}
\end{figure}

\begin{figure}[b]
    \centering
    \begin{tikzpicture}
    \begin{axis}[
    width=0.98\textwidth,
    xmin=1,xmax=1000,ymin=0,ymax=100,
    xlabel={$k$},
    ylabel={$y[k]$},
    legend pos=south east,
    y tick label style={/pgf/number format/1000 sep=},
    ] 
    \addlegendentry{$u$},
    \addlegendimage{no markers,blue}
    \addplot[blue, semithick] file{../data/lab/zad5/proba2_input.csv};  
    \end{axis}
    \end{tikzpicture}
    \caption{Przebieg sygnału sterującego podczas regulacji obiektu nieliniowego rozmytym regulatorem DMC}
    \label{dr_1_u}
\end{figure}
\FloatBarrier

\section{Rózne parametry kary}

W kolejnej próbie postanowiliśmy każdemu regulatorowi lokalnemu przypisać wartość parametru kary $\lambda_{i}$,
taką jak w poprzednio omawianym regulatorze liniowym $\lambda = \num{100}$ Wyniki tej próby wyszły jeszcze gorsze niż się spodziewaliśmy.
Z biegiem czasu, możemy stwierdzić że ten pomysł był bez sensu ponieważ dla mniejszych wartości tego parametru pojawiały się 
uchyby ustalone, dlatego też w tym przypadku powinniśmy zmniejszyć ten parametr a nie go zwiększać.

\begin{figure}[t]
    \centering
    \begin{tikzpicture}
    \begin{axis}[
    width=0.98\textwidth,
    xmin=1,xmax=1000,ymin=30,ymax=55,
    xlabel={$k$},
    ylabel={$y[k]$},
    legend pos=south east,
    y tick label style={/pgf/number format/1000 sep=},
    ] 
    \addlegendentry{$y[k]$},
    \addlegendentry{$y^{\mathrm{zad}}$},
    \addlegendimage{no markers,red},
    \addlegendimage{no markers,blue}

    \addplot[red, semithick] file{../data/lab/zad5/proba1_output.csv};
    \addplot[blue, semithick] file{../data/lab/zad5/proba1_stpt.csv};  
    \end{axis}
    \end{tikzpicture}
    \caption{Przebieg procesu regulacji obiektu nieliniowego rozmytym regulatorem DMC}
    \label{dr_21_y}
\end{figure}

\begin{figure}[b]
    \centering
    \begin{tikzpicture}
    \begin{axis}[
    width=0.98\textwidth,
    xmin=1,xmax=1000,ymin=0,ymax=100,
    xlabel={$k$},
    ylabel={$y[k]$},
    legend pos=south east,
    y tick label style={/pgf/number format/1000 sep=},
    ] 
    \addlegendentry{$u$},
    \addlegendimage{no markers,blue}
    \addplot[blue, semithick] file{../data/lab/zad5/proba1_input.csv};  
    \end{axis}
    \end{tikzpicture}
    \caption{Przebieg sygnału sterującego podczas regulacji obiektu nieliniowego rozmytym regulatorem DMC}
    \label{dr_2_u}
\end{figure}

\indent{}
W ostatniej próbie zdecydowaliśmy się zróżnicować wartości parametrów $\lambda_{i}$ pomiędzy regulatorami 
lokalnymi. Uznaliśmy że zastosujemy następujący zestaw parametrów: $\lambda_{1} = \num{1}$, $\lambda_{2} = \num{100}$,
$\lambda_{1} = \num{0.01}$. Wybraliśmy taki zestaw ponieważ uznaliśmy że regulacja w okolicy punktu pracy regulatora
pierwszego jest solidna a w okolicy większych wartości zadanych regulator niedomaga. Wartość środkowego
regulatora zostawiliśmy na poziomie z poprzedniej próby tak aby nie wprowadzała niepotrzebynch zakłóceń 
na przełamaniu charakterystyki obiektu.

\begin{figure}[t]
    \centering
    \begin{tikzpicture}
    \begin{axis}[
    width=0.98\textwidth,
    xmin=1,xmax=1000,ymin=30,ymax=55,
    xlabel={$k$},
    ylabel={$y[k]$},
    legend pos=south east,
    y tick label style={/pgf/number format/1000 sep=},
    ] 
    \addlegendentry{$y[k]$},
    \addlegendentry{$y^{\mathrm{zad}}$},
    \addlegendimage{no markers,red},
    \addlegendimage{no markers,blue}

    \addplot[red, semithick] file{../data/lab/zad5/proba3_output.csv};
    \addplot[blue, semithick] file{../data/lab/zad5/proba3_stpt.csv};  
    \end{axis}
    \end{tikzpicture}
    \caption{Przebieg procesu regulacji obiektu nieliniowego rozmytym regulatorem DMC}
    \label{dr_3_y}
\end{figure}

\begin{figure}[b]
    \centering
    \begin{tikzpicture}
    \begin{axis}[
    width=0.98\textwidth,
    xmin=1,xmax=1000,ymin=0,ymax=100,
    xlabel={$k$},
    ylabel={$y[k]$},
    legend pos=south east,
    y tick label style={/pgf/number format/1000 sep=},
    ] 
    \addlegendentry{$u$},
    \addlegendimage{no markers,blue}
    \addplot[blue, semithick] file{../data/lab/zad5/proba3_input.csv};  
    \end{axis}
    \end{tikzpicture}
    \caption{Przebieg sygnału sterującego podczas regulacji obiektu nieliniowego rozmytym regulatorem DMC}
    \label{dr_3_u}
\end{figure}

\subsubsection{Wnioski}
Podobnie jak w przypadku rozmytego regulatora PID, proces doboru parametrów regulatora DMC jest długi i złożony.
W trakcie laboratorium nie zdążyliśmy przetestować innych zestawów parametrów $\lambda$, a także działania
regulatora dla innych zmiennych decyzyjnych i funkcji przynależności.