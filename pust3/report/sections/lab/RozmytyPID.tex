\chapter{Regulacja z wykorzystaniem rozmytego regulatora PID}
\label{lab4}

Eksperymenty przeprowadzane były dla regulatora wykorzystującego trzy regulatory lokalne. 
Do każdego regulatora lokalnego staraliśmy się dobrać parametry minimalizujące wartość wskaźnika jakości.
Korzystając z doświadczenia zdobytego na projekcie, zdecydowaliśmy się na zastosowanie trapezowej funkcji przynależności,
ponieważ to dla niej uzyskaliśmy najlepsze efekty. Podobnie jak w zadaniu projektowym, zmienną decyzyjną
było aktualne wyjście procesu.

\subsubsection{Próba 1.}
Parametry regulatorów lokalnych:
\begin{enumerate}
\item $K = 15, T_{\mathrm{i}} = 300,  T_{\mathrm{d}} = 1$
\item $K = 17,5, T_{\mathrm{i}} = 250,  T_{\mathrm{d}} = 1$
\item $K = 20, T_{\mathrm{i}} = 200,  T_{\mathrm{d}} = 1$
\end{enumerate}

Dla pierwszej próby wskaźnik jakości wyniósł $E = \num{67034}$ co jest wynikiem znacznie gorszym od wyniku dla regulatora PID z rozdziału \ref{lab3}. Na wykresach również obserwujemy bardzo słabą jakość regulacji, brak jakiegokolwiek nadążania za wartością zadaną i duży błąd. Wyniki przedstawione na wykresach \ref{pr_1_y} i \ref{pr_1_u}.

\begin{figure}[t]
    \centering
    \begin{tikzpicture}
    \begin{axis}[
    width=0.98\textwidth,
    xmin=1,xmax=1000,ymin=30,ymax=55,
    xlabel={$k$},
    ylabel={$y[k]$},
    legend pos=south east,
    y tick label style={/pgf/number format/1000 sep=},
    ] 
    \addlegendentry{$y[k]$},
    \addlegendentry{$y^{\mathrm{zad}}$},
    \addlegendimage{no markers,red},
    \addlegendimage{no markers,blue}

    \addplot[red, semithick] file{../data/lab/zad4/proba_1_output.csv};
    \addplot[blue, semithick] file{../data/lab/zad4/proba_1_stpt.csv};  
    \end{axis}
    \end{tikzpicture}
    \caption{Próba 1.: Przebieg procesu regulacji obiektu nieliniowego rozmytym regulatorem PID.}
    \label{pr_1_y}
\end{figure}

\begin{figure}[b]
    \centering
    \begin{tikzpicture}
    \begin{axis}[
    width=0.98\textwidth,
    xmin=1,xmax=1000,ymin=0,ymax=100,
    xlabel={$k$},
    ylabel={$y[k]$},
    legend pos=south east,
    y tick label style={/pgf/number format/1000 sep=},
    ] 
    \addlegendentry{$u$},
    \addlegendimage{no markers,blue}
    \addplot[blue, semithick] file{../data/lab/zad4/proba_1_input.csv};  
    \end{axis}
    \end{tikzpicture}
    \caption{Próba 1.: Przebieg sygnału sterującego podczas regulacji obiektu nieliniowego rozmytym regulatorem PID.}
    \label{pr_1_u}
\end{figure}


\subsubsection{Próba 2.}
Parametry regulatorów lokalnych:
\begin{enumerate}
\item $K = 17, T_{\mathrm{i}} = 300,  T_{\mathrm{d}} = 1$
\item $K = 19, T_{\mathrm{i}} = 250,  T_{\mathrm{d}} = 1$
\item $K = 22, T_{\mathrm{i}} = 200,  T_{\mathrm{d}} = 1$
\end{enumerate}
Przy próbie drugiej możemy zaobserwować, że jakość regulacji uległa nieznacznej poprawie. Znajduje to potwierdzenie we wskaźniku jakości - $E = \num{37713}$. Jest to jednak wartość o rząd wielkości większa niż uzyskana w zadaniu poprzednim. Sygnały nie dążą do wartości zadanej, obserwujemy oscylacje i niestabilny sygnał sterujący. Wyniki przedstawione na wykresach \ref{pr_2_y} oraz \ref{pr_2_u}.

\begin{figure}[t]
    \centering
    \begin{tikzpicture}
    \begin{axis}[
    width=0.98\textwidth,
    xmin=1,xmax=1000,ymin=30,ymax=55,
    xlabel={$k$},
    ylabel={$y[k]$},
    legend pos=south east,
    y tick label style={/pgf/number format/1000 sep=},
    ] 
    \addlegendentry{$y[k]$},
    \addlegendentry{$y^{\mathrm{zad}}$},
    \addlegendimage{no markers,red},
    \addlegendimage{no markers,blue}

    \addplot[red, semithick] file{../data/lab/zad4/proba_2_output.csv};
    \addplot[blue, semithick] file{../data/lab/zad4/proba_2_stpt.csv}; 
    \end{axis}
    \end{tikzpicture}
    \caption{Próba 2.: Przebieg procesu regulacji obiektu nieliniowego rozmytym regulatorem PID.}
    \label{pr_2_y}
\end{figure}

\begin{figure}[b]
    \centering
    \begin{tikzpicture}
    \begin{axis}[
    width=0.98\textwidth,
    xmin=1, xmax=1000,ymin=0,ymax=100,
    xlabel={$k$},
    ylabel={$y[k]$},
    legend pos=south east,
    y tick label style={/pgf/number format/1000 sep=},
    ] 
    \addlegendentry{$u$},
    \addlegendimage{no markers,blue}
   \addplot[blue, semithick] file{../data/lab/zad4/proba_2_input.csv};  
    \end{axis}
    \end{tikzpicture}
    \caption{Próba 2.: Przebieg sygnału sterującego podczas regulacji obiektu nieliniowego rozmytym regulatorem PID.}
    \label{pr_2_u}
\end{figure}


\subsubsection{Próba 3.}
Parametry regulatorów lokalnych:
\begin{enumerate}
\item $K = 17, T_{\mathrm{i}} = 250,  T_{\mathrm{d}} = 18$
\item $K = 19, T_{\mathrm{i}} = 200,  T_{\mathrm{d}} = 20$
\item $K = 22, T_{\mathrm{i}} = 180,  T_{\mathrm{d}} = 15$
\end{enumerate}
Próba trzecia różniła się od drugiej jedynie gładszym przebiegiem sygnału wyjściowego. Zdecydowanie ostrzejszy przebieg możemy zaobserwować na wykresie sterowania, które oscyluje z krótkim okresem. Wskaźnik jakości nieznacznie się zmniejszył - $E = \num{33717}$. Przebiegi przedstawione na wykresach \ref{pr_3_y} i \ref{pr_3_u}.

\begin{figure}[t]
    \centering
    \begin{tikzpicture}
    \begin{axis}[
    width=0.98\textwidth,
    xmin=1,xmax=1000,ymin=30,ymax=55,
    xlabel={$k$},
    ylabel={$y[k]$},
    legend pos=south east,
    y tick label style={/pgf/number format/1000 sep=},
    ] 
    \addlegendentry{$y[k]$},
    \addlegendentry{$y^{\mathrm{zad}}$},
    \addlegendimage{no markers,red},
    \addlegendimage{no markers,blue}

    \addplot[red, semithick] file{../data/lab/zad4/proba_3_output.csv};
    \addplot[blue, semithick] file{../data/lab/zad4/proba_3_stpt.csv}; 
    \end{axis}
    \end{tikzpicture}
    \caption{Próba 3.: Przebieg procesu regulacji obiektu nieliniowego rozmytym regulatorem PID.}
    \label{pr_3_y}
\end{figure}

\begin{figure}[b]
    \centering
    \begin{tikzpicture}
    \begin{axis}[
    width=0.98\textwidth,
    xmin=1,xmax=1000,ymin=0,ymax=100,
    xlabel={$k$},
    ylabel={$y[k]$},
    legend pos=south east,
    y tick label style={/pgf/number format/1000 sep=},
    ] 
    \addlegendentry{$u$},
    \addlegendimage{no markers,blue}
    \addplot[blue, semithick] file{../data/lab/zad4/proba_3_input.csv};  
    \end{axis}
    \end{tikzpicture}
    \caption{Próba 3.: Przebieg sygnału sterującego podczas regulacji obiektu nieliniowego rozmytym regulatorem PID.}
    \label{pr_3_u}
\end{figure}


\subsubsection{Próba 4.}
Parametry regulatorów lokalnych:
\begin{enumerate}
\item $K = 18, T_{\mathrm{i}} = 300,  T_{\mathrm{d}} = 1$
\item $K = 19, T_{\mathrm{i}} = 250,  T_{\mathrm{d}} = 1$
\item $K = 20, T_{\mathrm{i}} = 200,  T_{\mathrm{d}} = 1$
\end{enumerate}
Próba czwarta również wypadła rozczarowująco. 
Wskaźnik jakości zwiększył się znacznie - $E = \num{49693}$. 
Obserwujemy powolne działanie i reakcje regulatora. 
Niepożądanym zjawiskiem są również oscylacje oraz ostre piki sygnału sterującego. 
Wyniki dla próby czwartej przedstawione na wykresach \ref{pr_4_y} i \ref{pr_4_u}.

\begin{figure}[t]
    \centering
    \begin{tikzpicture}
    \begin{axis}[
    width=0.98\textwidth,
    xmin=1,xmax=1000,ymin=30,ymax=55,
    xlabel={$k$},
    ylabel={$y[k]$},
    legend pos=south east,
    y tick label style={/pgf/number format/1000 sep=},
    ] 
    \addlegendentry{$y[k]$},
    \addlegendentry{$y^{\mathrm{zad}}$},
    \addlegendimage{no markers,red},
    \addlegendimage{no markers,blue}

    \addplot[red, semithick] file{../data/lab/zad4/proba_4_output.csv};
    \addplot[blue, semithick] file{../data/lab/zad4/proba_4_stpt.csv}; 
    \end{axis}
    \end{tikzpicture}
    \caption{Próba 4.: Przebieg procesu regulacji obiektu nieliniowego rozmytym regulatorem PID.}
    \label{pr_4_y}
\end{figure}

\begin{figure}[b]
    \centering
    \begin{tikzpicture}
    \begin{axis}[
    width=0.98\textwidth,
    xmin=1,xmax=1000,ymin=0,ymax=100,
    xlabel={$k$},
    ylabel={$y[k]$},
    legend pos=south east,
    y tick label style={/pgf/number format/1000 sep=},
    ] 
    \addlegendentry{$u$},
    \addlegendimage{no markers,blue}
    \addplot[blue, semithick] file{../data/lab/zad4/proba_4_input.csv};  
    \end{axis}
    \end{tikzpicture}
    \caption{Próba 4.: Przebieg sygnału sterującego podczas regulacji obiektu nieliniowego rozmytym regulatorem PID.}
    \label{pr_4_u}
\end{figure}

\subsubsection{Wnioski}
Strojenie rozmytego regulatora PID nie należy do zadań prostych.
Podczas czterech wykonanych prób udało nam się poprawić jakość regulacji jedynie w niewielkim stopniu. 
Zadanie to wymaga wielu eksperymentów i prób dla różnych parametrów każdego z regulatorów lokalnych. 
Podczas strojenia należy obserwować działanie regulatorów lokalnych w zależności od punktu pracy i 
dobieranie nastaw na podstawie obserwacji działania. 
Konieczne jest również dobre przemyślenie kolejnych kroków doboru parametrów. 
Warto uzbroić się również w cierpliwość. Najlepszym sposobem na dostrojenie regulatora PID jest 
pojedyncze strojenie każdego z regulatorów lokalnych w okolicy jego punktu pracy. Niestety, długi horyzont 
dynamiki obiektu wymusił na nas strojenie wszystkich regulatorów lokalnych równocześnie co sprawiło że 
żaden z badanych regulatorów nie spisał się choćby dostatecznie.
