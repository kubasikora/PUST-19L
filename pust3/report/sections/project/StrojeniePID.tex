\chapter{Dobór nastaw regulatora PID i DMC}
\label{pro4}

\section{Przebieg strojenia regulatora PID}
\label{pro4_PID}

\subsection{Próba dostrojenia regulatora metodą Zieglera - Nicholsa}
\label{pro4_PID_ZN}
W pierwszej kolejności spróbowaliśmy nastroić regulator PID klasyczną metodą Zieglera-Nicholsa.
W tym celu wykonaliśmy eksperyment polegający na pobudzeniu procesu sterowanego regulatorem 
proporcjonalnym, skokową zmianą sterowania. Poszukiwaliśmy takiego wzmocnienia krytycznego
$K_{\mathrm{Kr}}$ dla którego odpowiedź obiektu cechowała się niegasnącymi oscylacjami o 
okresie $T_{\mathrm{Kr}}$. Przebiegi eksperymentu zostały przedstawione na rysunkach 
\ref{pro_zad4_niegasnace_oscylacje_out} i \ref{pro_zad4_niegasnace_oscylacje_inp}.\\
\\
\indent Na podstawie przebiegów uzyskanych z eksperymentu udało nam się ustalić wartość wzmocnienia 
krytycznego $K_{\mathrm{Kr}} = \num{0,25}$ oraz okres oscylacji krytycznych $T_{\mathrm{Kr}} =
\num{14.5}s$. Na podstawie tabeli \ref{pro4_tabela_ZN} określiliśmy parametry regulatora PID:
$K = \num{0.15}$, $T_{\mathrm{i}} = \num{7.25}$ i $T_{\mathrm{d}} = \num{1.74}$

\begin{table}[h]
    \centering
    \begin{tabular}{|l|c|c|c|}
    \hline
    Typ & $K$ & $T_{\mathrm{i}}$ & $T_{\mathrm{d}}$ \\ \hline
    P   & $\num{0.5}K_{\mathrm{Kr}}$ & --- & --- \\
    PI  & $\num{0.45}K_{\mathrm{Kr}}$ & $\frac{T_{\mathrm{Kr}}}{\num{1.2}}$ & --- \\
    PID & $\num{0.6}K_{\mathrm{Kr}}$ & $\frac{T_{\mathrm{Kr}}}{\num{2}}$ & $\frac{T_{\mathrm{Kr}}}{\num{8}}$ \\ \hline
    \end{tabular}
    \caption{Tabela z proponowanymi nastawami regulatorów P, PI i PID zgodna z metodą Zieglera-Nicholsa}
    \label{pro4_tabela_ZN}
\end{table}

Mając obliczone parametry regulatora, przystąpiliśmy do zbadania jakości regulacji.
Badania przeprowadziliśmy dla jednolitej trajektorii zadanej.
Wskaźnik jakości regulacji wynosił $E = \num{980.75}$. Przebiegi wyjściowe i 
sterowania dla tego zestawu parametrów zostały przedstawione na wykresach
\ref{pro_zad4_ziegler_out} i \ref{pro_zad4_ziegler_inp}. 
Widzimy na nich niegasnące oscylacje, niepożądane podczas regulacji. \\
\\
\indent Jednogłośnie stwierdziliśmy że metoda ta, nie przyniosła zadowalających 
efektów. Uzyskiwane przebiegi charakteryzowały się niepożądanymi oscylacjami.
Z powodu niezadowalających rezultatów, postanowiliśmy ręcznie poprawić 
wartości parametrów.

\begin{figure}[t]
    \centering
    \begin{tikzpicture}
    \begin{axis}[
    width=0.98\textwidth,
    xmin=0.0,xmax=1000,ymin=-0.5,ymax=9,
    xlabel={$k$},
    ylabel={$y[k]$},
    legend pos=south east,
    y tick label style={/pgf/number format/1000 sep=},
    ] 
    \addlegendentry{$y[k]$},
    \addlegendentry{$y^{\mathrm{zad}}[k]$},

    \addlegendimage{no markers,red}
    \addlegendimage{no markers,blue}

    \addplot[red, semithick] file{../data/project/zad4/zad3PID_output_K_0.25_Ti_100000000000_Td_0.csv};  
    \addplot[blue, semithick] file{../data/project/zad4/zad3PID_setpoint_K_0.25_Ti_100000000000_Td_0.csv};
    
    \end{axis}
    \end{tikzpicture}
    \caption{Wyjście procesu podczas przeprowadzania eksperymentu Zieglera-Nicholsa}
    \label{pro_zad4_niegasnace_oscylacje_out}
\end{figure}

\begin{figure}[b]
    \centering
    \begin{tikzpicture}
    \begin{axis}[
    width=0.98\textwidth,
    xmin=0.0,xmax=1000,ymin=-1,ymax=1.2,
    xlabel={$k$},
    ylabel={$y[k]$},
    legend pos=south east,
    y tick label style={/pgf/number format/1000 sep=},
    ] 
    \addlegendentry{$u[k]$},
    
    \addlegendimage{no markers,blue}

    \addplot[const plot, blue, semithick] file{../data/project/zad4/zad3PID_input_K_0.25_Ti_100000000000_Td_0.csv};
    
    \end{axis}
    \end{tikzpicture}
    \caption{Przebieg sygnału sterującego podczas przeprowadzania eksperymentu Zieglera-Nicholsa}
    \label{pro_zad4_niegasnace_oscylacje_inp}
\end{figure}

\begin{figure}[t]
    \centering
    \begin{tikzpicture}
    \begin{axis}[
    width=0.98\textwidth,
    xmin=0.0,xmax=1000,ymin=-2,ymax=12.5,
    xlabel={$k$},
    ylabel={$y[k]$},
    legend pos=south east,
    y tick label style={/pgf/number format/1000 sep=},
    ] 
    \addlegendentry{$y[k]$},
    \addlegendentry{$y^{\mathrm{zad}}[k]$},
    \addlegendimage{no markers,red}
    \addlegendimage{no markers,blue}
    \addplot[red, semithick] file{../data/project/zad4/zad3PID_output_K_0.15_Ti_4.5_Td_1.08.csv};  
    \addplot[blue, semithick] file{../data/project/zad4/zad3PID_setpoint_K_0.15_Ti_4.5_Td_1.08.csv};
    \end{axis}
    \end{tikzpicture}
    \caption{Przebieg procesu sterowanego za pomocą regulatora z parametrami wyznaczonymi za pomocą metody Zieglera-Nicholsa}
    \label{pro_zad4_ziegler_out}
\end{figure}

\begin{figure}[b]
    \centering
    \begin{tikzpicture}
    \begin{axis}[
    width=0.98\textwidth,
    xmin=0.0,xmax=1000,ymin=-1.1,ymax=1.1,
    xlabel={$k$},
    ylabel={$y[k]$},
    legend pos=south east,
    y tick label style={/pgf/number format/1000 sep=},
    ] 
    \addlegendentry{$u[k]$},
    \addlegendimage{no markers,blue}
    \addplot[const plot, blue, semithick] file{../data/project/zad4/zad3PID_input_K_0.15_Ti_4.5_Td_1.08.csv};
    \end{axis}
    \end{tikzpicture}
    \caption{Przebieg sygnału sterującego regulatora z parametrami wyznaczonymi za pomocą metody Zieglera-Nicholsa}
    \label{pro_zad4_ziegler_inp}
\end{figure}
\FloatBarrier

\subsection{Ręczna poprawa nastawów regulatora PID}
\label{pro4_PID_metoda_inz}

\indent Próbując poprawić jakość regulacji zdecydowalismy się na zmniejszenie wszystkich 
parametrów regulatora badanego procesu.
Wartości początkowe parametrów przyjęliśmy mniejsze niż przy regulatorze uzyskanym w \ref{pro4_PID_ZN}
mianowicie $K = \num{0,05}$, $T_{\mathrm{i}} = \num{4.5}$ i $T_{\mathrm{d}} = \num{1.08}$. 
Przebiegi sygnałów dla tych wartości zaprezentowaliśmy na rysunkach \ref{pro4_pidinz_2out} i 
\ref{pro4_pidinz_2inp}. Wskaźnik jakości regulacji 
wyniósł $E = \num{3571.24}$. \\
\\
\indent Postanowiliśmy jeszcze bardziej zmniejszyć wartość
wzmocnienia  $K$ w celu zmniejszenia przeregulowania przy dużej wartości wyjścia procesu.. 
Wykresy \ref{pro4_pidinz_3out} i \ref{pro4_pidinz_3inp} 
prezentują przebiegi dla parametrów $K = \num{0,03}$, $T_{\mathrm{i}} = \num{4.5}$ 
i $T_{\mathrm{d}} = \num{1.08}$ Dla ostatnich nastawów uzyskaliśmy najmniejszy 
wskaźnik jakości regulacji równy $E = \num{1665.82}$. Udało się zniwelować 
niechciane przeregulowanie a wartości wyjścia dla dodatnich wartości zadanych osiągane są 
w zadowalającym czasie. Regulator ma spore problemy dla ujemnych wartości zadanych. W tym przypadku
liniowy regulator PID nie jest w stanie skutecznie regulować nieliniowym procesem w szybki sposób.

\begin{figure}[t]
    \centering
    \begin{tikzpicture}
    \begin{axis}[
    width=0.98\textwidth,
    xmin=0.0,xmax=1000,ymin=-2,ymax=12.5,
    xlabel={$k$},
    ylabel={$y[k]$},
    legend pos=south east,
    y tick label style={/pgf/number format/1000 sep=},
    ] 
    \addlegendentry{$y[k]$},
    \addlegendentry{$y^{\mathrm{zad}}[k]$},

    \addlegendimage{no markers,red}
    \addlegendimage{no markers,blue}

    \addplot[red, semithick] file{../data/project/zad4/zad3PID_output_K_0.05_Ti_4.5_Td_1.08.csv};  
    \addplot[blue, semithick] file{../data/project/zad4/zad3PID_setpoint_K_0.05_Ti_4.5_Td_1.08.csv};
    
    \end{axis}
    \end{tikzpicture}
    \caption{Przebieg procesu sterowanego za pomocą regulatora z parametrami $K = \num{0,05}$, $T_{\mathrm{i}} = \num{4.5}$ i $T_{\mathrm{d}} = \num{1.08}$}
    \label{pro4_pidinz_2out}
\end{figure}

\begin{figure}[b]
    \centering
    \begin{tikzpicture}
    \begin{axis}[
    width=0.98\textwidth,
    xmin=0.0,xmax=1000,ymin=-0.4,ymax=1.1,
    xlabel={$k$},
    ylabel={$y[k]$},
    legend pos=south east,
    y tick label style={/pgf/number format/1000 sep=},
    ] 
    \addlegendentry{$u[k]$},
    
    \addlegendimage{no markers,blue}

    \addplot[const plot, blue, semithick] file{../data/project/zad4/zad3PID_input_K_0.05_Ti_4.5_Td_1.08.csv};
    
    \end{axis}
    \end{tikzpicture}
    \caption{Przebieg sygnału sterującego przy parametrach $K = \num{0,05}$, $T_{\mathrm{i}} = \num{4.5}$ i $T_{\mathrm{d}} = \num{1.08}$}
    \label{pro4_pidinz_2inp}
\end{figure}

\begin{figure}[t]
    \centering
    \begin{tikzpicture}
    \begin{axis}[
    width=0.98\textwidth,
    xmin=0.0,xmax=1000,ymin=-2,ymax=12.5,
    xlabel={$k$},
    ylabel={$y[k]$},
    legend pos=south east,
    y tick label style={/pgf/number format/1000 sep=},
    ] 
    \addlegendentry{$y[k]$},
    \addlegendentry{$y^{\mathrm{zad}}[k]$},

    \addlegendimage{no markers,red}
    \addlegendimage{no markers,blue}

    \addplot[red, semithick] file{../data/project/zad4/zad3PID_output_K_0.03_Ti_4.5_Td_1.08.csv};  
    \addplot[blue, semithick] file{../data/project/zad4/zad3PID_setpoint_K_0.03_Ti_4.5_Td_1.08.csv};
    
    \end{axis}
    \end{tikzpicture}
    \caption{Przebieg procesu sterowanego za pomocą regulatora z parametrami $K = \num{0,03}$, $T_{\mathrm{i}} = \num{4.5}$ i $T_{\mathrm{d}} = \num{1.08}$}
    \label{pro4_pidinz_3out}
\end{figure}

\begin{figure}[b]
    \centering
    \begin{tikzpicture}
    \begin{axis}[
    width=0.98\textwidth,
    xmin=0.0,xmax=1000,ymin=-0.4,ymax=1.1,
    xlabel={$k$},
    ylabel={$y[k]$},
    legend pos=south east,
    y tick label style={/pgf/number format/1000 sep=},
    ] 
    \addlegendentry{$u[k]$},
    
    \addlegendimage{no markers,blue}

    \addplot[const plot, blue, semithick] file{../data/project/zad4/zad3PID_input_K_0.03_Ti_4.5_Td_1.08.csv};
    
    \end{axis}
    \end{tikzpicture}
    \caption{Przebieg sygnału sterującego przy parametrach $K = \num{0,03}$, $T_{\mathrm{i}} = \num{4.5}$ i $T_{\mathrm{d}} = \num{1.08}$}
    \label{pro4_pidinz_3inp}
\end{figure}
\FloatBarrier
