\chapter{Badanie odpowiedzi skokowych omawianego obiektu nieliniowego}
\label{pro2}


\section{Odpowiedzi skokowe}
Za pomocą skryptu \verb+zad2.m+ zebralismy dziesięć odpowiedzi skokowych, 
równomiernie badając dziedzinę dopuszczalnego sterowania. Zostały one 
przedstawione na rysunku \ref{pro2_skoki_nieznormalizowane}.
Już na podstawie tego wykresu można stwierdzić że obiekt nie jest liniowy. 
O wiele więcej możemy wynieść po normalizacji każdej z odpowiedzi. Porównanie 
znormalizowanych odpowiedzi znajduje się na rysunku \ref{pro2_skoki_znormalizowane}.
Na podstawie przebiegów odpowiedzi skokowych można stwierdzić że właściowści statyczne 
obiektu są nieliniowe. Wzmocnienie statyczne obiektu różni się w zależności punktu pracy.
Mimo to, właściowści dynamiczne obiektu są liniowe. Czas narastania nie zależy od punktu 
pracy i jest stały w każdym przypadku.

\begin{figure}[b]
    \centering
    \begin{tikzpicture}
    \begin{axis}[
    width=0.98\textwidth,
    xmin=0.0,xmax=200,ymin=-1.5,ymax=12,
    xlabel={$k$},
    ylabel={$y[k]$},
    legend pos=south east,
    y tick label style={/pgf/number format/1000 sep=},
    ] 
    \addlegendentry{$\Delta u = \num{-1}$},
    \addlegendentry{$\Delta u = \num{-0.8}$},
    \addlegendentry{$\Delta u = \num{-0.6}$},
    \addlegendentry{$\Delta u = \num{-0.4}$},
    \addlegendentry{$\Delta u = \num{-0.2}$},
    \addlegendentry{$\Delta u = \num{0.2}$},
    \addlegendentry{$\Delta u = \num{0.4}$},
    \addlegendentry{$\Delta u = \num{0.6}$},
    \addlegendentry{$\Delta u = \num{0.8}$},
    \addlegendentry{$\Delta u = \num{1}$},

    \addlegendimage{no markers,red}
    \addlegendimage{no markers,brown}
    \addlegendimage{no markers,blue}
    \addlegendimage{no markers,cyan}
    \addlegendimage{no markers,magenta}
    \addlegendimage{no markers,yellow}
    \addlegendimage{no markers,gray}
    \addlegendimage{no markers,green}
    \addlegendimage{no markers,lime}
    \addlegendimage{no markers,orange}

    \addplot[red, semithick] file{../data/project/zad2/odp_skok_u=-1.csv}; 
    \addplot[brown, semithick] file{../data/project/zad2/odp_skok_u=-0.8.csv}; 
    \addplot[blue, semithick] file{../data/project/zad2/odp_skok_u=-0.6.csv}; 
    \addplot[cyan, semithick] file{../data/project/zad2/odp_skok_u=-0.4.csv}; 
    \addplot[magenta, semithick] file{../data/project/zad2/odp_skok_u=-0.2.csv}; 
    \addplot[yellow, semithick] file{../data/project/zad2/odp_skok_u=0.2.csv}; 
    \addplot[gray, semithick] file{../data/project/zad2/odp_skok_u=0.4.csv}; 
    \addplot[green, semithick] file{../data/project/zad2/odp_skok_u=0.6.csv}; 
    \addplot[lime, semithick] file{../data/project/zad2/odp_skok_u=0.8.csv}; 
    \addplot[orange, semithick] file{../data/project/zad2/odp_skok_u=1.csv}; 
    
    \end{axis}
    \end{tikzpicture}
    \caption{Odpowiedzi procesu na skokową zmianę sygnału sterującego}
    \label{pro2_skoki_nieznormalizowane}
\end{figure}


\begin{figure}[t]
    \centering
    \begin{tikzpicture}
    \begin{axis}[
    width=0.98\textwidth,
    xmin=0.0,xmax=200,ymin=0,ymax=12,
    xlabel={$k$},
    ylabel={$s_{k}$},
    legend pos=south east,
    y tick label style={/pgf/number format/1000 sep=},
    ] 
    \addlegendentry{$\Delta u = \num{-1}$},
    \addlegendentry{$\Delta u = \num{-0.8}$},
    \addlegendentry{$\Delta u = \num{-0.6}$},
    \addlegendentry{$\Delta u = \num{-0.4}$},
    \addlegendentry{$\Delta u = \num{-0.2}$},
    \addlegendentry{$\Delta u = \num{0.2}$},
    \addlegendentry{$\Delta u = \num{0.4}$},
    \addlegendentry{$\Delta u = \num{0.6}$},
    \addlegendentry{$\Delta u = \num{0.8}$},
    \addlegendentry{$\Delta u = \num{1}$},

    \addlegendimage{no markers,red}
    \addlegendimage{no markers,brown}
    \addlegendimage{no markers,blue}
    \addlegendimage{no markers,cyan}
    \addlegendimage{no markers,magenta}
    \addlegendimage{no markers,yellow}
    \addlegendimage{no markers,gray}
    \addlegendimage{no markers,green}
    \addlegendimage{no markers,lime}
    \addlegendimage{no markers,orange}

    \addplot[red, semithick] file{../data/project/zad2/odp_skok_norm_u=-1.csv}; 
    \addplot[brown, semithick] file{../data/project/zad2/odp_skok_norm_u=-0.8.csv}; 
    \addplot[blue, semithick] file{../data/project/zad2/odp_skok_norm_u=-0.6.csv}; 
    \addplot[cyan, semithick] file{../data/project/zad2/odp_skok_norm_u=-0.4.csv}; 
    \addplot[magenta, semithick] file{../data/project/zad2/odp_skok_norm_u=-0.2.csv}; 
    \addplot[yellow, semithick] file{../data/project/zad2/odp_skok_norm_u=0.2.csv}; 
    \addplot[gray, semithick] file{../data/project/zad2/odp_skok_norm_u=0.4.csv}; 
    \addplot[green, semithick] file{../data/project/zad2/odp_skok_norm_u=0.6.csv}; 
    \addplot[lime, semithick] file{../data/project/zad2/odp_skok_norm_u=0.8.csv}; 
    \addplot[orange, semithick] file{../data/project/zad2/odp_skok_norm_u=1.csv}; 
    
    \end{axis}
    \end{tikzpicture}
    \caption{Znormalizowane odpowiedzi skokowe badanego procesu}
    \label{pro2_skoki_znormalizowane}
\end{figure}

\section{Charakterystyka statyczna badanego obiektu}
\label{pro2_zebranie_char_stat}

Druga część skryptu \verb+zad.m+ wyznaczała charakterystykę statyczną na całym dopuszczalnym
przedziale. Kształt tej charakterystyki, przedstawionej na \ref{pro2_char_stat}, jednoznacznie
potwierdza nieliniowość badanego procesu.

\begin{figure}[b]
    \centering
    \begin{tikzpicture}
    \begin{axis}[
    width=\textwidth,
    xmin=-1,xmax=1,ymin=-2,ymax=12,
    xlabel={$u$},
    ylabel={$y(u)$},
    legend pos=south east,
    y tick label style={/pgf/number format/1000 sep=},
    ]
    \addplot[blue, semithick] file{../data/project/zad2/char_stat_long.csv};
    \legend{$y(u)$}
    \end{axis} 
    \end{tikzpicture}
    \caption{Charakterystyka statyczna badanego obiektu}
    \label{pro2_char_stat}
\end{figure}
