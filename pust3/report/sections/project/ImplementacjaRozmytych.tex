\chapter{Implementacja rozmytych regulatorów PID i DMC}
\label{pro5}

\section{Rozmyty regulator PID}
\subsection{Implementacja}
Program zawierający rozmyty algorytm PID rozpoczyna się przypisaniem zadanej przez użytkownika ilości lokalnych regulatorów oraz wyboru funkcji przynależności jako jednej z dostępnych: \verb+trapezoid+, \verb+bell+, \verb+triangle+.

\begin{lstlisting}[style=custommatlab,frame=single]
LOCAL_REGS = 2;
membershipFunction = 'trapezoid'; 
\end{lstlisting}

Następnie w programie tworzona jest macierz parametrów dla lokalnych regulatorów PID - \verb+fuzzyMatrix+ (pod warunkiem, że nie istnieje już macierz o odpowiednim rozmiarze).

\begin{lstlisting}[style=custommatlab,frame=single]
if exist('fuzzyMatrix', 'var') == 0 || size(fuzzyMatrix,1) ~= LOCAL_REGS
    fuzzyMatrix = fuzzyPIDParameters(LOCAL_REGS);
end
\end{lstlisting}

Funkcja \verb+fuzzyPIDParameters+ dobiera odpowiednie parametry dla każdego regulatora lokalnego: $K, T_{\mathrm{i}}, T_{\mathrm{d}}$  poprzez wykorzystanie optymalizacji gradientowej w funkcji \verb+fmincon+. Funkcja rozpoczyna się doborem punktu pracy dla każdego regulatora. Wykorzystujemy do tego optymalizację gradientową i funkcję \verb+upp_target.m+, która dopasowuje sterowanie do wyliczonego wcześniej wyjścia na podstawie błędu zwracanego przez funkcję. Wyznaczenie punktów pracy przedstawia listing \ref{PID_pkt_pracy}. 
\\

\begin{lstlisting}[style=custommatlab,frame=single,label={PID_pkt_pracy},caption={Funkcja dobierająca punktów pracy},captionpos=b]
function FUZZY_PID_PARAMETERS = fuzzyPIDParameters(REGULATOR_NUM)
options = optimoptions('fmincon', 'Algorithm', 'sqp', 'Display', 'iter', 
			'MaxFunctionEvaluations', 600);
addpath ../ 
Y_MIN = -0.31546;
Y_MAX = 11.839;
DY = Y_MAX - Y_MIN;

OFFSET = DY/(REGULATOR_NUM+1);

FUZZY_YPPs = zeros(REGULATOR_NUM,1);
FUZZY_UPPs = zeros(REGULATOR_NUM,1);

for i=1:REGULATOR_NUM
    FUZZY_YPPs(i) = Y_MIN + (i*OFFSET);
    FUZZY_UPPs(i) = fmincon(@(Upp)upp_target(Upp, FUZZY_YPPs(i)), 0, [], [], 
    			[], [], -1, 1, [], options);

end
\end{lstlisting}

Po zdefiniowaniu ograniczeń dla parametrów regulatorów następuje symulacja dla każdego regulatora w odpowiadającym mu punkcie pracy. Do tego również wykorzystywana jest funkcja \verb+fmincon+, która z kolei wywołuje funkcje \verb+pid_target_fun+ będącą faktycznie symulacją regulatora PID i zwracającą wartość wskaźnika jakości. Dobór parametrów przedstawia listing \ref{PID_param}. Po wyliczeniu parametrów następuje symulacja dla danego regulatora i zapisanie parametrów oraz punktów pracy do zwracanej przez funkcję macierzy \verb+FUZZY_PID_PARAMETERS+. 
\\

\begin{lstlisting}[style=custommatlab,frame=single,label={PID_param},caption={Funkcja dobierająca parametry regulatorów lokalnych},captionpos=b]
for i=1:REGULATOR_NUM

Ypp = FUZZY_YPPs(i);
Upp = FUZZY_UPPs(i);

base_values = [0.5 7.25 1.62];

parameters = fmincon(@(parameters)pid_target_fun(parameters, Upp, Ypp), 
			base_values, [], [], [], [], [K_min Ti_min Td_min], 
					[K_max Ti_max Td_max], [], options);

% ...

FUZZY_PID_PARAMETERS(i,1:3) = parameters;
FUZZY_PID_PARAMETERS(i,4) = FUZZY_UPPs(i);
FUZZY_PID_PARAMETERS(i,5) = FUZZY_YPPs(i);
end

end
\end{lstlisting}

Następnym krokiem w implementacji rozmytego algorytmu PID była inicjalizacja wektorów przechowujących zmienne procesowe. Kolejno zaimplementowana została pętla symulująca działanie regulatora rozmytego. W programie zostały utworzone dodatkowe wektory \verb+local_inputs+ i \verb+memberships+ przechowujące kolejno wartości sterowań generowanych przez regulatory lokalne i wagi z jakimi będą one uwzględniane. Pętla składa się z kilku etapów:
\begin{enumerate}
\item Pomiar wyjścia obiektu oraz obliczenie uchybu,
\item Pętla, która dla każdego regulatora pobiera parametry oraz oblicza sterowanie zapisywane w wektorze \verb+local_inputs+,
\item Pętla obliczająca wartości funkcji przynależnośći dla każdego regulatora lokalnego,
\item Pętla sumująca sterowania regulatorów przemnożone przez ich wagi,
\item Dzielenie sterowania przez sumę wszystkich wag,
\item Uwzględnienie ograniczeń i obliczenie wskaźnika jakości.
\end{enumerate}

\begin{lstlisting}[style=custommatlab,frame=single,label={FPID_sim},caption={Symulacja rozmytego algorytmu PID},captionpos=b]
local_inputs = zeros(LOCAL_REGS,1);
memberships = zeros(LOCAL_REGS,1);
%% symulacja obiektu
for k = 7:SIM_LEN
    output(k) = symulacja_obiektu1y(input(k-5), input(k-6),
    	 output(k-1), output(k-2));    % pomiar wyjscia
    error(k) = stpt(k) - output(k);   % obliczenie uchyby
    
    for i=1:LOCAL_REGS
        parameters = fuzzyMatrix(i, 1:3);
        K = parameters(1);
        Ti = parameters(2);
        Td = parameters(3);
        r0 = K*(1 + T/(2*Ti) + Td/T);
        r1 = K*(T/(2*Ti) - (2*Td)/T - 1);
        r2 = (K*Td)/T;
        
        local_inputs(i) = r2 * error(k-2) + r1 * error(k-1) + r0 * error(k) 
        		+ input(k-1);  % wyliczenie lokalnego sterowania
    end
    
    for i=1:LOCAL_REGS
        memberships(i) = eval(strcat(membershipFunction, 
        		'(output(k), fuzzyMatrix(i,5));'));
    end
    
    for i=1:LOCAL_REGS
        input(k) = input(k) + local_inputs(i)*memberships(i);
    end
    
    if sum(memberships) ~= 0
        input(k) = input(k)/sum(memberships);
    end
    
    if input(k) >= Umax
        input(k) = Umax;
    elseif input(k) <= Umin
        input(k) = Umin;
    end
    
    error_sum = error_sum + error(k)^2;
end
\end{lstlisting}
\subsection{Zmienna wybierająca}
Jako zmienną wybierającą przyjęliśmy sygnał wyjściowy w chwili $k$.

\section{Rozmyty regulator DMC}
Implementacja rozmytego algorytmu DMC była zadaniem o wiele prostszym, ze względu na jednakowe parametry każdego regulatora lokalnego.\\
\indent{} Dobór punktów pracy przebiegał podobnie jak w przypadku regulatora PID - z wykorzysaniem funkcji \verb+fmincon+ oraz funkcji \verb+upp_target+. Po inicjalizacji potrzebnych zmiennych takich jak parametry algorytmu oraz macierze nastąpiło wyliczenie poszczególnych elementów każdego z regulatorów lokalnych.\\
\indent{} W pętli wykonywane były eksperymenty mające na celu pozyskanie odpowiedzi skokowych dla każdego, wcześniej wyliczonego punktu pracy. Następnie wyliczane były macierze: $M, M^{\mathrm{P}}, K$ oraz wartości $k_{e}$ i wektory $k_{u}$. Wyliczenie macierzy algorytmu dla każdego regulatora lokalnego przedstawia listing \ref{DMC_matrices}.\\


\begin{lstlisting}[style=custommatlab,frame=single,label={DMC_matrices},caption={Wyliczenie macierzy algorytmu DMC dla regulatorów lokalnych},captionpos=b]
for i = 1:REGULATOR_NUM
    M = zeros(N, Nu);
    Mp = zeros(N, D-1);

%% symulacja obiektu w celu otrzymania odp skokowej dla danego pkt pracy    
    step = simulate_steps(FUZZY_UPPs(i), FUZZY_YPPs(i));
    STEPS(:,i) = step(1:D);
    
%% wyliczenie macierzy 
    % macierz M
    for p = 1:Nu
        k = 1;
        for q = 1:N
            if q < p
                M(q, p) = 0;
            else
                M(q, p) = step(k);
                k = k + 1;
            end
        end
    end

    % macierz Mp
    for p = 1:N
        for q = 1:(D-1)
            k = p + q;
            if k > D
                Mp(p,q) = step(D) - step(q);
            else
                Mp(p,q) = step(k) - step(q);
            end
        end
    end

    % macierz K
    K = inv((M' * M + lambda(i) * eye(Nu)))*M';
    Ke(i) = sum(K(1,:));
    Ku(i,:) = K(1,:) * Mp;
end    
\end{lstlisting}

Pętla symulująca proces regulacji ponownie podzielona była na kilka etapów:

\begin{enumerate}
\item Wyliczenie wektora $\Delta U^{\mathrm{p}}$,
\item Pomiar wyjścia obiektu i obliczenie uchybu,
\item Obliczenie sterowań lokalnych regulatorów z wykorzystaniem wcześniej obliczonych $k_{e}$ i $k_{u}$,
\item Wyliczenie wag dla poszczególnych sterowań,
\item Pętla sumująca sterowania regulatorów przemnożone przez ich wagi,
\item Dzielenie sterowania przez sumę wszystkich wag,
\item Uwzględnienie ograniczeń i obliczenie wskaźnika jakości.
\end{enumerate}

Pętlę symulującą proces regulacji zawiera listing \ref{DMC_sim}.

\begin{lstlisting}[style=custommatlab,frame=single,label={DMC_sim},caption={Pętla symulująca proces regulacji z wykorzystaniem rozmytego algorytmu DMC.},captionpos=b]
for k=7:SIM_LEN    
    % wektor dUp
    for i = 1:(D-1)
        if (k-i) <= 0
            du1 = 0;
        else
            du1 = input(k - i);
        end
        if (k-i-1) <= 0
            du2 = 0;
        else
            du2 = input(k - i - 1);
        end 
        dUp(i) = du1 - du2;
    end
    
    
    output(k) = symulacja_obiektu1y(input(k-5), input(k-6), 
    	output(k-1), output(k-2)); 
    error(k) = stpt(k) - output(k);   
    error_sum = error_sum + error(k)^2;
    
    for i=1:REGULATOR_NUM        
        local_inputs(i,k) = Ke(i) * error(k) - Ku(i, :) * dUp; 
        if local_inputs(i,k) >= Umax
            local_inputs(i,k) = Umax;
        elseif local_inputs(i,k) <= Umin
            local_inputs(i,k) = Umin;
        end
    end
    
    for i=1:REGULATOR_NUM      
        local_inputs(i,k) = Ke(i) * error(k) - Ku(i, :) * dUp; 
        if local_inputs(i,k) >= Umax
            local_inputs(i,k) = Umax;
        elseif local_inputs(i,k) <= Umin
            local_inputs(i,k) = Umin;
        end
    end
    
    for i=1:REGULATOR_NUM
        memberships(i, k) = trapezoid(stpt(k), FUZZY_YPPs(i));
    end
    
    input(k) = 0;
    for i=1:REGULATOR_NUM
        input(k) = input(k) + (local_inputs(i,k) * memberships(i,k));
        
    end
    
    if sum(memberships(:,k)) ~= 0
        input(k) = input(k)/sum(memberships(:,k));
    end
    
    input(k) = input(k) + input(k-1);
    
    if input(k) >= Umax
        input(k) = Umax;
    elseif input(k) <= Umin
        input(k) = Umin;
    end
end
\end{lstlisting}