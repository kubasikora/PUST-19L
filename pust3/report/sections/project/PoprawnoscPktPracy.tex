\chapter{Sprawdzenie poprawności punktu pracy}
\label{pro1}

Aby sprawdzić poprawność podanego punktu pracy wykonaliśmy ekspe-
ryment polegający na pobudzeniu wejścia obiektu stałym sygnałem wejściowym o wartości
$U_{\mathrm{pp}} = \num{0}$ i sprawdzeniu czy sygnał wyjściowy stabilizuje się 
na wartości $Y_{\mathrm{pp}} = \num{0}$. Symulację obiektu przeprowadziliśmy
za pomocą funkcji \verb+symulacja_obiektu1y+. 

Eksperyment przeprowadziliśmy uruchamiając dostarczoną funkcję z następującymi parametrami:

\begin{center}
\verb+symulacja_obiektu1y(Upp, Upp, Ypp, Ypp)+ 
\end{center}

Uzyskany wynik \verb+ans+ $= \num{0}$ jest równy co do wartości $Y_{\mathrm{pp}}$ co potwierdza poprawność podanego
punktu pracy.

\begin{figure}[b]
    \centering
    \begin{tikzpicture}
    \begin{axis}[
    width=\textwidth,
    xmin=0,xmax=200,ymin=-2,ymax=2,
    xlabel={$k$},
    ylabel={$y[k]$},
    legend pos=south east,
    y tick label style={/pgf/number format/1000 sep=},
    ]
    \addplot[red, semithick] file{../data/project/zad1/zad1_output.csv};
    \legend{$y[k]$, }
    \end{axis} 
    \end{tikzpicture}
    \caption{Przebieg wyjścia obiektu sterowanego sygnałem stałym o wartości \mbox{$U_{\mathrm{pp}} = \num{0}$}}
    \label{pro1}
\end{figure}