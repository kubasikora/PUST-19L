\section{Przebieg strojenia regulatora DMC}
\label{pro4_DMC}
Dla badanego obiektu, spróbowaliśmy dobrać parametry regulatora predykcyjnego DMC, który 
sterując danym procesem minimalizowałby uchyby wyjścia. 
W pierwszej kolejności założyliśmy że wszystkie horyzonty
$D, N, N_{\mathrm{u}}$ są równe i wynoszą $D = N = N_{\mathrm{u}} = \num{303}$. 
Dodatkowo, założylismy że wartość parametru kary będzie równa $\lambda = 1$.
Tak zaprojektowany regulator miał problemy z osiągnięciem wartości zadanej, 
a także charakteryzował się niegasnącymi oscylacjami. Przebiegi wyjściowe procesu 
i generowane sterowanie przez regulator zostały przedstawione na rysunkach  
\ref{pro_zad4_d_303_n_303_nu_303_out} i \ref{pro_zad4_d_303_n_303_nu_303_inp}). Dla 
przeprowadzonego eksperymentu uzyskaliśmy wskaźnik jakości równy $E = \num{1829.6}$.

Oscylacyjny charaketer przebiegu wyjściowego obiektu był dla nas alarmujący.
Stwierdziliśmy że charakter wynika z szybkozmiennego przebiegu sygnału sterującego.
Z tego powodu zdecydowaliśmy się podnieść wartość współczynnika kary $\lambda$. 
Zrobiliśmy to trzykrotnie, najpierw do wartości $\num{100}$, potem $\num{1000}$ i 
skończyliśmy na wielkości $\lambda = \num{10000})$. 
Dla ostatniej próby przebiegi sygnałów prezentowały się w naszych oczach najlepiej.

Wskaźnik jakości zaskakująco rośnie wraz z zwiększaniem parametru $\lambda$. Dzieje się
tak ponieważ całkowany błąd rośnie z powodu wolniejszego nadążania wyjścia do wartości zadanej.
Wskaźnik jakości zdefiniowany jako całka z kwadratów uchybów faworyzuje oscylujące regulatory,
ponieważ suma kwadratów małych uchybów pozostających przez długi czas i tak będzie mniejsza niż 
suma kwadratów dużych uchybów występujących przez znacznie krótszy okres czasu. W tym przypadku 
większy nacisk kładlismy na czas regulacji, czyli okres po którym wyjście procesu stabilizuje się na 
wartości zadanej, dlatego tez naturalnym jest fakt że ilościowy wskaźnik jakości pogarsza się, mimo iż rzeczywista
niemierzalna jakość regulacji oceniania w sposób jakościowy polepsza się.  

\begin{table}[h]
    \centering
    \begin{tabular}{|c|c|}
    \hline
    Wartość parametru $\lambda$& $K$ \\ \hline
    $\num{100}$ & $\num{1033.9}$ \\
    $\num{1000}$ & $\num{1212.8}$ \\
    $\num{10000}$& $\num{1660.8}$  \\ \hline
    \end{tabular}
    \caption{Wartości wskaźnika jakości przy strojeniu parametru kary $\lambda$}
    \label{pro4_dmc_lambdy}
\end{table}

W kolejnym kroku przeszliśmy do dobrania wartości parametrów $ N, N_{\mathrm{u}}$.
Zaczęliśmy od zmniejszenia wartości parametrów $N$ i $N_{\mathrm{u}}$ do $\num{220}$, 
a potem do $\num{50}$. Nie wpłynęło to w sposób znaczny na poprawę przebiegów sygnałów. 
Możemy to zauważyć na rysunkach \ref{pro_zad4_d_303_n_220_nu_220_out} i 
\ref{pro_zad4_d_303_n_220_nu_220_inp}. Wskaźnik jakości regulacji dla takiej samej trajektorii zadanej wyniósł 
$E = \num{1680,8}$. Ostatecznie zdecydowaliśmy się na pozostawienie
horyzontu predykcji równego $N = \num{50}$ i 
przeszliśmy do doboru wartości $N_{u}$. Horyzont sterowania zmniejszyliśmy do $\num{10}$, 
o nie spowodowało znaczących zmian w przebiegu regulacji. 
Po porównaniu wykresów dla wartości $N_{u} = 50$ i $N_{u} = 10$ 
zobaczymy, że wyjście procesu przy skróconym horyzoncie sterowaniu szybciej osiąga wartość zadaną, co potwierdza
lepszy wskaźnik jakości regulacji $E = \num{1633,6}$.

Podczas strojenia regulatora DMC doszliśmy do wniosku, że największy wpływ na przebieg regulacji miała wartość 
współczynnika kary. Zwiększenie parametru $\lambda$ pozwoliło na znaczącą poprawę regulacji. Mimo to, regulator w
każdym badanym przypadku słabo radził sobie z ujemnymi wartościami zadanymi. Wynika to z faktu, iż tak jak regulator
PID, regulator DMC jest liniowym algorytmem regulacji i nie reguluje nieliniowych obiektów w zadowalający sposób. 
Warto jednak wspomnieć że regulator DMC jest odporny na zmianę modelu dlatego jeżeli nie jest to znaczna zmiana 
charakterystyki obiektu, to algorytm DMC powinien dać sobie radę z regulacją takiego procesu. W tym przypadku nieliniowość
obiektu była zbyt duża aby regulator w sensowny sposób obsługiwał ujemne wartości zadane. 

\begin{figure}[t]
    \centering
    \begin{tikzpicture}
    \begin{axis}[
    width=0.98\textwidth,
    xmin=0.0,xmax=1000,ymin=-2,ymax=12.5,
    xlabel={$k$},
    ylabel={$y[k]$},
    legend pos=south east,
    y tick label style={/pgf/number format/1000 sep=},
    ] 
    \addlegendentry{$y[k]$},
    \addlegendentry{$y^{\mathrm{zad}}[k]$},

    \addlegendimage{no markers,red}
    \addlegendimage{no markers,blue}

    \addplot[red, semithick] file{../data/project/zad4/zad4DMC_output_D_303_N_303_Nu_303_lam_1.csv};  
    \addplot[blue, semithick] file{../data/project/zad4/zad4DMC_setpoint_D_303_N_303_Nu_303_lam_1.csv};
    
    \end{axis}
    \end{tikzpicture}
    \caption{Przebieg procesu sterowanego za pomocą regulatora z parametrami $K = \num{303}$, $N = \num{303}$, $N_{\mathrm{u}} = \num{303}$ i $\lambda = 1$}
    \label{pro_zad4_d_303_n_303_nu_303_out}
\end{figure}

\begin{figure}[b]
    \centering
    \begin{tikzpicture}
    \begin{axis}[
    width=0.98\textwidth,
    xmin=0.0,xmax=1000,ymin=-1.2,ymax=1.2,
    xlabel={$k$},
    ylabel={$y[k]$},
    legend pos=south east,
    y tick label style={/pgf/number format/1000 sep=},
    ] 
    \addlegendentry{$u[k]$},
    
    \addlegendimage{no markers,blue}

    \addplot[blue, semithick] file{../data/project/zad4/zad4DMC_input_D_303_N_303_Nu_303_lam_1.csv};
    
    \end{axis}
    \end{tikzpicture}
    \caption{Przebieg sygnału sterującego przy parametrach $D = \num{303}$, $N = \num{303}$, $N_{\mathrm{u}} = \num{303}$ i $\lambda = 1$}
    \label{pro_zad4_d_303_n_303_nu_303_inp}
\end{figure}


\begin{figure}[t]
    \centering
    \begin{tikzpicture}
    \begin{axis}[
    width=0.98\textwidth,
    xmin=0.0,xmax=1000,ymin=-2,ymax=12.5,
    xlabel={$k$},
    ylabel={$y[k]$},
    legend pos=south east,
    y tick label style={/pgf/number format/1000 sep=},
    ] 
    \addlegendentry{$y[k]$},
    \addlegendentry{$y^{\mathrm{zad}}[k]$},

    \addlegendimage{no markers,red}
    \addlegendimage{no markers,blue}

    \addplot[red, semithick] file{../data/project/zad4/zad4DMC_output_D_303_N_303_Nu_303_lam_100.csv};  
    \addplot[blue, semithick] file{../data/project/zad4/zad4DMC_setpoint_D_303_N_303_Nu_303_lam_100.csv};
    
    \end{axis}
    \end{tikzpicture}
    \caption{Przebieg procesu sterowanego za pomocą regulatora z parametrami $K = \num{303}$, $N = \num{303}$, $N_{\mathrm{u}} = \num{303}$ i $\lambda = 100$}
    \label{pro_zad4_d_303_n_303_nu_303_lam_100out}
\end{figure}

\begin{figure}[b]
    \centering
    \begin{tikzpicture}
    \begin{axis}[
    width=0.98\textwidth,
    xmin=0.0,xmax=1000,ymin=-1.2,ymax=1.2,
    xlabel={$k$},
    ylabel={$y[k]$},
    legend pos=south east,
    y tick label style={/pgf/number format/1000 sep=},
    ] 
    \addlegendentry{$u[k]$},
    
    \addlegendimage{no markers,blue}

    \addplot[blue, semithick] file{../data/project/zad4/zad4DMC_input_D_303_N_303_Nu_303_lam_100.csv};
    
    \end{axis}
    \end{tikzpicture}
    \caption{Przebieg sygnału sterującego przy parametrach $D = \num{303}$, $N = \num{303}$, $N_{\mathrm{u}} = \num{303} $ i $\lambda = 100$}
    \label{pro_zad4_d_303_n_303_nu_303_lam_100inp}
\end{figure}

\begin{figure}[t]
    \centering
    \begin{tikzpicture}
    \begin{axis}[
    width=0.98\textwidth,
    xmin=0.0,xmax=1000,ymin=-2,ymax=12.5,
    xlabel={$k$},
    ylabel={$y[k]$},
    legend pos=south east,
    y tick label style={/pgf/number format/1000 sep=},
    ] 
    \addlegendentry{$y[k]$},
    \addlegendentry{$y^{\mathrm{zad}}[k]$},

    \addlegendimage{no markers,red}
    \addlegendimage{no markers,blue}

    \addplot[red, semithick] file{../data/project/zad4/zad4DMC_output_D_303_N_303_Nu_303_lam_1000.csv};  
    \addplot[blue, semithick] file{../data/project/zad4/zad4DMC_setpoint_D_303_N_303_Nu_303_lam_1000.csv};
    
    \end{axis}
    \end{tikzpicture}
    \caption{Przebieg procesu sterowanego za pomocą regulatora z parametrami $K = \num{303}$, $N = \num{303}$, $N_{\mathrm{u}} = \num{303}$ i $\lambda = 1000$}
    \label{pro_zad4_d_303_n_303_nu_303_lam_1000out}
\end{figure}

\begin{figure}[b]
    \centering
    \begin{tikzpicture}
    \begin{axis}[
    width=0.98\textwidth,
    xmin=0.0,xmax=1000,ymin=-1,ymax=1.2,
    xlabel={$k$},
    ylabel={$y[k]$},
    legend pos=south east,
    y tick label style={/pgf/number format/1000 sep=},
    ] 
    \addlegendentry{$u[k]$},
    
    \addlegendimage{no markers,blue}

    \addplot[blue, semithick] file{../data/project/zad4/zad4DMC_input_D_303_N_303_Nu_303_lam_1000.csv};
    
    \end{axis}
    \end{tikzpicture}
    \caption{Przebieg sygnału sterującego przy parametrach $D = \num{303}$, $N = \num{303}$, $N_{\mathrm{u}} = \num{303} $ i $\lambda = 1000$}
    \label{pro_zad4_d_303_n_303_nu_303_lam_1000inp}
\end{figure}

\begin{figure}[t]
    \centering
    \begin{tikzpicture}
    \begin{axis}[
    width=0.98\textwidth,
    xmin=0.0,xmax=1000,ymin=-2,ymax=12.5,
    xlabel={$k$},
    ylabel={$y[k]$},
    legend pos=south east,
    y tick label style={/pgf/number format/1000 sep=},
    ] 
    \addlegendentry{$y[k]$},
    \addlegendentry{$y^{\mathrm{zad}}[k]$},

    \addlegendimage{no markers,red}
    \addlegendimage{no markers,blue}

    \addplot[red, semithick] file{../data/project/zad4/zad4DMC_output_D_303_N_303_Nu_303_lam_10000.csv};  
    \addplot[blue, semithick] file{../data/project/zad4/zad4DMC_setpoint_D_303_N_303_Nu_303_lam_10000.csv};
    
    \end{axis}
    \end{tikzpicture}
    \caption{Przebieg procesu sterowanego za pomocą regulatora z parametrami $K = \num{303}$, $N = \num{303}$, $N_{\mathrm{u}} = \num{303}$ i $\lambda = 10000$}
    \label{pro_zad4_d_303_n_303_nu_303_lam_1000out}
\end{figure}

\begin{figure}[b]
    \centering
    \begin{tikzpicture}
    \begin{axis}[
    width=0.98\textwidth,
    xmin=0.0,xmax=1000,ymin=-0.8,ymax=1.2,
    xlabel={$k$},
    ylabel={$y[k]$},
    legend pos=south east,
    y tick label style={/pgf/number format/1000 sep=},
    ] 
    \addlegendentry{$u[k]$},
    
    \addlegendimage{no markers,blue}

    \addplot[blue, semithick] file{../data/project/zad4/zad4DMC_input_D_303_N_303_Nu_303_lam_10000.csv};
    
    \end{axis}
    \end{tikzpicture}
    \caption{Przebieg sygnału sterującego przy parametrach $D = \num{303}$, $N = \num{303}$, $N_{\mathrm{u}} = \num{303} $ i $\lambda = 10000$}
    \label{pro_zad4_d_303_n_303_nu_303_lam_10000inp}
\end{figure}

\begin{figure}[t]
    \centering
    \begin{tikzpicture}
    \begin{axis}[
    width=0.98\textwidth,
    xmin=0.0,xmax=1000,ymin=-2,ymax=12.5,
    xlabel={$k$},
    ylabel={$y[k]$},
    legend pos=south east,
    y tick label style={/pgf/number format/1000 sep=},
    ] 
    \addlegendentry{$y[k]$},
    \addlegendentry{$y^{\mathrm{zad}}[k]$},

    \addlegendimage{no markers,red}
    \addlegendimage{no markers,blue}

    \addplot[red, semithick] file{../data/project/zad4/zad4DMC_output_D_303_N_220_Nu_220_lam_10000.csv};  
    \addplot[blue, semithick] file{../data/project/zad4/zad4DMC_setpoint_D_303_N_220_Nu_220_lam_10000.csv};
    
    \end{axis}
    \end{tikzpicture}
    \caption{Przebieg procesu sterowanego za pomocą regulatora z parametrami $K = \num{303}$, $N = \num{220}$, $N_{\mathrm{u}} = \num{220}$ i $\lambda = 10000$}
    \label{pro_zad4_d_303_n_220_nu_220_out}
\end{figure}

\begin{figure}[b]
    \centering
    \begin{tikzpicture}
    \begin{axis}[
    width=0.98\textwidth,
    xmin=0.0,xmax=1000,ymin=-0.6,ymax=1.2,
    xlabel={$k$},
    ylabel={$y[k]$},
    legend pos=south east,
    y tick label style={/pgf/number format/1000 sep=},
    ] 
    \addlegendentry{$u[k]$},
    
    \addlegendimage{no markers,blue}

    \addplot[blue, semithick] file{../data/project/zad4/zad4DMC_input_D_303_N_220_Nu_220_lam_10000.csv};
    
    \end{axis}
    \end{tikzpicture}
    \caption{Przebieg sygnału sterującego przy parametrach $D = \num{303}$, $N = \num{220}$, $N_{\mathrm{u}} = \num{220}$ i $\lambda = 10000$}
    \label{pro_zad4_d_303_n_220_nu_220_inp}
\end{figure}

\begin{figure}[t]
    \centering
    \begin{tikzpicture}
    \begin{axis}[
    width=0.98\textwidth,
    xmin=0.0,xmax=1000,ymin=-2,ymax=12.5,
    xlabel={$k$},
    ylabel={$y[k]$},
    legend pos=south east,
    y tick label style={/pgf/number format/1000 sep=},
    ] 
    \addlegendentry{$y[k]$},
    \addlegendentry{$y^{\mathrm{zad}}[k]$},

    \addlegendimage{no markers,red}
    \addlegendimage{no markers,blue}

    \addplot[red, semithick] file{../data/project/zad4/zad4DMC_output_D_303_N_50_Nu_50_lam_10000.csv};  
    \addplot[blue, semithick] file{../data/project/zad4/zad4DMC_setpoint_D_303_N_50_Nu_50_lam_10000.csv};
    
    \end{axis}
    \end{tikzpicture}
    \caption{Przebieg procesu sterowanego za pomocą regulatora z parametrami $K = \num{303}$, $N = \num{50}$, $N_{\mathrm{u}} = \num{50}$ i $\lambda = 10000$}
    \label{pro_zad4_d_303_n_50_nu_50_out}
\end{figure}

\begin{figure}[b]
    \centering
    \begin{tikzpicture}
    \begin{axis}[
    width=0.98\textwidth,
    xmin=0.0,xmax=1000,ymin=-0.6,ymax=1.2,
    xlabel={$k$},
    ylabel={$y[k]$},
    legend pos=south east,
    y tick label style={/pgf/number format/1000 sep=},
    ] 
    \addlegendentry{$u[k]$},
    
    \addlegendimage{no markers,blue}

    \addplot[blue, semithick] file{../data/project/zad4/zad4DMC_input_D_303_N_50_Nu_50_lam_10000.csv};
    
    \end{axis}
    \end{tikzpicture}
    \caption{Przebieg sygnału sterującego przy parametrach $D = \num{303}$, $N = \num{50}$, $N_{\mathrm{u}} = \num{50} $ i $\lambda = 10000$}
    \label{pro_zad4_d_303_n_50_nu_50_inp}
\end{figure}

\begin{figure}[t]
    \centering
    \begin{tikzpicture}
    \begin{axis}[
    width=0.98\textwidth,
    xmin=0.0,xmax=1000,ymin=-2,ymax=12.5,
    xlabel={$k$},
    ylabel={$y[k]$},
    legend pos=south east,
    y tick label style={/pgf/number format/1000 sep=},
    ] 
    \addlegendentry{$y[k]$},
    \addlegendentry{$y^{\mathrm{zad}}[k]$},

    \addlegendimage{no markers,red}
    \addlegendimage{no markers,blue}

    \addplot[red, semithick] file{../data/project/zad4/zad4DMC_output_D_303_N_50_Nu_10_lam_10000.csv};  
    \addplot[blue, semithick] file{../data/project/zad4/zad4DMC_setpoint_D_303_N_50_Nu_10_lam_10000.csv};
    
    \end{axis}
    \end{tikzpicture}
    \caption{Przebieg procesu sterowanego za pomocą regulatora z parametrami $K = \num{303}$, $N = \num{50}$, $N_{\mathrm{u}} = \num{10}$ i $\lambda = 10000$}
    \label{pro_zad4_d_303_n_50_nu_10_out}
\end{figure}

\begin{figure}[b]
    \centering
    \begin{tikzpicture}
    \begin{axis}[
    width=0.98\textwidth,
    xmin=0.0,xmax=1000,ymin=-0.8,ymax=1.2,
    xlabel={$k$},
    ylabel={$y[k]$},
    legend pos=south east,
    y tick label style={/pgf/number format/1000 sep=},
    ] 
    \addlegendentry{$u[k]$},
    
    \addlegendimage{no markers,blue}

    \addplot[blue, semithick] file{../data/project/zad4/zad4DMC_input_D_303_N_50_Nu_10_lam_10000.csv};
    
    \end{axis}
    \end{tikzpicture}
    \caption{Przebieg sygnału sterującego przy parametrach $D = \num{303}$, $N = \num{50}$, $N_{\mathrm{u}} = \num{10} $ i $\lambda = 10000$}
    \label{pro_zad4_d_303_n_50_nu_10_inp}
\end{figure}