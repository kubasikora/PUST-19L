\documentclass{mwrep}

% Polskie znaki
\usepackage{polski}
\usepackage[polish]{babel}
\usepackage[utf8]{inputenc}
\usepackage[T1]{fontenc}
\usepackage[utf8]{luainputenc}
\usepackage{lmodern}
\usepackage{indentfirst}

% Strona tytułowa
\usepackage{pgfplots}
\usepackage{siunitx}
\usepackage{paracol}
\usepackage{gensymb}

% Pływające obrazki
\usepackage{float}
\usepackage{svg}
\usepackage{graphicx}

% table of contents refs
\usepackage{hyperref}
\usepackage{cleveref}
\usepackage{booktabs}
\usepackage{listings}
\usepackage{placeins}
\usepackage{xcolor}

\usetikzlibrary{pgfplots.groupplots}
\sisetup{detect-weight,exponent-product=\cdot,output-decimal-marker={,},per-mode=symbol,binary-units=true,range-phrase={-},range-units=single}
\definecolor{szary}{rgb}{0.95,0.95,0.95}
%konfiguracje pakietu listings
\lstset{
	backgroundcolor=\color{szary},
	frame=single,
	breaklines=true,
}
\lstdefinestyle{customlatex}{
	basicstyle=\footnotesize\ttfamily,
	%basicstyle=\small\ttfamily,
}
\lstdefinestyle{customc}{
	breaklines=true,
	frame=tb,
	language=C,
	xleftmargin=0pt,
	showstringspaces=false,
	basicstyle=\small\ttfamily,
	keywordstyle=\bfseries\color{green!40!black},
	commentstyle=\itshape\color{purple!40!black},
	identifierstyle=\color{blue},
	stringstyle=\color{orange},
}
\lstdefinestyle{custommatlab}{
	captionpos=t,
	breaklines=true,
	frame=tb,
	xleftmargin=0pt,
	language=matlab,
	showstringspaces=false,
	%basicstyle=\footnotesize\ttfamily,
	basicstyle=\scriptsize\ttfamily,
	keywordstyle=\bfseries\color{green!40!black},
	commentstyle=\itshape\color{purple!40!black},
	identifierstyle=\color{blue},
	stringstyle=\color{orange},
}

%wymiar tekstu
\def\figurename{Rys.}
\def\tablename{Tab.}

%konfiguracja liczby p�ywaj�cych element�w
\setcounter{topnumber}{0}%2
\setcounter{bottomnumber}{3}%1
\setcounter{totalnumber}{5}%3
\renewcommand{\textfraction}{0.01}%0.2
\renewcommand{\topfraction}{0.95}%0.7
\renewcommand{\bottomfraction}{0.95}%0.3
\renewcommand{\floatpagefraction}{0.35}%0.5

\SendSettingsToPgf
\title{\bf Sprawozdanie z projektu i ćwiczenia laboratoryjnego nr 3, zadanie nr 1 \vskip 0.1cm}
\author{Marcin Dolicher \\ Jakub Sikora \\ Robert Wojtaś}
\date{\today}
\pgfplotsset{compat=1.15}	
\begin{document}
\frenchspacing
\pagestyle{uheadings}

\makeatletter
\renewcommand{\maketitle}{\begin{titlepage}
		\begin{center}{
				\LARGE {\bf Politechnika Warszawska}}\\
            \vspace{0.4cm}
            \leftskip-0.9cm
            {\LARGE {\bf \mbox{Wydział Elektroniki i Technik Informacyjnych}}}\\
            \vspace{0.2cm}
            {\LARGE {\bf \mbox{Instytut Automatyki i Informatyki Stosowanej}}}\\
            
            \vspace{5cm}
            \leftskip-1.3cm
			{\bf \Huge \mbox{Projektowanie układów sterowania} \vskip 0.1cm}
		\end{center}
		\vspace{0.1cm}

		\begin{center}
			{\bf \LARGE \@title}
		\end{center}

		\vspace{9cm}
		\begin{paracol}{2}
			\addtocontents{toc}{\protect\setcounter{tocdepth}{1}}
			\subsection*{Zespół:}
			\bf{ \Large{ \noindent\@author \par}}
			\addtocontents{toc}{\protect\setcounter{tocdepth}{2}}

			\switchcolumn \addtocontents{toc}{\protect\setcounter{tocdepth}{1}}
			\subsection*{Prowadzący:}
			\bf{\Large{\noindent dr inż. Patryk Chaber}}
			\addtocontents{toc}{\protect\setcounter{tocdepth}{2}}

		\end{paracol}
		\vspace*{\stretch{6}}
		\begin{center}
			\bf{\large{Warszawa, \@date\vskip 0.1cm}}
		\end{center}
	\end{titlepage}
}
\makeatother
\maketitle
\tableofcontents

\part{Projekt}
\chapter{Sprawdzenie poprawności punktu pracy}
\label{pro1}

Aby sprawdzić poprawność podanego punktu pracy wykonaliśmy ekspe-
ryment polegający na pobudzeniu wejścia obiektu stałym sygnałem wejściowym o wartości
$U_{\mathrm{pp}} = \num{0}$ i sprawdzeniu czy sygnał wyjściowy stabilizuje się 
na wartości $Y_{\mathrm{pp}} = \num{0}$. Symulację obiektu przeprowadziliśmy
za pomocą funkcji \verb+symulacja_obiektu1y+. 

Eksperyment przeprowadziliśmy uruchamiając dostarczoną funkcję z następującymi parametrami:

\begin{center}
\verb+symulacja_obiektu1y(Upp, Upp, Ypp, Ypp)+ 
\end{center}

Uzyskany wynik \verb+ans+ $= \num{0}$ jest równy co do wartości $Y_{\mathrm{pp}}$ co potwierdza poprawność podanego
punktu pracy.

\vskip3cm

\begin{figure}[H]
    \centering
    \begin{tikzpicture}
    \begin{axis}[
    width=\textwidth,
    xmin=0,xmax=200,ymin=-2,ymax=2,
    xlabel={$k$},
    ylabel={$y[k]$},
    legend pos=south east,
    y tick label style={/pgf/number format/1000 sep=},
    ]
    \addplot[red, semithick] file{../data/project/zad1/zad1_output.csv};
    \legend{$y[k]$, }
    \end{axis} 
    \end{tikzpicture}
    \caption{Przebieg wyjścia obiektu sterowanego sygnałem stałym o wartości \mbox{$U_{\mathrm{pp}} = \num{0}$}}
    \label{pro1}
\end{figure}
\chapter{Wyznaczenie odpowiedzi skokowych obiektu}
\label{zad2}

\section{Odpowiedź skokowa toru wejście-wyjście}
Eksperymenty wykonywane były z punktu pracy zdefiniowanego w zadaniu, przy sygnale zakłócenia $Z_{\mathrm{pp}} = 0$. Na wykresie \ref{zad2_odp_skok_input} możemy zaobserwować, że wraz z wzrostem wartości skoku rośnie również wartość odpowiedzi procesu $y[k]$. Zachowanie procesu jest zgodne 
z typową charakterystyką obiektów dynamicznych liniowych, w których wartość ustalona zmienia się liniowo wraz ze zmianą sygnału sterującego. 

\begin{figure}[t]
    \centering
    \begin{tikzpicture}
    \begin{axis}[
    width=0.98\textwidth,
    xmin=0.0,xmax=200,ymin=-1.5,ymax=1.5,
    xlabel={$k$},
    ylabel={$y[k]$},
    legend pos=south east,
    y tick label style={/pgf/number format/1000 sep=},
    ] 
    \addlegendentry{$\Delta u = \num{0,1}$},
    \addlegendentry{$\Delta u = \num{0,2}$}
    \addlegendentry{$\Delta u = \num{0,35}$},
    \addlegendentry{$\Delta u = \num{0,5}$},
    \addlegendentry{$\Delta u = \num{-0,1}$},
    \addlegendentry{$\Delta u = \num{-0,2}$},
    \addlegendentry{$\Delta u = \num{-0,35}$},
    \addlegendentry{$\Delta u = \num{-0,5}$},
    \addlegendimage{no markers,green}
	\addlegendimage{no markers,red}
	\addlegendimage{no markers,yellow}
	\addlegendimage{no markers,blue}
	\addlegendimage{no markers,black}
	\addlegendimage{no markers,orange}
	\addlegendimage{no markers,brown}
	\addlegendimage{no markers,violet}
    \addplot[green, semithick, thick] file{../data/zad2/zad2_input_output0.1.csv};
    \addplot[red, semithick, thick] file{../data/zad2/zad2_input_output0.2.csv};
    \addplot[yellow, semithick, thick] file{../data/zad2/zad2_input_output0.35.csv};
    \addplot[blue, semithick, thick] file{../data/zad2/zad2_input_output0.5.csv};
    \addplot[black, semithick, thick] file{../data/zad2/zad2_input_output-0.1.csv};
    \addplot[orange, semithick, thick] file{../data/zad2/zad2_input_output-0.2.csv};
    \addplot[brown, semithick, thick] file{../data/zad2/zad2_input_output-0.35.csv};
    \addplot[violet, semithick, thick] file{../data/zad2/zad2_input_output-0.5.csv};    
    \end{axis}
    \end{tikzpicture}
    \caption{Odpowiedzi procesu na skokową zmianę sygnału sterującego}
    \label{zad2_odp_skok_input}
\end{figure}

\section{Odpowiedź skokowa toru zakłócenie-wyjście}

Odpowiedź skokową toru zakłócenie-wyjście otrzymywaliśmy poprzez skokową zmianę zakłócenia. Eksperymenty wykonywane były ze stanu punktu pracy, w którym wszystkie sygnały wynosiły 0. Sygnał sterowania był stały podczas całej symulacji. Na wykresie \ref{zad2_odp_skok_disturbance} obserwujemy reakcję na zmiany zakłóceń i możemy stwierdzić, że jest ona podobna do tej otrzymanej przy zmianach sterowania, co tylko potwierdza nasze przypuszczenia o liniowości obiektu. 

\begin{figure}[b]
    \centering
    \begin{tikzpicture}
    \begin{axis}[
    width=0.98\textwidth,
    xmin=0.0,xmax=200,ymin=-1,ymax=1,
    xlabel={$k$},
    ylabel={$y[k]$},
    legend pos=south east,
    y tick label style={/pgf/number format/1000 sep=},
    ] 
    \addlegendentry{$\Delta z = \num{0,05}$},
    \addlegendentry{$\Delta z = \num{0,2}$}
    \addlegendentry{$\Delta z = \num{0,35}$},
    \addlegendentry{$\Delta z = \num{0,5}$},
    \addlegendentry{$\Delta z = \num{-0,05}$},
    \addlegendentry{$\Delta z = \num{-0,2}$},
    \addlegendentry{$\Delta z = \num{-0,35}$},
    \addlegendentry{$\Delta z = \num{-0,5}$},
    \addlegendimage{no markers,green}
	\addlegendimage{no markers,red}
	\addlegendimage{no markers,yellow}
	\addlegendimage{no markers,blue}
	\addlegendimage{no markers,black}
	\addlegendimage{no markers,orange}
	\addlegendimage{no markers,brown}
	\addlegendimage{no markers,violet}
    \addplot[green, semithick, thick] file{../data/zad2/zad2_disturbance_output0.05.csv};
    \addplot[red, semithick, thick] file{../data/zad2/zad2_disturbance_output0.2.csv};
    \addplot[yellow, semithick, thick] file{../data/zad2/zad2_disturbance_output0.35.csv};
    \addplot[blue, semithick, thick] file{../data/zad2/zad2_disturbance_output0.5.csv};
    \addplot[black, semithick, thick] file{../data/zad2/zad2_disturbance_output-0.05.csv};
    \addplot[orange, semithick, thick] file{../data/zad2/zad2_disturbance_output-0.2.csv};
    \addplot[brown, semithick, thick] file{../data/zad2/zad2_disturbance_output-0.35.csv};
    \addplot[violet, semithick, thick] file{../data/zad2/zad2_disturbance_output-0.5.csv};    
    \end{axis}
    \end{tikzpicture}
    \caption{Odpowiedzi procesu na skokową zmianę sygnału zaklócenia}
    \label{zad2_odp_skok_disturbance}
\end{figure}

\section{Wyznaczenie charakterystyki statycznej $y(u,z)$}
Wyznaczenie charakterystyki $y(u,z)$ rozpoczęliśmy od wyznaczenia charakterystyk $y(u)$ i $y(z)$. 
Na dalszym etapie zadania pomogło to w określeniu wzmocnień statycznych torów wejście-wyjście, zakłócenie-wyjśćie. 

Aby otrzymać wykres charakterystyki statycznej obiektu w zależności od 
dwóch argumentów należało przeprowadzić symulacje dla każdej 
wartości sygnałów $u$ oraz $d$ i zapamiętaniu wartości 
nasycenia sygnału $y$. Program, który realizujący zadanie to 
\verb+zad2_static_surface.m+ wykonywał wspomniane działania, 
a wyniki zapisywał do tablicy z wartościami charakterystyki statycznej. 
Wynik można zobaczyć na rysunku \ref{zad2_static_surf}.\\
\indent{} Wykres przedstawia płaszczyznę $y(u,z)$. Skoro powstały wykres 
jest płaszczyzną to można wywnioskować, że mamy do czynienia z obiektem liniowym. 


\chapter{Dobór nastaw regulatora PID i DMC.}
\label{pro4_PID}

\section{Przebieg strojenia regulatora PID.}
\subsection{Próba dostrojenia regulatora metoda Zieglera - Nicholsa}
Nasze strojenie rozpoczęliśmy od wykorzystania metody Zieglera - Nicholsa. Parametry regulatora dobraliśmy według reguł stosowanych w tej metodzie. Niestety nie przyniosła ona zadowalających efektów. Dla końcowych parametrów uzyskiwaliśmy zawsze niegasnące oscylacje. Zostawały one nawet po ręcznym skorygowaniu przez nas wartości członów regulatora (wartości obliczone za pomocą metody Zieglera - Nicholsa były wtedy traktowane jako dobry punkt wyjścia do dalszego strojenia).

~\\\\Na rysunkach \ref{pro_zad4_niegasnace_oscylacje_out} i \ref{pro_zad4_niegasnace_oscylacje_inp} zaprezentowaliśmy przebiegi sygnałów dla niegasnących oscylacji. Otrzymane przez nas wzmocnienie krytyczne wyniosło $K_{\mathrm{Kr}} = 2.2$, a okres oscylacji był równy $T_{\mathrm{Kr}} = \num{13.5}$ s. 

~\\\\ Następnie wartości $K_{\mathrm{Kr}}$ i $T_{\mathrm{Kr}}$ podstawiliśmy do tabeli z wzorami na $K$, $T_{\mathrm{i}}$ i $T_{\mathrm{d}}$ wykorzystywanej w metodzie Zieglera - Nicholsa, dzięki temu uzyskaliśmy wartości liczbowe tych parametrów równe $K = \num{2.2}$, $T_{\mathrm{i}} = \num{6.75}$ i $T_{\mathrm{d}} = \num{1.62}$. Wskaźnik jakości regulacji wynosił $E = 50084.0886$. Wykresy dla tych ustawień to \ref{pro_zad4_ziegler_out} i \ref{pro_zad4_ziegler_inp}. Widzimy na nich niegasnące oscylacje, które są niepożądane podczas regulacji. Podjęliśmy próbę lekkiej zmiany parametrów regulatora, jednak nie poprawiły one w żaden sposób jakości regulacji. 

~\\\\ Z powodu niezadowalających rezultatów po użyciu metody Zieglera - Nicholsa postanowiliśmy, że podczas szukania najlepszych nastawów oprzemy się w dużej mierze o nasze doświadczenie zdobyte na poprzednich laboratoriach i innych przedmiotach, których tematyką była regulacja. Zastosowaliśmy metodę inżynierską w której oceniamy jakościowo regulację i zmieniamy parametry tak, żeby nowe nastawy były lepsze od poprzednich. 

\subsection{Dobór nastawów regulatora metodą inżynierską}
Wiedzieliśmy, że musimy zlikwidować niegasnące oscylacje, aby to uzyskać zmniejszaliśmy K, a pozostałe człony wyłączyliśmy. Najlepsze rezultaty dostaliśmy dla $ K = \num{0.03} $ i $ K = \num{0.05} $. Przebiegi sygnałów dla tych wartości zaprezentowaliśmy na rysunkach \ref{pro_zad4_k_0.03_out}, \ref{pro_zad4_k_0.03_inp} oraz \ref{pro_zad4_k_0.05_out}, \ref{pro_zad4_k_0.05_inp}.

~\\\\ W następnym etapie włączyliśmy człon całkujący i różniczkujący. Ich wartości początkowe przyjęliśmy takie jak w metodzie Zieglera - Nicholsa $T_{\mathrm{i}} = \num{6.75}$ i $T_{\mathrm{d}} = \num{1.62}$. Przebiegi sygnałów dla tych wartości zaprezentowaliśmy na rysunkach \ref{pro_zad4_k_0.03_6.75_1.62_out} i \ref{pro_zad4_k_0.03_6.75_1.62_inp}. Możemy zauważyć, że regulator nie był w stanie osiągnąć pierwszych zmian wartości zadanej. Poprawna regulacja rozpoczęła się od około 2000 próbki, wtedy większość wartości zadanych została osiągnięta. Wskaźnik jakości regulacji wyniósł $E = 344443.9$. 

~\\\\ Postanowiliśmy jeszcze bardziej zmniejszyć wartości $T_{\mathrm{i}}$ i $T_{\mathrm{d}}$ w celu uzyskania jeszcze lepszej regulacji. Wykresy \ref{pro_zad4_k_0.03_6.5_1.56_out} i \ref{pro_zad4_k_0.03_6.5_1.56_inp} prezentują przebiegi dla parametrów $K = \num{0,03}$, $T_{\mathrm{i}} = \num{6.5}$ i $T_{\mathrm{d}} = \num{1.56}$, a rysunki \ref{pro_zad4_k_0.03_5.5_1.32_out} i \ref{pro_zad4_k_0.03_5.5_1.32_inp} dla $K = \num{0,03}$, $T_{\mathrm{i}} = \num{5.5}$ i $T_{\mathrm{d}} = \num{1.32}$. Dla ostatnich nastawów uzyskaliśmy najmniejszy wskaźnik jakości regulacji równy $E = \num{33955,70}$. Przebiegi sygnałów nie zmieniły się znacząco, ale po uważnej analizie możemy zauważyć, że dla ostatnich wartości parametrów niektóre wielkości zadane są szybciej osiągane. 


\begin{figure}[t]
    \centering
    \begin{tikzpicture}
    \begin{axis}[
    width=0.98\textwidth,
    xmin=0.0,xmax=5000,ymin=-4.5,ymax=11.5,
    xlabel={$k$},
    ylabel={$y[k]$},
    legend pos=south east,
    y tick label style={/pgf/number format/1000 sep=},
    ] 
    \addlegendentry{$y[k]$},
    \addlegendentry{$y^{\mathrm{zad}}[k]$},

    \addlegendimage{no markers,red}
    \addlegendimage{no markers,blue}

    \addplot[red, semithick] file{../data/project/zad4/zad3PID_output_K_2.2_Ti_100000000000_Td_0.csv};  
    \addplot[blue, semithick] file{../data/project/zad4/zad3PID_setpoint_K_2.2_Ti_100000000000_Td_0.csv};
    
    \end{axis}
    \end{tikzpicture}
    \caption{Przebieg procesu sterowanego za pomocą regulatora}
    \label{pro_zad4_niegasnace_oscylacje_out}
\end{figure}

\begin{figure}[b]
    \centering
    \begin{tikzpicture}
    \begin{axis}[
    width=0.98\textwidth,
    xmin=0.0,xmax=5000,ymin=-1.2,ymax=1.2,
    xlabel={$k$},
    ylabel={$y[k]$},
    legend pos=south east,
    y tick label style={/pgf/number format/1000 sep=},
    ] 
    \addlegendentry{$u[k]$},
    
    \addlegendimage{no markers,blue}

    \addplot[blue, semithick] file{../data/project/zad4/zad3PID_input_K_2.2_Ti_100000000000_Td_0.csv};
    
    \end{axis}
    \end{tikzpicture}
    \caption{Przebieg sygnału sterującego regulatora}
    \label{pro_zad4_niegasnace_oscylacje_inp}
\end{figure}

\begin{figure}[t]
    \centering
    \begin{tikzpicture}
    \begin{axis}[
    width=0.98\textwidth,
    xmin=0.0,xmax=5000,ymin=-4.5,ymax=11.5,
    xlabel={$k$},
    ylabel={$y[k]$},
    legend pos=south east,
    y tick label style={/pgf/number format/1000 sep=},
    ] 
    \addlegendentry{$y[k]$},
    \addlegendentry{$y^{\mathrm{zad}}[k]$},

    \addlegendimage{no markers,red}
    \addlegendimage{no markers,blue}

    \addplot[red, semithick] file{../data/project/zad4/zad3PID_output_K_2.2_Ti_6.75_Td_1.62.csv};  
    \addplot[blue, semithick] file{../data/project/zad4/zad3PID_setpoint_K_2.2_Ti_6.75_Td_1.62.csv};
    
    \end{axis}
    \end{tikzpicture}
    \caption{Przebieg procesu sterowanego za pomocą regulatora z parametrami wyznaczonymi za pomocą metody Zieglera-Nicholsa}
    \label{pro_zad4_ziegler_out}
\end{figure}

\begin{figure}[b]
    \centering
    \begin{tikzpicture}
    \begin{axis}[
    width=0.98\textwidth,
    xmin=0.0,xmax=5000,ymin=-1.2,ymax=1.2,
    xlabel={$k$},
    ylabel={$y[k]$},
    legend pos=south east,
    y tick label style={/pgf/number format/1000 sep=},
    ] 
    \addlegendentry{$u[k]$},
    
    \addlegendimage{no markers,blue}

    \addplot[blue, semithick] file{../data/project/zad4/zad3PID_input_K_2.2_Ti_6.75_Td_1.62.csv};
    
    \end{axis}
    \end{tikzpicture}
    \caption{Przebieg sygnału sterującego regulatora z parametrami wyznaczonymi za pomocą metody Zieglera-Nicholsa}
    \label{pro_zad4_ziegler_inp}
\end{figure}

\begin{figure}[t]
    \centering
    \begin{tikzpicture}
    \begin{axis}[
    width=0.98\textwidth,
    xmin=0.0,xmax=5000,ymin=-4.5,ymax=11.5,
    xlabel={$k$},
    ylabel={$y[k]$},
    legend pos=south east,
    y tick label style={/pgf/number format/1000 sep=},
    ] 
    \addlegendentry{$y[k]$},
    \addlegendentry{$y^{\mathrm{zad}}[k]$},

    \addlegendimage{no markers,red}
    \addlegendimage{no markers,blue}

    \addplot[red, semithick] file{../data/project/zad4/zad3PID_output_K_0.03_Ti_10000000_Td_0.csv};  
    \addplot[blue, semithick] file{../data/project/zad4/zad3PID_setpoint_K_0.03_Ti_10000000_Td_0.csv};
    
    \end{axis}
    \end{tikzpicture}
    \caption{Odpowiedzi procesu na skokową zmianę sygnału sterującego}
    \label{pro_zad4_k_0.03_out}
\end{figure}

\begin{figure}[b]
    \centering
    \begin{tikzpicture}
    \begin{axis}[
    width=0.98\textwidth,
    xmin=0.0,xmax=5000,ymin=-1.2,ymax=1.2,
    xlabel={$k$},
    ylabel={$y[k]$},
    legend pos=south east,
    y tick label style={/pgf/number format/1000 sep=},
    ] 
    \addlegendentry{$u[k]$},
    
    \addlegendimage{no markers,blue}

    \addplot[blue, semithick] file{../data/project/zad4/zad3PID_input_K_0.03_Ti_10000000_Td_0.csv};
    
    \end{axis}
    \end{tikzpicture}
    \caption{Odpowiedzi procesu na skokową zmianę sygnału sterującego}
    \label{pro_zad4_k_0.03_inp}
\end{figure}

\begin{figure}[t]
    \centering
    \begin{tikzpicture}
    \begin{axis}[
    width=0.98\textwidth,
    xmin=0.0,xmax=5000,ymin=-4.5,ymax=11.5,
    xlabel={$k$},
    ylabel={$y[k]$},
    legend pos=south east,
    y tick label style={/pgf/number format/1000 sep=},
    ] 
    \addlegendentry{$y[k]$},
    \addlegendentry{$y^{\mathrm{zad}}[k]$},

    \addlegendimage{no markers,red}
    \addlegendimage{no markers,blue}

    \addplot[red, semithick] file{../data/project/zad4/zad3PID_output_K_0.05_Ti_10000000_Td_0.csv};  
    \addplot[blue, semithick] file{../data/project/zad4/zad3PID_setpoint_K_0.05_Ti_10000000_Td_0.csv};
    
    \end{axis}
    \end{tikzpicture}
    \caption{Odpowiedzi procesu na skokową zmianę sygnału sterującego}
    \label{pro_zad4_k_0.05_out}
\end{figure}

\begin{figure}[b]
    \centering
    \begin{tikzpicture}
    \begin{axis}[
    width=0.98\textwidth,
    xmin=0.0,xmax=5000,ymin=-1.2,ymax=1.2,
    xlabel={$k$},
    ylabel={$y[k]$},
    legend pos=south east,
    y tick label style={/pgf/number format/1000 sep=},
    ] 
    \addlegendentry{$u[k]$},
    
    \addlegendimage{no markers,blue}

    \addplot[blue, semithick] file{../data/project/zad4/zad3PID_input_K_0.05_Ti_10000000_Td_0.csv};
    
    \end{axis}
    \end{tikzpicture}
    \caption{Odpowiedzi procesu na skokową zmianę sygnału sterującego}
    \label{pro_zad4_k_0.05_inp}
\end{figure}

\begin{figure}[t]
    \centering
    \begin{tikzpicture}
    \begin{axis}[
    width=0.98\textwidth,
    xmin=0.0,xmax=5000,ymin=-4.5,ymax=11.5,
    xlabel={$k$},
    ylabel={$y[k]$},
    legend pos=south east,
    y tick label style={/pgf/number format/1000 sep=},
    ] 
    \addlegendentry{$y[k]$},
    \addlegendentry{$y^{\mathrm{zad}}[k]$},

    \addlegendimage{no markers,red}
    \addlegendimage{no markers,blue}

    \addplot[red, semithick] file{../data/project/zad4/zad3PID_output_K_0.03_Ti_6.75_Td_1.62.csv};  
    \addplot[blue, semithick] file{../data/project/zad4/zad3PID_setpoint_K_0.03_Ti_6.75_Td_1.62.csv};
    
    \end{axis}
    \end{tikzpicture}
    \caption{Odpowiedzi procesu na skokową zmianę sygnału sterującego dla $K = \num{0,03}$, $T_{\mathrm{i}} = \num{6.75}$ i $T_{\mathrm{d}} = \num{1.62}$}
    \label{pro_zad4_k_0.03_6.75_1.62_out}
\end{figure}

\begin{figure}[b]
    \centering
    \begin{tikzpicture}
    \begin{axis}[
    width=0.98\textwidth,
    xmin=0.0,xmax=5000,ymin=-1.2,ymax=1.2,
    xlabel={$k$},
    ylabel={$y[k]$},
    legend pos=south east,
    y tick label style={/pgf/number format/1000 sep=},
    ] 
    \addlegendentry{$u[k]$},
    
    \addlegendimage{no markers,blue}

    \addplot[blue, semithick] file{../data/project/zad4/zad3PID_input_K_0.03_Ti_6.75_Td_1.62.csv};
    
    \end{axis}
    \end{tikzpicture}
    \caption{Odpowiedzi procesu na skokową zmianę sygnału sterującego dla $K = \num{0,03}$, $T_{\mathrm{i}} = \num{6.75}$ i $T_{\mathrm{d}} = \num{1.62}$}
    \label{pro_zad4_k_0.03_6.75_1.62_inp}
\end{figure}

\begin{figure}[t]
    \centering
    \begin{tikzpicture}
    \begin{axis}[
    width=0.98\textwidth,
    xmin=0.0,xmax=5000,ymin=-4.5,ymax=11.5,
    xlabel={$k$},
    ylabel={$y[k]$},
    legend pos=south east,
    y tick label style={/pgf/number format/1000 sep=},
    ] 
    \addlegendentry{$y[k]$},
    \addlegendentry{$y^{\mathrm{zad}}[k]$},

    \addlegendimage{no markers,red}
    \addlegendimage{no markers,blue}

    \addplot[red, semithick] file{../data/project/zad4/zad3PID_output_K_0.03_Ti_6.5_Td_1.56.csv};  
    \addplot[blue, semithick] file{../data/project/zad4/zad3PID_setpoint_K_0.03_Ti_6.5_Td_1.56.csv};
    
    \end{axis}
    \end{tikzpicture}
    \caption{Odpowiedzi procesu na skokową zmianę sygnału sterującego dla $K = \num{0,03}$, $T_{\mathrm{i}} = \num{6.5}$ i $T_{\mathrm{d}} = \num{1.56}$}
    \label{pro_zad4_k_0.03_6.5_1.56_out}
\end{figure}

\begin{figure}[b]
    \centering
    \begin{tikzpicture}
    \begin{axis}[
    width=0.98\textwidth,
    xmin=0.0,xmax=5000,ymin=-1.2,ymax=1.2,
    xlabel={$k$},
    ylabel={$y[k]$},
    legend pos=south east,
    y tick label style={/pgf/number format/1000 sep=},
    ] 
    \addlegendentry{$u[k]$},
    
    \addlegendimage{no markers,blue}

    \addplot[blue, semithick] file{../data/project/zad4/zad3PID_input_K_0.03_Ti_6.5_Td_1.56.csv};
    
    \end{axis}
    \end{tikzpicture}
    \caption{Odpowiedzi procesu na skokową zmianę sygnału sterującego dla $K = \num{0,03}$, $T_{\mathrm{i}} = \num{6.5}$ i $T_{\mathrm{d}} = \num{1.56}$}
    \label{pro_zad4_k_0.03_6.5_1.56_inp}
\end{figure}

\begin{figure}[t]
    \centering
    \begin{tikzpicture}
    \begin{axis}[
    width=0.98\textwidth,
    xmin=0.0,xmax=5000,ymin=-4.5,ymax=11.5,
    xlabel={$k$},
    ylabel={$y[k]$},
    legend pos=south east,
    y tick label style={/pgf/number format/1000 sep=},
    ] 
    \addlegendentry{$y[k]$},
    \addlegendentry{$y^{\mathrm{zad}}[k]$},

    \addlegendimage{no markers,red}
    \addlegendimage{no markers,blue}

    \addplot[red, semithick] file{../data/project/zad4/zad3PID_output_K_0.03_Ti_5.5_Td_1.32.csv};  
    \addplot[blue, semithick] file{../data/project/zad4/zad3PID_setpoint_K_0.03_Ti_5.5_Td_1.32.csv};
    
    \end{axis}
    \end{tikzpicture}
    \caption{Odpowiedzi procesu na skokową zmianę sygnału sterującego dla $K = \num{0,03}$, $T_{\mathrm{i}} = \num{5.5}$ i $T_{\mathrm{d}} = \num{1.32}$}
    \label{pro_zad4_k_0.03_5.5_1.32_out}
\end{figure}

\begin{figure}[b]
    \centering
    \begin{tikzpicture}
    \begin{axis}[
    width=0.98\textwidth,
    xmin=0.0,xmax=5000,ymin=-1.2,ymax=1.2,
    xlabel={$k$},
    ylabel={$y[k]$},
    legend pos=south east,
    y tick label style={/pgf/number format/1000 sep=},
    ] 
    \addlegendentry{$u[k]$},
    
    \addlegendimage{no markers,blue}

    \addplot[blue, semithick] file{../data/project/zad4/zad3PID_input_K_0.03_Ti_5.5_Td_1.32.csv};
    
    \end{axis}
    \end{tikzpicture}
    \caption{Odpowiedzi procesu na skokową zmianę sygnału sterującego dla $K = \num{0,03}$, $T_{\mathrm{i}} = \num{5.5}$ i $T_{\mathrm{d}} = \num{1.32}$}
    \label{pro_zad4_k_0.03_5.5_1.32_inp}
\end{figure}
\section{Przebieg strojenia regulatora DMC}
\label{pro4_DMC}

\begin{figure}[t]
    \centering
    \begin{tikzpicture}
    \begin{axis}[
    width=0.98\textwidth,
    xmin=0.0,xmax=3000,ymin=-4.2,ymax=10,
    xlabel={$k$},
    ylabel={$y[k]$},
    legend pos=south east,
    y tick label style={/pgf/number format/1000 sep=},
    ] 
    \addlegendentry{$y[k]$},
    \addlegendentry{$y^{\mathrm{zad}}[k]$},

    \addlegendimage{no markers,red}
    \addlegendimage{no markers,blue}

    \addplot[red, semithick] file{../data/project/zad4/zad4DMC_output_D_303_N_303_Nu_303_lam_1.csv};  
    \addplot[blue, semithick] file{../data/project/zad4/zad4DMC_setpoint_D_303_N_303_Nu_303_lam_1.csv};
    
    \end{axis}
    \end{tikzpicture}
    \caption{Odpowiedzi procesu na skokową zmianę sygnału sterującego dla $K = \num{303}$, $N = \num{303}$, $N_{\mathrm{u}} = \num{303}$ i $\lambda = 1$}
    \label{pro_zad4_d_303_n_303_nu_303_out}
\end{figure}

\begin{figure}[b]
    \centering
    \begin{tikzpicture}
    \begin{axis}[
    width=0.98\textwidth,
    xmin=0.0,xmax=3000,ymin=-1.2,ymax=1.2,
    xlabel={$k$},
    ylabel={$y[k]$},
    legend pos=south east,
    y tick label style={/pgf/number format/1000 sep=},
    ] 
    \addlegendentry{$u[k]$},
    
    \addlegendimage{no markers,blue}

    \addplot[blue, semithick] file{../data/project/zad4/zad4DMC_input_D_303_N_303_Nu_303_lam_1.csv};
    
    \end{axis}
    \end{tikzpicture}
    \caption{Odpowiedzi procesu na skokową zmianę sygnału sterującego dla $D = \num{303}$, $N = \num{303}$, $N_{\mathrm{u}} = \num{303}$ i $\lambda = 1$}
    \label{pro_zad4_d_303_n_303_nu_303_inp}
\end{figure}

\begin{figure}[t]
    \centering
    \begin{tikzpicture}
    \begin{axis}[
    width=0.98\textwidth,
    xmin=0.0,xmax=5000,ymin=-4.2,ymax=10,
    xlabel={$k$},
    ylabel={$y[k]$},
    legend pos=south east,
    y tick label style={/pgf/number format/1000 sep=},
    ] 
    \addlegendentry{$y[k]$},
    \addlegendentry{$y^{\mathrm{zad}}[k]$},

    \addlegendimage{no markers,red}
    \addlegendimage{no markers,blue}

    \addplot[red, semithick] file{../data/project/zad4/zad4DMC_output_D_303_N_220_Nu_220_lam_1_E_35395.6.csv};  
    \addplot[blue, semithick] file{../data/project/zad4/zad4DMC_setpoint_D_303_N_220_Nu_220_lam_1_E_35395.6.csv};
    
    \end{axis}
    \end{tikzpicture}
    \caption{Odpowiedzi procesu na skokową zmianę sygnału sterującego dla $K = \num{303}$, $N = \num{220}$, $N_{\mathrm{u}} = \num{220}$ i $\lambda = 1$}
    \label{pro_zad4_d_303_n_220_nu_220_out}
\end{figure}

\begin{figure}[b]
    \centering
    \begin{tikzpicture}
    \begin{axis}[
    width=0.98\textwidth,
    xmin=0.0,xmax=5000,ymin=-1.2,ymax=1.2,
    xlabel={$k$},
    ylabel={$y[k]$},
    legend pos=south east,
    y tick label style={/pgf/number format/1000 sep=},
    ] 
    \addlegendentry{$u[k]$},
    
    \addlegendimage{no markers,blue}

    \addplot[blue, semithick] file{../data/project/zad4/zad4DMC_input_D_303_N_220_Nu_220_lam_1_E_35395.6.csv};
    
    \end{axis}
    \end{tikzpicture}
    \caption{Odpowiedzi procesu na skokową zmianę sygnału sterującego dla $D = \num{303}$, $N = \num{220}$, $N_{\mathrm{u}} = \num{220}$ i $\lambda = 1$}
    \label{pro_zad4_d_303_n_220_nu_220_inp}
\end{figure}

\begin{figure}[t]
    \centering
    \begin{tikzpicture}
    \begin{axis}[
    width=0.98\textwidth,
    xmin=0.0,xmax=5000,ymin=-4.2,ymax=10,
    xlabel={$k$},
    ylabel={$y[k]$},
    legend pos=south east,
    y tick label style={/pgf/number format/1000 sep=},
    ] 
    \addlegendentry{$y[k]$},
    \addlegendentry{$y^{\mathrm{zad}}[k]$},

    \addlegendimage{no markers,red}
    \addlegendimage{no markers,blue}

    \addplot[red, semithick] file{../data/project/zad4/zad4DMC_output_D_303_N_50_Nu_50_lam_1_E_35395.6.csv};  
    \addplot[blue, semithick] file{../data/project/zad4/zad4DMC_setpoint_D_303_N_50_Nu_50_lam_1_E_35395.6.csv};
    
    \end{axis}
    \end{tikzpicture}
    \caption{Odpowiedzi procesu na skokową zmianę sygnału sterującego dla $K = \num{303}$, $N = \num{50}$, $N_{\mathrm{u}} = \num{50}$ i $\lambda = 1$}
    \label{pro_zad4_d_303_n_50_nu_50_out}
\end{figure}

\begin{figure}[b]
    \centering
    \begin{tikzpicture}
    \begin{axis}[
    width=0.98\textwidth,
    xmin=0.0,xmax=5000,ymin=-1.2,ymax=1.2,
    xlabel={$k$},
    ylabel={$y[k]$},
    legend pos=south east,
    y tick label style={/pgf/number format/1000 sep=},
    ] 
    \addlegendentry{$u[k]$},
    
    \addlegendimage{no markers,blue}

    \addplot[blue, semithick] file{../data/project/zad4/zad4DMC_input_D_303_N_50_Nu_50_lam_1_E_35395.6.csv};
    
    \end{axis}
    \end{tikzpicture}
    \caption{Odpowiedzi procesu na skokową zmianę sygnału sterującego dla $D = \num{303}$, $N = \num{50}$, $N_{\mathrm{u}} = \num{50} $ i $\lambda = 1$}
    \label{pro_zad4_d_303_n_50_nu_50_inp}
\end{figure}

\begin{figure}[t]
    \centering
    \begin{tikzpicture}
    \begin{axis}[
    width=0.98\textwidth,
    xmin=0.0,xmax=5000,ymin=-4.2,ymax=10,
    xlabel={$k$},
    ylabel={$y[k]$},
    legend pos=south east,
    y tick label style={/pgf/number format/1000 sep=},
    ] 
    \addlegendentry{$y[k]$},
    \addlegendentry{$y^{\mathrm{zad}}[k]$},

    \addlegendimage{no markers,red}
    \addlegendimage{no markers,blue}

    \addplot[red, semithick] file{../data/project/zad4/zad4DMC_output_D_303_N_50_Nu_10_lam_1_E_35377.3163.csv};  
    \addplot[blue, semithick] file{../data/project/zad4/zad4DMC_setpoint_D_303_N_50_Nu_10_lam_1_E_35377.3163.csv};
    
    \end{axis}
    \end{tikzpicture}
    \caption{Odpowiedzi procesu na skokową zmianę sygnału sterującego dla $K = \num{303}$, $N = \num{50}$, $N_{\mathrm{u}} = \num{10}$ i $\lambda = 1$}
    \label{pro_zad4_d_303_n_50_nu_10_out}
\end{figure}

\begin{figure}[b]
    \centering
    \begin{tikzpicture}
    \begin{axis}[
    width=0.98\textwidth,
    xmin=0.0,xmax=5000,ymin=-1.2,ymax=1.2,
    xlabel={$k$},
    ylabel={$y[k]$},
    legend pos=south east,
    y tick label style={/pgf/number format/1000 sep=},
    ] 
    \addlegendentry{$u[k]$},
    
    \addlegendimage{no markers,blue}

    \addplot[blue, semithick] file{../data/project/zad4/zad4DMC_input_D_303_N_50_Nu_10_lam_1_E_35377.3163.csv};
    
    \end{axis}
    \end{tikzpicture}
    \caption{Odpowiedzi procesu na skokową zmianę sygnału sterującego dla $D = \num{303}$, $N = \num{50}$, $N_{\mathrm{u}} = \num{10} $ i $\lambda = 1$}
    \label{pro_zad4_d_303_n_50_nu_10_inp}
\end{figure}


\begin{figure}[t]
    \centering
    \begin{tikzpicture}
    \begin{axis}[
    width=0.98\textwidth,
    xmin=0.0,xmax=5000,ymin=-4.2,ymax=10,
    xlabel={$k$},
    ylabel={$y[k]$},
    legend pos=south east,
    y tick label style={/pgf/number format/1000 sep=},
    ] 
    \addlegendentry{$y[k]$},
    \addlegendentry{$y^{\mathrm{zad}}[k]$},

    \addlegendimage{no markers,red}
    \addlegendimage{no markers,blue}

    \addplot[red, semithick] file{../data/project/zad4/zad4DMC_output_D_303_N_50_Nu_10_lam_10_E_32550.4148.csv};  
    \addplot[blue, semithick] file{../data/project/zad4/zad4DMC_setpoint_D_303_N_50_Nu_10_lam_10_E_32550.4148.csv};
    
    \end{axis}
    \end{tikzpicture}
    \caption{Odpowiedzi procesu na skokową zmianę sygnału sterującego dla $K = \num{303}$, $N = \num{50}$, $N_{\mathrm{u}} = \num{10}$ i $\lambda = 10$}
    \label{pro_zad4_d_303_n_50_nu_10_lam_10out}
\end{figure}

\begin{figure}[b]
    \centering
    \begin{tikzpicture}
    \begin{axis}[
    width=0.98\textwidth,
    xmin=0.0,xmax=5000,ymin=-1.2,ymax=1.2,
    xlabel={$k$},
    ylabel={$y[k]$},
    legend pos=south east,
    y tick label style={/pgf/number format/1000 sep=},
    ] 
    \addlegendentry{$u[k]$},
    
    \addlegendimage{no markers,blue}

    \addplot[blue, semithick] file{../data/project/zad4/zad4DMC_input_D_303_N_50_Nu_10_lam_10_E_32550.4148.csv};
    
    \end{axis}
    \end{tikzpicture}
    \caption{Odpowiedzi procesu na skokową zmianę sygnału sterującego dla $D = \num{303}$, $N = \num{50}$, $N_{\mathrm{u}} = \num{10} $ i $\lambda = 10$}
    \label{pro_zad4_d_303_n_50_nu_10_lam_10inp}
\end{figure}

\begin{figure}[t]
    \centering
    \begin{tikzpicture}
    \begin{axis}[
    width=0.98\textwidth,
    xmin=0.0,xmax=5000,ymin=-4.2,ymax=10,
    xlabel={$k$},
    ylabel={$y[k]$},
    legend pos=south east,
    y tick label style={/pgf/number format/1000 sep=},
    ] 
    \addlegendentry{$y[k]$},
    \addlegendentry{$y^{\mathrm{zad}}[k]$},

    \addlegendimage{no markers,red}
    \addlegendimage{no markers,blue}

    \addplot[red, semithick] file{../data/project/zad4/zad4DMC_output_D_303_N_50_Nu_10_lam_50_E_32109.6505.csv};  
    \addplot[blue, semithick] file{../data/project/zad4/zad4DMC_setpoint_D_303_N_50_Nu_10_lam_50_E_32109.6505.csv};
    
    \end{axis}
    \end{tikzpicture}
    \caption{Odpowiedzi procesu na skokową zmianę sygnału sterującego dla $K = \num{303}$, $N = \num{50}$, $N_{\mathrm{u}} = \num{10}$ i $\lambda = 50$}
    \label{pro_zad4_d_303_n_50_nu_10_lam_50out}
\end{figure}

\begin{figure}[b]
    \centering
    \begin{tikzpicture}
    \begin{axis}[
    width=0.98\textwidth,
    xmin=0.0,xmax=5000,ymin=-1.2,ymax=1.2,
    xlabel={$k$},
    ylabel={$y[k]$},
    legend pos=south east,
    y tick label style={/pgf/number format/1000 sep=},
    ] 
    \addlegendentry{$u[k]$},
    
    \addlegendimage{no markers,blue}

    \addplot[blue, semithick] file{../data/project/zad4/zad4DMC_input_D_303_N_50_Nu_10_lam_50_E_32109.6505.csv};
    
    \end{axis}
    \end{tikzpicture}
    \caption{Odpowiedzi procesu na skokową zmianę sygnału sterującego dla $D = \num{303}$, $N = \num{50}$, $N_{\mathrm{u}} = \num{10} $ i $\lambda = 50$}
    \label{pro_zad4_d_303_n_50_nu_10_lam_50inp}
\end{figure}

\begin{figure}[t]
    \centering
    \begin{tikzpicture}
    \begin{axis}[
    width=0.98\textwidth,
    xmin=0.0,xmax=5000,ymin=-4.2,ymax=10,
    xlabel={$k$},
    ylabel={$y[k]$},
    legend pos=south east,
    y tick label style={/pgf/number format/1000 sep=},
    ] 
    \addlegendentry{$y[k]$},
    \addlegendentry{$y^{\mathrm{zad}}[k]$},

    \addlegendimage{no markers,red}
    \addlegendimage{no markers,blue}

    \addplot[red, semithick] file{../data/project/zad4/zad4DMC_output_D_303_N_50_Nu_10_lam_120_E_32189.1237.csv};  
    \addplot[blue, semithick] file{../data/project/zad4/zad4DMC_setpoint_D_303_N_50_Nu_10_lam_120_E_32189.1237.csv};
    
    \end{axis}
    \end{tikzpicture}
    \caption{Odpowiedzi procesu na skokową zmianę sygnału sterującego dla $K = \num{303}$, $N = \num{50}$, $N_{\mathrm{u}} = \num{10}$ i $\lambda = 120$}
    \label{pro_zad4_d_303_n_50_nu_10_lam_120out}
\end{figure}

\begin{figure}[b]
    \centering
    \begin{tikzpicture}
    \begin{axis}[
    width=0.98\textwidth,
    xmin=0.0,xmax=5000,ymin=-1.2,ymax=1.2,
    xlabel={$k$},
    ylabel={$y[k]$},
    legend pos=south east,
    y tick label style={/pgf/number format/1000 sep=},
    ] 
    \addlegendentry{$u[k]$},
    
    \addlegendimage{no markers,blue}

    \addplot[blue, semithick] file{../data/project/zad4/zad4DMC_input_D_303_N_50_Nu_10_lam_120_E_32189.1237.csv};
    
    \end{axis}
    \end{tikzpicture}
    \caption{Odpowiedzi procesu na skokową zmianę sygnału sterującego dla $D = \num{303}$, $N = \num{50}$, $N_{\mathrm{u}} = \num{10} $ i $\lambda = 120$}
    \label{pro_zad4_d_303_n_50_nu_10_lam_120inp}
\end{figure}
\section{Dobór parametrów lokalnych}
Do badania i generacji lokalnych regulatorów napisaliśmy skrypty \verb|fuzzy_pid.m| i \verb|fuzzy_dmc.m|. Za pomocą zmiennej \verb|LOCAL_REGS| wybieraliśmy ilość lokalnych regulatorów. Do znalezienia optymalnych parametrów naszych regulatorów lokalnych posłużyła nam funkcja \textit{fmincon}, która tak dobiera parametry regulatorów, żeby wskaźnik jakości był jak najmniejszy. Symulację przeprowadziliśmy $2, 3, 4, 5, 6$ i $12$ regulatorów lokalnych. Dla każdego przypadku dokonaliśmy testów dla trzech różnych funkcji przynależności funkcji gausowskiej, trójkątnej i trapezowej. 
\section{Strojenie regulatorów lokalnych PID}

\subsection{Przebieg strojenia dla dwóch lokalnych regulatorów PID}
Na rysunkach \ref{pro_zad5_bell_2_inp}, \ref{pro_zad5_bell_2_out}, \ref{pro_zad5_triangle_2_inp}, \ref{pro_zad5_triangle_2_out} i \ref{pro_zad5_trapezoid_2_inp}, \ref{pro_zad5_trapezoid_2_out} możemy zobaczyć jak przebiega proces regulacji w przypadku dwóch lokalnych regulatorów PID. Do 700 próbki regulacja przebiega pomyślnie, po tym czasie zaczynają się pojawiać gasnące oscylacje przy osiąganiu wartości zadanej. Najprawdopodobniej wynika to z tego, że nastawy jednego z lokalnych regulatorów nie są wystarczająco dobre, aby zapewnić satysfakcjonującą regulację w okolicach tego punktu pacy. 
\begin{figure}[b]
    \centering
    \begin{tikzpicture}
    \begin{axis}[
    width=0.98\textwidth,
    xmin=0.0,xmax=1000,ymin=-1,ymax=11.5,
    xlabel={$k$},
    ylabel={$y[k]$},
    legend pos=south east,
    y tick label style={/pgf/number format/1000 sep=},
    ] 
    \addlegendentry{$y[k]$},
    \addlegendentry{$y^{\mathrm{zad}}[k]$},

    \addlegendimage{no markers,red}
    \addlegendimage{no markers,blue}

    \addplot[red, semithick] file{../data/project/zad5/output_trapezoid_2.csv};  
    \addplot[blue, semithick] file{../data/project/zad5/stpt_trapezoid_2.csv};
    
    \end{axis}
    \end{tikzpicture}
    \caption{Wyjście procesu dwóch regulatorów lokalnych PID z trapezową funkcją przynależności}
    \label{pro_zad5_trapezoid_2_out}
\end{figure}

\begin{figure}[b]
    \centering
    \begin{tikzpicture}
    \begin{axis}[
    width=0.98\textwidth,
    xmin=0.0,xmax=1000,ymin=-0.4,ymax=1.2,
    xlabel={$k$},
    ylabel={$y[k]$},
    legend pos=south east,
    y tick label style={/pgf/number format/1000 sep=},
    ] 
    \addlegendentry{$u[k]$},
    
    \addlegendimage{no markers,blue}

    \addplot[const plot, blue, semithick] file{../data/project/zad5/input_trapezoid_2.csv};
    
    \end{axis}
    \end{tikzpicture}
    \caption{Przebieg sygnału sterującego dla dwóch regulatorów lokalnych PID z trapezową funkcją przynależności}
    \label{pro_zad5_trapezoid_2_inp}
\end{figure}


\begin{figure}[t]
    \centering
    \begin{tikzpicture}
    \begin{axis}[
    width=0.98\textwidth,
    xmin=0.0,xmax=1000,ymin=-1,ymax=11.5,
    xlabel={$k$},
    ylabel={$y[k]$},
    legend pos=south east,
    y tick label style={/pgf/number format/1000 sep=},
    ] 
    \addlegendentry{$y[k]$},
    \addlegendentry{$y^{\mathrm{zad}}[k]$},

    \addlegendimage{no markers,red}
    \addlegendimage{no markers,blue}

    \addplot[red, semithick] file{../data/project/zad5/output_bell_2.csv};  
    \addplot[blue, semithick] file{../data/project/zad5/stpt_bell_2.csv};
    
    \end{axis}
    \end{tikzpicture}
    \caption{Wyjście procesu dwóch regulatorów lokalnych PID z gaussowską funkcją przynależności}
    \label{pro_zad5_bell_2_out}
\end{figure}

\begin{figure}[b]
    \centering
    \begin{tikzpicture}
    \begin{axis}[
    width=0.98\textwidth,
    xmin=0.0,xmax=1000,ymin=-0.4,ymax=1.2,
    xlabel={$k$},
    ylabel={$y[k]$},
    legend pos=south east,
    y tick label style={/pgf/number format/1000 sep=},
    ] 
    \addlegendentry{$u[k]$},
    
    \addlegendimage{no markers,blue}

    \addplot[const plot, blue, semithick] file{../data/project/zad5/input_bell_2.csv};
    
    \end{axis}
    \end{tikzpicture}
    \caption{Przebieg sygnału sterującego dla dwóch regulatorów lokalnych PID z gaussowską funkcją przynależności}
    \label{pro_zad5_bell_2_inp}
\end{figure}


\begin{figure}[t]
    \centering
    \begin{tikzpicture}
    \begin{axis}[
    width=0.98\textwidth,
    xmin=0.0,xmax=1000,ymin=-1,ymax=11.5,
    xlabel={$k$},
    ylabel={$y[k]$},
    legend pos=south east,
    y tick label style={/pgf/number format/1000 sep=},
    ] 
    \addlegendentry{$y[k]$},
    \addlegendentry{$y^{\mathrm{zad}}[k]$},

    \addlegendimage{no markers,red}
    \addlegendimage{no markers,blue}

    \addplot[red, semithick] file{../data/project/zad5/output_triangle_2.csv};  
    \addplot[blue, semithick] file{../data/project/zad5/stpt_triangle_2.csv};
    
    \end{axis}
    \end{tikzpicture}
    \caption{Wyjście procesu dwóch regulatorów lokalnych PID z trójkątną funkcją przynależności}
    \label{pro_zad5_triangle_2_out}
\end{figure}

\begin{figure}[b]
    \centering
    \begin{tikzpicture}
    \begin{axis}[
    width=0.98\textwidth,
    xmin=0.0,xmax=1000,ymin=-0.4,ymax=1.2,
    xlabel={$k$},
    ylabel={$y[k]$},
    legend pos=south east,
    y tick label style={/pgf/number format/1000 sep=},
    ] 
    \addlegendentry{$u[k]$},
    
    \addlegendimage{no markers,blue}

    \addplot[const plot, blue, semithick] file{../data/project/zad5/input_triangle_2.csv};
    
    \end{axis}
    \end{tikzpicture}
    \caption{Przebieg sygnału sterującego dla dwóch regulatorów lokalnych PID z trójkątną  funkcją przynależności}
    \label{pro_zad5_triangle_2_inp}
\end{figure}
\FloatBarrier

\subsection{Przebieg strojenia dla trzech lokalnych regulatorów PID}
Na rysunkach \ref{pro_zad5_bell_3_inp}, \ref{pro_zad5_bell_3_out}, \ref{pro_zad5_triangle_3_inp}, \ref{pro_zad5_triangle_3_out} i \ref{pro_zad5_trapezoid_3_inp}, \ref{pro_zad5_trapezoid_3_out} możemy zobaczyć jak przebiega proces regulacji w przypadku trzech lokalnych regulatorów PID. Jest on gorszy niż dla dwóch regulatorów bo dla pewnych wartości zadanych sygnał ten charakteryzuje się niegasnącymi oscylacjami. Sytuacja ta występuje dla wszystkich rodzajów funkcji przynależności. 
\begin{figure}[b]
    \centering
    \begin{tikzpicture}
    \begin{axis}[
    width=0.98\textwidth,
    xmin=0.0,xmax=1000,ymin=-1,ymax=11.5,
    xlabel={$k$},
    ylabel={$y[k]$},
    legend pos=south east,
    y tick label style={/pgf/number format/1000 sep=},
    ] 
    \addlegendentry{$y[k]$},
    \addlegendentry{$y^{\mathrm{zad}}[k]$},

    \addlegendimage{no markers,red}
    \addlegendimage{no markers,blue}

    \addplot[red, semithick] file{../data/project/zad5/output_trapezoid_3.csv};  
    \addplot[blue, semithick] file{../data/project/zad5/stpt_trapezoid_3.csv};
    
    \end{axis}
    \end{tikzpicture}
    \caption{Wyjście procesu trzech regulatorów lokalnych PID z trapezową funkcją przynależności}
    \label{pro_zad5_trapezoid_3_out}
\end{figure}

\begin{figure}[b]
    \centering
    \begin{tikzpicture}
    \begin{axis}[
    width=0.98\textwidth,
    xmin=0.0,xmax=1000,ymin=-0.4,ymax=1.2,
    xlabel={$k$},
    ylabel={$y[k]$},
    legend pos=south east,
    y tick label style={/pgf/number format/1000 sep=},
    ] 
    \addlegendentry{$u[k]$},
    
    \addlegendimage{no markers,blue}

    \addplot[const plot, blue, semithick] file{../data/project/zad5/input_trapezoid_3.csv};
    
    \end{axis}
    \end{tikzpicture}
    \caption{Przebieg sygnału sterującego dla trzech regulatorów lokalnych PID z trapezową funkcją przynależności}
    \label{pro_zad5_trapezoid_3_inp}
\end{figure}


\begin{figure}[t]
    \centering
    \begin{tikzpicture}
    \begin{axis}[
    width=0.98\textwidth,
    xmin=0.0,xmax=1000,ymin=-1,ymax=11.5,
    xlabel={$k$},
    ylabel={$y[k]$},
    legend pos=south east,
    y tick label style={/pgf/number format/1000 sep=},
    ] 
    \addlegendentry{$y[k]$},
    \addlegendentry{$y^{\mathrm{zad}}[k]$},

    \addlegendimage{no markers,red}
    \addlegendimage{no markers,blue}

    \addplot[red, semithick] file{../data/project/zad5/output_bell_3.csv};  
    \addplot[blue, semithick] file{../data/project/zad5/stpt_bell_3.csv};
    
    \end{axis}
    \end{tikzpicture}
    \caption{Wyjście procesu trzech regulatorów lokalnych PID z gaussowską funkcją przynależności}
    \label{pro_zad5_bell_3_out}
\end{figure}

\begin{figure}[b]
    \centering
    \begin{tikzpicture}
    \begin{axis}[
    width=0.98\textwidth,
    xmin=0.0,xmax=1000,ymin=-0.4,ymax=1.2,
    xlabel={$k$},
    ylabel={$y[k]$},
    legend pos=south east,
    y tick label style={/pgf/number format/1000 sep=},
    ] 
    \addlegendentry{$u[k]$},
    
    \addlegendimage{no markers,blue}

    \addplot[const plot, blue, semithick] file{../data/project/zad5/input_bell_3.csv};
    
    \end{axis}
    \end{tikzpicture}
    \caption{Przebieg sygnału sterującego dla trzech regulatorów lokalnych PID z gaussowską funkcją przynależności}
    \label{pro_zad5_bell_3_inp}
\end{figure}

\begin{figure}[t]
    \centering
    \begin{tikzpicture}
    \begin{axis}[
    width=0.98\textwidth,
    xmin=0.0,xmax=1000,ymin=-1,ymax=11.5,
    xlabel={$k$},
    ylabel={$y[k]$},
    legend pos=south east,
    y tick label style={/pgf/number format/1000 sep=},
    ] 
    \addlegendentry{$y[k]$},
    \addlegendentry{$y^{\mathrm{zad}}[k]$},

    \addlegendimage{no markers,red}
    \addlegendimage{no markers,blue}

    \addplot[red, semithick] file{../data/project/zad5/output_triangle_3.csv};  
    \addplot[blue, semithick] file{../data/project/zad5/stpt_triangle_3.csv};
    
    \end{axis}
    \end{tikzpicture}
    \caption{Wyjście procesu trzech regulatorów lokalnych PID z trójkątną funkcją przynależności}
    \label{pro_zad5_triangle_3_out}
\end{figure}

\begin{figure}[b]
    \centering
    \begin{tikzpicture}
    \begin{axis}[
    width=0.98\textwidth,
    xmin=0.0,xmax=1000,ymin=-0.4,ymax=1.2,
    xlabel={$k$},
    ylabel={$y[k]$},
    legend pos=south east,
    y tick label style={/pgf/number format/1000 sep=},
    ] 
    \addlegendentry{$u[k]$},
    
    \addlegendimage{no markers,blue}

    \addplot[const plot, blue, semithick] file{../data/project/zad5/input_triangle_3.csv};
    
    \end{axis}
    \end{tikzpicture}
    \caption{Przebieg sygnału sterującego dla trzech regulatorów lokalnych PID z trójkątną  funkcją przynależności}
    \label{pro_zad5_triangle_3_inp}
\end{figure}
\FloatBarrier

\subsection{Przebieg strojenia dla czterech lokalnych regulatorów PID}
Dla czterech regulatorów lokalnych regulacja przebiega najlepiej w porównaniu do poprzednich przypadków. Jedynym minusem jest fakt, że tak dobrany regulator rozmyty ma tendencję do generowania krótkich przeregulowań podczas zmiany wartości zadanej. Jednak nie są one na tyle duże, żeby dyskwalifikowały taki regulator rozmyty. Przebiegi sygnałów zostały zaprezentowane na rysunkach \ref{pro_zad5_bell_4_inp}, \ref{pro_zad5_bell_4_out}, \ref{pro_zad5_triangle_4_inp}, \ref{pro_zad5_triangle_4_out} i \ref{pro_zad5_trapezoid_4_inp}, \ref{pro_zad5_trapezoid_4_out}.  

\begin{figure}[b]
    \centering
    \begin{tikzpicture}
    \begin{axis}[
    width=0.98\textwidth,
    xmin=0.0,xmax=1000,ymin=-1,ymax=11.5,
    xlabel={$k$},
    ylabel={$y[k]$},
    legend pos=south east,
    y tick label style={/pgf/number format/1000 sep=},
    ] 
    \addlegendentry{$y[k]$},
    \addlegendentry{$y^{\mathrm{zad}}[k]$},

    \addlegendimage{no markers,red}
    \addlegendimage{no markers,blue}

    \addplot[red, semithick] file{../data/project/zad5/output_trapezoid_4.csv};  
    \addplot[blue, semithick] file{../data/project/zad5/stpt_trapezoid_4.csv};
    
    \end{axis}
    \end{tikzpicture}
    \caption{Wyjście procesu czterech regulatorów lokalnych PID z trapezową funkcją przynależności}
    \label{pro_zad5_trapezoid_4_out}
\end{figure}

\begin{figure}[b]
    \centering
    \begin{tikzpicture}
    \begin{axis}[
    width=0.98\textwidth,
    xmin=0.0,xmax=1000,ymin=-0.4,ymax=1.2,
    xlabel={$k$},
    ylabel={$y[k]$},
    legend pos=south east,
    y tick label style={/pgf/number format/1000 sep=},
    ] 
    \addlegendentry{$u[k]$},
    
    \addlegendimage{no markers,blue}

    \addplot[const plot, blue, semithick] file{../data/project/zad5/input_trapezoid_4.csv};
    
    \end{axis}
    \end{tikzpicture}
    \caption{Przebieg sygnału sterującego dla czterech regulatorów lokalnych PID z trapezową funkcją przynależności}
    \label{pro_zad5_trapezoid_4_inp}
\end{figure}


\begin{figure}[t]
    \centering
    \begin{tikzpicture}
    \begin{axis}[
    width=0.98\textwidth,
    xmin=0.0,xmax=1000,ymin=-1,ymax=11.5,
    xlabel={$k$},
    ylabel={$y[k]$},
    legend pos=south east,
    y tick label style={/pgf/number format/1000 sep=},
    ] 
    \addlegendentry{$y[k]$},
    \addlegendentry{$y^{\mathrm{zad}}[k]$},

    \addlegendimage{no markers,red}
    \addlegendimage{no markers,blue}

    \addplot[red, semithick] file{../data/project/zad5/output_bell_4.csv};  
    \addplot[blue, semithick] file{../data/project/zad5/stpt_bell_4.csv};
    
    \end{axis}
    \end{tikzpicture}
    \caption{Wyjście procesu czterech regulatorów lokalnych PID z gaussowską funkcją przynależności}
    \label{pro_zad5_bell_4_out}
\end{figure}

\begin{figure}[b]
    \centering
    \begin{tikzpicture}
    \begin{axis}[
    width=0.98\textwidth,
    xmin=0.0,xmax=1000,ymin=-0.4,ymax=1.2,
    xlabel={$k$},
    ylabel={$y[k]$},
    legend pos=south east,
    y tick label style={/pgf/number format/1000 sep=},
    ] 
    \addlegendentry{$u[k]$},
    
    \addlegendimage{no markers,blue}

    \addplot[const plot, blue, semithick] file{../data/project/zad5/input_bell_4.csv};
    
    \end{axis}
    \end{tikzpicture}
    \caption{Przebieg sygnału sterującego dla czterech regulatorów lokalnych PID z gaussowską funkcją przynależności}
    \label{pro_zad5_bell_4_inp}
\end{figure}

\begin{figure}[t]
    \centering
    \begin{tikzpicture}
    \begin{axis}[
    width=0.98\textwidth,
    xmin=0.0,xmax=1000,ymin=-1,ymax=11.5,
    xlabel={$k$},
    ylabel={$y[k]$},
    legend pos=south east,
    y tick label style={/pgf/number format/1000 sep=},
    ] 
    \addlegendentry{$y[k]$},
    \addlegendentry{$y^{\mathrm{zad}}[k]$},

    \addlegendimage{no markers,red}
    \addlegendimage{no markers,blue}

    \addplot[red, semithick] file{../data/project/zad5/output_triangle_4.csv};  
    \addplot[blue, semithick] file{../data/project/zad5/stpt_triangle_4.csv};
    
    \end{axis}
    \end{tikzpicture}
    \caption{Wyjście procesu czterech regulatorów lokalnych PID z trójkątną funkcją przynależności}
    \label{pro_zad5_triangle_4_out}
\end{figure}

\begin{figure}[b]
    \centering
    \begin{tikzpicture}
    \begin{axis}[
    width=0.98\textwidth,
    xmin=0.0,xmax=1000,ymin=-0.4,ymax=1.2,
    xlabel={$k$},
    ylabel={$y[k]$},
    legend pos=south east,
    y tick label style={/pgf/number format/1000 sep=},
    ] 
    \addlegendentry{$u[k]$},
    
    \addlegendimage{no markers,blue}

    \addplot[const plot, blue, semithick] file{../data/project/zad5/input_triangle_4.csv};
    
    \end{axis}
    \end{tikzpicture}
    \caption{Przebieg sygnału sterującego dla czterech regulatorów lokalnych PID z trójkątną  funkcją przynależności}
    \label{pro_zad5_triangle_4_inp}
\end{figure}
\FloatBarrier

\subsection{Przebieg strojenia dla pięciu lokalnych regulatorów PID}
Dla pięciu regulatorów lokalnych regulacja przebiega zadowalająco tylko dla wartości zadanej równej 11. W przypadku innych wielkości pojawiają się oscylacje i duże przeregulowania. Po analizie wyników doszliśmy do wniosku, że regulacja dla pięciu lokalnych regulatorów PID nie jest poprawna. Przebiegi sygnałów zostały zaprezentowane na rysunkach \ref{pro_zad5_bell_5_inp}, \ref{pro_zad5_bell_5_out}, \ref{pro_zad5_triangle_5_inp}, \ref{pro_zad5_triangle_5_out} i \ref{pro_zad5_trapezoid_5_inp}, \ref{pro_zad5_trapezoid_5_out}.
\begin{figure}[b]
    \centering
    \begin{tikzpicture}
    \begin{axis}[
    width=0.98\textwidth,
    xmin=0.0,xmax=1000,ymin=-1,ymax=11.5,
    xlabel={$k$},
    ylabel={$y[k]$},
    legend pos=south east,
    y tick label style={/pgf/number format/1000 sep=},
    ] 
    \addlegendentry{$y[k]$},
    \addlegendentry{$y^{\mathrm{zad}}[k]$},

    \addlegendimage{no markers,red}
    \addlegendimage{no markers,blue}

    \addplot[red, semithick] file{../data/project/zad5/output_trapezoid_5.csv};  
    \addplot[blue, semithick] file{../data/project/zad5/stpt_trapezoid_5.csv};
    
    \end{axis}
    \end{tikzpicture}
    \caption{Wyjście procesu pięciu regulatorów lokalnych PID z trapezową funkcją przynależności}
    \label{pro_zad5_trapezoid_5_out}
\end{figure}

\begin{figure}[b]
    \centering
    \begin{tikzpicture}
    \begin{axis}[
    width=0.98\textwidth,
    xmin=0.0,xmax=1000,ymin=-0.4,ymax=1.2,
    xlabel={$k$},
    ylabel={$y[k]$},
    legend pos=south east,
    y tick label style={/pgf/number format/1000 sep=},
    ] 
    \addlegendentry{$u[k]$},
    
    \addlegendimage{no markers,blue}

    \addplot[const plot, blue, semithick] file{../data/project/zad5/input_trapezoid_5.csv};
    
    \end{axis}
    \end{tikzpicture}
    \caption{Przebieg sygnału sterującego dla pięciu regulatorów lokalnych PID z trapezową funkcją przynależności}
    \label{pro_zad5_trapezoid_5_inp}
\end{figure}


\begin{figure}[t]
    \centering
    \begin{tikzpicture}
    \begin{axis}[
    width=0.98\textwidth,
    xmin=0.0,xmax=1000,ymin=-1,ymax=11.5,
    xlabel={$k$},
    ylabel={$y[k]$},
    legend pos=south east,
    y tick label style={/pgf/number format/1000 sep=},
    ] 
    \addlegendentry{$y[k]$},
    \addlegendentry{$y^{\mathrm{zad}}[k]$},

    \addlegendimage{no markers,red}
    \addlegendimage{no markers,blue}

    \addplot[red, semithick] file{../data/project/zad5/output_bell_5.csv};  
    \addplot[blue, semithick] file{../data/project/zad5/stpt_bell_5.csv};
    
    \end{axis}
    \end{tikzpicture}
    \caption{Wyjście procesu dla pięciu regulatorów lokalnych PID z gaussowską funkcją przynależności}
    \label{pro_zad5_bell_5_out}
\end{figure}

\begin{figure}[b]
    \centering
    \begin{tikzpicture}
    \begin{axis}[
    width=0.98\textwidth,
    xmin=0.0,xmax=1000,ymin=-0.4,ymax=1.2,
    xlabel={$k$},
    ylabel={$y[k]$},
    legend pos=south east,
    y tick label style={/pgf/number format/1000 sep=},
    ] 
    \addlegendentry{$u[k]$},
    
    \addlegendimage{no markers,blue}

    \addplot[const plot, blue, semithick] file{../data/project/zad5/input_bell_5.csv};
    
    \end{axis}
    \end{tikzpicture}
    \caption{Przebieg sygnału sterującego dla pięciu regulatorów lokalnych PID z gaussowską funkcją przynależności}
    \label{pro_zad5_bell_5_inp}
\end{figure}

\begin{figure}[t]
    \centering
    \begin{tikzpicture}
    \begin{axis}[
    width=0.98\textwidth,
    xmin=0.0,xmax=1000,ymin=-1,ymax=11.5,
    xlabel={$k$},
    ylabel={$y[k]$},
    legend pos=south east,
    y tick label style={/pgf/number format/1000 sep=},
    ] 
    \addlegendentry{$y[k]$},
    \addlegendentry{$y^{\mathrm{zad}}[k]$},

    \addlegendimage{no markers,red}
    \addlegendimage{no markers,blue}

    \addplot[red, semithick] file{../data/project/zad5/output_triangle_5.csv};  
    \addplot[blue, semithick] file{../data/project/zad5/stpt_triangle_5.csv};
    
    \end{axis}
    \end{tikzpicture}
    \caption{Wyjście procesu dla pięciu regulatorów lokalnych PID z trójkątną funkcją przynależności}
    \label{pro_zad5_triangle_5_out}
\end{figure}

\begin{figure}[b]
    \centering
    \begin{tikzpicture}
    \begin{axis}[
    width=0.98\textwidth,
    xmin=0.0,xmax=1000,ymin=-0.4,ymax=1.2,
    xlabel={$k$},
    ylabel={$y[k]$},
    legend pos=south east,
    y tick label style={/pgf/number format/1000 sep=},
    ] 
    \addlegendentry{$u[k]$},
    
    \addlegendimage{no markers,blue}

    \addplot[const plot, blue, semithick] file{../data/project/zad5/input_triangle_5.csv};
    
    \end{axis}
    \end{tikzpicture}
    \caption{Przebieg sygnału sterującego dla pięciu regulatorów lokalnych PID z trójkątną  funkcją przynależności}
    \label{pro_zad5_triangle_5_inp}
\end{figure}
\FloatBarrier

\subsection{Przebieg strojenia dla sześciu lokalnych regulatorów PID}
Jak możemy zauważyć na rysunkach \ref{pro_zad5_bell_6_inp}, \ref{pro_zad5_bell_6_out}, \ref{pro_zad5_triangle_6_inp}, \ref{pro_zad5_triangle_6_out} i \ref{pro_zad5_trapezoid_6_inp}, \ref{pro_zad5_trapezoid_6_out} możemy zobaczyć jak przebiega proces regulacji w przypadku sześciu lokalnych regulatorów PID. Wyniki oraz przebiegi sygnałów regulacji są zbliżone do tych otrzymanych dla pięciu lokalnych regulatorów. Pojawiają się te same problemy czyli oscylacje i zbyt duże przeregulowanie. 
\begin{figure}[b]
    \centering
    \begin{tikzpicture}
    \begin{axis}[
    width=0.98\textwidth,
    xmin=0.0,xmax=1000,ymin=-1,ymax=11.5,
    xlabel={$k$},
    ylabel={$y[k]$},
    legend pos=south east,
    y tick label style={/pgf/number format/1000 sep=},
    ] 
    \addlegendentry{$y[k]$},
    \addlegendentry{$y^{\mathrm{zad}}[k]$},

    \addlegendimage{no markers,red}
    \addlegendimage{no markers,blue}

    \addplot[red, semithick] file{../data/project/zad5/output_trapezoid_6.csv};  
    \addplot[blue, semithick] file{../data/project/zad5/stpt_trapezoid_6.csv};
    
    \end{axis}
    \end{tikzpicture}
    \caption{Wyjście procesu sześciu regulatorów lokalnych PID z trapezową funkcją przynależności}
    \label{pro_zad5_trapezoid_6_out}
\end{figure}

\begin{figure}[b]
    \centering
    \begin{tikzpicture}
    \begin{axis}[
    width=0.98\textwidth,
    xmin=0.0,xmax=1000,ymin=-0.4,ymax=1.2,
    xlabel={$k$},
    ylabel={$y[k]$},
    legend pos=south east,
    y tick label style={/pgf/number format/1000 sep=},
    ] 
    \addlegendentry{$u[k]$},
    
    \addlegendimage{no markers,blue}

    \addplot[const plot, blue, semithick] file{../data/project/zad5/input_trapezoid_6.csv};
    
    \end{axis}
    \end{tikzpicture}
    \caption{Przebieg sygnału sterującego dla sześciu regulatorów lokalnych PID z trapezową funkcją przynależności}
    \label{pro_zad5_trapezoid_6_inp}
\end{figure}


\begin{figure}[t]
    \centering
    \begin{tikzpicture}
    \begin{axis}[
    width=0.98\textwidth,
    xmin=0.0,xmax=1000,ymin=-1,ymax=11.5,
    xlabel={$k$},
    ylabel={$y[k]$},
    legend pos=south east,
    y tick label style={/pgf/number format/1000 sep=},
    ] 
    \addlegendentry{$y[k]$},
    \addlegendentry{$y^{\mathrm{zad}}[k]$},

    \addlegendimage{no markers,red}
    \addlegendimage{no markers,blue}

    \addplot[red, semithick] file{../data/project/zad5/output_bell_6.csv};  
    \addplot[blue, semithick] file{../data/project/zad5/stpt_bell_6.csv};
    
    \end{axis}
    \end{tikzpicture}
    \caption{Wyjście procesu dla sześciu regulatorów lokalnych PID z gaussowską funkcją przynależności}
    \label{pro_zad5_bell_6_out}
\end{figure}

\begin{figure}[b]
    \centering
    \begin{tikzpicture}
    \begin{axis}[
    width=0.98\textwidth,
    xmin=0.0,xmax=1000,ymin=-0.4,ymax=1.2,
    xlabel={$k$},
    ylabel={$y[k]$},
    legend pos=south east,
    y tick label style={/pgf/number format/1000 sep=},
    ] 
    \addlegendentry{$u[k]$},
    
    \addlegendimage{no markers,blue}

    \addplot[const plot, blue, semithick] file{../data/project/zad5/input_bell_6.csv};
    
    \end{axis}
    \end{tikzpicture}
    \caption{Przebieg sygnału sterującego dla sześciu regulatorów lokalnych PID z gaussowską funkcją przynależności}
    \label{pro_zad5_bell_6_inp}
\end{figure}

\begin{figure}[t]
    \centering
    \begin{tikzpicture}
    \begin{axis}[
    width=0.98\textwidth,
    xmin=0.0,xmax=1000,ymin=-1,ymax=11.5,
    xlabel={$k$},
    ylabel={$y[k]$},
    legend pos=south east,
    y tick label style={/pgf/number format/1000 sep=},
    ] 
    \addlegendentry{$y[k]$},
    \addlegendentry{$y^{\mathrm{zad}}[k]$},

    \addlegendimage{no markers,red}
    \addlegendimage{no markers,blue}

    \addplot[red, semithick] file{../data/project/zad5/output_triangle_6.csv};  
    \addplot[blue, semithick] file{../data/project/zad5/stpt_triangle_6.csv};
    
    \end{axis}
    \end{tikzpicture}
    \caption{Wyjście procesu dla sześciu regulatorów lokalnych PID z trójkątną funkcją przynależności}
    \label{pro_zad5_triangle_6_out}
\end{figure}

\begin{figure}[b]
    \centering
    \begin{tikzpicture}
    \begin{axis}[
    width=0.98\textwidth,
    xmin=0.0,xmax=1000,ymin=-0.4,ymax=1.2,
    xlabel={$k$},
    ylabel={$y[k]$},
    legend pos=south east,
    y tick label style={/pgf/number format/1000 sep=},
    ] 
    \addlegendentry{$u[k]$},
    
    \addlegendimage{no markers,blue}

    \addplot[const plot, blue, semithick] file{../data/project/zad5/input_triangle_6.csv};
    
    \end{axis}
    \end{tikzpicture}
    \caption{Przebieg sygnału sterującego dla sześciu regulatorów lokalnych PID z trójkątną  funkcją przynależności}
    \label{pro_zad5_triangle_6_inp}
\end{figure}
\FloatBarrier

\subsection{Przebieg strojenia dla dwunastu lokalnych regulatorów PID}
Na rysunkach \ref{pro_zad5_bell_12_inp}, \ref{pro_zad5_bell_12_out}, \ref{pro_zad5_triangle_12_inp}, \ref{pro_zad5_triangle_12_out} i \ref{pro_zad5_trapezoid_12_inp}, \ref{pro_zad5_trapezoid_12_out} możemy zobaczyć jak przebiega proces regulacji w przypadku dwunastu lokalnych regulatorów PID. W czasie całej symulacji pojawiają się bardzo duże niegasnące oscylacje. Doszliśmy do wniosku, że dla tego obiektu liczba dwunastu lokalnych regulatorów PID jest wyraźnie za duża, a zwiększanie liczebności regulatorów lokalnych nie prowadzi do poprawy regulacji.  
\begin{figure}[b]
    \centering
    \begin{tikzpicture}
    \begin{axis}[
    width=0.98\textwidth,
    xmin=0.0,xmax=1000,ymin=-1,ymax=11.5,
    xlabel={$k$},
    ylabel={$y[k]$},
    legend pos=south east,
    y tick label style={/pgf/number format/1000 sep=},
    ] 
    \addlegendentry{$y[k]$},
    \addlegendentry{$y^{\mathrm{zad}}[k]$},

    \addlegendimage{no markers,red}
    \addlegendimage{no markers,blue}

    \addplot[red, semithick] file{../data/project/zad5/output_trapezoid_12.csv};  
    \addplot[blue, semithick] file{../data/project/zad5/stpt_trapezoid_12.csv};
    
    \end{axis}
    \end{tikzpicture}
    \caption{Wyjście procesu dwunastu regulatorów lokalnych PID z trapezową funkcją przynależności}
    \label{pro_zad5_trapezoid_12_out}
\end{figure}

\begin{figure}[b]
    \centering
    \begin{tikzpicture}
    \begin{axis}[
    width=0.98\textwidth,
    xmin=0.0,xmax=1000,ymin=-1.2,ymax=1.2,
    xlabel={$k$},
    ylabel={$y[k]$},
    legend pos=south east,
    y tick label style={/pgf/number format/1000 sep=},
    ] 
    \addlegendentry{$u[k]$},
    
    \addlegendimage{no markers,blue}

    \addplot[const plot, blue, semithick] file{../data/project/zad5/input_trapezoid_12.csv};
    
    \end{axis}
    \end{tikzpicture}
    \caption{Przebieg sygnału sterującego dla dwunastu regulatorów lokalnych PID z trapezową funkcją przynależności}
    \label{pro_zad5_trapezoid_12_inp}
\end{figure}


\begin{figure}[t]
    \centering
    \begin{tikzpicture}
    \begin{axis}[
    width=0.98\textwidth,
    xmin=0.0,xmax=1000,ymin=-1,ymax=11.5,
    xlabel={$k$},
    ylabel={$y[k]$},
    legend pos=south east,
    y tick label style={/pgf/number format/1000 sep=},
    ] 
    \addlegendentry{$y[k]$},
    \addlegendentry{$y^{\mathrm{zad}}[k]$},

    \addlegendimage{no markers,red}
    \addlegendimage{no markers,blue}

    \addplot[red, semithick] file{../data/project/zad5/output_bell_12.csv};  
    \addplot[blue, semithick] file{../data/project/zad5/stpt_bell_12.csv};
    
    \end{axis}
    \end{tikzpicture}
    \caption{Wyjście procesu dla dwunastu regulatorów lokalnych PID z gaussowską funkcją przynależności}
    \label{pro_zad5_bell_12_out}
\end{figure}

\begin{figure}[b]
    \centering
    \begin{tikzpicture}
    \begin{axis}[
    width=0.98\textwidth,
    xmin=0.0,xmax=1000,ymin=-1.2,ymax=1.2,
    xlabel={$k$},
    ylabel={$y[k]$},
    legend pos=south east,
    y tick label style={/pgf/number format/1000 sep=},
    ] 
    \addlegendentry{$u[k]$},
    
    \addlegendimage{no markers,blue}

    \addplot[const plot, blue, semithick] file{../data/project/zad5/input_bell_12.csv};
    
    \end{axis}
    \end{tikzpicture}
    \caption{Przebieg sygnału sterującego dla dwunastu regulatorów lokalnych PID z gaussowską funkcją przynależności}
    \label{pro_zad5_bell_12_inp}
\end{figure}

\begin{figure}[t]
    \centering
    \begin{tikzpicture}
    \begin{axis}[
    width=0.98\textwidth,
    xmin=0.0,xmax=1000,ymin=-1,ymax=11.5,
    xlabel={$k$},
    ylabel={$y[k]$},
    legend pos=south east,
    y tick label style={/pgf/number format/1000 sep=},
    ] 
    \addlegendentry{$y[k]$},
    \addlegendentry{$y^{\mathrm{zad}}[k]$},

    \addlegendimage{no markers,red}
    \addlegendimage{no markers,blue}

    \addplot[red, semithick] file{../data/project/zad5/output_triangle_12.csv};  
    \addplot[blue, semithick] file{../data/project/zad5/stpt_triangle_12.csv};
    
    \end{axis}
    \end{tikzpicture}
    \caption{Wyjście procesu dla dwunastu regulatorów lokalnych PID z trójkątną funkcją przynależności}
    \label{pro_zad5_triangle_12_out}
\end{figure}

\begin{figure}[b]
    \centering
    \begin{tikzpicture}
    \begin{axis}[
    width=0.98\textwidth,
    xmin=0.0,xmax=1000,ymin=-1.2,ymax=1.2,
    xlabel={$k$},
    ylabel={$y[k]$},
    legend pos=south east,
    y tick label style={/pgf/number format/1000 sep=},
    ] 
    \addlegendentry{$u[k]$},
    
    \addlegendimage{no markers,blue}

    \addplot[const plot, blue, semithick] file{../data/project/zad5/input_triangle_12.csv};
    
    \end{axis}
    \end{tikzpicture}
    \caption{Przebieg sygnału sterującego dla dwunastu regulatorów lokalnych PID z trójkątną  funkcją przynależności}
    \label{pro_zad5_triangle_12_inp}
\end{figure}
\FloatBarrier

Jednogłośnie całym zespołem podjęliśmy decyzję, że najlepsze przebiegi sygnałów, a co za tym idzie najlepsza jakość regulacji występuje dla 4 regulatorów lokalnych PID.  
\section{Strojenie regulatorów lokalnych DMC}
Podczas strojenia i poszukiwania najlepszych nastawów regulatorów lokalnych DMC zmienialiśmy parametry D, N i $N_{u}$ . Jako funkcje przynależności posłużyły nam funkcja gaussowska, trójkątna i trapezowa. Wybór wartości współczynników D, N i $N_{u}$ był optymalizowany przez matlabową funkcję \textit{fmincon} tak aby wskaźnik jakości regulacji E był jak najmniejszy. Symulacje przeprowadziliśmy dla różnych liczb regulatorów lokalnych (2, 3, 4, 5, 6 i 12). Wartość współczynnika kary dla wszystkich przeprowadzonych eksperymentów była taka sama i wynosiła $\lambda = 1$.

\section{Podsumowanie strojenia regulatorów lokalnych DMC}
Poniżej prezentujemy przebiegi sygnałów na których możemy zobaczyć jak przebiega proces regulacji dla różnej ilości regulatorów lokalnych DMC. Gdy porównamy ze sobą wszystkie wykresy bardzo szybko dojdziemy do wniosku, że wyjścia procesu i przebiegi sygnału sterującego dla różnych nastawów są bardzo zbliżone do siebie. Wspólnie w zespole doszliśmy do wniosku, że różnice są praktycznie niezauważalne i nie mają znaczącego wpływu na cały proces regulacji. Identyczne wnioski wyciągnęliśmy dla wartości wskaźnika jakości regulacji E. Jego średnia wartość dla wszystkich eksperymentów wynosiła $E = 2030$. W zależności od ilości regulatorów lokalnych wskaźnik ten zmieniał się o $\Delta E = \pm 5$. Gdy weźmiemy pod uwagę długość symulacji wynoszącą 1000 próbek okazuje się, że zmiana ta jest bardzo mała. Regulatory DMC zachowują się na tyle elastycznie, że dla badanego obiektu ilość regulatorów lokalnych nie ma tak dużego wpływu. W tym przypadku nie zauważymy znaczących różnic pomiędzy użyciem dwóch, czy też dwunastu regulatorów rozmytych. 

Poniższe symulacje zostały wykonane dla wartości parametru $\lambda = 10000$, ponieważ dla tej wartości nasz regulator DMC był wstanie odpowiednio dobrze dokonać regulacji. Przy wartości $\lambda = 1$ zachodziły niegasnące oscylacje i wspólnie w zespole uznaliśmy, że prezentacja takich wyników i ich omawianie nie wnosi wartości merytorycznej do sprawozdania. Szerzej opis doboru współczynnika kary i jego wpływu na regulację, wraz z opisem zachowania dla $\lambda = 1$ zostanie zaprezentowany w rozdziale \ref{pro_zad7_opis}. 
\subsection{Przebieg strojenia dla dwóch lokalnych regulatorów DMC} 
\begin{figure}[h]
    \centering
    \begin{tikzpicture}
    \begin{axis}[
    width=0.98\textwidth,
    xmin=0.0,xmax=1000,ymin=-1,ymax=11.5,
    xlabel={$k$},
    ylabel={$y[k]$},
    legend pos=south east,
    y tick label style={/pgf/number format/1000 sep=},
    ] 
    \addlegendentry{$y[k]$},
    \addlegendentry{$y^{\mathrm{zad}}[k]$},

    \addlegendimage{no markers,red}
    \addlegendimage{no markers,blue}

    \addplot[red, semithick] file{../data/project/zad5_DMC/output_trapezoid_2.csv};  
    \addplot[blue, semithick] file{../data/project/zad5_DMC/stpt_trapezoid_2.csv};
    
    \end{axis}
    \end{tikzpicture}
    \caption{Wyjście procesu dwóch regulatorów lokalnych DMC z trapezową funkcją przynależności}
    \label{pro_zad5_DMC_trapezoid_2_out}
\end{figure}

\begin{figure}[b]
    \centering
    \begin{tikzpicture}
    \begin{axis}[
    width=0.98\textwidth,
    xmin=0.0,xmax=1000,ymin=-0.4,ymax=1.2,
    xlabel={$k$},
    ylabel={$y[k]$},
    legend pos=south east,
    y tick label style={/pgf/number format/1000 sep=},
    ] 
    \addlegendentry{$u[k]$},
    
    \addlegendimage{no markers,blue}

    \addplot[const plot, blue, semithick] file{../data/project/zad5_DMC/input_trapezoid_2.csv};
    
    \end{axis}
    \end{tikzpicture}
    \caption{Przebieg sygnału sterującego dla dwóch regulatorów lokalnych DMC z trapezową funkcją przynależności}
    \label{pro_zad5_DMC_trapezoid_2_inp}
\end{figure}


\begin{figure}[t]
    \centering
    \begin{tikzpicture}
    \begin{axis}[
    width=0.98\textwidth,
    xmin=0.0,xmax=1000,ymin=-1,ymax=11.5,
    xlabel={$k$},
    ylabel={$y[k]$},
    legend pos=south east,
    y tick label style={/pgf/number format/1000 sep=},
    ] 
    \addlegendentry{$y[k]$},
    \addlegendentry{$y^{\mathrm{zad}}[k]$},

    \addlegendimage{no markers,red}
    \addlegendimage{no markers,blue}

    \addplot[red, semithick] file{../data/project/zad5_DMC/output_bell_2.csv};  
    \addplot[blue, semithick] file{../data/project/zad5_DMC/stpt_bell_2.csv};
    
    \end{axis}
    \end{tikzpicture}
    \caption{Wyjście procesu dwóch regulatorów lokalnych DMC z gaussowską funkcją przynależności}
    \label{pro_zad5_DMC_bell_2_out}
\end{figure}

\begin{figure}[b]
    \centering
    \begin{tikzpicture}
    \begin{axis}[
    width=0.98\textwidth,
    xmin=0.0,xmax=1000,ymin=-0.4,ymax=1.2,
    xlabel={$k$},
    ylabel={$y[k]$},
    legend pos=south east,
    y tick label style={/pgf/number format/1000 sep=},
    ] 
    \addlegendentry{$u[k]$},
    
    \addlegendimage{no markers,blue}

    \addplot[const plot, blue, semithick] file{../data/project/zad5_DMC/input_bell_2.csv};
    
    \end{axis}
    \end{tikzpicture}
    \caption{Przebieg sygnału sterującego dla dwóch regulatorów lokalnych DMC z gaussowską funkcją przynależności}
    \label{pro_zad5_DMC_bell_2_inp}
\end{figure}


\begin{figure}[t]
    \centering
    \begin{tikzpicture}
    \begin{axis}[
    width=0.98\textwidth,
    xmin=0.0,xmax=1000,ymin=-1,ymax=11.5,
    xlabel={$k$},
    ylabel={$y[k]$},
    legend pos=south east,
    y tick label style={/pgf/number format/1000 sep=},
    ] 
    \addlegendentry{$y[k]$},
    \addlegendentry{$y^{\mathrm{zad}}[k]$},

    \addlegendimage{no markers,red}
    \addlegendimage{no markers,blue}

    \addplot[red, semithick] file{../data/project/zad5_DMC/output_triangle_2.csv};  
    \addplot[blue, semithick] file{../data/project/zad5_DMC/stpt_triangle_2.csv};
    
    \end{axis}
    \end{tikzpicture}
    \caption{Wyjście procesu dwóch regulatorów lokalnych DMC z trójkątną funkcją przynależności}
    \label{pro_zad5_DMC_triangle_2_out}
\end{figure}

\begin{figure}[b]
    \centering
    \begin{tikzpicture}
    \begin{axis}[
    width=0.98\textwidth,
    xmin=0.0,xmax=1000,ymin=-0.4,ymax=1.2,
    xlabel={$k$},
    ylabel={$y[k]$},
    legend pos=south east,
    y tick label style={/pgf/number format/1000 sep=},
    ] 
    \addlegendentry{$u[k]$},
    
    \addlegendimage{no markers,blue}

    \addplot[const plot, blue, semithick] file{../data/project/zad5_DMC/input_triangle_2.csv};
    
    \end{axis}
    \end{tikzpicture}
    \caption{Przebieg sygnału sterującego dla dwóch regulatorów lokalnych DMC z trójkątną  funkcją przynależności}
    \label{pro_zad5_DMC_triangle_2_inp}
\end{figure}
\FloatBarrier

\subsection{Przebieg strojenia dla trzech lokalnych regulatorów DMC}
\begin{figure}[h]
    \centering
    \begin{tikzpicture}
    \begin{axis}[
    width=0.98\textwidth,
    xmin=0.0,xmax=1000,ymin=-1,ymax=11.5,
    xlabel={$k$},
    ylabel={$y[k]$},
    legend pos=south east,
    y tick label style={/pgf/number format/1000 sep=},
    ] 
    \addlegendentry{$y[k]$},
    \addlegendentry{$y^{\mathrm{zad}}[k]$},

    \addlegendimage{no markers,red}
    \addlegendimage{no markers,blue}

    \addplot[red, semithick] file{../data/project/zad5_DMC/output_trapezoid_3.csv};  
    \addplot[blue, semithick] file{../data/project/zad5_DMC/stpt_trapezoid_3.csv};
    
    \end{axis}
    \end{tikzpicture}
    \caption{Wyjście procesu trzech regulatorów lokalnych DMC z trapezową funkcją przynależności}
    \label{pro_zad5_DMC_trapezoid_3_out}
\end{figure}

\begin{figure}[b]
    \centering
    \begin{tikzpicture}
    \begin{axis}[
    width=0.98\textwidth,
    xmin=0.0,xmax=1000,ymin=-0.4,ymax=1.2,
    xlabel={$k$},
    ylabel={$y[k]$},
    legend pos=south east,
    y tick label style={/pgf/number format/1000 sep=},
    ] 
    \addlegendentry{$u[k]$},
    
    \addlegendimage{no markers,blue}

    \addplot[const plot, blue, semithick] file{../data/project/zad5_DMC/input_trapezoid_3.csv};
    
    \end{axis}
    \end{tikzpicture}
    \caption{Przebieg sygnału sterującego dla trzech regulatorów lokalnych DMC z trapezową funkcją przynależności}
    \label{pro_zad5_DMC_trapezoid_3_inp}
\end{figure}


\begin{figure}[t]
    \centering
    \begin{tikzpicture}
    \begin{axis}[
    width=0.98\textwidth,
    xmin=0.0,xmax=1000,ymin=-1,ymax=11.5,
    xlabel={$k$},
    ylabel={$y[k]$},
    legend pos=south east,
    y tick label style={/pgf/number format/1000 sep=},
    ] 
    \addlegendentry{$y[k]$},
    \addlegendentry{$y^{\mathrm{zad}}[k]$},

    \addlegendimage{no markers,red}
    \addlegendimage{no markers,blue}

    \addplot[red, semithick] file{../data/project/zad5_DMC/output_bell_3.csv};  
    \addplot[blue, semithick] file{../data/project/zad5_DMC/stpt_bell_3.csv};
    
    \end{axis}
    \end{tikzpicture}
    \caption{Wyjście procesu trzech regulatorów lokalnych DMC z gaussowską funkcją przynależności}
    \label{pro_zad5_DMC_bell_3_out}
\end{figure}

\begin{figure}[b]
    \centering
    \begin{tikzpicture}
    \begin{axis}[
    width=0.98\textwidth,
    xmin=0.0,xmax=1000,ymin=-0.4,ymax=1.2,
    xlabel={$k$},
    ylabel={$y[k]$},
    legend pos=south east,
    y tick label style={/pgf/number format/1000 sep=},
    ] 
    \addlegendentry{$u[k]$},
    
    \addlegendimage{no markers,blue}

    \addplot[const plot, blue, semithick] file{../data/project/zad5_DMC/input_bell_3.csv};
    
    \end{axis}
    \end{tikzpicture}
    \caption{Przebieg sygnału sterującego dla trzech regulatorów lokalnych DMC z gaussowską funkcją przynależności}
    \label{pro_zad5_DMC_bell_3_inp}
\end{figure}

\begin{figure}[t]
    \centering
    \begin{tikzpicture}
    \begin{axis}[
    width=0.98\textwidth,
    xmin=0.0,xmax=1000,ymin=-1,ymax=11.5,
    xlabel={$k$},
    ylabel={$y[k]$},
    legend pos=south east,
    y tick label style={/pgf/number format/1000 sep=},
    ] 
    \addlegendentry{$y[k]$},
    \addlegendentry{$y^{\mathrm{zad}}[k]$},

    \addlegendimage{no markers,red}
    \addlegendimage{no markers,blue}

    \addplot[red, semithick] file{../data/project/zad5_DMC/output_triangle_3.csv};  
    \addplot[blue, semithick] file{../data/project/zad5_DMC/stpt_triangle_3.csv};
    
    \end{axis}
    \end{tikzpicture}
    \caption{Wyjście procesu trzech regulatorów lokalnych DMC z trójkątną funkcją przynależności}
    \label{pro_zad5_DMC_triangle_3_out}
\end{figure}

\begin{figure}[b]
    \centering
    \begin{tikzpicture}
    \begin{axis}[
    width=0.98\textwidth,
    xmin=0.0,xmax=1000,ymin=-0.4,ymax=1.2,
    xlabel={$k$},
    ylabel={$y[k]$},
    legend pos=south east,
    y tick label style={/pgf/number format/1000 sep=},
    ] 
    \addlegendentry{$u[k]$},
    
    \addlegendimage{no markers,blue}

    \addplot[const plot, blue, semithick] file{../data/project/zad5_DMC/input_triangle_3.csv};
    
    \end{axis}
    \end{tikzpicture}
    \caption{Przebieg sygnału sterującego dla trzech regulatorów lokalnych DMC z trójkątną  funkcją przynależności}
    \label{pro_zad5_DMC_triangle_3_inp}
\end{figure}
\FloatBarrier

\subsection{Przebieg strojenia dla czterech lokalnych regulatorów DMC}
\begin{figure}[h]
    \centering
    \begin{tikzpicture}
    \begin{axis}[
    width=0.98\textwidth,
    xmin=0.0,xmax=1000,ymin=-1,ymax=11.5,
    xlabel={$k$},
    ylabel={$y[k]$},
    legend pos=south east,
    y tick label style={/pgf/number format/1000 sep=},
    ] 
    \addlegendentry{$y[k]$},
    \addlegendentry{$y^{\mathrm{zad}}[k]$},

    \addlegendimage{no markers,red}
    \addlegendimage{no markers,blue}

    \addplot[red, semithick] file{../data/project/zad5_DMC/output_trapezoid_4.csv};  
    \addplot[blue, semithick] file{../data/project/zad5_DMC/stpt_trapezoid_4.csv};
    
    \end{axis}
    \end{tikzpicture}
    \caption{Wyjście procesu czterech regulatorów lokalnych DMC z trapezową funkcją przynależności}
    \label{pro_zad5_DMC_trapezoid_4_out}
\end{figure}

\begin{figure}[b]
    \centering
    \begin{tikzpicture}
    \begin{axis}[
    width=0.98\textwidth,
    xmin=0.0,xmax=1000,ymin=-0.4,ymax=1.2,
    xlabel={$k$},
    ylabel={$y[k]$},
    legend pos=south east,
    y tick label style={/pgf/number format/1000 sep=},
    ] 
    \addlegendentry{$u[k]$},
    
    \addlegendimage{no markers,blue}

    \addplot[const plot, blue, semithick] file{../data/project/zad5_DMC/input_trapezoid_4.csv};
    
    \end{axis}
    \end{tikzpicture}
    \caption{Przebieg sygnału sterującego dla czterech regulatorów lokalnych DMC z trapezową funkcją przynależności}
    \label{pro_zad5_DMC_trapezoid_4_inp}
\end{figure}


\begin{figure}[t]
    \centering
    \begin{tikzpicture}
    \begin{axis}[
    width=0.98\textwidth,
    xmin=0.0,xmax=1000,ymin=-1,ymax=11.5,
    xlabel={$k$},
    ylabel={$y[k]$},
    legend pos=south east,
    y tick label style={/pgf/number format/1000 sep=},
    ] 
    \addlegendentry{$y[k]$},
    \addlegendentry{$y^{\mathrm{zad}}[k]$},

    \addlegendimage{no markers,red}
    \addlegendimage{no markers,blue}

    \addplot[red, semithick] file{../data/project/zad5_DMC/output_bell_4.csv};  
    \addplot[blue, semithick] file{../data/project/zad5_DMC/stpt_bell_4.csv};
    
    \end{axis}
    \end{tikzpicture}
    \caption{Wyjście procesu czterech regulatorów lokalnych DMC z gaussowską funkcją przynależności}
    \label{pro_zad5_DMC_bell_4_out}
\end{figure}

\begin{figure}[b]
    \centering
    \begin{tikzpicture}
    \begin{axis}[
    width=0.98\textwidth,
    xmin=0.0,xmax=1000,ymin=-0.4,ymax=1.2,
    xlabel={$k$},
    ylabel={$y[k]$},
    legend pos=south east,
    y tick label style={/pgf/number format/1000 sep=},
    ] 
    \addlegendentry{$u[k]$},
    
    \addlegendimage{no markers,blue}

    \addplot[const plot, blue, semithick] file{../data/project/zad5_DMC/input_bell_4.csv};
    
    \end{axis}
    \end{tikzpicture}
    \caption{Przebieg sygnału sterującego dla czterech regulatorów lokalnych DMC z gaussowską funkcją przynależności}
    \label{pro_zad5_DMC_bell_4_inp}
\end{figure}

\begin{figure}[t]
    \centering
    \begin{tikzpicture}
    \begin{axis}[
    width=0.98\textwidth,
    xmin=0.0,xmax=1000,ymin=-1,ymax=11.5,
    xlabel={$k$},
    ylabel={$y[k]$},
    legend pos=south east,
    y tick label style={/pgf/number format/1000 sep=},
    ] 
    \addlegendentry{$y[k]$},
    \addlegendentry{$y^{\mathrm{zad}}[k]$},

    \addlegendimage{no markers,red}
    \addlegendimage{no markers,blue}

    \addplot[red, semithick] file{../data/project/zad5_DMC/output_triangle_4.csv};  
    \addplot[blue, semithick] file{../data/project/zad5_DMC/stpt_triangle_4.csv};
    
    \end{axis}
    \end{tikzpicture}
    \caption{Wyjście procesu czterech regulatorów lokalnych DMC z trójkątną funkcją przynależności}
    \label{pro_zad5_DMC_triangle_4_out}
\end{figure}

\begin{figure}[b]
    \centering
    \begin{tikzpicture}
    \begin{axis}[
    width=0.98\textwidth,
    xmin=0.0,xmax=1000,ymin=-0.4,ymax=1.2,
    xlabel={$k$},
    ylabel={$y[k]$},
    legend pos=south east,
    y tick label style={/pgf/number format/1000 sep=},
    ] 
    \addlegendentry{$u[k]$},
    
    \addlegendimage{no markers,blue}

    \addplot[const plot, blue, semithick] file{../data/project/zad5_DMC/input_triangle_4.csv};
    
    \end{axis}
    \end{tikzpicture}
    \caption{Przebieg sygnału sterującego dla czterech regulatorów lokalnych DMC z trójkątną  funkcją przynależności}
    \label{pro_zad5_DMC_triangle_4_inp}
\end{figure}
\FloatBarrier

\subsection{Przebieg strojenia dla pięciu lokalnych regulatorów DMC}
\begin{figure}[h]
    \centering
    \begin{tikzpicture}
    \begin{axis}[
    width=0.98\textwidth,
    xmin=0.0,xmax=1000,ymin=-1,ymax=11.5,
    xlabel={$k$},
    ylabel={$y[k]$},
    legend pos=south east,
    y tick label style={/pgf/number format/1000 sep=},
    ] 
    \addlegendentry{$y[k]$},
    \addlegendentry{$y^{\mathrm{zad}}[k]$},

    \addlegendimage{no markers,red}
    \addlegendimage{no markers,blue}

    \addplot[red, semithick] file{../data/project/zad5_DMC/output_trapezoid_5.csv};  
    \addplot[blue, semithick] file{../data/project/zad5_DMC/stpt_trapezoid_5.csv};
    
    \end{axis}
    \end{tikzpicture}
    \caption{Wyjście procesu pięciu regulatorów lokalnych DMC z trapezową funkcją przynależności}
    \label{pro_zad5_DMC_trapezoid_5_out}
\end{figure}

\begin{figure}[b]
    \centering
    \begin{tikzpicture}
    \begin{axis}[
    width=0.98\textwidth,
    xmin=0.0,xmax=1000,ymin=-0.4,ymax=1.2,
    xlabel={$k$},
    ylabel={$y[k]$},
    legend pos=south east,
    y tick label style={/pgf/number format/1000 sep=},
    ] 
    \addlegendentry{$u[k]$},
    
    \addlegendimage{no markers,blue}

    \addplot[const plot, blue, semithick] file{../data/project/zad5_DMC/input_trapezoid_5.csv};
    
    \end{axis}
    \end{tikzpicture}
    \caption{Przebieg sygnału sterującego dla pięciu regulatorów lokalnych DMC z trapezową funkcją przynależności}
    \label{pro_zad5_DMC_trapezoid_5_inp}
\end{figure}


\begin{figure}[t]
    \centering
    \begin{tikzpicture}
    \begin{axis}[
    width=0.98\textwidth,
    xmin=0.0,xmax=1000,ymin=-1,ymax=11.5,
    xlabel={$k$},
    ylabel={$y[k]$},
    legend pos=south east,
    y tick label style={/pgf/number format/1000 sep=},
    ] 
    \addlegendentry{$y[k]$},
    \addlegendentry{$y^{\mathrm{zad}}[k]$},

    \addlegendimage{no markers,red}
    \addlegendimage{no markers,blue}

    \addplot[red, semithick] file{../data/project/zad5_DMC/output_bell_5.csv};  
    \addplot[blue, semithick] file{../data/project/zad5_DMC/stpt_bell_5.csv};
    
    \end{axis}
    \end{tikzpicture}
    \caption{Wyjście procesu dla pięciu regulatorów lokalnych DMC z gaussowską funkcją przynależności}
    \label{pro_zad5_DMC_bell_5_out}
\end{figure}

\begin{figure}[b]
    \centering
    \begin{tikzpicture}
    \begin{axis}[
    width=0.98\textwidth,
    xmin=0.0,xmax=1000,ymin=-0.4,ymax=1.2,
    xlabel={$k$},
    ylabel={$y[k]$},
    legend pos=south east,
    y tick label style={/pgf/number format/1000 sep=},
    ] 
    \addlegendentry{$u[k]$},
    
    \addlegendimage{no markers,blue}

    \addplot[const plot, blue, semithick] file{../data/project/zad5_DMC/input_bell_5.csv};
    
    \end{axis}
    \end{tikzpicture}
    \caption{Przebieg sygnału sterującego dla pięciu regulatorów lokalnych DMC z gaussowską funkcją przynależności}
    \label{pro_zad5_DMC_bell_5_inp}
\end{figure}

\begin{figure}[t]
    \centering
    \begin{tikzpicture}
    \begin{axis}[
    width=0.98\textwidth,
    xmin=0.0,xmax=1000,ymin=-1,ymax=11.5,
    xlabel={$k$},
    ylabel={$y[k]$},
    legend pos=south east,
    y tick label style={/pgf/number format/1000 sep=},
    ] 
    \addlegendentry{$y[k]$},
    \addlegendentry{$y^{\mathrm{zad}}[k]$},

    \addlegendimage{no markers,red}
    \addlegendimage{no markers,blue}

    \addplot[red, semithick] file{../data/project/zad5_DMC/output_triangle_5.csv};  
    \addplot[blue, semithick] file{../data/project/zad5_DMC/stpt_triangle_5.csv};
    
    \end{axis}
    \end{tikzpicture}
    \caption{Wyjście procesu dla pięciu regulatorów lokalnych DMC z trójkątną funkcją przynależności}
    \label{pro_zad5_DMC_triangle_5_out}
\end{figure}

\begin{figure}[b]
    \centering
    \begin{tikzpicture}
    \begin{axis}[
    width=0.98\textwidth,
    xmin=0.0,xmax=1000,ymin=-0.4,ymax=1.2,
    xlabel={$k$},
    ylabel={$y[k]$},
    legend pos=south east,
    y tick label style={/pgf/number format/1000 sep=},
    ] 
    \addlegendentry{$u[k]$},
    
    \addlegendimage{no markers,blue}

    \addplot[const plot, blue, semithick] file{../data/project/zad5_DMC/input_triangle_5.csv};
    
    \end{axis}
    \end{tikzpicture}
    \caption{Przebieg sygnału sterującego dla pięciu regulatorów lokalnych DMC z trójkątną  funkcją przynależności}
    \label{pro_zad5_DMC_triangle_5_inp}
\end{figure}
\FloatBarrier

\subsection{Przebieg strojenia dla sześciu lokalnych regulatorów DMC}
\begin{figure}[h]
    \centering
    \begin{tikzpicture}
    \begin{axis}[
    width=0.98\textwidth,
    xmin=0.0,xmax=1000,ymin=-1,ymax=11.5,
    xlabel={$k$},
    ylabel={$y[k]$},
    legend pos=south east,
    y tick label style={/pgf/number format/1000 sep=},
    ] 
    \addlegendentry{$y[k]$},
    \addlegendentry{$y^{\mathrm{zad}}[k]$},

    \addlegendimage{no markers,red}
    \addlegendimage{no markers,blue}

    \addplot[red, semithick] file{../data/project/zad5_DMC/output_trapezoid_6.csv};  
    \addplot[blue, semithick] file{../data/project/zad5_DMC/stpt_trapezoid_6.csv};
    
    \end{axis}
    \end{tikzpicture}
    \caption{Wyjście procesu sześciu regulatorów lokalnych DMC z trapezową funkcją przynależności}
    \label{pro_zad5_DMC_trapezoid_6_out}
\end{figure}

\begin{figure}[b]
    \centering
    \begin{tikzpicture}
    \begin{axis}[
    width=0.98\textwidth,
    xmin=0.0,xmax=1000,ymin=-0.4,ymax=1.2,
    xlabel={$k$},
    ylabel={$y[k]$},
    legend pos=south east,
    y tick label style={/pgf/number format/1000 sep=},
    ] 
    \addlegendentry{$u[k]$},
    
    \addlegendimage{no markers,blue}

    \addplot[const plot, blue, semithick] file{../data/project/zad5_DMC/input_trapezoid_6.csv};
    
    \end{axis}
    \end{tikzpicture}
    \caption{Przebieg sygnału sterującego dla sześciu regulatorów lokalnych DMC z trapezową funkcją przynależności}
    \label{pro_zad5_DMC_trapezoid_6_inp}
\end{figure}


\begin{figure}[t]
    \centering
    \begin{tikzpicture}
    \begin{axis}[
    width=0.98\textwidth,
    xmin=0.0,xmax=1000,ymin=-1,ymax=11.5,
    xlabel={$k$},
    ylabel={$y[k]$},
    legend pos=south east,
    y tick label style={/pgf/number format/1000 sep=},
    ] 
    \addlegendentry{$y[k]$},
    \addlegendentry{$y^{\mathrm{zad}}[k]$},

    \addlegendimage{no markers,red}
    \addlegendimage{no markers,blue}

    \addplot[red, semithick] file{../data/project/zad5_DMC/output_bell_6.csv};  
    \addplot[blue, semithick] file{../data/project/zad5_DMC/stpt_bell_6.csv};
    
    \end{axis}
    \end{tikzpicture}
    \caption{Wyjście procesu dla sześciu regulatorów lokalnych DMC z gaussowską funkcją przynależności}
    \label{pro_zad5_DMC_bell_6_out}
\end{figure}

\begin{figure}[b]
    \centering
    \begin{tikzpicture}
    \begin{axis}[
    width=0.98\textwidth,
    xmin=0.0,xmax=1000,ymin=-0.4,ymax=1.2,
    xlabel={$k$},
    ylabel={$y[k]$},
    legend pos=south east,
    y tick label style={/pgf/number format/1000 sep=},
    ] 
    \addlegendentry{$u[k]$},
    
    \addlegendimage{no markers,blue}

    \addplot[const plot, blue, semithick] file{../data/project/zad5_DMC/input_bell_6.csv};
    
    \end{axis}
    \end{tikzpicture}
    \caption{Przebieg sygnału sterującego dla sześciu regulatorów lokalnych DMC z gaussowską funkcją przynależności}
    \label{pro_zad5_DMC_bell_6_inp}
\end{figure}

\begin{figure}[t]
    \centering
    \begin{tikzpicture}
    \begin{axis}[
    width=0.98\textwidth,
    xmin=0.0,xmax=1000,ymin=-1,ymax=11.5,
    xlabel={$k$},
    ylabel={$y[k]$},
    legend pos=south east,
    y tick label style={/pgf/number format/1000 sep=},
    ] 
    \addlegendentry{$y[k]$},
    \addlegendentry{$y^{\mathrm{zad}}[k]$},

    \addlegendimage{no markers,red}
    \addlegendimage{no markers,blue}

    \addplot[red, semithick] file{../data/project/zad5_DMC/output_triangle_6.csv};  
    \addplot[blue, semithick] file{../data/project/zad5_DMC/stpt_triangle_6.csv};
    
    \end{axis}
    \end{tikzpicture}
    \caption{Wyjście procesu dla sześciu regulatorów lokalnych DMC z trójkątną funkcją przynależności}
    \label{pro_zad5_DMC_triangle_6_out}
\end{figure}

\begin{figure}[b]
    \centering
    \begin{tikzpicture}
    \begin{axis}[
    width=0.98\textwidth,
    xmin=0.0,xmax=1000,ymin=-0.4,ymax=1.2,
    xlabel={$k$},
    ylabel={$y[k]$},
    legend pos=south east,
    y tick label style={/pgf/number format/1000 sep=},
    ] 
    \addlegendentry{$u[k]$},
    
    \addlegendimage{no markers,blue}

    \addplot[const plot, blue, semithick] file{../data/project/zad5_DMC/input_triangle_6.csv};
    
    \end{axis}
    \end{tikzpicture}
    \caption{Przebieg sygnału sterującego dla sześciu regulatorów lokalnych DMC z trójkątną  funkcją przynależności}
    \label{pro_zad5_DMC_triangle_6_inp}
\end{figure}
\FloatBarrier

\subsection{Przebieg strojenia dla dwunastu lokalnych regulatorów DMC}
\begin{figure}[h]
    \centering
    \begin{tikzpicture}
    \begin{axis}[
    width=0.98\textwidth,
    xmin=0.0,xmax=1000,ymin=-1,ymax=11.5,
    xlabel={$k$},
    ylabel={$y[k]$},
    legend pos=south east,
    y tick label style={/pgf/number format/1000 sep=},
    ] 
    \addlegendentry{$y[k]$},
    \addlegendentry{$y^{\mathrm{zad}}[k]$},

    \addlegendimage{no markers,red}
    \addlegendimage{no markers,blue}

    \addplot[red, semithick] file{../data/project/zad5_DMC/output_trapezoid_12.csv};  
    \addplot[blue, semithick] file{../data/project/zad5_DMC/stpt_trapezoid_12.csv};
    
    \end{axis}
    \end{tikzpicture}
    \caption{Wyjście procesu dwunastu regulatorów lokalnych DMC z trapezową funkcją przynależności}
    \label{pro_zad5_DMC_trapezoid_12_out}
\end{figure}

\begin{figure}[b]
    \centering
    \begin{tikzpicture}
    \begin{axis}[
    width=0.98\textwidth,
    xmin=0.0,xmax=1000,ymin=-0.4,ymax=1.2,
    xlabel={$k$},
    ylabel={$y[k]$},
    legend pos=south east,
    y tick label style={/pgf/number format/1000 sep=},
    ] 
    \addlegendentry{$u[k]$},
    
    \addlegendimage{no markers,blue}

    \addplot[const plot, blue, semithick] file{../data/project/zad5_DMC/input_trapezoid_12.csv};
    
    \end{axis}
    \end{tikzpicture}
    \caption{Przebieg sygnału sterującego dla dwunastu regulatorów lokalnych DMC z trapezową funkcją przynależności}
    \label{pro_zad5_DMC_trapezoid_12_inp}
\end{figure}


\begin{figure}[t]
    \centering
    \begin{tikzpicture}
    \begin{axis}[
    width=0.98\textwidth,
    xmin=0.0,xmax=1000,ymin=-1,ymax=11.5,
    xlabel={$k$},
    ylabel={$y[k]$},
    legend pos=south east,
    y tick label style={/pgf/number format/1000 sep=},
    ] 
    \addlegendentry{$y[k]$},
    \addlegendentry{$y^{\mathrm{zad}}[k]$},

    \addlegendimage{no markers,red}
    \addlegendimage{no markers,blue}

    \addplot[red, semithick] file{../data/project/zad5_DMC/output_bell_12.csv};  
    \addplot[blue, semithick] file{../data/project/zad5_DMC/stpt_bell_12.csv};
    
    \end{axis}
    \end{tikzpicture}
    \caption{Wyjście procesu dla dwunastu regulatorów lokalnych DMC z gaussowską funkcją przynależności}
    \label{pro_zad5_DMC_bell_12_out}
\end{figure}

\begin{figure}[b]
    \centering
    \begin{tikzpicture}
    \begin{axis}[
    width=0.98\textwidth,
    xmin=0.0,xmax=1000,ymin=-0.4,ymax=1.2,
    xlabel={$k$},
    ylabel={$y[k]$},
    legend pos=south east,
    y tick label style={/pgf/number format/1000 sep=},
    ] 
    \addlegendentry{$u[k]$},
    
    \addlegendimage{no markers,blue}

    \addplot[const plot, blue, semithick] file{../data/project/zad5_DMC/input_bell_12.csv};
    
    \end{axis}
    \end{tikzpicture}
    \caption{Przebieg sygnału sterującego dla dwunastu regulatorów lokalnych DMC z gaussowską funkcją przynależności}
    \label{pro_zad5_DMC_bell_12_inp}
\end{figure}

\begin{figure}[t]
    \centering
    \begin{tikzpicture}
    \begin{axis}[
    width=0.98\textwidth,
    xmin=0.0,xmax=1000,ymin=-1,ymax=11.5,
    xlabel={$k$},
    ylabel={$y[k]$},
    legend pos=south east,
    y tick label style={/pgf/number format/1000 sep=},
    ] 
    \addlegendentry{$y[k]$},
    \addlegendentry{$y^{\mathrm{zad}}[k]$},

    \addlegendimage{no markers,red}
    \addlegendimage{no markers,blue}

    \addplot[red, semithick] file{../data/project/zad5_DMC/output_triangle_12.csv};  
    \addplot[blue, semithick] file{../data/project/zad5_DMC/stpt_triangle_12.csv};
    
    \end{axis}
    \end{tikzpicture}
    \caption{Wyjście procesu dla dwunastu regulatorów lokalnych DMC z trójkątną funkcją przynależności}
    \label{pro_zad5_DMC_triangle_12_out}
\end{figure}

\begin{figure}[b]
    \centering
    \begin{tikzpicture}
    \begin{axis}[
    width=0.98\textwidth,
    xmin=0.0,xmax=1000,ymin=-0.4,ymax=1.2,
    xlabel={$k$},
    ylabel={$y[k]$},
    legend pos=south east,
    y tick label style={/pgf/number format/1000 sep=},
    ] 
    \addlegendentry{$u[k]$},
    
    \addlegendimage{no markers,blue}

    \addplot[const plot, blue, semithick] file{../data/project/zad5_DMC/input_triangle_12.csv};
    
    \end{axis}
    \end{tikzpicture}
    \caption{Przebieg sygnału sterującego dla dwunastu regulatorów lokalnych DMC z trójkątną  funkcją przynależności}
    \label{pro_zad5_DMC_triangle_12_inp}
\end{figure}
\FloatBarrier
\section{Dobór parametru $\lambda$ dla różnej liczby regulatorów lokalnych DMC}
\label{pro_zad7_opis}

\subsection{Przebiegi sygnałów dla $\lambda = 1$}
Tak jak wspominaliśmy wcześniej, w tym rozdziale dodatkowo pokażemy przebiegi sygnału dla wartości współczynnika kary równego $\lambda = 1$. Zaprezentujemy wykresy dla trzech różnych używanych przez nas funkcji przynależności dla czterech regulatorów lokalnych DMC. Jak możemy zauważyć na poniższych wykresach przebiegi te są bardzo mocno oscylacyjne przez co bardziej przypominają generator szumu niż regulator. Możemy uznać, że wartości zadane praktycznie nie są osiągane, a w dodatku wyjście procesu oscyluje przez cały czas regulacji. W zespole zgodnie uznaliśmy, że taki regulator nie nadaje się do regulacji obiektu i z tego powodu nie opisywaliśmy, ani go nie prezentowaliśmy szczegółowo w poprzednim punkcie. Takie same przebiegi sygnałów dla wartości $\lambda = 1$ uzyskiwaliśmy dla 2, 3, 5, 6 i 12 regulatorów lokalnych DMC. 
\begin{figure}[h]
    \centering
    \begin{tikzpicture}
    \begin{axis}[
    width=0.98\textwidth,
    xmin=0.0,xmax=1000,ymin=-1,ymax=11.5,
    xlabel={$k$},
    ylabel={$y[k]$},
    legend pos=south east,
    y tick label style={/pgf/number format/1000 sep=},
    ] 
    \addlegendentry{$y[k]$},
    \addlegendentry{$y^{\mathrm{zad}}[k]$},

    \addlegendimage{no markers,red}
    \addlegendimage{no markers,blue}

    \addplot[red, semithick] file{../data/project/zad5_DMC_his_1/output_trapezoid_4.csv};  
    \addplot[blue, semithick] file{../data/project/zad5_DMC_his_1/stpt_trapezoid_4.csv};
    
    \end{axis}
    \end{tikzpicture}
    \caption{Wyjście procesu czterech regulatorów lokalnych DMC z trapezową funkcją przynależności}
    \label{pro_zad7_DMC_trapezoid_4_out}
\end{figure}

\begin{figure}[b]
    \centering
    \begin{tikzpicture}
    \begin{axis}[
    width=0.98\textwidth,
    xmin=0.0,xmax=1000,ymin=-1.1,ymax=1.1,
    xlabel={$k$},
    ylabel={$y[k]$},
    legend pos=south east,
    y tick label style={/pgf/number format/1000 sep=},
    ] 
    \addlegendentry{$u[k]$},
    
    \addlegendimage{no markers,blue}

    \addplot[const plot, blue, semithick] file{../data/project/zad5_DMC_his_1/input_trapezoid_4.csv};
    
    \end{axis}
    \end{tikzpicture}
    \caption{Przebieg sygnału sterującego dla czterech regulatorów lokalnych DMC z trapezową funkcją przynależności}
    \label{pro_zad7_DMC_trapezoid_4_inp}
\end{figure}


\begin{figure}[t]
    \centering
    \begin{tikzpicture}
    \begin{axis}[
    width=0.98\textwidth,
    xmin=0.0,xmax=1000,ymin=-1,ymax=11.5,
    xlabel={$k$},
    ylabel={$y[k]$},
    legend pos=south east,
    y tick label style={/pgf/number format/1000 sep=},
    ] 
    \addlegendentry{$y[k]$},
    \addlegendentry{$y^{\mathrm{zad}}[k]$},

    \addlegendimage{no markers,red}
    \addlegendimage{no markers,blue}

    \addplot[red, semithick] file{../data/project/zad5_DMC_his_1/output_bell_4.csv};  
    \addplot[blue, semithick] file{../data/project/zad5_DMC_his_1/stpt_bell_4.csv};
    
    \end{axis}
    \end{tikzpicture}
    \caption{Wyjście procesu czterech regulatorów lokalnych DMC z gaussowską funkcją przynależności}
    \label{pro_zad7_DMC_bell_4_out}
\end{figure}

\begin{figure}[b]
    \centering
    \begin{tikzpicture}
    \begin{axis}[
    width=0.98\textwidth,
    xmin=0.0,xmax=1000,ymin=-1.1,ymax=1.1,
    xlabel={$k$},
    ylabel={$y[k]$},
    legend pos=south east,
    y tick label style={/pgf/number format/1000 sep=},
    ] 
    \addlegendentry{$u[k]$},
    
    \addlegendimage{no markers,blue}

    \addplot[const plot, blue, semithick] file{../data/project/zad5_DMC_his_1/input_bell_4.csv};
    
    \end{axis}
    \end{tikzpicture}
    \caption{Przebieg sygnału sterującego dla czterech regulatorów lokalnych DMC z gaussowską funkcją przynależności}
    \label{pro_zad7_DMC_bell_4_inp}
\end{figure}

\begin{figure}[t]
    \centering
    \begin{tikzpicture}
    \begin{axis}[
    width=0.98\textwidth,
    xmin=0.0,xmax=1000,ymin=-1,ymax=11.5,
    xlabel={$k$},
    ylabel={$y[k]$},
    legend pos=south east,
    y tick label style={/pgf/number format/1000 sep=},
    ] 
    \addlegendentry{$y[k]$},
    \addlegendentry{$y^{\mathrm{zad}}[k]$},

    \addlegendimage{no markers,red}
    \addlegendimage{no markers,blue}

    \addplot[red, semithick] file{../data/project/zad5_DMC_his_1/output_triangle_4.csv};  
    \addplot[blue, semithick] file{../data/project/zad5_DMC_his_1/stpt_triangle_4.csv};
    
    \end{axis}
    \end{tikzpicture}
    \caption{Wyjście procesu czterech regulatorów lokalnych DMC z trójkątną funkcją przynależności}
    \label{pro_zad7_DMC_triangle_4_out}
\end{figure}

\begin{figure}[b]
    \centering
    \begin{tikzpicture}
    \begin{axis}[
    width=0.98\textwidth,
    xmin=0.0,xmax=1000,ymin=-1.1,ymax=1.1,
    xlabel={$k$},
    ylabel={$y[k]$},
    legend pos=south east,
    y tick label style={/pgf/number format/1000 sep=},
    ] 
    \addlegendentry{$u[k]$},
    
    \addlegendimage{no markers,blue}

    \addplot[const plot, blue, semithick] file{../data/project/zad5_DMC_his_1/input_triangle_4.csv};
    
    \end{axis}
    \end{tikzpicture}
    \caption{Przebieg sygnału sterującego dla czterech regulatorów lokalnych DMC z trójkątną  funkcją przynależności}
    \label{pro_zad7_DMC_triangle_4_inp}
\end{figure}
\FloatBarrier

\subsection{Dobór parametru $\lambda$ dla trzech regulatorów lokalnych DMC}
\begin{figure}[h]
    \centering
    \begin{tikzpicture}
    \begin{axis}[
    width=0.98\textwidth,
    xmin=0.0,xmax=1000,ymin=-1,ymax=11.5,
    xlabel={$k$},
    ylabel={$y[k]$},
    legend pos=south east,
    y tick label style={/pgf/number format/1000 sep=},
    ] 
    \addlegendentry{$y[k]$},
    \addlegendentry{$y^{\mathrm{zad}}[k]$},

    \addlegendimage{no markers,red}
    \addlegendimage{no markers,blue}

    \addplot[red, semithick] file{../data/project/zad7/output_trapezoid_3.csv};  
    \addplot[blue, semithick] file{../data/project/zad7/stpt_trapezoid_3.csv};
    
    \end{axis}
    \end{tikzpicture}
    \caption{Wyjście procesu trzech regulatorów lokalnych DMC z trapezową funkcją przynależności}
    \label{pro_zad7_DMC_trapezoid_3_out}
\end{figure}

\begin{figure}[b]
    \centering
    \begin{tikzpicture}
    \begin{axis}[
    width=0.98\textwidth,
    xmin=0.0,xmax=1000,ymin=-1.1,ymax=1.1,
    xlabel={$k$},
    ylabel={$y[k]$},
    legend pos=south east,
    y tick label style={/pgf/number format/1000 sep=},
    ] 
    \addlegendentry{$u[k]$},
    
    \addlegendimage{no markers,blue}

    \addplot[const plot, blue, semithick] file{../data/project/zad7/input_trapezoid_3.csv};
    
    \end{axis}
    \end{tikzpicture}
    \caption{Przebieg sygnału sterującego dla trzech regulatorów lokalnych DMC z trapezową funkcją przynależności}
    \label{pro_zad7_DMC_trapezoid_3_inp}
\end{figure}


\begin{figure}[t]
    \centering
    \begin{tikzpicture}
    \begin{axis}[
    width=0.98\textwidth,
    xmin=0.0,xmax=1000,ymin=-1,ymax=11.5,
    xlabel={$k$},
    ylabel={$y[k]$},
    legend pos=south east,
    y tick label style={/pgf/number format/1000 sep=},
    ] 
    \addlegendentry{$y[k]$},
    \addlegendentry{$y^{\mathrm{zad}}[k]$},

    \addlegendimage{no markers,red}
    \addlegendimage{no markers,blue}

    \addplot[red, semithick] file{../data/project/zad7/output_bell_3.csv};  
    \addplot[blue, semithick] file{../data/project/zad7/stpt_bell_3.csv};
    
    \end{axis}
    \end{tikzpicture}
    \caption{Wyjście procesu trzech regulatorów lokalnych DMC z gaussowską funkcją przynależności}
    \label{pro_zad7_DMC_bell_3_out}
\end{figure}

\begin{figure}[b]
    \centering
    \begin{tikzpicture}
    \begin{axis}[
    width=0.98\textwidth,
    xmin=0.0,xmax=1000,ymin=-1.1,ymax=1.1,
    xlabel={$k$},
    ylabel={$y[k]$},
    legend pos=south east,
    y tick label style={/pgf/number format/1000 sep=},
    ] 
    \addlegendentry{$u[k]$},
    
    \addlegendimage{no markers,blue}

    \addplot[const plot, blue, semithick] file{../data/project/zad7/input_bell_3.csv};
    
    \end{axis}
    \end{tikzpicture}
    \caption{Przebieg sygnału sterującego dla trzech regulatorów lokalnych DMC z gaussowską funkcją przynależności}
    \label{pro_zad7_DMC_bell_3_inp}
\end{figure}

\begin{figure}[t]
    \centering
    \begin{tikzpicture}
    \begin{axis}[
    width=0.98\textwidth,
    xmin=0.0,xmax=1000,ymin=-1,ymax=11.5,
    xlabel={$k$},
    ylabel={$y[k]$},
    legend pos=south east,
    y tick label style={/pgf/number format/1000 sep=},
    ] 
    \addlegendentry{$y[k]$},
    \addlegendentry{$y^{\mathrm{zad}}[k]$},

    \addlegendimage{no markers,red}
    \addlegendimage{no markers,blue}

    \addplot[red, semithick] file{../data/project/zad7/output_triangle_3.csv};  
    \addplot[blue, semithick] file{../data/project/zad7/stpt_triangle_3.csv};
    
    \end{axis}
    \end{tikzpicture}
    \caption{Wyjście procesu czterech regulatorów lokalnych DMC z trójkątną funkcją przynależności}
    \label{pro_zad7_DMC_triangle_3_out}
\end{figure}

\begin{figure}[b]
    \centering
    \begin{tikzpicture}
    \begin{axis}[
    width=0.98\textwidth,
    xmin=0.0,xmax=1000,ymin=-1.1,ymax=1.1,
    xlabel={$k$},
    ylabel={$y[k]$},
    legend pos=south east,
    y tick label style={/pgf/number format/1000 sep=},
    ] 
    \addlegendentry{$u[k]$},
    
    \addlegendimage{no markers,blue}

    \addplot[const plot, blue, semithick] file{../data/project/zad7/input_triangle_3.csv};
    
    \end{axis}
    \end{tikzpicture}
    \caption{Przebieg sygnału sterującego dla trzech regulatorów lokalnych DMC z trójkątną  funkcją przynależności}
    \label{pro_zad7_DMC_triangle_3_inp}
\end{figure}
\FloatBarrier

\subsection{Dobór parametru $\lambda$ dla czterech regulatorów lokalnych DMC}
\begin{figure}[h]
    \centering
    \begin{tikzpicture}
    \begin{axis}[
    width=0.98\textwidth,
    xmin=0.0,xmax=1000,ymin=-1,ymax=11.5,
    xlabel={$k$},
    ylabel={$y[k]$},
    legend pos=south east,
    y tick label style={/pgf/number format/1000 sep=},
    ] 
    \addlegendentry{$y[k]$},
    \addlegendentry{$y^{\mathrm{zad}}[k]$},

    \addlegendimage{no markers,red}
    \addlegendimage{no markers,blue}

    \addplot[red, semithick] file{../data/project/zad7/output_trapezoid_4.csv};  
    \addplot[blue, semithick] file{../data/project/zad7/stpt_trapezoid_4.csv};
    
    \end{axis}
    \end{tikzpicture}
    \caption{Wyjście procesu czterech regulatorów lokalnych DMC z trapezową funkcją przynależności}
    \label{pro_zad7_DMC_trapezoid_4_out}
\end{figure}

\begin{figure}[b]
    \centering
    \begin{tikzpicture}
    \begin{axis}[
    width=0.98\textwidth,
    xmin=0.0,xmax=1000,ymin=-1.1,ymax=1.1,
    xlabel={$k$},
    ylabel={$y[k]$},
    legend pos=south east,
    y tick label style={/pgf/number format/1000 sep=},
    ] 
    \addlegendentry{$u[k]$},
    
    \addlegendimage{no markers,blue}

    \addplot[const plot, blue, semithick] file{../data/project/zad7/input_trapezoid_4.csv};
    
    \end{axis}
    \end{tikzpicture}
    \caption{Przebieg sygnału sterującego dla czterech regulatorów lokalnych DMC z trapezową funkcją przynależności}
    \label{pro_zad7_DMC_trapezoid_4_inp}
\end{figure}


\begin{figure}[t]
    \centering
    \begin{tikzpicture}
    \begin{axis}[
    width=0.98\textwidth,
    xmin=0.0,xmax=1000,ymin=-1,ymax=11.5,
    xlabel={$k$},
    ylabel={$y[k]$},
    legend pos=south east,
    y tick label style={/pgf/number format/1000 sep=},
    ] 
    \addlegendentry{$y[k]$},
    \addlegendentry{$y^{\mathrm{zad}}[k]$},

    \addlegendimage{no markers,red}
    \addlegendimage{no markers,blue}

    \addplot[red, semithick] file{../data/project/zad7/output_bell_4.csv};  
    \addplot[blue, semithick] file{../data/project/zad7/stpt_bell_4.csv};
    
    \end{axis}
    \end{tikzpicture}
    \caption{Wyjście procesu czterech regulatorów lokalnych DMC z gaussowską funkcją przynależności}
    \label{pro_zad7_DMC_bell_4_out}
\end{figure}

\begin{figure}[b]
    \centering
    \begin{tikzpicture}
    \begin{axis}[
    width=0.98\textwidth,
    xmin=0.0,xmax=1000,ymin=-1.1,ymax=1.1,
    xlabel={$k$},
    ylabel={$y[k]$},
    legend pos=south east,
    y tick label style={/pgf/number format/1000 sep=},
    ] 
    \addlegendentry{$u[k]$},
    
    \addlegendimage{no markers,blue}

    \addplot[const plot, blue, semithick] file{../data/project/zad7/input_bell_4.csv};
    
    \end{axis}
    \end{tikzpicture}
    \caption{Przebieg sygnału sterującego dla czterech regulatorów lokalnych DMC z gaussowską funkcją przynależności}
    \label{pro_zad7_DMC_bell_4_inp}
\end{figure}

\begin{figure}[t]
    \centering
    \begin{tikzpicture}
    \begin{axis}[
    width=0.98\textwidth,
    xmin=0.0,xmax=1000,ymin=-1,ymax=11.5,
    xlabel={$k$},
    ylabel={$y[k]$},
    legend pos=south east,
    y tick label style={/pgf/number format/1000 sep=},
    ] 
    \addlegendentry{$y[k]$},
    \addlegendentry{$y^{\mathrm{zad}}[k]$},

    \addlegendimage{no markers,red}
    \addlegendimage{no markers,blue}

    \addplot[red, semithick] file{../data/project/zad7/output_triangle_4.csv};  
    \addplot[blue, semithick] file{../data/project/zad7/stpt_triangle_4.csv};
    
    \end{axis}
    \end{tikzpicture}
    \caption{Wyjście procesu czterech regulatorów lokalnych DMC z trójkątną funkcją przynależności}
    \label{pro_zad7_DMC_triangle_4_out}
\end{figure}

\begin{figure}[b]
    \centering
    \begin{tikzpicture}
    \begin{axis}[
    width=0.98\textwidth,
    xmin=0.0,xmax=1000,ymin=-1.1,ymax=1.1,
    xlabel={$k$},
    ylabel={$y[k]$},
    legend pos=south east,
    y tick label style={/pgf/number format/1000 sep=},
    ] 
    \addlegendentry{$u[k]$},
    
    \addlegendimage{no markers,blue}

    \addplot[const plot, blue, semithick] file{../data/project/zad7/input_triangle_4.csv};
    
    \end{axis}
    \end{tikzpicture}
    \caption{Przebieg sygnału sterującego dla czterech regulatorów lokalnych DMC z trójkątną  funkcją przynależności}
    \label{pro_zad7_DMC_triangle_4_inp}
\end{figure}
\FloatBarrier

\subsection{Dobór parametru $\lambda$ dla pięciu regulatorów lokalnych DMC}
\begin{figure}[h]
    \centering
    \begin{tikzpicture}
    \begin{axis}[
    width=0.98\textwidth,
    xmin=0.0,xmax=1000,ymin=-1,ymax=11.5,
    xlabel={$k$},
    ylabel={$y[k]$},
    legend pos=south east,
    y tick label style={/pgf/number format/1000 sep=},
    ] 
    \addlegendentry{$y[k]$},
    \addlegendentry{$y^{\mathrm{zad}}[k]$},

    \addlegendimage{no markers,red}
    \addlegendimage{no markers,blue}

    \addplot[red, semithick] file{../data/project/zad7/output_trapezoid_5.csv};  
    \addplot[blue, semithick] file{../data/project/zad7/stpt_trapezoid_5.csv};
    
    \end{axis}
    \end{tikzpicture}
    \caption{Wyjście procesu pięciu regulatorów lokalnych DMC z trapezową funkcją przynależności}
    \label{pro_zad7_DMC_trapezoid_5_out}
\end{figure}

\begin{figure}[b]
    \centering
    \begin{tikzpicture}
    \begin{axis}[
    width=0.98\textwidth,
    xmin=0.0,xmax=1000,ymin=-1.1,ymax=1.1,
    xlabel={$k$},
    ylabel={$y[k]$},
    legend pos=south east,
    y tick label style={/pgf/number format/1000 sep=},
    ] 
    \addlegendentry{$u[k]$},
    
    \addlegendimage{no markers,blue}

    \addplot[const plot, blue, semithick] file{../data/project/zad7/input_trapezoid_5.csv};
    
    \end{axis}
    \end{tikzpicture}
    \caption{Przebieg sygnału sterującego dla pięciu regulatorów lokalnych DMC z trapezową funkcją przynależności}
    \label{pro_zad7_DMC_trapezoid_5_inp}
\end{figure}


\begin{figure}[t]
    \centering
    \begin{tikzpicture}
    \begin{axis}[
    width=0.98\textwidth,
    xmin=0.0,xmax=1000,ymin=-1,ymax=11.5,
    xlabel={$k$},
    ylabel={$y[k]$},
    legend pos=south east,
    y tick label style={/pgf/number format/1000 sep=},
    ] 
    \addlegendentry{$y[k]$},
    \addlegendentry{$y^{\mathrm{zad}}[k]$},

    \addlegendimage{no markers,red}
    \addlegendimage{no markers,blue}

    \addplot[red, semithick] file{../data/project/zad7/output_bell_5.csv};  
    \addplot[blue, semithick] file{../data/project/zad7/stpt_bell_5.csv};
    
    \end{axis}
    \end{tikzpicture}
    \caption{Wyjście procesu pięciu regulatorów lokalnych DMC z gaussowską funkcją przynależności}
    \label{pro_zad7_DMC_bell_5_out}
\end{figure}

\begin{figure}[b]
    \centering
    \begin{tikzpicture}
    \begin{axis}[
    width=0.98\textwidth,
    xmin=0.0,xmax=1000,ymin=-1.1,ymax=1.1,
    xlabel={$k$},
    ylabel={$y[k]$},
    legend pos=south east,
    y tick label style={/pgf/number format/1000 sep=},
    ] 
    \addlegendentry{$u[k]$},
    
    \addlegendimage{no markers,blue}

    \addplot[const plot, blue, semithick] file{../data/project/zad7/input_bell_5.csv};
    
    \end{axis}
    \end{tikzpicture}
    \caption{Przebieg sygnału sterującego dla pięciu regulatorów lokalnych DMC z gaussowską funkcją przynależności}
    \label{pro_zad7_DMC_bell_5_inp}
\end{figure}

\begin{figure}[t]
    \centering
    \begin{tikzpicture}
    \begin{axis}[
    width=0.98\textwidth,
    xmin=0.0,xmax=1000,ymin=-1,ymax=11.5,
    xlabel={$k$},
    ylabel={$y[k]$},
    legend pos=south east,
    y tick label style={/pgf/number format/1000 sep=},
    ] 
    \addlegendentry{$y[k]$},
    \addlegendentry{$y^{\mathrm{zad}}[k]$},

    \addlegendimage{no markers,red}
    \addlegendimage{no markers,blue}

    \addplot[red, semithick] file{../data/project/zad7/output_triangle_5.csv};  
    \addplot[blue, semithick] file{../data/project/zad7/stpt_triangle_5.csv};
    
    \end{axis}
    \end{tikzpicture}
    \caption{Wyjście procesu pięciu regulatorów lokalnych DMC z trójkątną funkcją przynależności}
    \label{pro_zad7_DMC_triangle_5_out}
\end{figure}

\begin{figure}[b]
    \centering
    \begin{tikzpicture}
    \begin{axis}[
    width=0.98\textwidth,
    xmin=0.0,xmax=1000,ymin=-1.1,ymax=1.1,
    xlabel={$k$},
    ylabel={$y[k]$},
    legend pos=south east,
    y tick label style={/pgf/number format/1000 sep=},
    ] 
    \addlegendentry{$u[k]$},
    
    \addlegendimage{no markers,blue}

    \addplot[const plot, blue, semithick] file{../data/project/zad7/input_triangle_5.csv};
    
    \end{axis}
    \end{tikzpicture}
    \caption{Przebieg sygnału sterującego dla pięciu regulatorów lokalnych DMC z trójkątną  funkcją przynależności}
    \label{pro_zad7_DMC_triangle_5_inp}
\end{figure}
\FloatBarrier
\part{Laboratorium}
\chapter{Weryfikacja poprawności działania i dobór parametrów
algorytmów regulacji jednowymiarowego procesu laboratoryjnego o
istotnie nieliniowych właściwościach.}

\section{Wyznaczenie punktu pracy}
Po sprawdzeniu możliwości sterowania i pomiaru w komunikacji ze stanowiskiem
przystąpiliśmy do wyznaczenia punktu pracy. Od prowadzącego otrzymaliśmy
informacje o wartości sygnału sterującego w punkcie pracy równej
$Upp = 26\%$. W celu wyznaczenia wartości sygnału wyjściowego w punkcie pracy
Ypp na wejście obiektu podaliśmy stałe wejście o wartości Upp i zaczekaliśmy aż
wyjście ustabilizuje się. Po odczekaniu około 5 minut, wyjście obiektu ustabilizowało
się na wartości $ Ypp = 32^{\circ} C$.

\section{Określenia wzmocnienia funkcji sterowania}

\chapter{Wyznaczenie charakterystyki statycznej obiektu}
\label{lab2}

Aby wyznaczyć charakterystykę statyczną wykonaliśmy eksperyment mający na celu zbadanie wzmocnienia w zależności od sterowania. Dla kolejnych wartości sygnału sterującego: $20, 30, 40...$ oczekiwaliśmy stabilizacji sygnału wyjściowego obiektu. Wynik eksperymentu przedstawia wykres \ref{eksperyment}. Na wykresie możemy zauważyć, że przyrost wartości sygnału wyjściowego zmniejsza się wraz ze zwiększaniem wartości sterowania. Punkty stabilizacji sygnału posłużyły do wyznacznia charakterystyki statycznej obiektu. Wynik aproksymacji przedstawia wykrest \ref{char_stat}.\\
\indent{} Przegięcie wykresu wskazuje na to, że mamy do czynienia z obiektem nieliniowym, dlatego niemożliwe jest wyznaczenie jednego wzmocnienia statycznego 
opisującego cały proces.

\begin{figure}[b]
    \centering
    \begin{tikzpicture}
    \begin{axis}[
    width=0.98\textwidth,
    xmin=0.0,xmax=2040,ymin=30,ymax=55,
    xlabel={$k$},
    ylabel={$y[k]$},
    legend pos=south east,
    y tick label style={/pgf/number format/1000 sep=},
    ] 
    \addlegendentry{$y[k]$},
    \addlegendimage{no markers,red}

    \addplot[red, semithick] file{../data/lab/zad2/eksperyment.csv};  
    \end{axis}
    \end{tikzpicture}
    \caption{Przebieg eksperymentu przy wartościach sterowania: $20, 30, 40, 50, 60, 70, 80$}
    \label{eksperyment}
\end{figure}

\begin{figure}[t]
    \centering
    \begin{tikzpicture}
    \begin{axis}[
    width=0.98\textwidth,
    xmin=20.0,xmax=80,ymin=32,ymax=52,
    xlabel={$u$},
    ylabel={$y[u]$},
    legend pos=south east,
    y tick label style={/pgf/number format/1000 sep=},
    ] 
    \addlegendentry{$y[u]$},
    \addlegendimage{no markers,red}

    \addplot[red, semithick] file{../data/lab/zad2/char_stat.csv};  
    \end{axis}
    \end{tikzpicture}
    \caption{Charakterystyka statyczna obiektu}
    \label{char_stat}
\end{figure}
\chapter{Regulacja liniowa obiektu nieliniowego}
\label{lab3}

Zadanie polegało na wypróbowaniu zwykłych regulatorów, wykorzystywanych przy obiektach liniowych, do regulacji obiektu nieliniowego.

\section{Regulator PID}
Jako pierwszy testowany był algorytm PID. Na wykresach obserwujemy przebiegi sygnałów przy kilku skokach wartości zadanej. Patrząc na wykres można stwierdzić, że regulator radzi sobie przyzwoicie. Osiągane są wszystkie wartości zadane w niedługim czasie po skoku. Występują jednak niewielkie oscylacje sygnału wyjściowego oraz spore oscylacje sterowania. Wskaźnik jakości wyniósł $E = \num{12794.09}$. Wyniki testu przedstawione zostały na wykresach \ref{reg_PID} oraz \ref{reg_PID_u}.

\begin{figure}[t]
    \centering
    \begin{tikzpicture}
    \begin{axis}[
    width=0.98\textwidth,
    xmin=0.0,xmax=800,ymin=30,ymax=55,
    xlabel={$k$},
    ylabel={$y[k]$},
    legend pos=south east,
    y tick label style={/pgf/number format/1000 sep=},
    ] 
    \addlegendentry{$y[k]$},
    \addlegendentry{$y^{\mathrm{zad}}$},
    \addlegendimage{no markers,red},
    \addlegendimage{no markers,blue}

    \addplot[red, semithick] file{../data/lab/zad3/pid_output.csv};
    \addplot[blue, semithick] file{../data/lab/zad3/pid_stpt.csv};  
    \end{axis}
    \end{tikzpicture}
    \caption{Przebieg procesu regulacji obiektu nieliniowego zwykłym regulatorem PID.}
    \label{reg_PID}
\end{figure}

\begin{figure}[b]
    \centering
    \begin{tikzpicture}
    \begin{axis}[
    width=0.98\textwidth,
    xmin=0.0,xmax=800,ymin=0,ymax=100,
    xlabel={$k$},
    ylabel={$y[k]$},
    legend pos=south east,
    y tick label style={/pgf/number format/1000 sep=},
    ] 
    \addlegendentry{$u$},
    \addlegendimage{no markers,blue}
    \addplot[blue, semithick] file{../data/lab/zad3/pid_input.csv};  
    \end{axis}
    \end{tikzpicture}
    \caption{Przebieg sygnału sterującego podczas regulacji obiektu nieliniowego zwykłym regulatorem PID.}
    \label{reg_PID_u}
\end{figure}

\section{Regulator DMC}
Następnie nastąpił test regulatora DMC. W tym przypadku wyniki nie były już tak zadowalające jak przy regulatorze PID. Wykresy przedstawiają wolne działanie regulatora. Co więcej, wyjście nie osiąga wartości zadanej przy największym skoku. Sygnały przebiegają spokojnie jednak podczas regulacji pożądana byłaby lepsza jakość regulacji. Wskaźnik jakości osiągnął niemalże dwukrotnie większą wartość niż przy eksperymencie z regulatorem PID - $E = \num{24071.15}$. Przebiegi sygnałów można zobaczyć na wykresach \ref{reg_DMC} oraz \ref{reg_DMC_u}. 

\begin{figure}[t]
    \centering
    \begin{tikzpicture}
    \begin{axis}[
    width=0.98\textwidth,
    xmin=0.0,xmax=800,ymin=30,ymax=55,
    xlabel={$k$},
    ylabel={$y[k]$},
    legend pos=south east,
    y tick label style={/pgf/number format/1000 sep=},
    ] 
    \addlegendentry{$y[k]$},
    \addlegendentry{$y^{\mathrm{zad}}$},
    \addlegendimage{no markers,red},
    \addlegendimage{no markers,blue}

    \addplot[red, semithick] file{../data/lab/zad3/dmc_output.csv};
    \addplot[blue, semithick] file{../data/lab/zad3/dmc_stpt.csv};  
    \end{axis}
    \end{tikzpicture}
    \caption{Przebieg procesu regulacji obiektu nieliniowego zwykłym regulatorem DMC.}
    \label{reg_DMC}
\end{figure}

\begin{figure}[b]
    \centering
    \begin{tikzpicture}
    \begin{axis}[
    width=0.98\textwidth,
    xmin=0.0,xmax=800,ymin=0,ymax=100,
    xlabel={$k$},
    ylabel={$y[k]$},
    legend pos=south east,
    y tick label style={/pgf/number format/1000 sep=},
    ] 
    \addlegendentry{$u$},
    \addlegendimage{no markers,blue}
    \addplot[blue, semithick] file{../data/lab/zad3/dmc_input.csv};  
    \end{axis}
    \end{tikzpicture}
    \caption{Przebieg sygnału sterującego podczas regulacji obiektu nieliniowego zwykłym regulatorem DMC.}
    \label{reg_DMC_u}
\end{figure}
\chapter{Regulacja z wykorzystaniem rozmytego regulatora PID}
\label{lab4}

Eksperymenty przeprowadzane były dla regulatora wykorzystującego trzy regulatory lokalne. Do każdego regulatora lokalnego staraliśmy się dobrać parametry minimalizujące wartość wskaźnika jakośći.

\subsubsection{Próba 1.}
Parametry regulatorów lokalnych:
\begin{enumerate}
\item $K = 15, T_{\mathrm{i}} = 300,  T_{\mathrm{d}} = 1$
\item $K = 17,5, T_{\mathrm{i}} = 250,  T_{\mathrm{d}} = 1$
\item $K = 20, T_{\mathrm{i}} = 200,  T_{\mathrm{d}} = 1$
\end{enumerate}

Dla pierwszej próby wskaźnik jakości wyniósł $E = \num{67034}$ co jest wynikiem znacznie gorszym od wyniku dla regulatora PID z rozdziału \ref{lab3}. Na wykresach również obserwujemy bardzo słabą jakość regulacji, brak jakiegokolwiek nadążania za wartością zadaną i duży błąd. Wyniki przedstawione na wykresach \ref{pr_1_y} i \ref{pr_1_u}.

\begin{figure}[t]
    \centering
    \begin{tikzpicture}
    \begin{axis}[
    width=0.98\textwidth,
    xmin=0.0,xmax=800,ymin=30,ymax=55,
    xlabel={$k$},
    ylabel={$y[k]$},
    legend pos=south east,
    y tick label style={/pgf/number format/1000 sep=},
    ] 
    \addlegendentry{$y[k]$},
    \addlegendentry{$y^{\mathrm{zad}}$},
    \addlegendimage{no markers,red},
    \addlegendimage{no markers,blue}

    \addplot[red, semithick] file{../data/lab/zad4/proba_1_output.csv};
    \addplot[blue, semithick] file{../data/lab/zad4/proba_1_stpt.csv};  
    \end{axis}
    \end{tikzpicture}
    \caption{Próba 1.: Przebieg procesu regulacji obiektu nieliniowego rozmytym regulatorem PID.}
    \label{pr_1_y}
\end{figure}

\begin{figure}[b]
    \centering
    \begin{tikzpicture}
    \begin{axis}[
    width=0.98\textwidth,
    xmin=0.0,xmax=800,ymin=0,ymax=100,
    xlabel={$k$},
    ylabel={$y[k]$},
    legend pos=south east,
    y tick label style={/pgf/number format/1000 sep=},
    ] 
    \addlegendentry{$u$},
    \addlegendimage{no markers,blue}
    \addplot[blue, semithick] file{../data/lab/zad4/proba_1_input.csv};  
    \end{axis}
    \end{tikzpicture}
    \caption{Próba 1.: Przebieg sygnału sterującego podczas regulacji obiektu nieliniowego rozmytym regulatorem PID.}
    \label{pr_1_u}
\end{figure}


\subsubsection{Próba 2.}
Parametry regulatorów lokalnych:
\begin{enumerate}
\item $K = 17, T_{\mathrm{i}} = 300,  T_{\mathrm{d}} = 1$
\item $K = 19, T_{\mathrm{i}} = 250,  T_{\mathrm{d}} = 1$
\item $K = 22, T_{\mathrm{i}} = 200,  T_{\mathrm{d}} = 1$
\end{enumerate}
Przy próbie drugiej możemy zaobserwować, że jakość regulacji uległa nieznacznej poprawie. Znajduje to potwierdzenie we wskaźniku jakości - $E = \num{37713}$. Jest to jednak wartość o rząd wielkości większa niż uzyskana w zadaniu poprzednim. Sygnały nie dążą do wartości zadanej, obserwujemy oscylacje i niestabilny sygnał sterujący. Wyniki przedstawione na wykresach \ref{pr_2_y} oraz \ref{pr_2_u}.

\begin{figure}[t]
    \centering
    \begin{tikzpicture}
    \begin{axis}[
    width=0.98\textwidth,
    xmin=0.0,xmax=800,ymin=30,ymax=55,
    xlabel={$k$},
    ylabel={$y[k]$},
    legend pos=south east,
    y tick label style={/pgf/number format/1000 sep=},
    ] 
    \addlegendentry{$y[k]$},
    \addlegendentry{$y^{\mathrm{zad}}$},
    \addlegendimage{no markers,red},
    \addlegendimage{no markers,blue}

    \addplot[red, semithick] file{../data/lab/zad4/proba_2_output.csv};
    \addplot[blue, semithick] file{../data/lab/zad4/proba_2_stpt.csv}; 
    \end{axis}
    \end{tikzpicture}
    \caption{Próba 2.: Przebieg procesu regulacji obiektu nieliniowego rozmytym regulatorem PID.}
    \label{pr_2_y}
\end{figure}

\begin{figure}[b]
    \centering
    \begin{tikzpicture}
    \begin{axis}[
    width=0.98\textwidth,
    xmin=0.0,xmax=800,ymin=0,ymax=100,
    xlabel={$k$},
    ylabel={$y[k]$},
    legend pos=south east,
    y tick label style={/pgf/number format/1000 sep=},
    ] 
    \addlegendentry{$u$},
    \addlegendimage{no markers,blue}
   \addplot[blue, semithick] file{../data/lab/zad4/proba_2_input.csv};  
    \end{axis}
    \end{tikzpicture}
    \caption{Próba 2.: Przebieg sygnału sterującego podczas regulacji obiektu nieliniowego rozmytym regulatorem PID.}
    \label{pr_2_u}
\end{figure}


\subsubsection{Próba 3.}
Parametry regulatorów lokalnych:
\begin{enumerate}
\item $K = 17, T_{\mathrm{i}} = 250,  T_{\mathrm{d}} = 18$
\item $K = 19, T_{\mathrm{i}} = 200,  T_{\mathrm{d}} = 20$
\item $K = 22, T_{\mathrm{i}} = 180,  T_{\mathrm{d}} = 15$
\end{enumerate}
Próba trzecia różniła się od drugiej jedynie gładszym przebiegiem sygnału wyjściowego. Zdecydowanie ostrzejszy przebieg możemy zaobserwować na wykresie sterowania, które oscyluje z krótkim okresem. Wskaźnik jakości nieznacznie się zmniejszył - $E = \num{33717}$. Przebiegi przedstawione na wykresach \ref{pr_3_y} i \ref{pr_3_u}.

\begin{figure}[t]
    \centering
    \begin{tikzpicture}
    \begin{axis}[
    width=0.98\textwidth,
    xmin=0.0,xmax=800,ymin=30,ymax=55,
    xlabel={$k$},
    ylabel={$y[k]$},
    legend pos=south east,
    y tick label style={/pgf/number format/1000 sep=},
    ] 
    \addlegendentry{$y[k]$},
    \addlegendentry{$y^{\mathrm{zad}}$},
    \addlegendimage{no markers,red},
    \addlegendimage{no markers,blue}

    \addplot[red, semithick] file{../data/lab/zad4/proba_3_output.csv};
    \addplot[blue, semithick] file{../data/lab/zad4/proba_3_stpt.csv}; 
    \end{axis}
    \end{tikzpicture}
    \caption{Próba 3.: Przebieg procesu regulacji obiektu nieliniowego rozmytym regulatorem PID.}
    \label{pr_3_y}
\end{figure}

\begin{figure}[b]
    \centering
    \begin{tikzpicture}
    \begin{axis}[
    width=0.98\textwidth,
    xmin=0.0,xmax=800,ymin=0,ymax=100,
    xlabel={$k$},
    ylabel={$y[k]$},
    legend pos=south east,
    y tick label style={/pgf/number format/1000 sep=},
    ] 
    \addlegendentry{$u$},
    \addlegendimage{no markers,blue}
    \addplot[blue, semithick] file{../data/lab/zad4/proba_3_input.csv};  
    \end{axis}
    \end{tikzpicture}
    \caption{Próba 3.: Przebieg sygnału sterującego podczas regulacji obiektu nieliniowego rozmytym regulatorem PID.}
    \label{pr_3_u}
\end{figure}


\subsubsection{Próba 4.}
Parametry regulatorów lokalnych:
\begin{enumerate}
\item $K = 18, T_{\mathrm{i}} = 300,  T_{\mathrm{d}} = 1$
\item $K = 19, T_{\mathrm{i}} = 250,  T_{\mathrm{d}} = 1$
\item $K = 20, T_{\mathrm{i}} = 200,  T_{\mathrm{d}} = 1$
\end{enumerate}
Próba czwarta również wypadła rozczarowująco. Wskaźnik jakości zwiększył się znacznie - $E = \num{49693}$. Obserwujemy powolne działanie i reakcje regulatora. Niepożądanym zjawiskiem są również oscylacje oraz ostre piki sygnału sterującego. Wyniki dla próby czwartej przedstawione na wykresach \ref{pr_4_y} i \ref{pr_4_u}.

\begin{figure}[t]
    \centering
    \begin{tikzpicture}
    \begin{axis}[
    width=0.98\textwidth,
    xmin=0.0,xmax=800,ymin=30,ymax=55,
    xlabel={$k$},
    ylabel={$y[k]$},
    legend pos=south east,
    y tick label style={/pgf/number format/1000 sep=},
    ] 
    \addlegendentry{$y[k]$},
    \addlegendentry{$y^{\mathrm{zad}}$},
    \addlegendimage{no markers,red},
    \addlegendimage{no markers,blue}

    \addplot[red, semithick] file{../data/lab/zad4/proba_4_output.csv};
    \addplot[blue, semithick] file{../data/lab/zad4/proba_4_stpt.csv}; 
    \end{axis}
    \end{tikzpicture}
    \caption{Próba 4.: Przebieg procesu regulacji obiektu nieliniowego rozmytym regulatorem PID.}
    \label{pr_4_y}
\end{figure}

\begin{figure}[b]
    \centering
    \begin{tikzpicture}
    \begin{axis}[
    width=0.98\textwidth,
    xmin=0.0,xmax=800,ymin=0,ymax=100,
    xlabel={$k$},
    ylabel={$y[k]$},
    legend pos=south east,
    y tick label style={/pgf/number format/1000 sep=},
    ] 
    \addlegendentry{$u$},
    \addlegendimage{no markers,blue}
    \addplot[blue, semithick] file{../data/lab/zad4/proba_4_input.csv};  
    \end{axis}
    \end{tikzpicture}
    \caption{Próba 4.: Przebieg sygnału sterującego podczas regulacji obiektu nieliniowego rozmytym regulatorem PID.}
    \label{pr_4_u}
\end{figure}

\subsubsection{Wnioski}
Strojenie rozmytego regulatora PID nie należy do zadań prostych. Podczas czterech wykonanych prób udało nam się poprawić jakość regulacji jedynie w niewielkim stopniu. Zadanie to wymaga wielu eksperymentów i prób dla różnych parametrów każdego z regulatorów lokalnych. Podczas strojenia należy obserwować działanie regulatorów lokalnych w zależności od punktu pracy i dobieranie nastaw na podstawie obserwacji działania. Podczas strojenia konieczne jest dobre przemyślenie kolejnych kroków doboru parametrów. Warto uzbroić się również w cierpliwość.

\chapter{Regulacja z wykorzystaniem rozmytego regulatora DMC}
\label{lab5i6}

W ramach laboratorium przeprowadziliśmy próbę regulacji badanym 
nieliniowym obiektem za pomocą algorytmu predykcyjnego DMC. Podobnie jak w przypadku 
regulatora PID zdecydowaliśmy się na trapezową funkcję przynależności, jednak jako zmienną 
decyzjną wybraliśmy wartość zadaną a nie aktualną wartość wyjściową. 

Zgodnie z poleceniem, założyliśmy $D = N = N_{\mathrm{u}} = 300$ i $\lambda = 1$.
Wyniki eksperymentu zostały przedstawione na rysunkach \ref{dr_1_y} i \ref{dr_1_u}.
Zgodnie z naszymi oczekiwaniami,

\begin{figure}[t]
    \centering
    \begin{tikzpicture}
    \begin{axis}[
    width=0.98\textwidth,
    xmin=1,xmax=1000,ymin=30,ymax=55,
    xlabel={$k$},
    ylabel={$y[k]$},
    legend pos=south east,
    y tick label style={/pgf/number format/1000 sep=},
    ] 
    \addlegendentry{$y[k]$},
    \addlegendentry{$y^{\mathrm{zad}}$},
    \addlegendimage{no markers,red},
    \addlegendimage{no markers,blue}

    \addplot[red, semithick] file{../data/lab/zad5/proba2_output.csv};
    \addplot[blue, semithick] file{../data/lab/zad5/proba2_stpt.csv};  
    \end{axis}
    \end{tikzpicture}
    \caption{Przebieg procesu regulacji obiektu nieliniowego rozmytym regulatorem DMC}
    \label{dr_1_y}
\end{figure}

\begin{figure}[b]
    \centering
    \begin{tikzpicture}
    \begin{axis}[
    width=0.98\textwidth,
    xmin=1,xmax=1000,ymin=0,ymax=100,
    xlabel={$k$},
    ylabel={$y[k]$},
    legend pos=south east,
    y tick label style={/pgf/number format/1000 sep=},
    ] 
    \addlegendentry{$u$},
    \addlegendimage{no markers,blue}
    \addplot[blue, semithick] file{../data/lab/zad5/proba2_input.csv};  
    \end{axis}
    \end{tikzpicture}
    \caption{Przebieg sygnału sterującego podczas regulacji obiektu nieliniowego rozmytym regulatorem DMC}
    \label{dr_1_u}
\end{figure}
\FloatBarrier

\section{Rózne parametry kary}

W kolejnej próbie postanowiliśmy każdemu regulatorowi lokalnemu przypisać wartość parametru kary $\lambda_{i}$,
taką jak w poprzednio omawianym regulatorze liniowym $\lambda = \num{100}$ Wyniki tej próby wyszły jeszcze gorsze niż się spodziewaliśmy.
Z biegiem czasu, możemy stwierdzić że ten pomysł był bez sensu ponieważ dla mniejszych wartości tego parametru pojawiały się 
uchyby ustalone, dlatego też w tym przypadku powinniśmy zmniejszyć ten parametr a nie go zwiększać.

\begin{figure}[t]
    \centering
    \begin{tikzpicture}
    \begin{axis}[
    width=0.98\textwidth,
    xmin=1,xmax=1000,ymin=30,ymax=55,
    xlabel={$k$},
    ylabel={$y[k]$},
    legend pos=south east,
    y tick label style={/pgf/number format/1000 sep=},
    ] 
    \addlegendentry{$y[k]$},
    \addlegendentry{$y^{\mathrm{zad}}$},
    \addlegendimage{no markers,red},
    \addlegendimage{no markers,blue}

    \addplot[red, semithick] file{../data/lab/zad5/proba1_output.csv};
    \addplot[blue, semithick] file{../data/lab/zad5/proba1_stpt.csv};  
    \end{axis}
    \end{tikzpicture}
    \caption{Przebieg procesu regulacji obiektu nieliniowego rozmytym regulatorem DMC}
    \label{dr_21_y}
\end{figure}

\begin{figure}[b]
    \centering
    \begin{tikzpicture}
    \begin{axis}[
    width=0.98\textwidth,
    xmin=1,xmax=1000,ymin=0,ymax=100,
    xlabel={$k$},
    ylabel={$y[k]$},
    legend pos=south east,
    y tick label style={/pgf/number format/1000 sep=},
    ] 
    \addlegendentry{$u$},
    \addlegendimage{no markers,blue}
    \addplot[blue, semithick] file{../data/lab/zad5/proba1_input.csv};  
    \end{axis}
    \end{tikzpicture}
    \caption{Przebieg sygnału sterującego podczas regulacji obiektu nieliniowego rozmytym regulatorem DMC}
    \label{dr_2_u}
\end{figure}

\indent{}
W ostatniej próbie zdecydowaliśmy się zróżnicować wartości parametrów $\lambda_{i}$ pomiędzy regulatorami 
lokalnymi. Uznaliśmy że zastosujemy następujący zestaw parametrów: $\lambda_{1} = \num{1}$, $\lambda_{2} = \num{100}$,
$\lambda_{1} = \num{0.01}$. Wybraliśmy taki zestaw ponieważ uznaliśmy że regulacja w okolicy punktu pracy regulatora
pierwszego jest solidna a w okolicy większych wartości zadanych regulator niedomaga. Wartość środkowego
regulatora zostawiliśmy na poziomie z poprzedniej próby tak aby nie wprowadzała niepotrzebynch zakłóceń 
na przełamaniu charakterystyki obiektu.

\begin{figure}[t]
    \centering
    \begin{tikzpicture}
    \begin{axis}[
    width=0.98\textwidth,
    xmin=1,xmax=1000,ymin=30,ymax=55,
    xlabel={$k$},
    ylabel={$y[k]$},
    legend pos=south east,
    y tick label style={/pgf/number format/1000 sep=},
    ] 
    \addlegendentry{$y[k]$},
    \addlegendentry{$y^{\mathrm{zad}}$},
    \addlegendimage{no markers,red},
    \addlegendimage{no markers,blue}

    \addplot[red, semithick] file{../data/lab/zad5/proba3_output.csv};
    \addplot[blue, semithick] file{../data/lab/zad5/proba3_stpt.csv};  
    \end{axis}
    \end{tikzpicture}
    \caption{Przebieg procesu regulacji obiektu nieliniowego rozmytym regulatorem DMC}
    \label{dr_3_y}
\end{figure}

\begin{figure}[b]
    \centering
    \begin{tikzpicture}
    \begin{axis}[
    width=0.98\textwidth,
    xmin=1,xmax=1000,ymin=0,ymax=100,
    xlabel={$k$},
    ylabel={$y[k]$},
    legend pos=south east,
    y tick label style={/pgf/number format/1000 sep=},
    ] 
    \addlegendentry{$u$},
    \addlegendimage{no markers,blue}
    \addplot[blue, semithick] file{../data/lab/zad5/proba3_input.csv};  
    \end{axis}
    \end{tikzpicture}
    \caption{Przebieg sygnału sterującego podczas regulacji obiektu nieliniowego rozmytym regulatorem DMC}
    \label{dr_3_u}
\end{figure}

\subsubsection{Wnioski}
Podobnie jak w przypadku rozmytego regulatora PID, proces doboru parametrów regulatora DMC jest długi i złożony.
W trakcie laboratorium nie zdążyliśmy przetestować innych zestawów parametrów $\lambda$, a także działania
regulatora dla innych zmiennych decyzyjnych i funkcji przynależności.
\end{document}