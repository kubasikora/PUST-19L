\documentclass{mwrep}

% Polskie znaki
\usepackage{polski}
\usepackage[utf8]{inputenc}
\usepackage[T1]{fontenc}
\usepackage{lmodern}
\usepackage{indentfirst}

% Strona tytułowa
\usepackage{pgfplots}
\usepackage{siunitx}
\usepackage{paracol}
\usepackage{gensymb}

% Pływające obrazki
\usepackage{float}
\usepackage{svg}
\usepackage{graphicx}

% table of contents refs
\usepackage{hyperref}
\usepackage{cleveref}
\usepackage{booktabs}
\usepackage{listings}
\usepackage{placeins}
\usepackage{xcolor}

\usepgfplotslibrary{external}
\tikzexternalize

\usetikzlibrary{pgfplots.groupplots}
\sisetup{detect-weight,exponent-product=\cdot,output-decimal-marker={,},per-mode=symbol,binary-units=true,range-phrase={-},range-units=single}
\definecolor{szary}{rgb}{0.95,0.95,0.95}
%konfiguracje pakietu listings
\lstset{
	backgroundcolor=\color{szary},
	frame=single,
	breaklines=true,
}
\lstdefinestyle{customlatex}{
	basicstyle=\footnotesize\ttfamily,
	%basicstyle=\small\ttfamily,
}
\lstdefinestyle{customc}{
	breaklines=true,
	frame=tb,
	language=C,
	xleftmargin=0pt,
	showstringspaces=false,
	basicstyle=\small\ttfamily,
	keywordstyle=\bfseries\color{green!40!black},
	commentstyle=\itshape\color{purple!40!black},
	identifierstyle=\color{blue},
	stringstyle=\color{orange},
}
\lstdefinestyle{custommatlab}{
	captionpos=t,
	breaklines=true,
	frame=tb,
	xleftmargin=0pt,
	language=matlab,
	showstringspaces=false,
	%basicstyle=\footnotesize\ttfamily,
	basicstyle=\scriptsize\ttfamily,
	keywordstyle=\bfseries\color{green!40!black},
	commentstyle=\itshape\color{purple!40!black},
	identifierstyle=\color{blue},
	stringstyle=\color{orange},
}

%wymiar tekstu
\def\figurename{Rys.}
\def\tablename{Tab.}

%konfiguracja liczby p�ywaj�cych element�w
\setcounter{topnumber}{0}%2
\setcounter{bottomnumber}{3}%1
\setcounter{totalnumber}{5}%3
\renewcommand{\textfraction}{0.01}%0.2
\renewcommand{\topfraction}{0.95}%0.7
\renewcommand{\bottomfraction}{0.95}%0.3
\renewcommand{\floatpagefraction}{0.35}%0.5

\SendSettingsToPgf
\title{\bf Sprawozdanie z projektu i ćwiczenia laboratoryjnego nr 3, zadanie nr 1 \vskip 0.1cm}
\author{Marcin Dolicher \\ Jakub Sikora \\ Robert Wojtaś}
\date{\today}
\pgfplotsset{compat=1.15}	
\begin{document}
\frenchspacing
\pagestyle{uheadings}

\makeatletter
\renewcommand{\maketitle}{\begin{titlepage}
		\begin{center}{
				\LARGE {\bf Politechnika Warszawska}}\\
            \vspace{0.4cm}
            \leftskip-0.9cm
            {\LARGE {\bf \mbox{Wydział Elektroniki i Technik Informacyjnych}}}\\
            \vspace{0.2cm}
            {\LARGE {\bf \mbox{Instytut Automatyki i Informatyki Stosowanej}}}\\
            
            \vspace{5cm}
            \leftskip-1.3cm
			{\bf \Huge \mbox{Projektowanie układów sterowania} \vskip 0.1cm}
		\end{center}
		\vspace{0.1cm}

		\begin{center}
			{\bf \LARGE \@title}
		\end{center}

		\vspace{9cm}
		\begin{paracol}{2}
			\addtocontents{toc}{\protect\setcounter{tocdepth}{1}}
			\subsection*{Zespół:}
			\bf{ \Large{ \noindent\@author \par}}
			\addtocontents{toc}{\protect\setcounter{tocdepth}{2}}

			\switchcolumn \addtocontents{toc}{\protect\setcounter{tocdepth}{1}}
			\subsection*{Prowadzący:}
			\bf{\Large{\noindent dr inż. Patryk Chaber}}
			\addtocontents{toc}{\protect\setcounter{tocdepth}{2}}

		\end{paracol}
		\vspace*{\stretch{6}}
		\begin{center}
			\bf{\large{Warszawa, \@date\vskip 0.1cm}}
		\end{center}
	\end{titlepage}
}
\makeatother
\maketitle
\tableofcontents

\part{Projekt}
%\chapter{Sprawdzenie poprawności punktu pracy}
\label{pro1}

Aby sprawdzić poprawność podanego punktu pracy wykonaliśmy ekspe-
ryment polegający na pobudzeniu wejścia obiektu stałym sygnałem wejściowym o wartości
$U_{\mathrm{pp}} = \num{0}$ i sprawdzeniu czy sygnał wyjściowy stabilizuje się 
na wartości $Y_{\mathrm{pp}} = \num{0}$. Symulację obiektu przeprowadziliśmy
za pomocą funkcji \verb+symulacja_obiektu1y+. 

Eksperyment przeprowadziliśmy uruchamiając dostarczoną funkcję z następującymi parametrami:

\begin{center}
\verb+symulacja_obiektu1y(Upp, Upp, Ypp, Ypp)+ 
\end{center}

Uzyskany wynik \verb+ans+ $= \num{0}$ jest równy co do wartości $Y_{\mathrm{pp}}$ co potwierdza poprawność podanego
punktu pracy.

\vskip3cm

\begin{figure}[H]
    \centering
    \begin{tikzpicture}
    \begin{axis}[
    width=\textwidth,
    xmin=0,xmax=200,ymin=-2,ymax=2,
    xlabel={$k$},
    ylabel={$y[k]$},
    legend pos=south east,
    y tick label style={/pgf/number format/1000 sep=},
    ]
    \addplot[red, semithick] file{../data/project/zad1/zad1_output.csv};
    \legend{$y[k]$, }
    \end{axis} 
    \end{tikzpicture}
    \caption{Przebieg wyjścia obiektu sterowanego sygnałem stałym o wartości \mbox{$U_{\mathrm{pp}} = \num{0}$}}
    \label{pro1}
\end{figure}
%\chapter{Wyznaczenie odpowiedzi skokowych obiektu}
\label{zad2}

\section{Odpowiedź skokowa toru wejście-wyjście}
Eksperymenty wykonywane były z punktu pracy zdefiniowanego w zadaniu, przy sygnale zakłócenia $Z_{\mathrm{pp}} = 0$. Na wykresie \ref{zad2_odp_skok_input} możemy zaobserwować, że wraz z wzrostem wartości skoku rośnie również wartość odpowiedzi procesu $y[k]$. Zachowanie procesu jest zgodne 
z typową charakterystyką obiektów dynamicznych liniowych, w których wartość ustalona zmienia się liniowo wraz ze zmianą sygnału sterującego. 

\begin{figure}[t]
    \centering
    \begin{tikzpicture}
    \begin{axis}[
    width=0.98\textwidth,
    xmin=0.0,xmax=200,ymin=-1.5,ymax=1.5,
    xlabel={$k$},
    ylabel={$y[k]$},
    legend pos=south east,
    y tick label style={/pgf/number format/1000 sep=},
    ] 
    \addlegendentry{$\Delta u = \num{0,1}$},
    \addlegendentry{$\Delta u = \num{0,2}$}
    \addlegendentry{$\Delta u = \num{0,35}$},
    \addlegendentry{$\Delta u = \num{0,5}$},
    \addlegendentry{$\Delta u = \num{-0,1}$},
    \addlegendentry{$\Delta u = \num{-0,2}$},
    \addlegendentry{$\Delta u = \num{-0,35}$},
    \addlegendentry{$\Delta u = \num{-0,5}$},
    \addlegendimage{no markers,green}
	\addlegendimage{no markers,red}
	\addlegendimage{no markers,yellow}
	\addlegendimage{no markers,blue}
	\addlegendimage{no markers,black}
	\addlegendimage{no markers,orange}
	\addlegendimage{no markers,brown}
	\addlegendimage{no markers,violet}
    \addplot[green, semithick, thick] file{../data/zad2/zad2_input_output0.1.csv};
    \addplot[red, semithick, thick] file{../data/zad2/zad2_input_output0.2.csv};
    \addplot[yellow, semithick, thick] file{../data/zad2/zad2_input_output0.35.csv};
    \addplot[blue, semithick, thick] file{../data/zad2/zad2_input_output0.5.csv};
    \addplot[black, semithick, thick] file{../data/zad2/zad2_input_output-0.1.csv};
    \addplot[orange, semithick, thick] file{../data/zad2/zad2_input_output-0.2.csv};
    \addplot[brown, semithick, thick] file{../data/zad2/zad2_input_output-0.35.csv};
    \addplot[violet, semithick, thick] file{../data/zad2/zad2_input_output-0.5.csv};    
    \end{axis}
    \end{tikzpicture}
    \caption{Odpowiedzi procesu na skokową zmianę sygnału sterującego}
    \label{zad2_odp_skok_input}
\end{figure}

\section{Odpowiedź skokowa toru zakłócenie-wyjście}

Odpowiedź skokową toru zakłócenie-wyjście otrzymywaliśmy poprzez skokową zmianę zakłócenia. Eksperymenty wykonywane były ze stanu punktu pracy, w którym wszystkie sygnały wynosiły 0. Sygnał sterowania był stały podczas całej symulacji. Na wykresie \ref{zad2_odp_skok_disturbance} obserwujemy reakcję na zmiany zakłóceń i możemy stwierdzić, że jest ona podobna do tej otrzymanej przy zmianach sterowania, co tylko potwierdza nasze przypuszczenia o liniowości obiektu. 

\begin{figure}[b]
    \centering
    \begin{tikzpicture}
    \begin{axis}[
    width=0.98\textwidth,
    xmin=0.0,xmax=200,ymin=-1,ymax=1,
    xlabel={$k$},
    ylabel={$y[k]$},
    legend pos=south east,
    y tick label style={/pgf/number format/1000 sep=},
    ] 
    \addlegendentry{$\Delta z = \num{0,05}$},
    \addlegendentry{$\Delta z = \num{0,2}$}
    \addlegendentry{$\Delta z = \num{0,35}$},
    \addlegendentry{$\Delta z = \num{0,5}$},
    \addlegendentry{$\Delta z = \num{-0,05}$},
    \addlegendentry{$\Delta z = \num{-0,2}$},
    \addlegendentry{$\Delta z = \num{-0,35}$},
    \addlegendentry{$\Delta z = \num{-0,5}$},
    \addlegendimage{no markers,green}
	\addlegendimage{no markers,red}
	\addlegendimage{no markers,yellow}
	\addlegendimage{no markers,blue}
	\addlegendimage{no markers,black}
	\addlegendimage{no markers,orange}
	\addlegendimage{no markers,brown}
	\addlegendimage{no markers,violet}
    \addplot[green, semithick, thick] file{../data/zad2/zad2_disturbance_output0.05.csv};
    \addplot[red, semithick, thick] file{../data/zad2/zad2_disturbance_output0.2.csv};
    \addplot[yellow, semithick, thick] file{../data/zad2/zad2_disturbance_output0.35.csv};
    \addplot[blue, semithick, thick] file{../data/zad2/zad2_disturbance_output0.5.csv};
    \addplot[black, semithick, thick] file{../data/zad2/zad2_disturbance_output-0.05.csv};
    \addplot[orange, semithick, thick] file{../data/zad2/zad2_disturbance_output-0.2.csv};
    \addplot[brown, semithick, thick] file{../data/zad2/zad2_disturbance_output-0.35.csv};
    \addplot[violet, semithick, thick] file{../data/zad2/zad2_disturbance_output-0.5.csv};    
    \end{axis}
    \end{tikzpicture}
    \caption{Odpowiedzi procesu na skokową zmianę sygnału zaklócenia}
    \label{zad2_odp_skok_disturbance}
\end{figure}

\section{Wyznaczenie charakterystyki statycznej $y(u,z)$}
Wyznaczenie charakterystyki $y(u,z)$ rozpoczęliśmy od wyznaczenia charakterystyk $y(u)$ i $y(z)$. 
Na dalszym etapie zadania pomogło to w określeniu wzmocnień statycznych torów wejście-wyjście, zakłócenie-wyjśćie. 

Aby otrzymać wykres charakterystyki statycznej obiektu w zależności od 
dwóch argumentów należało przeprowadzić symulacje dla każdej 
wartości sygnałów $u$ oraz $d$ i zapamiętaniu wartości 
nasycenia sygnału $y$. Program, który realizujący zadanie to 
\verb+zad2_static_surface.m+ wykonywał wspomniane działania, 
a wyniki zapisywał do tablicy z wartościami charakterystyki statycznej. 
Wynik można zobaczyć na rysunku \ref{zad2_static_surf}.\\
\indent{} Wykres przedstawia płaszczyznę $y(u,z)$. Skoro powstały wykres 
jest płaszczyzną to można wywnioskować, że mamy do czynienia z obiektem liniowym. 


\chapter{Dobór nastaw regulatora PID i DMC.}
\label{pro4_PID}

\section{Przebieg strojenia regulatora PID.}
\subsection{Próba dostrojenia regulatora metoda Zieglera - Nicholsa}
Nasze strojenie rozpoczęliśmy od wykorzystania metody Zieglera - Nicholsa. Parametry regulatora dobraliśmy według reguł stosowanych w tej metodzie. Niestety nie przyniosła ona zadowalających efektów. Dla końcowych parametrów uzyskiwaliśmy zawsze niegasnące oscylacje. Zostawały one nawet po ręcznym skorygowaniu przez nas wartości członów regulatora (wartości obliczone za pomocą metody Zieglera - Nicholsa były wtedy traktowane jako dobry punkt wyjścia do dalszego strojenia).

~\\\\Na rysunkach \ref{pro_zad4_niegasnace_oscylacje_out} i \ref{pro_zad4_niegasnace_oscylacje_inp} zaprezentowaliśmy przebiegi sygnałów dla niegasnących oscylacji. Otrzymane przez nas wzmocnienie krytyczne wyniosło $K_{\mathrm{Kr}} = 2.2$, a okres oscylacji był równy $T_{\mathrm{Kr}} = \num{13.5}$ s. 

~\\\\ Następnie wartości $K_{\mathrm{Kr}}$ i $T_{\mathrm{Kr}}$ podstawiliśmy do tabeli z wzorami na $K$, $T_{\mathrm{i}}$ i $T_{\mathrm{d}}$ wykorzystywanej w metodzie Zieglera - Nicholsa, dzięki temu uzyskaliśmy wartości liczbowe tych parametrów równe $K = \num{2.2}$, $T_{\mathrm{i}} = \num{6.75}$ i $T_{\mathrm{d}} = \num{1.62}$. Wskaźnik jakości regulacji wynosił $E = 50084.0886$. Wykresy dla tych ustawień to \ref{pro_zad4_ziegler_out} i \ref{pro_zad4_ziegler_inp}. Widzimy na nich niegasnące oscylacje, które są niepożądane podczas regulacji. Podjęliśmy próbę lekkiej zmiany parametrów regulatora, jednak nie poprawiły one w żaden sposób jakości regulacji. 

~\\\\ Z powodu niezadowalających rezultatów po użyciu metody Zieglera - Nicholsa postanowiliśmy, że podczas szukania najlepszych nastawów oprzemy się w dużej mierze o nasze doświadczenie zdobyte na poprzednich laboratoriach i innych przedmiotach, których tematyką była regulacja. Zastosowaliśmy metodę inżynierską w której oceniamy jakościowo regulację i zmieniamy parametry tak, żeby nowe nastawy były lepsze od poprzednich. 

\subsection{Dobór nastawów regulatora metodą inżynierską}
Wiedzieliśmy, że musimy zlikwidować niegasnące oscylacje, aby to uzyskać zmniejszaliśmy K, a pozostałe człony wyłączyliśmy. Najlepsze rezultaty dostaliśmy dla $ K = \num{0.03} $ i $ K = \num{0.05} $. Przebiegi sygnałów dla tych wartości zaprezentowaliśmy na rysunkach \ref{pro_zad4_k_0.03_out}, \ref{pro_zad4_k_0.03_inp} oraz \ref{pro_zad4_k_0.05_out}, \ref{pro_zad4_k_0.05_inp}.

~\\\\ W następnym etapie włączyliśmy człon całkujący i różniczkujący. Ich wartości początkowe przyjęliśmy takie jak w metodzie Zieglera - Nicholsa $T_{\mathrm{i}} = \num{6.75}$ i $T_{\mathrm{d}} = \num{1.62}$. Przebiegi sygnałów dla tych wartości zaprezentowaliśmy na rysunkach \ref{pro_zad4_k_0.03_6.75_1.62_out} i \ref{pro_zad4_k_0.03_6.75_1.62_inp}. Możemy zauważyć, że regulator nie był w stanie osiągnąć pierwszych zmian wartości zadanej. Poprawna regulacja rozpoczęła się od około 2000 próbki, wtedy większość wartości zadanych została osiągnięta. Wskaźnik jakości regulacji wyniósł $E = 344443.9$. 

~\\\\ Postanowiliśmy jeszcze bardziej zmniejszyć wartości $T_{\mathrm{i}}$ i $T_{\mathrm{d}}$ w celu uzyskania jeszcze lepszej regulacji. Wykresy \ref{pro_zad4_k_0.03_6.5_1.56_out} i \ref{pro_zad4_k_0.03_6.5_1.56_inp} prezentują przebiegi dla parametrów $K = \num{0,03}$, $T_{\mathrm{i}} = \num{6.5}$ i $T_{\mathrm{d}} = \num{1.56}$, a rysunki \ref{pro_zad4_k_0.03_5.5_1.32_out} i \ref{pro_zad4_k_0.03_5.5_1.32_inp} dla $K = \num{0,03}$, $T_{\mathrm{i}} = \num{5.5}$ i $T_{\mathrm{d}} = \num{1.32}$. Dla ostatnich nastawów uzyskaliśmy najmniejszy wskaźnik jakości regulacji równy $E = \num{33955,70}$. Przebiegi sygnałów nie zmieniły się znacząco, ale po uważnej analizie możemy zauważyć, że dla ostatnich wartości parametrów niektóre wielkości zadane są szybciej osiągane. 


\begin{figure}[t]
    \centering
    \begin{tikzpicture}
    \begin{axis}[
    width=0.98\textwidth,
    xmin=0.0,xmax=5000,ymin=-4.5,ymax=11.5,
    xlabel={$k$},
    ylabel={$y[k]$},
    legend pos=south east,
    y tick label style={/pgf/number format/1000 sep=},
    ] 
    \addlegendentry{$y[k]$},
    \addlegendentry{$y^{\mathrm{zad}}[k]$},

    \addlegendimage{no markers,red}
    \addlegendimage{no markers,blue}

    \addplot[red, semithick] file{../data/project/zad4/zad3PID_output_K_2.2_Ti_100000000000_Td_0.csv};  
    \addplot[blue, semithick] file{../data/project/zad4/zad3PID_setpoint_K_2.2_Ti_100000000000_Td_0.csv};
    
    \end{axis}
    \end{tikzpicture}
    \caption{Przebieg procesu sterowanego za pomocą regulatora}
    \label{pro_zad4_niegasnace_oscylacje_out}
\end{figure}

\begin{figure}[b]
    \centering
    \begin{tikzpicture}
    \begin{axis}[
    width=0.98\textwidth,
    xmin=0.0,xmax=5000,ymin=-1.2,ymax=1.2,
    xlabel={$k$},
    ylabel={$y[k]$},
    legend pos=south east,
    y tick label style={/pgf/number format/1000 sep=},
    ] 
    \addlegendentry{$u[k]$},
    
    \addlegendimage{no markers,blue}

    \addplot[blue, semithick] file{../data/project/zad4/zad3PID_input_K_2.2_Ti_100000000000_Td_0.csv};
    
    \end{axis}
    \end{tikzpicture}
    \caption{Przebieg sygnału sterującego regulatora}
    \label{pro_zad4_niegasnace_oscylacje_inp}
\end{figure}

\begin{figure}[t]
    \centering
    \begin{tikzpicture}
    \begin{axis}[
    width=0.98\textwidth,
    xmin=0.0,xmax=5000,ymin=-4.5,ymax=11.5,
    xlabel={$k$},
    ylabel={$y[k]$},
    legend pos=south east,
    y tick label style={/pgf/number format/1000 sep=},
    ] 
    \addlegendentry{$y[k]$},
    \addlegendentry{$y^{\mathrm{zad}}[k]$},

    \addlegendimage{no markers,red}
    \addlegendimage{no markers,blue}

    \addplot[red, semithick] file{../data/project/zad4/zad3PID_output_K_2.2_Ti_6.75_Td_1.62.csv};  
    \addplot[blue, semithick] file{../data/project/zad4/zad3PID_setpoint_K_2.2_Ti_6.75_Td_1.62.csv};
    
    \end{axis}
    \end{tikzpicture}
    \caption{Przebieg procesu sterowanego za pomocą regulatora z parametrami wyznaczonymi za pomocą metody Zieglera-Nicholsa}
    \label{pro_zad4_ziegler_out}
\end{figure}

\begin{figure}[b]
    \centering
    \begin{tikzpicture}
    \begin{axis}[
    width=0.98\textwidth,
    xmin=0.0,xmax=5000,ymin=-1.2,ymax=1.2,
    xlabel={$k$},
    ylabel={$y[k]$},
    legend pos=south east,
    y tick label style={/pgf/number format/1000 sep=},
    ] 
    \addlegendentry{$u[k]$},
    
    \addlegendimage{no markers,blue}

    \addplot[blue, semithick] file{../data/project/zad4/zad3PID_input_K_2.2_Ti_6.75_Td_1.62.csv};
    
    \end{axis}
    \end{tikzpicture}
    \caption{Przebieg sygnału sterującego regulatora z parametrami wyznaczonymi za pomocą metody Zieglera-Nicholsa}
    \label{pro_zad4_ziegler_inp}
\end{figure}

\begin{figure}[t]
    \centering
    \begin{tikzpicture}
    \begin{axis}[
    width=0.98\textwidth,
    xmin=0.0,xmax=5000,ymin=-4.5,ymax=11.5,
    xlabel={$k$},
    ylabel={$y[k]$},
    legend pos=south east,
    y tick label style={/pgf/number format/1000 sep=},
    ] 
    \addlegendentry{$y[k]$},
    \addlegendentry{$y^{\mathrm{zad}}[k]$},

    \addlegendimage{no markers,red}
    \addlegendimage{no markers,blue}

    \addplot[red, semithick] file{../data/project/zad4/zad3PID_output_K_0.03_Ti_10000000_Td_0.csv};  
    \addplot[blue, semithick] file{../data/project/zad4/zad3PID_setpoint_K_0.03_Ti_10000000_Td_0.csv};
    
    \end{axis}
    \end{tikzpicture}
    \caption{Odpowiedzi procesu na skokową zmianę sygnału sterującego}
    \label{pro_zad4_k_0.03_out}
\end{figure}

\begin{figure}[b]
    \centering
    \begin{tikzpicture}
    \begin{axis}[
    width=0.98\textwidth,
    xmin=0.0,xmax=5000,ymin=-1.2,ymax=1.2,
    xlabel={$k$},
    ylabel={$y[k]$},
    legend pos=south east,
    y tick label style={/pgf/number format/1000 sep=},
    ] 
    \addlegendentry{$u[k]$},
    
    \addlegendimage{no markers,blue}

    \addplot[blue, semithick] file{../data/project/zad4/zad3PID_input_K_0.03_Ti_10000000_Td_0.csv};
    
    \end{axis}
    \end{tikzpicture}
    \caption{Odpowiedzi procesu na skokową zmianę sygnału sterującego}
    \label{pro_zad4_k_0.03_inp}
\end{figure}

\begin{figure}[t]
    \centering
    \begin{tikzpicture}
    \begin{axis}[
    width=0.98\textwidth,
    xmin=0.0,xmax=5000,ymin=-4.5,ymax=11.5,
    xlabel={$k$},
    ylabel={$y[k]$},
    legend pos=south east,
    y tick label style={/pgf/number format/1000 sep=},
    ] 
    \addlegendentry{$y[k]$},
    \addlegendentry{$y^{\mathrm{zad}}[k]$},

    \addlegendimage{no markers,red}
    \addlegendimage{no markers,blue}

    \addplot[red, semithick] file{../data/project/zad4/zad3PID_output_K_0.05_Ti_10000000_Td_0.csv};  
    \addplot[blue, semithick] file{../data/project/zad4/zad3PID_setpoint_K_0.05_Ti_10000000_Td_0.csv};
    
    \end{axis}
    \end{tikzpicture}
    \caption{Odpowiedzi procesu na skokową zmianę sygnału sterującego}
    \label{pro_zad4_k_0.05_out}
\end{figure}

\begin{figure}[b]
    \centering
    \begin{tikzpicture}
    \begin{axis}[
    width=0.98\textwidth,
    xmin=0.0,xmax=5000,ymin=-1.2,ymax=1.2,
    xlabel={$k$},
    ylabel={$y[k]$},
    legend pos=south east,
    y tick label style={/pgf/number format/1000 sep=},
    ] 
    \addlegendentry{$u[k]$},
    
    \addlegendimage{no markers,blue}

    \addplot[blue, semithick] file{../data/project/zad4/zad3PID_input_K_0.05_Ti_10000000_Td_0.csv};
    
    \end{axis}
    \end{tikzpicture}
    \caption{Odpowiedzi procesu na skokową zmianę sygnału sterującego}
    \label{pro_zad4_k_0.05_inp}
\end{figure}

\begin{figure}[t]
    \centering
    \begin{tikzpicture}
    \begin{axis}[
    width=0.98\textwidth,
    xmin=0.0,xmax=5000,ymin=-4.5,ymax=11.5,
    xlabel={$k$},
    ylabel={$y[k]$},
    legend pos=south east,
    y tick label style={/pgf/number format/1000 sep=},
    ] 
    \addlegendentry{$y[k]$},
    \addlegendentry{$y^{\mathrm{zad}}[k]$},

    \addlegendimage{no markers,red}
    \addlegendimage{no markers,blue}

    \addplot[red, semithick] file{../data/project/zad4/zad3PID_output_K_0.03_Ti_6.75_Td_1.62.csv};  
    \addplot[blue, semithick] file{../data/project/zad4/zad3PID_setpoint_K_0.03_Ti_6.75_Td_1.62.csv};
    
    \end{axis}
    \end{tikzpicture}
    \caption{Odpowiedzi procesu na skokową zmianę sygnału sterującego dla $K = \num{0,03}$, $T_{\mathrm{i}} = \num{6.75}$ i $T_{\mathrm{d}} = \num{1.62}$}
    \label{pro_zad4_k_0.03_6.75_1.62_out}
\end{figure}

\begin{figure}[b]
    \centering
    \begin{tikzpicture}
    \begin{axis}[
    width=0.98\textwidth,
    xmin=0.0,xmax=5000,ymin=-1.2,ymax=1.2,
    xlabel={$k$},
    ylabel={$y[k]$},
    legend pos=south east,
    y tick label style={/pgf/number format/1000 sep=},
    ] 
    \addlegendentry{$u[k]$},
    
    \addlegendimage{no markers,blue}

    \addplot[blue, semithick] file{../data/project/zad4/zad3PID_input_K_0.03_Ti_6.75_Td_1.62.csv};
    
    \end{axis}
    \end{tikzpicture}
    \caption{Odpowiedzi procesu na skokową zmianę sygnału sterującego dla $K = \num{0,03}$, $T_{\mathrm{i}} = \num{6.75}$ i $T_{\mathrm{d}} = \num{1.62}$}
    \label{pro_zad4_k_0.03_6.75_1.62_inp}
\end{figure}

\begin{figure}[t]
    \centering
    \begin{tikzpicture}
    \begin{axis}[
    width=0.98\textwidth,
    xmin=0.0,xmax=5000,ymin=-4.5,ymax=11.5,
    xlabel={$k$},
    ylabel={$y[k]$},
    legend pos=south east,
    y tick label style={/pgf/number format/1000 sep=},
    ] 
    \addlegendentry{$y[k]$},
    \addlegendentry{$y^{\mathrm{zad}}[k]$},

    \addlegendimage{no markers,red}
    \addlegendimage{no markers,blue}

    \addplot[red, semithick] file{../data/project/zad4/zad3PID_output_K_0.03_Ti_6.5_Td_1.56.csv};  
    \addplot[blue, semithick] file{../data/project/zad4/zad3PID_setpoint_K_0.03_Ti_6.5_Td_1.56.csv};
    
    \end{axis}
    \end{tikzpicture}
    \caption{Odpowiedzi procesu na skokową zmianę sygnału sterującego dla $K = \num{0,03}$, $T_{\mathrm{i}} = \num{6.5}$ i $T_{\mathrm{d}} = \num{1.56}$}
    \label{pro_zad4_k_0.03_6.5_1.56_out}
\end{figure}

\begin{figure}[b]
    \centering
    \begin{tikzpicture}
    \begin{axis}[
    width=0.98\textwidth,
    xmin=0.0,xmax=5000,ymin=-1.2,ymax=1.2,
    xlabel={$k$},
    ylabel={$y[k]$},
    legend pos=south east,
    y tick label style={/pgf/number format/1000 sep=},
    ] 
    \addlegendentry{$u[k]$},
    
    \addlegendimage{no markers,blue}

    \addplot[blue, semithick] file{../data/project/zad4/zad3PID_input_K_0.03_Ti_6.5_Td_1.56.csv};
    
    \end{axis}
    \end{tikzpicture}
    \caption{Odpowiedzi procesu na skokową zmianę sygnału sterującego dla $K = \num{0,03}$, $T_{\mathrm{i}} = \num{6.5}$ i $T_{\mathrm{d}} = \num{1.56}$}
    \label{pro_zad4_k_0.03_6.5_1.56_inp}
\end{figure}

\begin{figure}[t]
    \centering
    \begin{tikzpicture}
    \begin{axis}[
    width=0.98\textwidth,
    xmin=0.0,xmax=5000,ymin=-4.5,ymax=11.5,
    xlabel={$k$},
    ylabel={$y[k]$},
    legend pos=south east,
    y tick label style={/pgf/number format/1000 sep=},
    ] 
    \addlegendentry{$y[k]$},
    \addlegendentry{$y^{\mathrm{zad}}[k]$},

    \addlegendimage{no markers,red}
    \addlegendimage{no markers,blue}

    \addplot[red, semithick] file{../data/project/zad4/zad3PID_output_K_0.03_Ti_5.5_Td_1.32.csv};  
    \addplot[blue, semithick] file{../data/project/zad4/zad3PID_setpoint_K_0.03_Ti_5.5_Td_1.32.csv};
    
    \end{axis}
    \end{tikzpicture}
    \caption{Odpowiedzi procesu na skokową zmianę sygnału sterującego dla $K = \num{0,03}$, $T_{\mathrm{i}} = \num{5.5}$ i $T_{\mathrm{d}} = \num{1.32}$}
    \label{pro_zad4_k_0.03_5.5_1.32_out}
\end{figure}

\begin{figure}[b]
    \centering
    \begin{tikzpicture}
    \begin{axis}[
    width=0.98\textwidth,
    xmin=0.0,xmax=5000,ymin=-1.2,ymax=1.2,
    xlabel={$k$},
    ylabel={$y[k]$},
    legend pos=south east,
    y tick label style={/pgf/number format/1000 sep=},
    ] 
    \addlegendentry{$u[k]$},
    
    \addlegendimage{no markers,blue}

    \addplot[blue, semithick] file{../data/project/zad4/zad3PID_input_K_0.03_Ti_5.5_Td_1.32.csv};
    
    \end{axis}
    \end{tikzpicture}
    \caption{Odpowiedzi procesu na skokową zmianę sygnału sterującego dla $K = \num{0,03}$, $T_{\mathrm{i}} = \num{5.5}$ i $T_{\mathrm{d}} = \num{1.32}$}
    \label{pro_zad4_k_0.03_5.5_1.32_inp}
\end{figure}
\section{Przebieg strojenia regulatora DMC}
\label{pro4_DMC}

\begin{figure}[t]
    \centering
    \begin{tikzpicture}
    \begin{axis}[
    width=0.98\textwidth,
    xmin=0.0,xmax=3000,ymin=-4.2,ymax=10,
    xlabel={$k$},
    ylabel={$y[k]$},
    legend pos=south east,
    y tick label style={/pgf/number format/1000 sep=},
    ] 
    \addlegendentry{$y[k]$},
    \addlegendentry{$y^{\mathrm{zad}}[k]$},

    \addlegendimage{no markers,red}
    \addlegendimage{no markers,blue}

    \addplot[red, semithick] file{../data/project/zad4/zad4DMC_output_D_303_N_303_Nu_303_lam_1.csv};  
    \addplot[blue, semithick] file{../data/project/zad4/zad4DMC_setpoint_D_303_N_303_Nu_303_lam_1.csv};
    
    \end{axis}
    \end{tikzpicture}
    \caption{Odpowiedzi procesu na skokową zmianę sygnału sterującego dla $K = \num{303}$, $N = \num{303}$, $N_{\mathrm{u}} = \num{303}$ i $\lambda = 1$}
    \label{pro_zad4_d_303_n_303_nu_303_out}
\end{figure}

\begin{figure}[b]
    \centering
    \begin{tikzpicture}
    \begin{axis}[
    width=0.98\textwidth,
    xmin=0.0,xmax=3000,ymin=-1.2,ymax=1.2,
    xlabel={$k$},
    ylabel={$y[k]$},
    legend pos=south east,
    y tick label style={/pgf/number format/1000 sep=},
    ] 
    \addlegendentry{$u[k]$},
    
    \addlegendimage{no markers,blue}

    \addplot[blue, semithick] file{../data/project/zad4/zad4DMC_input_D_303_N_303_Nu_303_lam_1.csv};
    
    \end{axis}
    \end{tikzpicture}
    \caption{Odpowiedzi procesu na skokową zmianę sygnału sterującego dla $D = \num{303}$, $N = \num{303}$, $N_{\mathrm{u}} = \num{303}$ i $\lambda = 1$}
    \label{pro_zad4_d_303_n_303_nu_303_inp}
\end{figure}

\begin{figure}[t]
    \centering
    \begin{tikzpicture}
    \begin{axis}[
    width=0.98\textwidth,
    xmin=0.0,xmax=5000,ymin=-4.2,ymax=10,
    xlabel={$k$},
    ylabel={$y[k]$},
    legend pos=south east,
    y tick label style={/pgf/number format/1000 sep=},
    ] 
    \addlegendentry{$y[k]$},
    \addlegendentry{$y^{\mathrm{zad}}[k]$},

    \addlegendimage{no markers,red}
    \addlegendimage{no markers,blue}

    \addplot[red, semithick] file{../data/project/zad4/zad4DMC_output_D_303_N_220_Nu_220_lam_1_E_35395.6.csv};  
    \addplot[blue, semithick] file{../data/project/zad4/zad4DMC_setpoint_D_303_N_220_Nu_220_lam_1_E_35395.6.csv};
    
    \end{axis}
    \end{tikzpicture}
    \caption{Odpowiedzi procesu na skokową zmianę sygnału sterującego dla $K = \num{303}$, $N = \num{220}$, $N_{\mathrm{u}} = \num{220}$ i $\lambda = 1$}
    \label{pro_zad4_d_303_n_220_nu_220_out}
\end{figure}

\begin{figure}[b]
    \centering
    \begin{tikzpicture}
    \begin{axis}[
    width=0.98\textwidth,
    xmin=0.0,xmax=5000,ymin=-1.2,ymax=1.2,
    xlabel={$k$},
    ylabel={$y[k]$},
    legend pos=south east,
    y tick label style={/pgf/number format/1000 sep=},
    ] 
    \addlegendentry{$u[k]$},
    
    \addlegendimage{no markers,blue}

    \addplot[blue, semithick] file{../data/project/zad4/zad4DMC_input_D_303_N_220_Nu_220_lam_1_E_35395.6.csv};
    
    \end{axis}
    \end{tikzpicture}
    \caption{Odpowiedzi procesu na skokową zmianę sygnału sterującego dla $D = \num{303}$, $N = \num{220}$, $N_{\mathrm{u}} = \num{220}$ i $\lambda = 1$}
    \label{pro_zad4_d_303_n_220_nu_220_inp}
\end{figure}

\begin{figure}[t]
    \centering
    \begin{tikzpicture}
    \begin{axis}[
    width=0.98\textwidth,
    xmin=0.0,xmax=5000,ymin=-4.2,ymax=10,
    xlabel={$k$},
    ylabel={$y[k]$},
    legend pos=south east,
    y tick label style={/pgf/number format/1000 sep=},
    ] 
    \addlegendentry{$y[k]$},
    \addlegendentry{$y^{\mathrm{zad}}[k]$},

    \addlegendimage{no markers,red}
    \addlegendimage{no markers,blue}

    \addplot[red, semithick] file{../data/project/zad4/zad4DMC_output_D_303_N_50_Nu_50_lam_1_E_35395.6.csv};  
    \addplot[blue, semithick] file{../data/project/zad4/zad4DMC_setpoint_D_303_N_50_Nu_50_lam_1_E_35395.6.csv};
    
    \end{axis}
    \end{tikzpicture}
    \caption{Odpowiedzi procesu na skokową zmianę sygnału sterującego dla $K = \num{303}$, $N = \num{50}$, $N_{\mathrm{u}} = \num{50}$ i $\lambda = 1$}
    \label{pro_zad4_d_303_n_50_nu_50_out}
\end{figure}

\begin{figure}[b]
    \centering
    \begin{tikzpicture}
    \begin{axis}[
    width=0.98\textwidth,
    xmin=0.0,xmax=5000,ymin=-1.2,ymax=1.2,
    xlabel={$k$},
    ylabel={$y[k]$},
    legend pos=south east,
    y tick label style={/pgf/number format/1000 sep=},
    ] 
    \addlegendentry{$u[k]$},
    
    \addlegendimage{no markers,blue}

    \addplot[blue, semithick] file{../data/project/zad4/zad4DMC_input_D_303_N_50_Nu_50_lam_1_E_35395.6.csv};
    
    \end{axis}
    \end{tikzpicture}
    \caption{Odpowiedzi procesu na skokową zmianę sygnału sterującego dla $D = \num{303}$, $N = \num{50}$, $N_{\mathrm{u}} = \num{50} $ i $\lambda = 1$}
    \label{pro_zad4_d_303_n_50_nu_50_inp}
\end{figure}

\begin{figure}[t]
    \centering
    \begin{tikzpicture}
    \begin{axis}[
    width=0.98\textwidth,
    xmin=0.0,xmax=5000,ymin=-4.2,ymax=10,
    xlabel={$k$},
    ylabel={$y[k]$},
    legend pos=south east,
    y tick label style={/pgf/number format/1000 sep=},
    ] 
    \addlegendentry{$y[k]$},
    \addlegendentry{$y^{\mathrm{zad}}[k]$},

    \addlegendimage{no markers,red}
    \addlegendimage{no markers,blue}

    \addplot[red, semithick] file{../data/project/zad4/zad4DMC_output_D_303_N_50_Nu_10_lam_1_E_35377.3163.csv};  
    \addplot[blue, semithick] file{../data/project/zad4/zad4DMC_setpoint_D_303_N_50_Nu_10_lam_1_E_35377.3163.csv};
    
    \end{axis}
    \end{tikzpicture}
    \caption{Odpowiedzi procesu na skokową zmianę sygnału sterującego dla $K = \num{303}$, $N = \num{50}$, $N_{\mathrm{u}} = \num{10}$ i $\lambda = 1$}
    \label{pro_zad4_d_303_n_50_nu_10_out}
\end{figure}

\begin{figure}[b]
    \centering
    \begin{tikzpicture}
    \begin{axis}[
    width=0.98\textwidth,
    xmin=0.0,xmax=5000,ymin=-1.2,ymax=1.2,
    xlabel={$k$},
    ylabel={$y[k]$},
    legend pos=south east,
    y tick label style={/pgf/number format/1000 sep=},
    ] 
    \addlegendentry{$u[k]$},
    
    \addlegendimage{no markers,blue}

    \addplot[blue, semithick] file{../data/project/zad4/zad4DMC_input_D_303_N_50_Nu_10_lam_1_E_35377.3163.csv};
    
    \end{axis}
    \end{tikzpicture}
    \caption{Odpowiedzi procesu na skokową zmianę sygnału sterującego dla $D = \num{303}$, $N = \num{50}$, $N_{\mathrm{u}} = \num{10} $ i $\lambda = 1$}
    \label{pro_zad4_d_303_n_50_nu_10_inp}
\end{figure}


\begin{figure}[t]
    \centering
    \begin{tikzpicture}
    \begin{axis}[
    width=0.98\textwidth,
    xmin=0.0,xmax=5000,ymin=-4.2,ymax=10,
    xlabel={$k$},
    ylabel={$y[k]$},
    legend pos=south east,
    y tick label style={/pgf/number format/1000 sep=},
    ] 
    \addlegendentry{$y[k]$},
    \addlegendentry{$y^{\mathrm{zad}}[k]$},

    \addlegendimage{no markers,red}
    \addlegendimage{no markers,blue}

    \addplot[red, semithick] file{../data/project/zad4/zad4DMC_output_D_303_N_50_Nu_10_lam_10_E_32550.4148.csv};  
    \addplot[blue, semithick] file{../data/project/zad4/zad4DMC_setpoint_D_303_N_50_Nu_10_lam_10_E_32550.4148.csv};
    
    \end{axis}
    \end{tikzpicture}
    \caption{Odpowiedzi procesu na skokową zmianę sygnału sterującego dla $K = \num{303}$, $N = \num{50}$, $N_{\mathrm{u}} = \num{10}$ i $\lambda = 10$}
    \label{pro_zad4_d_303_n_50_nu_10_lam_10out}
\end{figure}

\begin{figure}[b]
    \centering
    \begin{tikzpicture}
    \begin{axis}[
    width=0.98\textwidth,
    xmin=0.0,xmax=5000,ymin=-1.2,ymax=1.2,
    xlabel={$k$},
    ylabel={$y[k]$},
    legend pos=south east,
    y tick label style={/pgf/number format/1000 sep=},
    ] 
    \addlegendentry{$u[k]$},
    
    \addlegendimage{no markers,blue}

    \addplot[blue, semithick] file{../data/project/zad4/zad4DMC_input_D_303_N_50_Nu_10_lam_10_E_32550.4148.csv};
    
    \end{axis}
    \end{tikzpicture}
    \caption{Odpowiedzi procesu na skokową zmianę sygnału sterującego dla $D = \num{303}$, $N = \num{50}$, $N_{\mathrm{u}} = \num{10} $ i $\lambda = 10$}
    \label{pro_zad4_d_303_n_50_nu_10_lam_10inp}
\end{figure}

\begin{figure}[t]
    \centering
    \begin{tikzpicture}
    \begin{axis}[
    width=0.98\textwidth,
    xmin=0.0,xmax=5000,ymin=-4.2,ymax=10,
    xlabel={$k$},
    ylabel={$y[k]$},
    legend pos=south east,
    y tick label style={/pgf/number format/1000 sep=},
    ] 
    \addlegendentry{$y[k]$},
    \addlegendentry{$y^{\mathrm{zad}}[k]$},

    \addlegendimage{no markers,red}
    \addlegendimage{no markers,blue}

    \addplot[red, semithick] file{../data/project/zad4/zad4DMC_output_D_303_N_50_Nu_10_lam_50_E_32109.6505.csv};  
    \addplot[blue, semithick] file{../data/project/zad4/zad4DMC_setpoint_D_303_N_50_Nu_10_lam_50_E_32109.6505.csv};
    
    \end{axis}
    \end{tikzpicture}
    \caption{Odpowiedzi procesu na skokową zmianę sygnału sterującego dla $K = \num{303}$, $N = \num{50}$, $N_{\mathrm{u}} = \num{10}$ i $\lambda = 50$}
    \label{pro_zad4_d_303_n_50_nu_10_lam_50out}
\end{figure}

\begin{figure}[b]
    \centering
    \begin{tikzpicture}
    \begin{axis}[
    width=0.98\textwidth,
    xmin=0.0,xmax=5000,ymin=-1.2,ymax=1.2,
    xlabel={$k$},
    ylabel={$y[k]$},
    legend pos=south east,
    y tick label style={/pgf/number format/1000 sep=},
    ] 
    \addlegendentry{$u[k]$},
    
    \addlegendimage{no markers,blue}

    \addplot[blue, semithick] file{../data/project/zad4/zad4DMC_input_D_303_N_50_Nu_10_lam_50_E_32109.6505.csv};
    
    \end{axis}
    \end{tikzpicture}
    \caption{Odpowiedzi procesu na skokową zmianę sygnału sterującego dla $D = \num{303}$, $N = \num{50}$, $N_{\mathrm{u}} = \num{10} $ i $\lambda = 50$}
    \label{pro_zad4_d_303_n_50_nu_10_lam_50inp}
\end{figure}

\begin{figure}[t]
    \centering
    \begin{tikzpicture}
    \begin{axis}[
    width=0.98\textwidth,
    xmin=0.0,xmax=5000,ymin=-4.2,ymax=10,
    xlabel={$k$},
    ylabel={$y[k]$},
    legend pos=south east,
    y tick label style={/pgf/number format/1000 sep=},
    ] 
    \addlegendentry{$y[k]$},
    \addlegendentry{$y^{\mathrm{zad}}[k]$},

    \addlegendimage{no markers,red}
    \addlegendimage{no markers,blue}

    \addplot[red, semithick] file{../data/project/zad4/zad4DMC_output_D_303_N_50_Nu_10_lam_120_E_32189.1237.csv};  
    \addplot[blue, semithick] file{../data/project/zad4/zad4DMC_setpoint_D_303_N_50_Nu_10_lam_120_E_32189.1237.csv};
    
    \end{axis}
    \end{tikzpicture}
    \caption{Odpowiedzi procesu na skokową zmianę sygnału sterującego dla $K = \num{303}$, $N = \num{50}$, $N_{\mathrm{u}} = \num{10}$ i $\lambda = 120$}
    \label{pro_zad4_d_303_n_50_nu_10_lam_120out}
\end{figure}

\begin{figure}[b]
    \centering
    \begin{tikzpicture}
    \begin{axis}[
    width=0.98\textwidth,
    xmin=0.0,xmax=5000,ymin=-1.2,ymax=1.2,
    xlabel={$k$},
    ylabel={$y[k]$},
    legend pos=south east,
    y tick label style={/pgf/number format/1000 sep=},
    ] 
    \addlegendentry{$u[k]$},
    
    \addlegendimage{no markers,blue}

    \addplot[blue, semithick] file{../data/project/zad4/zad4DMC_input_D_303_N_50_Nu_10_lam_120_E_32189.1237.csv};
    
    \end{axis}
    \end{tikzpicture}
    \caption{Odpowiedzi procesu na skokową zmianę sygnału sterującego dla $D = \num{303}$, $N = \num{50}$, $N_{\mathrm{u}} = \num{10} $ i $\lambda = 120$}
    \label{pro_zad4_d_303_n_50_nu_10_lam_120inp}
\end{figure}

\end{document}