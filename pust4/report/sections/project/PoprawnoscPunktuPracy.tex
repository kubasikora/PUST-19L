\chapter{Zbadanie poprawności podanego punktu pracy}
\label{pro1}

W pierwszej kolejności sprawdziliśmy poprawność podanego w projekcie 
punktu pracy. W stanie ustalonym spodziewamy się $u_{\mathrm{1}} = u_{\mathrm{2}} 
= u_{\mathrm{3}} = u_{\mathrm{4}} = y_{\mathrm{1}} = y_{\mathrm{2}} = 
y_{\mathrm{3}} = 0$. Za pomocą skryptu \verb+zad1.m+ dokonaliśmy sprawdzenia
czy podane wartości są poprawne. Skrypt ten wywołuje funkcję
\verb+symulacja_obiektu1+ z następującymi parametrami.

\lstset{language=Matlab,%
    breaklines=true,%
    morekeywords={matlab2tikz},
    keywordstyle=\color{blue},%
    morekeywords=[2]{1}, keywordstyle=[2]{\color{black}},
    identifierstyle=\color{black},%
    stringstyle=\color{mylilas},
    stringstyle=\color{magenta},
    commentstyle=\color{green},%
    showstringspaces=false,%without this there will be a symbol in the places where there is a space
    numbers=left,%
    numberstyle={\tiny \color{black}},% size of the numbers
    numbersep=9pt, % this defines how far the numbers are from the text
    emph=[1]{for,end,break},emphstyle=[1]\color{red}, %some words to emphasise
    %emph=[2]{word1,word2}, emphstyle=[2]{style},    
}

\begin{lstlisting}[style=custommatlab,frame=single,label={pro1},caption={Sprawdzenie poprawności podanego punktu pracy},captionpos=b]
    Upp1 = 0;
    Upp2 = 0;
    Upp3 = 0;
    Upp4 = 0;

    Ypp1 = 0;
    Ypp2 = 0;
    Ypp3 = 0;

    [y1, y2, y3] = symulacja_obiektu1(Upp1, Upp1, Upp1, Upp1, ...
                                      Upp2, Upp2, Upp2, Upp2, ...
                                      Upp3, Upp3, Upp3, Upp3, ...
                                      Upp4, Upp4, Upp4, Upp4, ...
                                      Ypp1, Ypp1, Ypp1, Ypp1, ...
                                      Ypp2, Ypp2, Ypp2, Ypp2, ...
                                      Ypp3, Ypp3, Ypp3, Ypp3);
\end{lstlisting}

W wyniku działania funkcji uzyskaliśmy wektor samych zer, które równe są wartościom wszystkich wyjść w punkcie pracy.
Jednoznacznie potwierdza to poprawność podanego punktu pracy.