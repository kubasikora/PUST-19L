\chapter{Badanie odpowiedzi skokowych obiektu}
\label{pro2}

W przypadku obiektu o wielu wejściach i wielu wyjściach, nie istnieje tylko jedna
odpowiedź skokowa opisująca obiekt. W badanych obiekcie, są trzy wyjścia i cztery wyjścia,
co daje dwanaście różnych odpowiedzi skokowych, z czego każda opisuje jeden tor procesu:
$$u_{i} \longleftrightarrow y_{j},  i = 1, 2, 3; j = 1, 2, 3, 4$$

Badania zautomatyzowaliśmy, wykorzystując przy tym autorski skrypt \verb+zad2.m+, który sam
wykonuje eksperymenty i zapisuje zebrane dane. Dodatkowo, zapisuje wszystkie odpowiedzi w postaci
jednej złożonej macierzy komórkowej. Ta postać będzie przydatna w dalszej części projektu, przy 
implementacji algorytmu regulacji DMC w wersji MIMO.
\newpage

\section{Odpowiedź obiektu na skokową zmianę pierwszego wejścia}
Na podstawie odpowiedzi skokowych obiektu określiliśmy wzmocnienia \\ \mbox{$K_{\mathrm{1},\mathrm{1}} = \num{2.5},
K_{\mathrm{2},\mathrm{1}} = \num{1.9}, K_{\mathrm{3},\mathrm{1}} = \num{0.25}$.}

\begin{figure}[h]
    \centering
    \begin{tikzpicture}
    \begin{axis}[
    width=0.95\textwidth,
    xmin=0.0,xmax=200,ymin=-0.5,ymax=1.5,
    xlabel={$k$},
    ylabel={$u[k]$},
    legend pos=north east,
    y tick label style={/pgf/number format/1000 sep=},
    ] 
    \addlegendentry{$u_{\mathrm{1}}[k]$},
    \addlegendentry{$u_{\mathrm{2}}[k]$},
    \addlegendentry{$u_{\mathrm{3}}[k]$},
    \addlegendentry{$u_{\mathrm{4}}[k]$},

    \addlegendimage{no markers, blue}
    \addlegendimage{no markers, red}    
    \addlegendimage{no markers, green}  
    \addlegendimage{no markers, orange}  
    \addplot[const plot, blue, semithick] file{../data/project/zad2/skok_wejscia_1_przebieg_wejscia_1.csv};
    \addplot[const plot, red, semithick] file{../data/project/zad2/skok_wejscia_1_przebieg_wejscia_2.csv};
    \addplot[const plot, green, semithick] file{../data/project/zad2/skok_wejscia_1_przebieg_wejscia_3.csv};
    \addplot[const plot, orange, semithick] file{../data/project/zad2/skok_wejscia_1_przebieg_wejscia_4.csv};
    
    \end{axis}
    \end{tikzpicture}
    \caption{Przebiegi wejściowe procesu podczas drugiego eksperymentu badania odpowiedzi skokowej}
    \label{pro2_odp_wej_1}
\end{figure}

\begin{figure}[b]
    \centering
    \begin{tikzpicture}
    \begin{axis}[
    width=0.95\textwidth,
    xmin=0.0,xmax=200,ymin=0,ymax=3,
    xlabel={$k$},
    ylabel={$s_{k}$},
    legend pos=north east,
    y tick label style={/pgf/number format/1000 sep=},
    ] 
    \addlegendentry{$y_{\mathrm{1}}[k]$},
    \addlegendentry{$y_{\mathrm{2}}[k]$},
    \addlegendentry{$y_{\mathrm{3}}[k]$},
    \addlegendimage{no markers, blue} 
    \addlegendimage{no markers, red}
    \addlegendimage{no markers, green}
    \addplot[blue, thick] file{../data/project/zad2/skok_wejscia_1_przebieg_wyjscia_1.csv};
    \addplot[red, thick] file{../data/project/zad2/skok_wejscia_1_przebieg_wyjscia_2.csv};
    \addplot[green, thick] file{../data/project/zad2/skok_wejscia_1_przebieg_wyjscia_3.csv};
    
    \end{axis}
    \end{tikzpicture}
    \caption{Przebieg wyjść procesu podczas skokowej zmiany pierwszego wejścia}
    \label{pro2_odp_wyj_1}
\end{figure}
\FloatBarrier

\section{Odpowiedź obiektu na skokową zmianę drugiego wejścia}
Na podstawie odpowiedzi skokowych obiektu określiliśmy wzmocnienia \\ \mbox{$K_{\mathrm{1},\mathrm{2}} = \num{1.5},
K_{\mathrm{2},\mathrm{2}} = \num{0.8}, K_{\mathrm{3},\mathrm{2}} = \num{1.1}$.}

\begin{figure}[h]
    \centering
    \begin{tikzpicture}
    \begin{axis}[
    width=0.95\textwidth,
    xmin=0.0,xmax=200,ymin=-0.5,ymax=1.5,
    xlabel={$k$},
    ylabel={$u[k]$},
    legend pos=north east,
    y tick label style={/pgf/number format/1000 sep=},
    ] 
    \addlegendentry{$u_{\mathrm{1}}[k]$},
    \addlegendentry{$u_{\mathrm{2}}[k]$},
    \addlegendentry{$u_{\mathrm{3}}[k]$},
    \addlegendentry{$u_{\mathrm{4}}[k]$},

    \addlegendimage{no markers, blue}
    \addlegendimage{no markers, red}    
    \addlegendimage{no markers, green}  
    \addlegendimage{no markers, orange}  
    \addplot[const plot, blue, semithick] file{../data/project/zad2/skok_wejscia_2_przebieg_wejscia_1.csv};
    \addplot[const plot, red, semithick] file{../data/project/zad2/skok_wejscia_2_przebieg_wejscia_2.csv};
    \addplot[const plot, green, semithick] file{../data/project/zad2/skok_wejscia_2_przebieg_wejscia_3.csv};
    \addplot[const plot, orange, semithick] file{../data/project/zad2/skok_wejscia_2_przebieg_wejscia_4.csv};

    \end{axis}
    \end{tikzpicture}
    \caption{Przebiegi wejściowe procesu podczas drugiego eksperymentu badania odpowiedzi skokowej}
    \label{pro2_odp_wej_2}
\end{figure}

\begin{figure}[b]
    \centering
    \begin{tikzpicture}
    \begin{axis}[
    width=0.95\textwidth,
    xmin=0.0,xmax=200,ymin=0,ymax=3,
    xlabel={$k$},
    ylabel={$s_{k}$},
    legend pos=north east,
    y tick label style={/pgf/number format/1000 sep=},
    ] 
    \addlegendentry{$y_{\mathrm{1}}[k]$},
    \addlegendentry{$y_{\mathrm{2}}[k]$},
    \addlegendentry{$y_{\mathrm{3}}[k]$},
    \addlegendimage{no markers, blue} 
    \addlegendimage{no markers, red}
    \addlegendimage{no markers, green}
    \addplot[blue, thick] file{../data/project/zad2/skok_wejscia_2_przebieg_wyjscia_1.csv};
    \addplot[red, thick] file{../data/project/zad2/skok_wejscia_2_przebieg_wyjscia_2.csv};
    \addplot[green, thick] file{../data/project/zad2/skok_wejscia_2_przebieg_wyjscia_3.csv};
    
    \end{axis}
    \end{tikzpicture}
    \caption{Przebieg wyjść procesu podczas skokowej zmiany drugiego wejścia}
    \label{pro2_odp_wyj_2}
\end{figure}
\FloatBarrier


\section{Odpowiedź obiektu na skokową zmianę trzeciego wejścia}
Na podstawie odpowiedzi skokowych obiektu określiliśmy wzmocnienia \\ \mbox{$K_{\mathrm{1},\mathrm{3}} = \num{0.1},
K_{\mathrm{2},\mathrm{3}} = \num{1.2}, K_{\mathrm{3},\mathrm{3}} = \num{1.7}$.}

\begin{figure}[h]
    \centering
    \begin{tikzpicture}
    \begin{axis}[
    width=0.95\textwidth,
    xmin=0.0,xmax=200,ymin=-0.5,ymax=1.5,
    xlabel={$k$},
    ylabel={$u[k]$},
    legend pos=north east,
    y tick label style={/pgf/number format/1000 sep=},
    ] 
    \addlegendentry{$u_{\mathrm{1}}[k]$},
    \addlegendentry{$u_{\mathrm{2}}[k]$},
    \addlegendentry{$u_{\mathrm{3}}[k]$},
    \addlegendentry{$u_{\mathrm{4}}[k]$},

    \addlegendimage{no markers, blue}
    \addlegendimage{no markers, red}    
    \addlegendimage{no markers, green}  
    \addlegendimage{no markers, orange}  
    \addplot[const plot, blue, semithick] file{../data/project/zad2/skok_wejscia_3_przebieg_wejscia_1.csv};
    \addplot[const plot, red, semithick] file{../data/project/zad2/skok_wejscia_3_przebieg_wejscia_2.csv};
    \addplot[const plot, green, semithick] file{../data/project/zad2/skok_wejscia_3_przebieg_wejscia_3.csv};
    \addplot[const plot, orange, semithick] file{../data/project/zad2/skok_wejscia_3_przebieg_wejscia_4.csv};

    \end{axis}
    \end{tikzpicture}
    \caption{Przebiegi wejściowe procesu podczas trzeciego eksperymentu badania odpowiedzi skokowej}
    \label{pro2_odp_wej_3}
\end{figure}

\begin{figure}[b]
    \centering
    \begin{tikzpicture}
    \begin{axis}[
    width=0.95\textwidth,
    xmin=0.0,xmax=200,ymin=0,ymax=3,
    xlabel={$k$},
    ylabel={$s_{k}$},
    legend pos=north east,
    y tick label style={/pgf/number format/1000 sep=},
    ] 
    \addlegendentry{$y_{\mathrm{1}}[k]$},
    \addlegendentry{$y_{\mathrm{2}}[k]$},
    \addlegendentry{$y_{\mathrm{3}}[k]$},
    \addlegendimage{no markers, blue} 
    \addlegendimage{no markers, red}
    \addlegendimage{no markers, green}
    \addplot[blue, thick] file{../data/project/zad2/skok_wejscia_3_przebieg_wyjscia_1.csv};
    \addplot[red, thick] file{../data/project/zad2/skok_wejscia_3_przebieg_wyjscia_2.csv};
    \addplot[green, thick] file{../data/project/zad2/skok_wejscia_3_przebieg_wyjscia_3.csv};
    
    \end{axis}
    \end{tikzpicture}
    \caption{Przebieg wyjść procesu podczas skokowej zmiany trzeciego wejścia}
    \label{pro2_odp_wyj_3}
\end{figure}
\FloatBarrier


\section{Odpowiedź obiektu na skokową zmianę czwartego wejścia}
Na podstawie odpowiedzi skokowych obiektu określiliśmy wzmocnienia \\ \mbox{$K_{\mathrm{1},\mathrm{2}} = \num{0.9},
K_{\mathrm{2},\mathrm{2}} = \num{0.25}, K_{\mathrm{3},\mathrm{2}} = \num{0.85}$.}

\begin{figure}[h]
    \centering
    \begin{tikzpicture}
    \begin{axis}[
    width=0.95\textwidth,
    xmin=0.0,xmax=200,ymin=-0.5,ymax=1.5,
    xlabel={$k$},
    ylabel={$u[k]$},
    legend pos=north east,
    y tick label style={/pgf/number format/1000 sep=},
    ] 
    \addlegendentry{$u_{\mathrm{1}}[k]$},
    \addlegendentry{$u_{\mathrm{2}}[k]$},
    \addlegendentry{$u_{\mathrm{3}}[k]$},
    \addlegendentry{$u_{\mathrm{4}}[k]$},

    \addlegendimage{no markers, blue}
    \addlegendimage{no markers, red}    
    \addlegendimage{no markers, green}  
    \addlegendimage{no markers, orange}  
    \addplot[const plot, blue, semithick] file{../data/project/zad2/skok_wejscia_4_przebieg_wejscia_1.csv};
    \addplot[const plot, red, semithick] file{../data/project/zad2/skok_wejscia_4_przebieg_wejscia_2.csv};
    \addplot[const plot, green, semithick] file{../data/project/zad2/skok_wejscia_4_przebieg_wejscia_3.csv};
    \addplot[const plot, orange, semithick] file{../data/project/zad2/skok_wejscia_4_przebieg_wejscia_4.csv};

    \end{axis}
    \end{tikzpicture}
    \caption{Przebiegi wejściowe procesu podczas czwartego eksperymentu badania odpowiedzi skokowej}
    \label{pro2_odp_wej_4}
\end{figure}

\begin{figure}[b]
    \centering
    \begin{tikzpicture}
    \begin{axis}[
    width=0.95\textwidth,
    xmin=0.0,xmax=200,ymin=0,ymax=3,
    xlabel={$k$},
    ylabel={$s_{k}$},
    legend pos=north east,
    y tick label style={/pgf/number format/1000 sep=},
    ] 
    \addlegendentry{$y_{\mathrm{1}}[k]$},
    \addlegendentry{$y_{\mathrm{2}}[k]$},
    \addlegendentry{$y_{\mathrm{3}}[k]$},
    \addlegendimage{no markers, blue} 
    \addlegendimage{no markers, red}
    \addlegendimage{no markers, green}
    \addplot[blue, thick] file{../data/project/zad2/skok_wejscia_4_przebieg_wyjscia_1.csv};
    \addplot[red, thick] file{../data/project/zad2/skok_wejscia_4_przebieg_wyjscia_2.csv};
    \addplot[green, thick] file{../data/project/zad2/skok_wejscia_4_przebieg_wyjscia_3.csv};
    
    \end{axis}
    \end{tikzpicture}
    \caption{Przebieg wyjść procesu podczas skokowej zmiany czwartego wejścia}
    \label{pro2_odp_wyj_4}
\end{figure}
\FloatBarrier