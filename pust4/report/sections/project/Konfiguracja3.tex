\subsection{Konfiguracje z ważoną macierzą połączeń}
\label{pro_pid_konf3}

W ramach strojenia postanowiliśmy sprawdzić jak zachowa się wielowymiarowy regulator
PID, którego struktura będzie opisana za pomocą macierzy o składnikach nie będącymi
zerem lub jedynką. Oznacza to że na swoim wejściu, regulator będzie otrzymywał nie jeden uchyb
a sumę ważoną wszystkich uchybów. Istotną zaletą takiego podejścia jest pełne wykorzystanie
wszystkich wejść procesu, aczkolwiek dobór współczynników macierzy połączeń może być problematyczny
a czasami wręcz niemożliwy. 

W przypadku badań nad regulatorami o takiej strukturze, będziemy posługiwać się
wyłącznie metodami optymalizacyjnymi. Postać macierzy połączeń dobieraliśmy 
ekseprymentalnie.


\subsubsection{Próba 1}
W pierwszym podejściu sprawdziliśmy zachowanie wielowymiarowego regulatora opisanego następującą 
macierzą połączeń:

\[
C =
\begin{bmatrix}
    0.5 & 0.5 & 0.1 \\
    0.33 & 0.33 & 0.33 \\
    0 & 0.5 & 0.5 \\
    0.45 & 0.1 & 0.45 
\end{bmatrix}
\]

W wyniku optymalizacji parametrów otrzymaliśmy:

\begin{center}
    $K^{\num{1}}_{\mathrm{p}} = \num{1.416}$, $T^{\num{1}}_{\mathrm{i}} = \num{1.0}$, $T^{\num{1}}_{\mathrm{d}} = \num{0.0}$ \\
    $K^{\num{2}}_{\mathrm{p}} = \num{2.819}$, $T^{\num{2}}_{\mathrm{i}} = \num{1.0}$, $T^{\num{2}}_{\mathrm{d}} = \num{0.0}$ \\
    $K^{\num{3}}_{\mathrm{p}} = \num{4.5388}$, $T^{\num{3}}_{\mathrm{i}} = \num{4.0917}$, $T^{\num{3}}_{\mathrm{d}} = \num{0.0}$ \\
    $K^{\num{4}}_{\mathrm{p}} = \num{2.7207}$, $T^{\num{4}}_{\mathrm{i}} = \num{1.0}$, $T^{\num{4}}_{\mathrm{d}} = \num{0.0}$ \\
\end{center}

W wyniku symulacji otrzymaliśmy następujące wskaźniki jakości:

\[
\begin{bmatrix}
    E_{\mathrm{1}} \\
    E_{\mathrm{2}} \\
    E_{\mathrm{3}} 
\end{bmatrix}
= 
\begin{bmatrix}
    \num{2.8588} \\
    \num{2.8639} \\
    \num{3.8668}
\end{bmatrix}
\]

Sumaryczny wskaźnik jakości, liczony jako suma wskaźników składowych, 
wyniósł $\bar{E} = \num{9.5894}$\\

Wyniki symulacji można uznać za zadowalające, mimo tego że nie udało się uzyskać
lepszego wyniku niż w przypadku binarnej macierzy połączeń. Przy takiej trajektorii 
zadanej, każdy wynik z wskaźnikiem jakości poniżej $\num{10}$ to wynik bardzo optymistyczny.
 
%% wyjscia 

\begin{figure}
    \centering
    \begin{subfigure}[b]{\textwidth}
        \centering
        \resizebox{\linewidth}{!}{
            \begin{tikzpicture}
                \begin{axis}[
                    width=0.98\textwidth,
                    height=7cm,
                    xmin=0.0,xmax=1000,ymin=-0.5,ymax=2,
                    xlabel={$k$},
                    ylabel={$y_{\mathrm{1}}[k]$},
                    legend pos=south east,
                    y tick label style={/pgf/number format/1000 sep=},
                    ] 
                    \addlegendentry{$y_{\mathrm{1}}[k]$},
                    \addlegendentry{$y_{\mathrm{1}}^{\mathrm{zad}}[k]$},
                
                    \addlegendimage{no markers,red}
                    \addlegendimage{no markers,blue}
                
                    \addplot[red, semithick] file{../data/project/zad4/pid/pid_mieszany_1/output_1.csv};  
                    \addplot[blue, semithick] file{../data/project/zad4/pid/pid_mieszany_1/stpt_1.csv};
                    
                    \end{axis}
            \end{tikzpicture}
        }
        \end{subfigure}

        \begin{subfigure}[b]{\textwidth}
            \centering
            \resizebox{\linewidth}{!}{
                \begin{tikzpicture}
                    \begin{axis}[
                        width=0.98\textwidth,
                        height=7cm,
                        xmin=0.0,xmax=1000,ymin=-0.5,ymax=2,
                        xlabel={$k$},
                        ylabel={$y_{\mathrm{2}}[k]$},
                        legend pos=south east,
                        y tick label style={/pgf/number format/1000 sep=},
                        ] 
                        \addlegendentry{$y_{\mathrm{2}}[k]$},
                        \addlegendentry{$y_{\mathrm{2}}^{\mathrm{zad}}[k]$},
                    
                        \addlegendimage{no markers,red}
                        \addlegendimage{no markers,blue}
                    
                        \addplot[red, semithick] file{../data/project/zad4/pid/pid_mieszany_1/output_2.csv};  
                        \addplot[blue, semithick] file{../data/project/zad4/pid/pid_mieszany_1/stpt_2.csv};
                        
                        \end{axis}
                \end{tikzpicture}
            }
        \end{subfigure}

        \begin{subfigure}[b]{\textwidth}
            \centering
            \resizebox{\linewidth}{!}{
                \begin{tikzpicture}
                    \begin{axis}[
                        width=0.98\textwidth,
                        height=7cm,
                        xmin=0.0,xmax=1000,ymin=-0.5,ymax=2,
                        xlabel={$k$},
                        ylabel={$y_{\mathrm{3}}[k]$},
                        legend pos=south east,
                        y tick label style={/pgf/number format/1000 sep=},
                        ] 
                        \addlegendentry{$y_{\mathrm{3}}[k]$},
                        \addlegendentry{$y_{\mathrm{3}}^{\mathrm{zad}}[k]$},
                    
                        \addlegendimage{no markers,red}
                        \addlegendimage{no markers,blue}
                    
                        \addplot[red, semithick] file{../data/project/zad4/pid/pid_mieszany_1/output_3.csv};  
                        \addplot[blue, semithick] file{../data/project/zad4/pid/pid_mieszany_1/stpt_3.csv};
                        
                        \end{axis}
                \end{tikzpicture}
            }
        \end{subfigure}
        \caption{Wyjścia procesu wielowymiarowego}
        \label{pro_pid_konf3_out1}
\end{figure}

%% wejscia 

\begin{figure}
    \centering
    \begin{subfigure}[b]{\textwidth}
        \centering
        \resizebox{\linewidth}{!}{
            \begin{tikzpicture}
                \begin{axis}[
                    width=0.98\textwidth,
                    height=6cm,
                    xmin=0.0,xmax=1000,ymin=-3,ymax=3,
                    xlabel={$k$},
                    ylabel={$u_{\mathrm{1}}[k]$},
                    legend pos=south east,
                    y tick label style={/pgf/number format/1000 sep=},
                    ] 
                    \addlegendentry{$u_{\mathrm{1}}[k]$},
                    \addlegendimage{no markers,blue}
                    \addplot[const plot, blue, semithick] file{../data/project/zad4/pid/pid_mieszany_1/input_1.csv};  
                \end{axis}
            \end{tikzpicture}
        }
    \end{subfigure}

    \begin{subfigure}[b]{\textwidth}
        \centering
        \resizebox{\linewidth}{!}{
            \begin{tikzpicture}
                \begin{axis}[
                    width=0.98\textwidth,
                    height=6cm,
                    xmin=0.0,xmax=1000,ymin=-3,ymax=3,
                    xlabel={$k$},
                    ylabel={$u_{\mathrm{2}}[k]$},
                    legend pos=south east,
                    y tick label style={/pgf/number format/1000 sep=},
                    ] 
                    \addlegendentry{$u_{\mathrm{2}}[k]$},
                    \addlegendimage{no markers,blue}
                    \addplot[const plot, blue, semithick] file{../data/project/zad4/pid/pid_mieszany_1/input_2.csv};  
                \end{axis}
            \end{tikzpicture}
        }
    \end{subfigure}

    \begin{subfigure}[b]{\textwidth}
        \centering
        \resizebox{\linewidth}{!}{
            \begin{tikzpicture}
                \begin{axis}[
                    width=0.98\textwidth,
                    height=6cm,
                    xmin=0.0,xmax=1000,ymin=-3,ymax=3,
                    xlabel={$k$},
                    ylabel={$u_{\mathrm{3}}[k]$},
                    legend pos=south east,
                    y tick label style={/pgf/number format/1000 sep=},
                    ] 
                    \addlegendentry{$u_{\mathrm{3}}[k]$},
                    \addlegendimage{no markers,blue}
                    \addplot[const plot, blue, semithick] file{../data/project/zad4/pid/pid_mieszany_1/input_3.csv};  
                \end{axis}
            \end{tikzpicture}
        }
    \end{subfigure}

    \begin{subfigure}[b]{\textwidth}
        \centering
        \resizebox{\linewidth}{!}{
            \begin{tikzpicture}
                \begin{axis}[
                    width=0.98\textwidth,
                    height=6cm,
                    xmin=0.0,xmax=1000,ymin=-3,ymax=3,
                    xlabel={$k$},
                    ylabel={$u_{\mathrm{4}}[k]$},
                    legend pos=south east,
                    y tick label style={/pgf/number format/1000 sep=},
                    ] 
                    \addlegendentry{$u_{\mathrm{4}}[k]$},
                    \addlegendimage{no markers,blue}
                    \addplot[const plot, blue, semithick] file{../data/project/zad4/pid/pid_mieszany_1/input_4.csv};  
                \end{axis}
            \end{tikzpicture}
        }
    \end{subfigure}

    \caption{Wejścia procesu wielowymiarowego}
    \label{pro_pid_konf3_in1}
\end{figure}
\FloatBarrier

\subsubsection{Próba 2}

Mając za cel zmniejszenie sumarycznego wskaźnika jakości spróbowaliśmy jeszcze raz
dobrać parametry macierzy połączeń. W kolejnej próbie zdecydowaliśmy się na następującą
strukturę regulatora:

\[
C =
\begin{bmatrix}
    0.8 & 0.1 & 0.1 \\
    0.1 & 0.4 & 0.5 \\
    0.1 & 0.75 & 0.15 \\
    0.1 & 0.6 & 0.3 
\end{bmatrix}
\]

W wyniku optymalizacji parametrów otrzymaliśmy:

\begin{center}
    $K^{\num{1}}_{\mathrm{p}} = \num{0.97718}$, $T^{\num{1}}_{\mathrm{i}} = \num{2.376}$, $T^{\num{1}}_{\mathrm{d}} = \num{0.0}$ \\
    $K^{\num{2}}_{\mathrm{p}} = \num{5.3702}$, $T^{\num{2}}_{\mathrm{i}} = \num{2.6245}$, $T^{\num{2}}_{\mathrm{d}} = \num{0.0}$ \\
    $K^{\num{3}}_{\mathrm{p}} = \num{7.5156}$, $T^{\num{3}}_{\mathrm{i}} = \num{1.0}$, $T^{\num{3}}_{\mathrm{d}} = \num{0.0}$ \\
    $K^{\num{4}}_{\mathrm{p}} = \num{0.01}$, $T^{\num{4}}_{\mathrm{i}} = \num{52.455}$, $T^{\num{4}}_{\mathrm{d}} = \num{0.0}$ \\
\end{center}

W wyniku symulacji otrzymaliśmy następujące wskaźniki jakości:

\[
\begin{bmatrix}
    E_{\mathrm{1}} \\
    E_{\mathrm{2}} \\
    E_{\mathrm{3}} 
\end{bmatrix}
= 
\begin{bmatrix}
    \num{3.9058} \\
    \num{2.4811} \\
    \num{4.087}
\end{bmatrix}
\]

Sumaryczny wskaźnik jakości, liczony jako suma wskaźników składowych, 
wyniósł $\bar{E} = \num{10.474}$\\

W przypadku tak opisanej struktury połączeń regulatora wielowymiarowego PID, doszło do ciekawego
zjawiska. W wyniku działania algorytmu optymalizacji regulator sterujący wejściem czwartym 
został zdegenerowany. Takie działanie jest dla nas niepożądane.

\begin{figure}
    \centering
    \begin{subfigure}[b]{\textwidth}
        \centering
        \resizebox{\linewidth}{!}{
            \begin{tikzpicture}
                \begin{axis}[
                    width=0.98\textwidth,
                    height=7cm,
                    xmin=0.0,xmax=1000,ymin=-0.5,ymax=2,
                    xlabel={$k$},
                    ylabel={$y_{\mathrm{1}}[k]$},
                    legend pos=south east,
                    y tick label style={/pgf/number format/1000 sep=},
                    ] 
                    \addlegendentry{$y_{\mathrm{1}}[k]$},
                    \addlegendentry{$y_{\mathrm{1}}^{\mathrm{zad}}[k]$},
                
                    \addlegendimage{no markers,red}
                    \addlegendimage{no markers,blue}
                
                    \addplot[red, semithick] file{../data/project/zad4/pid/pid_mieszany_2/output_1.csv};  
                    \addplot[blue, semithick] file{../data/project/zad4/pid/pid_mieszany_2/stpt_1.csv};
                    
                    \end{axis}
            \end{tikzpicture}
        }
        \end{subfigure}

        \begin{subfigure}[b]{\textwidth}
            \centering
            \resizebox{\linewidth}{!}{
                \begin{tikzpicture}
                    \begin{axis}[
                        width=0.98\textwidth,
                        height=7cm,
                        xmin=0.0,xmax=1000,ymin=-0.5,ymax=2,
                        xlabel={$k$},
                        ylabel={$y_{\mathrm{2}}[k]$},
                        legend pos=south east,
                        y tick label style={/pgf/number format/1000 sep=},
                        ] 
                        \addlegendentry{$y_{\mathrm{2}}[k]$},
                        \addlegendentry{$y_{\mathrm{2}}^{\mathrm{zad}}[k]$},
                    
                        \addlegendimage{no markers,red}
                        \addlegendimage{no markers,blue}
                    
                        \addplot[red, semithick] file{../data/project/zad4/pid/pid_mieszany_2/output_2.csv};  
                        \addplot[blue, semithick] file{../data/project/zad4/pid/pid_mieszany_2/stpt_2.csv};
                        
                        \end{axis}
                \end{tikzpicture}
            }
        \end{subfigure}

        \begin{subfigure}[b]{\textwidth}
            \centering
            \resizebox{\linewidth}{!}{
                \begin{tikzpicture}
                    \begin{axis}[
                        width=0.98\textwidth,
                        height=7cm,
                        xmin=0.0,xmax=1000,ymin=-0.5,ymax=2,
                        xlabel={$k$},
                        ylabel={$y_{\mathrm{3}}[k]$},
                        legend pos=south east,
                        y tick label style={/pgf/number format/1000 sep=},
                        ] 
                        \addlegendentry{$y_{\mathrm{3}}[k]$},
                        \addlegendentry{$y_{\mathrm{3}}^{\mathrm{zad}}[k]$},
                    
                        \addlegendimage{no markers,red}
                        \addlegendimage{no markers,blue}
                    
                        \addplot[red, semithick] file{../data/project/zad4/pid/pid_mieszany_2/output_3.csv};  
                        \addplot[blue, semithick] file{../data/project/zad4/pid/pid_mieszany_2/stpt_3.csv};
                        
                        \end{axis}
                \end{tikzpicture}
            }
        \end{subfigure}
        \caption{Wyjścia procesu wielowymiarowego}
        \label{pro_pid_konf3_out2}
\end{figure}

%% wejscia 

\begin{figure}
    \centering
    \begin{subfigure}[b]{\textwidth}
        \centering
        \resizebox{\linewidth}{!}{
            \begin{tikzpicture}
                \begin{axis}[
                    width=0.98\textwidth,
                    height=6cm,
                    xmin=0.0,xmax=1000,ymin=-3,ymax=3,
                    xlabel={$k$},
                    ylabel={$u_{\mathrm{1}}[k]$},
                    legend pos=south east,
                    y tick label style={/pgf/number format/1000 sep=},
                    ] 
                    \addlegendentry{$u_{\mathrm{1}}[k]$},
                    \addlegendimage{no markers,blue}
                    \addplot[const plot, blue, semithick] file{../data/project/zad4/pid/pid_mieszany_2/input_1.csv};  
                \end{axis}
            \end{tikzpicture}
        }
    \end{subfigure}

    \begin{subfigure}[b]{\textwidth}
        \centering
        \resizebox{\linewidth}{!}{
            \begin{tikzpicture}
                \begin{axis}[
                    width=0.98\textwidth,
                    height=6cm,
                    xmin=0.0,xmax=1000,ymin=-3,ymax=3,
                    xlabel={$k$},
                    ylabel={$u_{\mathrm{2}}[k]$},
                    legend pos=south east,
                    y tick label style={/pgf/number format/1000 sep=},
                    ] 
                    \addlegendentry{$u_{\mathrm{2}}[k]$},
                    \addlegendimage{no markers,blue}
                    \addplot[const plot, blue, semithick] file{../data/project/zad4/pid/pid_mieszany_2/input_2.csv};  
                \end{axis}
            \end{tikzpicture}
        }
    \end{subfigure}

    \begin{subfigure}[b]{\textwidth}
        \centering
        \resizebox{\linewidth}{!}{
            \begin{tikzpicture}
                \begin{axis}[
                    width=0.98\textwidth,
                    height=6cm,
                    xmin=0.0,xmax=1000,ymin=-3,ymax=3,
                    xlabel={$k$},
                    ylabel={$u_{\mathrm{3}}[k]$},
                    legend pos=south east,
                    y tick label style={/pgf/number format/1000 sep=},
                    ] 
                    \addlegendentry{$u_{\mathrm{3}}[k]$},
                    \addlegendimage{no markers,blue}
                    \addplot[const plot, blue, semithick] file{../data/project/zad4/pid/pid_mieszany_2/input_3.csv};  
                \end{axis}
            \end{tikzpicture}
        }
    \end{subfigure}

    \begin{subfigure}[b]{\textwidth}
        \centering
        \resizebox{\linewidth}{!}{
            \begin{tikzpicture}
                \begin{axis}[
                    width=0.98\textwidth,
                    height=6cm,
                    xmin=0.0,xmax=1000,ymin=-3,ymax=3,
                    xlabel={$k$},
                    ylabel={$u_{\mathrm{4}}[k]$},
                    legend pos=south east,
                    y tick label style={/pgf/number format/1000 sep=},
                    ] 
                    \addlegendentry{$u_{\mathrm{4}}[k]$},
                    \addlegendimage{no markers,blue}
                    \addplot[const plot, blue, semithick] file{../data/project/zad4/pid/pid_mieszany_2/input_4.csv};  
                \end{axis}
            \end{tikzpicture}
        }
    \end{subfigure}

    \caption{Wejścia procesu wielowymiarowego}
    \label{pro_pid_konf3_in2}
\end{figure}
\FloatBarrier

\subsubsection{Próba 3}

Ostatnią próbę wykonaliśmy dla następującej macierzy połączeń:

\[
C =
\begin{bmatrix}
    0.8 & 0.15 & 0.05 \\
    0.1 & 0.3 & 0.6 \\
    0.05 & 0.8 & 0.15 \\
    0.1 & 0.2 & 0.7 
\end{bmatrix}
\]

W wyniku optymalizacji parametrów otrzymaliśmy:

\begin{center}
    $K^{\num{1}}_{\mathrm{p}} = \num{0.54945}$, $T^{\num{1}}_{\mathrm{i}} = \num{1.0}$, $T^{\num{1}}_{\mathrm{d}} = \num{0.0}$ \\
    $K^{\num{2}}_{\mathrm{p}} = \num{4.9163}$, $T^{\num{2}}_{\mathrm{i}} = \num{1.1727}$, $T^{\num{2}}_{\mathrm{d}} = \num{0.0}$ \\
    $K^{\num{3}}_{\mathrm{p}} = \num{6.1998}$, $T^{\num{3}}_{\mathrm{i}} = \num{2.1187}$, $T^{\num{3}}_{\mathrm{d}} = \num{0.0}$ \\
    $K^{\num{4}}_{\mathrm{p}} = \num{2.848}$, $T^{\num{4}}_{\mathrm{i}} = \num{1.6849}$, $T^{\num{4}}_{\mathrm{d}} = \num{0.0}$ \\
\end{center}

W wyniku symulacji otrzymaliśmy następujące wskaźniki jakości:

\[
\begin{bmatrix}
    E_{\mathrm{1}} \\
    E_{\mathrm{2}} \\
    E_{\mathrm{3}} 
\end{bmatrix}
= 
\begin{bmatrix}
    \num{2.2882} \\
    \num{1.774} \\
    \num{3.1935}
\end{bmatrix}
\]

Sumaryczny wskaźnik jakości, liczony jako suma wskaźników składowych, 
wyniósł $\bar{E} = \num{7.2558}$\\

W przypadku tak zaprojektowanego regulatora udało nam się uzyskać wynik porównywalny z najlepszym.
Prawdopodobnie jest możliwe aby uzyskać jeszcze lepszy wynik poprzez optymalizowanie samej
macierzy połączeń, jednak uznaliśmy że wykracza to poza temat projektu.

\begin{figure}
    \centering
    \begin{subfigure}[b]{\textwidth}
        \centering
        \resizebox{\linewidth}{!}{
            \begin{tikzpicture}
                \begin{axis}[
                    width=0.98\textwidth,
                    height=7cm,
                    xmin=0.0,xmax=1000,ymin=-0.5,ymax=2,
                    xlabel={$k$},
                    ylabel={$y_{\mathrm{1}}[k]$},
                    legend pos=south east,
                    y tick label style={/pgf/number format/1000 sep=},
                    ] 
                    \addlegendentry{$y_{\mathrm{1}}[k]$},
                    \addlegendentry{$y_{\mathrm{1}}^{\mathrm{zad}}[k]$},
                
                    \addlegendimage{no markers,red}
                    \addlegendimage{no markers,blue}
                
                    \addplot[red, semithick] file{../data/project/zad4/pid/pid_mieszany_3/output_1.csv};  
                    \addplot[blue, semithick] file{../data/project/zad4/pid/pid_mieszany_3/stpt_1.csv};
                    
                    \end{axis}
            \end{tikzpicture}
        }
        \end{subfigure}

        \begin{subfigure}[b]{\textwidth}
            \centering
            \resizebox{\linewidth}{!}{
                \begin{tikzpicture}
                    \begin{axis}[
                        width=0.98\textwidth,
                        height=7cm,
                        xmin=0.0,xmax=1000,ymin=-0.5,ymax=2,
                        xlabel={$k$},
                        ylabel={$y_{\mathrm{2}}[k]$},
                        legend pos=south east,
                        y tick label style={/pgf/number format/1000 sep=},
                        ] 
                        \addlegendentry{$y_{\mathrm{2}}[k]$},
                        \addlegendentry{$y_{\mathrm{2}}^{\mathrm{zad}}[k]$},
                    
                        \addlegendimage{no markers,red}
                        \addlegendimage{no markers,blue}
                    
                        \addplot[red, semithick] file{../data/project/zad4/pid/pid_mieszany_3/output_2.csv};  
                        \addplot[blue, semithick] file{../data/project/zad4/pid/pid_mieszany_3/stpt_2.csv};
                        
                        \end{axis}
                \end{tikzpicture}
            }
        \end{subfigure}

        \begin{subfigure}[b]{\textwidth}
            \centering
            \resizebox{\linewidth}{!}{
                \begin{tikzpicture}
                    \begin{axis}[
                        width=0.98\textwidth,
                        height=7cm,
                        xmin=0.0,xmax=1000,ymin=-0.5,ymax=2,
                        xlabel={$k$},
                        ylabel={$y_{\mathrm{3}}[k]$},
                        legend pos=south east,
                        y tick label style={/pgf/number format/1000 sep=},
                        ] 
                        \addlegendentry{$y_{\mathrm{3}}[k]$},
                        \addlegendentry{$y_{\mathrm{3}}^{\mathrm{zad}}[k]$},
                    
                        \addlegendimage{no markers,red}
                        \addlegendimage{no markers,blue}
                    
                        \addplot[red, semithick] file{../data/project/zad4/pid/pid_mieszany_3/output_3.csv};  
                        \addplot[blue, semithick] file{../data/project/zad4/pid/pid_mieszany_3/stpt_3.csv};
                        
                        \end{axis}
                \end{tikzpicture}
            }
        \end{subfigure}
        \caption{Wyjścia procesu wielowymiarowego}
        \label{pro_pid_konf3_out3}
\end{figure}

%% wejscia 

\begin{figure}
    \centering
    \begin{subfigure}[b]{\textwidth}
        \centering
        \resizebox{\linewidth}{!}{
            \begin{tikzpicture}
                \begin{axis}[
                    width=0.98\textwidth,
                    height=6cm,
                    xmin=0.0,xmax=1000,ymin=-3,ymax=3,
                    xlabel={$k$},
                    ylabel={$u_{\mathrm{1}}[k]$},
                    legend pos=south east,
                    y tick label style={/pgf/number format/1000 sep=},
                    ] 
                    \addlegendentry{$u_{\mathrm{1}}[k]$},
                    \addlegendimage{no markers,blue}
                    \addplot[const plot, blue, semithick] file{../data/project/zad4/pid/pid_mieszany_3/input_1.csv};  
                \end{axis}
            \end{tikzpicture}
        }
    \end{subfigure}

    \begin{subfigure}[b]{\textwidth}
        \centering
        \resizebox{\linewidth}{!}{
            \begin{tikzpicture}
                \begin{axis}[
                    width=0.98\textwidth,
                    height=6cm,
                    xmin=0.0,xmax=1000,ymin=-3,ymax=3,
                    xlabel={$k$},
                    ylabel={$u_{\mathrm{2}}[k]$},
                    legend pos=south east,
                    y tick label style={/pgf/number format/1000 sep=},
                    ] 
                    \addlegendentry{$u_{\mathrm{2}}[k]$},
                    \addlegendimage{no markers,blue}
                    \addplot[const plot, blue, semithick] file{../data/project/zad4/pid/pid_mieszany_3/input_2.csv};  
                \end{axis}
            \end{tikzpicture}
        }
    \end{subfigure}

    \begin{subfigure}[b]{\textwidth}
        \centering
        \resizebox{\linewidth}{!}{
            \begin{tikzpicture}
                \begin{axis}[
                    width=0.98\textwidth,
                    height=6cm,
                    xmin=0.0,xmax=1000,ymin=-3,ymax=3,
                    xlabel={$k$},
                    ylabel={$u_{\mathrm{3}}[k]$},
                    legend pos=south east,
                    y tick label style={/pgf/number format/1000 sep=},
                    ] 
                    \addlegendentry{$u_{\mathrm{3}}[k]$},
                    \addlegendimage{no markers,blue}
                    \addplot[const plot, blue, semithick] file{../data/project/zad4/pid/pid_mieszany_3/input_3.csv};  
                \end{axis}
            \end{tikzpicture}
        }
    \end{subfigure}

    \begin{subfigure}[b]{\textwidth}
        \centering
        \resizebox{\linewidth}{!}{
            \begin{tikzpicture}
                \begin{axis}[
                    width=0.98\textwidth,
                    height=6cm,
                    xmin=0.0,xmax=1000,ymin=-3,ymax=3,
                    xlabel={$k$},
                    ylabel={$u_{\mathrm{4}}[k]$},
                    legend pos=south east,
                    y tick label style={/pgf/number format/1000 sep=},
                    ] 
                    \addlegendentry{$u_{\mathrm{4}}[k]$},
                    \addlegendimage{no markers,blue}
                    \addplot[const plot, blue, semithick] file{../data/project/zad4/pid/pid_mieszany_3/input_4.csv};  
                \end{axis}
            \end{tikzpicture}
        }
    \end{subfigure}

    \caption{Wejścia procesu wielowymiarowego}
    \label{pro_pid_konf3_in3}
\end{figure}
\FloatBarrier