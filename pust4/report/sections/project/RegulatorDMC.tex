\chapter{Wielowymiarowy regulator DMC}
\label{pro_dmc}

Kolejnym etapem projektu było zaimplementowanie oraz dobranie parametrów
wielowymiarowego regulatora DMC dla rozważanego obiektu.


\section{Implementacja wielowymiarowego regulatora DMC}
\label{pro_dmc_implementacja}
Implementacja wielowymiarowego regulatora DMC nie różni się wiele od wersji
jednowymiarowej. Zmienia się tylko postać macierzy potrzebnych do
znalezienia parametrów regulatora. 

\subsection{Model w postaci złożonej odpowiedzi skokowej}
\label{pro_dmc_impl_odp_skok}
W przypadku jednowymiarowego algorytmu DMC, model w postaci odpowiedzi skokowej
był postaci:

\begin{equation*}
    s = s_{\mathrm{1}}, s_{\mathrm{2}}, \hdots, s_{\mathrm{D}}
\end{equation*}

gdzie $s_{i}$ jest skalarną wartością znormalizowanej odpowiedzi skokowej toru
wejście-wyjście.\\

W przypadku wersji wielowymiarowej, $s_{i}$ jest macierzą wartości znormalizowanych
odpowiedzi skokowych wszystkich możliwych torów

\[
s_{i} =
\begin{bmatrix}
    s_{i}^{1,1} & s_{i}^{1,2} & \hdots & s_{i}^{1,n_{u}} \\
    s_{i}^{2,1} & s_{i}^{2,2} & \hdots & \vdots \\
    \vdots & \vdots & \ddots & \vdots \\
    s_{i}^{n_{y},1} & s_{i}^{2,1} & \hdots & s_{i}^{n_{y},n_{u}} 
\end{bmatrix}
\]

gdzie $n_{u}$ to liczba wejść obiektu a $n_{y}$ to liczba wyjść.\\

W przypadku rozważanego obiektu zapisaliśmy odpowiedź skokową jako 
macierz komórkową o $D = \num{400}$ elementach. Macierz komórkowa
to taka specjalna macierz, której elementami mogą być dowolne elementy:
liczby, napisy, obiekty oraz macierze. Tak zapisaną odpowiedź skokową zapisaliśmy
w do pliku \verb+zlozona_odp_skokowa.mat+.

\section{Strojenie wielowymiarowego regulatora DMC metodą eksperymentalną}
\label{pro_dmc_strojenie}


\section{Strojenie wielowymiarowego regulatora DMC metodą optymalizacji parametrów}
\label{pro_dmc_optymalizacja}


\section{Wnioski}
\label{pro_dmc_wnioski}
