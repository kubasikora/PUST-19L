\chapter{Wielowymiarowy regulator DMC}
\label{pro_dmc}

Kolejnym etapem projektu było zaimplementowanie oraz dobranie parametrów
wielowymiarowego regulatora DMC dla rozważanego obiektu.


\section{Implementacja wielowymiarowego regulatora DMC}
\label{pro_dmc_implementacja}
Implementacja wielowymiarowego regulatora DMC nie różni się wiele od wersji
jednowymiarowej. Zmienia się tylko postać macierzy potrzebnych do
znalezienia parametrów regulatora. 

\subsection{Model w postaci złożonej odpowiedzi skokowej}
\label{pro_dmc_impl_odp_skok}
W przypadku jednowymiarowego algorytmu DMC, model w postaci odpowiedzi skokowej
był postaci:

\begin{equation*}
    s = s_{\mathrm{1}}, s_{\mathrm{2}}, \hdots, s_{\mathrm{D}}
\end{equation*}

gdzie $s_{i}$ jest skalarną wartością znormalizowanej odpowiedzi skokowej toru
wejście-wyjście.\\

W przypadku wersji wielowymiarowej, $s_{i}$ jest macierzą wartości znormalizowanych
odpowiedzi skokowych wszystkich możliwych torów

\[
s_{i} =
\begin{bmatrix}
    s_{i}^{1,1} & s_{i}^{1,2} & \hdots & s_{i}^{1,n_{u}} \\
    s_{i}^{2,1} & s_{i}^{2,2} & \hdots & \vdots \\
    \vdots & \vdots & \ddots & \vdots \\
    s_{i}^{n_{y},1} & s_{i}^{2,1} & \hdots & s_{i}^{n_{y},n_{u}} 
\end{bmatrix}
\]

gdzie $n_{u}$ to liczba wejść obiektu a $n_{y}$ to liczba wyjść.\\

W przypadku rozważanego obiektu zapisaliśmy odpowiedź skokową jako 
macierz komórkową o $D$ elementach. Macierz komórkowa
to taka specjalna macierz, której elementami mogą być dowolne elementy:
liczby, napisy, obiekty oraz macierze. Tak zapisaną odpowiedź skokową zapisaliśmy
w do pliku \verb+zlozona_odp_skokowa.mat+.

\subsection{Obliczanie macierzy \bf{K}}
\label{pro_dmc_impl_macierze}

W przypadku wielowymiarowego regulatora DMC, wzór na macierz \textbf{K}
różni się nieznacznie od wersji jednowymiarowej.

\begin{center}
    $\bf{K} = (M^{\mathrm{T}} \Psi M + \Lambda)^{-1} (M^{T} \Psi)\/$
\end{center}

Postać macierzy $\textbf{M}$ oraz $\textbf{M}^{P}$ jest identyczna jak w przypadku
jednowymiarowym, z tą różnicą że teraz każdy element $s_{i}$ nie jest skalarem
a macierzą. 
Macierz $\textbf{M}$ jest rozmiaru $n_{y} N $ wierszy na $n_{u} N_{u}$ kolumn a macierz
$\textbf{M}^{P}$ jest wielkości $n_{y}N$ na $n_{u}(D -1)$.\\

Nowymi symbolami znajdującymi się we wzorze są macierze $\bf{} \Lambda \/$ i $\bf{} \Psi \/$.
Macierz $\bf{} \Lambda \/$ jest macierzą współczynników $\lambda_{i}$, które opisują karę
za zmienność sterowania wejścia $i$-tego. Jest ona postaci:

\[
\bf{} \Lambda \/ =
\begin{bmatrix}
    \lambda_{1} & \hdots & 0 & 0 & 0 & 0 & 0 & 0 & 0\\
    \vdots & \ddots & \vdots & 0 & 0 & 0 & 0 & 0 & 0  \\
    0 & \hdots & \lambda_{n_{u}} & 0 & 0 & 0 & 0 & 0 & 0 \\
    0 & 0 & 0 & \ddots & 0 & 0 & 0 & 0 & 0 \\
    0 & 0 & 0 & 0 & \ddots & 0 & 0 & 0 & 0 \\
    0 & 0 & 0 & 0 & 0 & \ddots & 0 & 0 & 0 \\
    0 & 0 & 0 & 0 & 0 & 0 & \lambda_{1} & 0  & \hdots  \\
    0 & 0 & 0 & 0 & 0 & 0 & \vdots & \ddots & \vdots \\
    0 & 0 & 0 & 0 & 0 & 0 & 0 & \hdots & \lambda_{n_{u}}\\
\end{bmatrix}
\]

Wszystkie elementy nie będące na diagonali są zerowe. Na diagonali znadują się
elementy od $\lambda_{1}$ do $\lambda_{n_{u}}$, powtórzone $n_{u}$ razy. \\

Zupełnie nową jest macierz $\bf{} \Psi \/$, której współczynniki $\mu$
opisują ważność wyjścia. Jej postać jest analogiczna do $\bf{} \Lambda \/$:

\[
\bf{} \Psi \/ =
\begin{bmatrix}
    \mu_{1} & \hdots & 0 & 0 & 0 & 0 & 0 & 0 & 0\\
    \vdots & \ddots & \vdots & 0 & 0 & 0 & 0 & 0 & 0  \\
    0 & \hdots & \mu_{n_{y}} & 0 & 0 & 0 & 0 & 0 & 0 \\
    0 & 0 & 0 & \ddots & 0 & 0 & 0 & 0 & 0 \\
    0 & 0 & 0 & 0 & \ddots & 0 & 0 & 0 & 0 \\
    0 & 0 & 0 & 0 & 0 & \ddots & 0 & 0 & 0 \\
    0 & 0 & 0 & 0 & 0 & 0 & \mu_{1} & 0  & \hdots  \\
    0 & 0 & 0 & 0 & 0 & 0 & \vdots & \ddots & \vdots \\
    0 & 0 & 0 & 0 & 0 & 0 & 0 & \hdots & \mu_{n_{y}}\\
\end{bmatrix}
\]

Na diagonali znajdują się elementy od $\mu_{1}$ do $\mu_{n_{y}}$, powtórzone $n_{y}$
razy.\\

W celu zaimplementowania algorytmu DMC w Matlabie, stworzyliśmy szereg funkcji generujących 
odpowiednie macierze na podstawie złożonej odpowiedzi skokowej. Dzięki temu, obliczenie 
wszystkich potrzebnych macierzy w programie zajęło tylko kilka linijek.
\begin{lstlisting}[style=custommatlab,frame=single,label={pro_pid_parametry},caption={Obliczenie macierzy \textbf{K} regulatora DMC},captionpos=b]
%% Generacja macierzy algorytmu DMC
M = createMMatrix(N, Nu, s);
Mp = createMpMatrix(N, D, s);
Lambda = createLambdaMatrix(Nu, lambda);
Psi = createPsiMatrix(N, mi);
dUp = zeros((D-1)*nu, 1);
    
K = inv( (M') * Psi * M + Lambda )* ( (M') * Psi );
\end{lstlisting}

\subsection{Obliczanie nowego sterowania}

Samo obliczanie nowych przyrostów nie różni się zbytnio od wersji jednowymiarowej. Należy wziąć pod uwagę 
że parametry $k_{\mathrm{e}}$ i $k_{\mathrm{u}}$ obliczane dla wersji oszczędnej mają teraz więcej wymiarów.
$k_{\mathrm{e}}$ jest teraz macierzą $n_{\mathrm{u}}$ na $n_{\mathrm{y}}$ a $k_{\mathrm{u}}$ jest 
macierzą komórkową o długości $D-1$, której elementami są macierze o rozmiarze $k_{\mathrm{u}}$ na 
$k_{\mathrm{u}}$. 
\newpage

\begin{lstlisting}[style=custommatlab,frame=single,label={pro_pid_parametry},caption={Obliczanie nowych sterowań w wielowymiarowym algorytmie DMC},captionpos=b]
% obliczenie nowych sterowan
second_part = 0;
for j=1:(D-1)
    second_part = second_part + ku{j}*dUp(1+(j-1)*nu: (j-1)*nu + nu);  
end
inputs(k, :) = (inputs(k-1, :))' +  ke*((setpoints(k,:))' - (outputs(k,:))')
- second_part; 
\end{lstlisting}

Warto zwrócić uwagę na fakt, że wynik obliczeń jest wektorem a nie skalarem.

\section{Strojenie wielowymiarowego regulatora DMC metodą eksperymentalną}
\label{pro_dmc_strojenie}
Mając zaimplementowany wielowymiarowy algorytm DMC w wersji oszczędnej, przystąpiliśmy
do procesu dobierania parametrów. W pierwszej kolejności dobraliśmy wartość parametru $D$ 
czyli horyzontu dynamiki procesu. Ponieważ odpowiedzi skokowe pojedynczych torów sterowania
miały różne horyzonty dynamiki, zdecydowaliśmy się na wybór najdłuższego z wszystkich.
Dzięki temu ustaliliśmy wartość $D = \num{200}$.\\


\subsection{Strojenie horyzontów predykcji i sterowania}
\label{pro_dmc_strojenie_horyzontow}
Mając wartość horyzontu dynamiki, ustawiliśmy wartości horyzontu predykcji i horyzontu
sterowania na tą samą wartość $D = N = N_{\mathrm{u}} = \num{200}$. Wszystkie parametry
$\lambda_{i}$ oraz $\mu_{j}$ ustawiliśmy na $\num{1}$. \\

Po wykonaniu symulacji otrzymaliśmy następujące wyniki. Wartości poszczególnych
wskaźników jakości wyniosły:

\[
\begin{bmatrix}
    E_{\mathrm{1}} \\
    E_{\mathrm{2}} \\
    E_{\mathrm{3}} 
\end{bmatrix}
= 
\begin{bmatrix}
    \num{3.749} \\
    \num{2.4814} \\
    \num{4.5936}
\end{bmatrix}
\]

Sumaryczny wskaźnik jakości, liczony jako suma wskaźników składowych, 
wyniósł $\bar{E} = \num{10.824}$\\

Wielowymiarowy regulator DMC bez specjalnego strojenia swoich parametrów, doskonale
radzi sobie ze sterowaniem procesu. Na przedstawionych poniżej przebiegach widać
jak regulator szybko nadąża za wartością zadaną. Efekt zakłócania wyjścia w trakcie
zmiany wartości zadanej innego wyjścia, który był bardzo mocno widoczny w przypadku
wielowymiarowego regulatora PID, został bardzo mocno zniwelowany.

%% wyjscia 

\begin{figure}
    \centering
    \begin{subfigure}[b]{\textwidth}
        \centering
        \resizebox{\linewidth}{!}{
            \begin{tikzpicture}
                \begin{axis}[
                    width=0.98\textwidth,
                    height=7cm,
                    xmin=0.0,xmax=1000,ymin=-0.5,ymax=2,
                    xlabel={$k$},
                    ylabel={$y_{\mathrm{1}}[k]$},
                    legend pos=south east,
                    y tick label style={/pgf/number format/1000 sep=},
                    ] 
                    \addlegendentry{$y_{\mathrm{1}}[k]$},
                    \addlegendentry{$y_{\mathrm{1}}^{\mathrm{zad}}[k]$},
                
                    \addlegendimage{no markers,red}
                    \addlegendimage{no markers,blue}
                
                    \addplot[red, semithick] file{../data/project/zad4/dmc/najprostszy_dmc/output_1.csv};  
                    \addplot[blue, semithick] file{../data/project/zad4/dmc/najprostszy_dmc/stpt_1.csv};
                    
                    \end{axis}
            \end{tikzpicture}
        }
        \end{subfigure}

        \begin{subfigure}[b]{\textwidth}
            \centering
            \resizebox{\linewidth}{!}{
                \begin{tikzpicture}
                    \begin{axis}[
                        width=0.98\textwidth,
                        height=7cm,
                        xmin=0.0,xmax=1000,ymin=-0.5,ymax=2,
                        xlabel={$k$},
                        ylabel={$y_{\mathrm{2}}[k]$},
                        legend pos=south east,
                        y tick label style={/pgf/number format/1000 sep=},
                        ] 
                        \addlegendentry{$y_{\mathrm{2}}[k]$},
                        \addlegendentry{$y_{\mathrm{2}}^{\mathrm{zad}}[k]$},
                    
                        \addlegendimage{no markers,red}
                        \addlegendimage{no markers,blue}
                    
                        \addplot[red, semithick] file{../data/project/zad4/dmc/najprostszy_dmc/output_2.csv};  
                        \addplot[blue, semithick] file{../data/project/zad4/dmc/najprostszy_dmc/stpt_2.csv};
                        
                        \end{axis}
                \end{tikzpicture}
            }
        \end{subfigure}

        \begin{subfigure}[b]{\textwidth}
            \centering
            \resizebox{\linewidth}{!}{
                \begin{tikzpicture}
                    \begin{axis}[
                        width=0.98\textwidth,
                        height=7cm,
                        xmin=0.0,xmax=1000,ymin=-0.5,ymax=2,
                        xlabel={$k$},
                        ylabel={$y_{\mathrm{3}}[k]$},
                        legend pos=south east,
                        y tick label style={/pgf/number format/1000 sep=},
                        ] 
                        \addlegendentry{$y_{\mathrm{3}}[k]$},
                        \addlegendentry{$y_{\mathrm{3}}^{\mathrm{zad}}[k]$},
                    
                        \addlegendimage{no markers,red}
                        \addlegendimage{no markers,blue}
                    
                        \addplot[red, semithick] file{../data/project/zad4/dmc/najprostszy_dmc/output_3.csv};  
                        \addplot[blue, semithick] file{../data/project/zad4/dmc/najprostszy_dmc/stpt_3.csv};
                        
                        \end{axis}
                \end{tikzpicture}
            }
        \end{subfigure}
        \caption{Wyjścia procesu wielowymiarowego sterowanego regulatorem DMC o horyzontach $N = \num{200}$ i
        $N_{\mathrm{u}} = \num{200}$}
        \label{pro_dmc_1_out}
\end{figure}

%% wejscia 

\begin{figure}
    \centering
    \begin{subfigure}[b]{\textwidth}
        \centering
        \resizebox{\linewidth}{!}{
            \begin{tikzpicture}
                \begin{axis}[
                    width=0.98\textwidth,
                    height=6cm,
                    xmin=0.0,xmax=1000,ymin=-2,ymax=2,
                    xlabel={$k$},
                    ylabel={$u_{\mathrm{1}}[k]$},
                    legend pos=south east,
                    y tick label style={/pgf/number format/1000 sep=},
                    ] 
                    \addlegendentry{$u_{\mathrm{1}}[k]$},
                    \addlegendimage{no markers,blue}
                    \addplot[const plot, blue, semithick] file{../data/project/zad4/dmc/najprostszy_dmc/input_1.csv};  
                \end{axis}
            \end{tikzpicture}
        }
    \end{subfigure}

    \begin{subfigure}[b]{\textwidth}
        \centering
        \resizebox{\linewidth}{!}{
            \begin{tikzpicture}
                \begin{axis}[
                    width=0.98\textwidth,
                    height=6cm,
                    xmin=0.0,xmax=1000,ymin=-2,ymax=2,
                    xlabel={$k$},
                    ylabel={$u_{\mathrm{2}}[k]$},
                    legend pos=south east,
                    y tick label style={/pgf/number format/1000 sep=},
                    ] 
                    \addlegendentry{$u_{\mathrm{2}}[k]$},
                    \addlegendimage{no markers,blue}
                    \addplot[const plot, blue, semithick] file{../data/project/zad4/dmc/najprostszy_dmc/input_2.csv};  
                \end{axis}
            \end{tikzpicture}
        }
    \end{subfigure}

    \begin{subfigure}[b]{\textwidth}
        \centering
        \resizebox{\linewidth}{!}{
            \begin{tikzpicture}
                \begin{axis}[
                    width=0.98\textwidth,
                    height=6cm,
                    xmin=0.0,xmax=1000,ymin=-2,ymax=2,
                    xlabel={$k$},
                    ylabel={$u_{\mathrm{3}}[k]$},
                    legend pos=south east,
                    y tick label style={/pgf/number format/1000 sep=},
                    ] 
                    \addlegendentry{$u_{\mathrm{3}}[k]$},
                    \addlegendimage{no markers,blue}
                    \addplot[const plot, blue, semithick] file{../data/project/zad4/dmc/najprostszy_dmc/input_3.csv};  
                \end{axis}
            \end{tikzpicture}
        }
    \end{subfigure}

    \begin{subfigure}[b]{\textwidth}
        \centering
        \resizebox{\linewidth}{!}{
            \begin{tikzpicture}
                \begin{axis}[
                    width=0.98\textwidth,
                    height=6cm,
                    xmin=0.0,xmax=1000,ymin=-2,ymax=2,
                    xlabel={$k$},
                    ylabel={$u_{\mathrm{4}}[k]$},
                    legend pos=south east,
                    y tick label style={/pgf/number format/1000 sep=},
                    ] 
                    \addlegendentry{$u_{\mathrm{4}}[k]$},
                    \addlegendimage{no markers,blue}
                    \addplot[const plot, blue, semithick] file{../data/project/zad4/dmc/najprostszy_dmc/input_4.csv};  
                \end{axis}
            \end{tikzpicture}
        }
    \end{subfigure}
    \caption{Wejścia procesu wielowymiarowego sterowane regulatorem DMC o horyzontach $N = \num{200}$ i
    $N_{\mathrm{u}} = \num{200}$}
    \label{pro_dmc_1_in}
\end{figure}
\FloatBarrier


Zmniejszanie wartości horyzontów $N$ i $N_{\mathrm{u}}$ nie przynosiło żadnych dodatkowych
korzyści. Dopiero po przekroczeniu $N = N_{\mathrm{u}} = \num{20}$ sumaryczny wskaźnik jakości
zaczął zwiększać się, dlatego też uznaliśmy że zostaniemy przy $N = \num{20}$ aby zmniejszyć liczbę
obliczeń potrzebnych do generacji nowego sterowania.\\

W następnej kolejności zaczęliśmy zmniejszać sam horyzont sterowania $N_{\mathrm{u}}$.
W pierwszej kolejności sprawdziliśmy jak zachowa się wskaźnik jakości przy zmniejszeniu
horyzontu do $\num{10}$ próbek.

\[
\begin{bmatrix}
    E_{\mathrm{1}} \\
    E_{\mathrm{2}} \\
    E_{\mathrm{3}} 
\end{bmatrix}
= 
\begin{bmatrix}
    \num{3.741} \\
    \num{2.4798} \\
    \num{4.5874}
\end{bmatrix}
\]

Sumaryczny wskaźnik jakości wyniósł $\bar{E} = \num{10.808}$\\

Wskaźniki jakości uległy niewielkiej poprawie. Sama zmiana była niewielka ale może 
wskazywać że przy dalszym zmniejszaniu $N_{\mathrm{u}}$, może dojść do większej
poprawy.

%% wyjscia 

\begin{figure}
    \centering
    \begin{subfigure}[b]{\textwidth}
        \centering
        \resizebox{\linewidth}{!}{
            \begin{tikzpicture}
                \begin{axis}[
                    width=0.98\textwidth,
                    height=7cm,
                    xmin=0.0,xmax=1000,ymin=-0.5,ymax=2,
                    xlabel={$k$},
                    ylabel={$y_{\mathrm{1}}[k]$},
                    legend pos=south east,
                    y tick label style={/pgf/number format/1000 sep=},
                    ] 
                    \addlegendentry{$y_{\mathrm{1}}[k]$},
                    \addlegendentry{$y_{\mathrm{1}}^{\mathrm{zad}}[k]$},
                
                    \addlegendimage{no markers,red}
                    \addlegendimage{no markers,blue}
                
                    \addplot[red, semithick] file{../data/project/zad4/dmc/horyzonty_20_10/output_1.csv};  
                    \addplot[blue, semithick] file{../data/project/zad4/dmc/horyzonty_20_10/stpt_1.csv};
                    
                    \end{axis}
            \end{tikzpicture}
        }
        \end{subfigure}

        \begin{subfigure}[b]{\textwidth}
            \centering
            \resizebox{\linewidth}{!}{
                \begin{tikzpicture}
                    \begin{axis}[
                        width=0.98\textwidth,
                        height=7cm,
                        xmin=0.0,xmax=1000,ymin=-0.5,ymax=2,
                        xlabel={$k$},
                        ylabel={$y_{\mathrm{2}}[k]$},
                        legend pos=south east,
                        y tick label style={/pgf/number format/1000 sep=},
                        ] 
                        \addlegendentry{$y_{\mathrm{2}}[k]$},
                        \addlegendentry{$y_{\mathrm{2}}^{\mathrm{zad}}[k]$},
                    
                        \addlegendimage{no markers,red}
                        \addlegendimage{no markers,blue}
                    
                        \addplot[red, semithick] file{../data/project/zad4/dmc/horyzonty_20_10/output_2.csv};  
                        \addplot[blue, semithick] file{../data/project/zad4/dmc/horyzonty_20_10/stpt_2.csv};
                        
                        \end{axis}
                \end{tikzpicture}
            }
        \end{subfigure}

        \begin{subfigure}[b]{\textwidth}
            \centering
            \resizebox{\linewidth}{!}{
                \begin{tikzpicture}
                    \begin{axis}[
                        width=0.98\textwidth,
                        height=7cm,
                        xmin=0.0,xmax=1000,ymin=-0.5,ymax=2,
                        xlabel={$k$},
                        ylabel={$y_{\mathrm{3}}[k]$},
                        legend pos=south east,
                        y tick label style={/pgf/number format/1000 sep=},
                        ] 
                        \addlegendentry{$y_{\mathrm{3}}[k]$},
                        \addlegendentry{$y_{\mathrm{3}}^{\mathrm{zad}}[k]$},
                    
                        \addlegendimage{no markers,red}
                        \addlegendimage{no markers,blue}
                    
                        \addplot[red, semithick] file{../data/project/zad4/dmc/horyzonty_20_10/output_3.csv};  
                        \addplot[blue, semithick] file{../data/project/zad4/dmc/horyzonty_20_10/stpt_3.csv};
                        
                        \end{axis}
                \end{tikzpicture}
            }
        \end{subfigure}
        \caption{Wyjścia procesu wielowymiarowego sterowanego regulatorem DMC o horyzontach $N = \num{20}$ i
        $N_{\mathrm{u}} = \num{10}$}
        \label{pro_dmc_2_out}
\end{figure}

%% wejscia 

\begin{figure}
    \centering
    \begin{subfigure}[b]{\textwidth}
        \centering
        \resizebox{\linewidth}{!}{
            \begin{tikzpicture}
                \begin{axis}[
                    width=0.98\textwidth,
                    height=6cm,
                    xmin=0.0,xmax=1000,ymin=-2,ymax=2,
                    xlabel={$k$},
                    ylabel={$u_{\mathrm{1}}[k]$},
                    legend pos=south east,
                    y tick label style={/pgf/number format/1000 sep=},
                    ] 
                    \addlegendentry{$u_{\mathrm{1}}[k]$},
                    \addlegendimage{no markers,blue}
                    \addplot[const plot, blue, semithick] file{../data/project/zad4/dmc/horyzonty_20_10/input_1.csv};  
                \end{axis}
            \end{tikzpicture}
        }
    \end{subfigure}

    \begin{subfigure}[b]{\textwidth}
        \centering
        \resizebox{\linewidth}{!}{
            \begin{tikzpicture}
                \begin{axis}[
                    width=0.98\textwidth,
                    height=6cm,
                    xmin=0.0,xmax=1000,ymin=-2,ymax=2,
                    xlabel={$k$},
                    ylabel={$u_{\mathrm{2}}[k]$},
                    legend pos=south east,
                    y tick label style={/pgf/number format/1000 sep=},
                    ] 
                    \addlegendentry{$u_{\mathrm{2}}[k]$},
                    \addlegendimage{no markers,blue}
                    \addplot[const plot, blue, semithick] file{../data/project/zad4/dmc/horyzonty_20_10/input_2.csv};  
                \end{axis}
            \end{tikzpicture}
        }
    \end{subfigure}

    \begin{subfigure}[b]{\textwidth}
        \centering
        \resizebox{\linewidth}{!}{
            \begin{tikzpicture}
                \begin{axis}[
                    width=0.98\textwidth,
                    height=6cm,
                    xmin=0.0,xmax=1000,ymin=-2,ymax=2,
                    xlabel={$k$},
                    ylabel={$u_{\mathrm{3}}[k]$},
                    legend pos=south east,
                    y tick label style={/pgf/number format/1000 sep=},
                    ] 
                    \addlegendentry{$u_{\mathrm{3}}[k]$},
                    \addlegendimage{no markers,blue}
                    \addplot[const plot, blue, semithick] file{../data/project/zad4/dmc/horyzonty_20_10/input_3.csv};  
                \end{axis}
            \end{tikzpicture}
        }
    \end{subfigure}

    \begin{subfigure}[b]{\textwidth}
        \centering
        \resizebox{\linewidth}{!}{
            \begin{tikzpicture}
                \begin{axis}[
                    width=0.98\textwidth,
                    height=6cm,
                    xmin=0.0,xmax=1000,ymin=-2,ymax=2,
                    xlabel={$k$},
                    ylabel={$u_{\mathrm{4}}[k]$},
                    legend pos=south east,
                    y tick label style={/pgf/number format/1000 sep=},
                    ] 
                    \addlegendentry{$u_{\mathrm{4}}[k]$},
                    \addlegendimage{no markers,blue}
                    \addplot[const plot, blue, semithick] file{../data/project/zad4/dmc/horyzonty_20_10/input_4.csv};  
                \end{axis}
            \end{tikzpicture}
        }
    \end{subfigure}
    \caption{Wejścia procesu wielowymiarowego sterowanego regulatorem DMC o horyzontach $N = \num{20}$ i
    $N_{\mathrm{u}} = \num{10}$}
    \label{pro_dmc_2_in}
\end{figure}
\FloatBarrier

Najlepsze rezulaty osiągneliśmy dla parametrów: $N = \num{20}$ i
$N_{\mathrm{u}} = \num{5}$. W przypadku dalszego zmniejszania horyzontu sterowania, sumaryczny błąd
zaczynał rosnąć, co jest efektem niepożądanym. Dla najlepszej konfiguracji horyzontów uzyskaliśmy 
następujące wskaźniki jakości:

\[
\begin{bmatrix}
    E_{\mathrm{1}} \\
    E_{\mathrm{2}} \\
    E_{\mathrm{3}} 
\end{bmatrix}
= 
\begin{bmatrix}
    \num{3.7734} \\
    \num{2.4821} \\
    \num{4.4375}
\end{bmatrix}
\]

Sumaryczny wskaźnik jakości 
wyniósł $\bar{E} = \num{10.693}$\\

%% wyjscia 

\begin{figure}
    \centering
    \begin{subfigure}[b]{\textwidth}
        \centering
        \resizebox{\linewidth}{!}{
            \begin{tikzpicture}
                \begin{axis}[
                    width=0.98\textwidth,
                    height=7cm,
                    xmin=0.0,xmax=1000,ymin=-0.5,ymax=2,
                    xlabel={$k$},
                    ylabel={$y_{\mathrm{1}}[k]$},
                    legend pos=south east,
                    y tick label style={/pgf/number format/1000 sep=},
                    ] 
                    \addlegendentry{$y_{\mathrm{1}}[k]$},
                    \addlegendentry{$y_{\mathrm{1}}^{\mathrm{zad}}[k]$},
                
                    \addlegendimage{no markers,red}
                    \addlegendimage{no markers,blue}
                
                    \addplot[red, semithick] file{../data/project/zad4/dmc/horyzonty_20_5/output_1.csv};  
                    \addplot[blue, semithick] file{../data/project/zad4/dmc/horyzonty_20_5/stpt_1.csv};
                    
                    \end{axis}
            \end{tikzpicture}
        }
        \end{subfigure}

        \begin{subfigure}[b]{\textwidth}
            \centering
            \resizebox{\linewidth}{!}{
                \begin{tikzpicture}
                    \begin{axis}[
                        width=0.98\textwidth,
                        height=7cm,
                        xmin=0.0,xmax=1000,ymin=-0.5,ymax=2,
                        xlabel={$k$},
                        ylabel={$y_{\mathrm{2}}[k]$},
                        legend pos=south east,
                        y tick label style={/pgf/number format/1000 sep=},
                        ] 
                        \addlegendentry{$y_{\mathrm{2}}[k]$},
                        \addlegendentry{$y_{\mathrm{2}}^{\mathrm{zad}}[k]$},
                    
                        \addlegendimage{no markers,red}
                        \addlegendimage{no markers,blue}
                    
                        \addplot[red, semithick] file{../data/project/zad4/dmc/horyzonty_20_5/output_2.csv};  
                        \addplot[blue, semithick] file{../data/project/zad4/dmc/horyzonty_20_5/stpt_2.csv};
                        
                        \end{axis}
                \end{tikzpicture}
            }
        \end{subfigure}

        \begin{subfigure}[b]{\textwidth}
            \centering
            \resizebox{\linewidth}{!}{
                \begin{tikzpicture}
                    \begin{axis}[
                        width=0.98\textwidth,
                        height=7cm,
                        xmin=0.0,xmax=1000,ymin=-0.5,ymax=2,
                        xlabel={$k$},
                        ylabel={$y_{\mathrm{3}}[k]$},
                        legend pos=south east,
                        y tick label style={/pgf/number format/1000 sep=},
                        ] 
                        \addlegendentry{$y_{\mathrm{3}}[k]$},
                        \addlegendentry{$y_{\mathrm{3}}^{\mathrm{zad}}[k]$},
                    
                        \addlegendimage{no markers,red}
                        \addlegendimage{no markers,blue}
                    
                        \addplot[red, semithick] file{../data/project/zad4/dmc/horyzonty_20_5/output_3.csv};  
                        \addplot[blue, semithick] file{../data/project/zad4/dmc/horyzonty_20_5/stpt_3.csv};
                        
                        \end{axis}
                \end{tikzpicture}
            }
        \end{subfigure}
        \caption{Wyjścia procesu wielowymiarowego sterowanego regulatorem DMC o horyzontach $N = \num{20}$ i
        $N_{\mathrm{u}} = \num{5}$}
        \label{pro_dmc_2_out}
\end{figure}

%% wejscia 

\begin{figure}
    \centering
    \begin{subfigure}[b]{\textwidth}
        \centering
        \resizebox{\linewidth}{!}{
            \begin{tikzpicture}
                \begin{axis}[
                    width=0.98\textwidth,
                    height=6cm,
                    xmin=0.0,xmax=1000,ymin=-2,ymax=2,
                    xlabel={$k$},
                    ylabel={$u_{\mathrm{1}}[k]$},
                    legend pos=south east,
                    y tick label style={/pgf/number format/1000 sep=},
                    ] 
                    \addlegendentry{$u_{\mathrm{1}}[k]$},
                    \addlegendimage{no markers,blue}
                    \addplot[const plot, blue, semithick] file{../data/project/zad4/dmc/horyzonty_20_5/input_1.csv};  
                \end{axis}
            \end{tikzpicture}
        }
    \end{subfigure}

    \begin{subfigure}[b]{\textwidth}
        \centering
        \resizebox{\linewidth}{!}{
            \begin{tikzpicture}
                \begin{axis}[
                    width=0.98\textwidth,
                    height=6cm,
                    xmin=0.0,xmax=1000,ymin=-2,ymax=2,
                    xlabel={$k$},
                    ylabel={$u_{\mathrm{2}}[k]$},
                    legend pos=south east,
                    y tick label style={/pgf/number format/1000 sep=},
                    ] 
                    \addlegendentry{$u_{\mathrm{2}}[k]$},
                    \addlegendimage{no markers,blue}
                    \addplot[const plot, blue, semithick] file{../data/project/zad4/dmc/horyzonty_20_5/input_2.csv};  
                \end{axis}
            \end{tikzpicture}
        }
    \end{subfigure}

    \begin{subfigure}[b]{\textwidth}
        \centering
        \resizebox{\linewidth}{!}{
            \begin{tikzpicture}
                \begin{axis}[
                    width=0.98\textwidth,
                    height=6cm,
                    xmin=0.0,xmax=1000,ymin=-2,ymax=2,
                    xlabel={$k$},
                    ylabel={$u_{\mathrm{3}}[k]$},
                    legend pos=south east,
                    y tick label style={/pgf/number format/1000 sep=},
                    ] 
                    \addlegendentry{$u_{\mathrm{3}}[k]$},
                    \addlegendimage{no markers,blue}
                    \addplot[const plot, blue, semithick] file{../data/project/zad4/dmc/horyzonty_20_5/input_3.csv};  
                \end{axis}
            \end{tikzpicture}
        }
    \end{subfigure}

    \begin{subfigure}[b]{\textwidth}
        \centering
        \resizebox{\linewidth}{!}{
            \begin{tikzpicture}
                \begin{axis}[
                    width=0.98\textwidth,
                    height=6cm,
                    xmin=0.0,xmax=1000,ymin=-2,ymax=2,
                    xlabel={$k$},
                    ylabel={$u_{\mathrm{4}}[k]$},
                    legend pos=south east,
                    y tick label style={/pgf/number format/1000 sep=},
                    ] 
                    \addlegendentry{$u_{\mathrm{4}}[k]$},
                    \addlegendimage{no markers,blue}
                    \addplot[const plot, blue, semithick] file{../data/project/zad4/dmc/horyzonty_20_5/input_4.csv};  
                \end{axis}
            \end{tikzpicture}
        }
    \end{subfigure}
    \caption{Wejścia procesu wielowymiarowego sterowanego regulatorem DMC o horyzontach $N = \num{20}$ i
    $N_{\mathrm{u}} = \num{5}$}
    \label{pro_dmc_2_in}
\end{figure}
\FloatBarrier


\subsection{Dobór parametrów $\mu$ i $\lambda$}
\label{pro_dmc_strojenie_mu_lambda}

Po dobraniu długości horyzontów, rozpoczęliśmy dobieranie parametrów $\mu$ i $\lambda$.
W pierwszej kolejności postanowiliśmy spróbowaliśmy dobrać wartości parametrów $\mu$. Uznaliśmy że 
zwiększamy wartość parametru $\mu_{3}$, ponieważ to właśnie błąd trzeciego wyjścia jest największy.
Rzeczywiście, zwiększenie tego parametru do $\mu_{3} = \num{5}$ pomogło zmniejszyć błąd trzeciego wyjścia oraz
sumaryczny błąd wszystkich wyjść, powodując jedynie nieznaczy wzrost błędów pozostałych wyjść.


\[
\begin{bmatrix}
    E_{\mathrm{1}} \\
    E_{\mathrm{2}} \\
    E_{\mathrm{3}} 
\end{bmatrix}
= 
\begin{bmatrix}
    \num{3.8972} \\
    \num{2.4908} \\
    \num{3.2131}
\end{bmatrix}
\]

Sumaryczny błąd wyniósł $\num{9.0068}$.\\

Zwiększenie parametru $\mu_{3}$ pomogło zmniejszyć dysproporcję pomiędzy błędami poszczególnych wyjść.
W kolejnych iteracjach staraliśmy się tak dobrać pozostałe parametry $\mu$ aby wyrównać błędy
wszystkich wyjść.

%% wyjscia 

\begin{figure}
    \centering
    \begin{subfigure}[b]{\textwidth}
        \centering
        \resizebox{\linewidth}{!}{
            \begin{tikzpicture}
                \begin{axis}[
                    width=0.98\textwidth,
                    height=7cm,
                    xmin=0.0,xmax=1000,ymin=-0.5,ymax=2,
                    xlabel={$k$},
                    ylabel={$y_{\mathrm{1}}[k]$},
                    legend pos=south east,
                    y tick label style={/pgf/number format/1000 sep=},
                    ] 
                    \addlegendentry{$y_{\mathrm{1}}[k]$},
                    \addlegendentry{$y_{\mathrm{1}}^{\mathrm{zad}}[k]$},
                
                    \addlegendimage{no markers,red}
                    \addlegendimage{no markers,blue}
                
                    \addplot[red, semithick] file{../data/project/zad4/dmc/mu3_na_5/output_1.csv};  
                    \addplot[blue, semithick] file{../data/project/zad4/dmc/mu3_na_5/stpt_1.csv};
                    
                    \end{axis}
            \end{tikzpicture}
        }
        \end{subfigure}

        \begin{subfigure}[b]{\textwidth}
            \centering
            \resizebox{\linewidth}{!}{
                \begin{tikzpicture}
                    \begin{axis}[
                        width=0.98\textwidth,
                        height=7cm,
                        xmin=0.0,xmax=1000,ymin=-0.5,ymax=2,
                        xlabel={$k$},
                        ylabel={$y_{\mathrm{2}}[k]$},
                        legend pos=south east,
                        y tick label style={/pgf/number format/1000 sep=},
                        ] 
                        \addlegendentry{$y_{\mathrm{2}}[k]$},
                        \addlegendentry{$y_{\mathrm{2}}^{\mathrm{zad}}[k]$},
                    
                        \addlegendimage{no markers,red}
                        \addlegendimage{no markers,blue}
                    
                        \addplot[red, semithick] file{../data/project/zad4/dmc/mu3_na_5/output_2.csv};  
                        \addplot[blue, semithick] file{../data/project/zad4/dmc/mu3_na_5/stpt_2.csv};
                        
                        \end{axis}
                \end{tikzpicture}
            }
        \end{subfigure}

        \begin{subfigure}[b]{\textwidth}
            \centering
            \resizebox{\linewidth}{!}{
                \begin{tikzpicture}
                    \begin{axis}[
                        width=0.98\textwidth,
                        height=7cm,
                        xmin=0.0,xmax=1000,ymin=-0.5,ymax=2,
                        xlabel={$k$},
                        ylabel={$y_{\mathrm{3}}[k]$},
                        legend pos=south east,
                        y tick label style={/pgf/number format/1000 sep=},
                        ] 
                        \addlegendentry{$y_{\mathrm{3}}[k]$},
                        \addlegendentry{$y_{\mathrm{3}}^{\mathrm{zad}}[k]$},
                    
                        \addlegendimage{no markers,red}
                        \addlegendimage{no markers,blue}
                    
                        \addplot[red, semithick] file{../data/project/zad4/dmc/mu3_na_5/output_3.csv};  
                        \addplot[blue, semithick] file{../data/project/zad4/dmc/mu3_na_5/stpt_3.csv};
                        
                        \end{axis}
                \end{tikzpicture}
            }
        \end{subfigure}
        \caption{Wyjścia procesu wielowymiarowego sterowanego regulatorem DMC}
        \label{pro_dmc_2_out}
\end{figure}

%% wejscia 

\begin{figure}
    \centering
    \begin{subfigure}[b]{\textwidth}
        \centering
        \resizebox{\linewidth}{!}{
            \begin{tikzpicture}
                \begin{axis}[
                    width=0.98\textwidth,
                    height=6cm,
                    xmin=0.0,xmax=1000,ymin=-2,ymax=2,
                    xlabel={$k$},
                    ylabel={$u_{\mathrm{1}}[k]$},
                    legend pos=south east,
                    y tick label style={/pgf/number format/1000 sep=},
                    ] 
                    \addlegendentry{$u_{\mathrm{1}}[k]$},
                    \addlegendimage{no markers,blue}
                    \addplot[const plot, blue, semithick] file{../data/project/zad4/dmc/mu3_na_5/input_1.csv};  
                \end{axis}
            \end{tikzpicture}
        }
    \end{subfigure}

    \begin{subfigure}[b]{\textwidth}
        \centering
        \resizebox{\linewidth}{!}{
            \begin{tikzpicture}
                \begin{axis}[
                    width=0.98\textwidth,
                    height=6cm,
                    xmin=0.0,xmax=1000,ymin=-2,ymax=2,
                    xlabel={$k$},
                    ylabel={$u_{\mathrm{2}}[k]$},
                    legend pos=south east,
                    y tick label style={/pgf/number format/1000 sep=},
                    ] 
                    \addlegendentry{$u_{\mathrm{2}}[k]$},
                    \addlegendimage{no markers,blue}
                    \addplot[const plot, blue, semithick] file{../data/project/zad4/dmc/mu3_na_5/input_2.csv};  
                \end{axis}
            \end{tikzpicture}
        }
    \end{subfigure}

    \begin{subfigure}[b]{\textwidth}
        \centering
        \resizebox{\linewidth}{!}{
            \begin{tikzpicture}
                \begin{axis}[
                    width=0.98\textwidth,
                    height=6cm,
                    xmin=0.0,xmax=1000,ymin=-2,ymax=2,
                    xlabel={$k$},
                    ylabel={$u_{\mathrm{3}}[k]$},
                    legend pos=south east,
                    y tick label style={/pgf/number format/1000 sep=},
                    ] 
                    \addlegendentry{$u_{\mathrm{3}}[k]$},
                    \addlegendimage{no markers,blue}
                    \addplot[const plot, blue, semithick] file{../data/project/zad4/dmc/mu3_na_5/input_3.csv};  
                \end{axis}
            \end{tikzpicture}
        }
    \end{subfigure}

    \begin{subfigure}[b]{\textwidth}
        \centering
        \resizebox{\linewidth}{!}{
            \begin{tikzpicture}
                \begin{axis}[
                    width=0.98\textwidth,
                    height=6cm,
                    xmin=0.0,xmax=1000,ymin=-2,ymax=2,
                    xlabel={$k$},
                    ylabel={$u_{\mathrm{4}}[k]$},
                    legend pos=south east,
                    y tick label style={/pgf/number format/1000 sep=},
                    ] 
                    \addlegendentry{$u_{\mathrm{4}}[k]$},
                    \addlegendimage{no markers,blue}
                    \addplot[const plot, blue, semithick] file{../data/project/zad4/dmc/mu3_na_5/input_4.csv};  
                \end{axis}
            \end{tikzpicture}
        }
    \end{subfigure}
    \caption{Wejścia procesu wielowymiarowego sterowanego regulatorem DMC}
    \label{pro_dmc_2_in}
\end{figure}
\FloatBarrier

Po kilkunastu próbach dobrania wektora parametrów $\mu$, metodą prób i błędów uzyskaliśmy następujący zestaw parametrów

\begin{center}
    $\mu_{1} = \num{3.1}$, $\mu_{2} = \num{1.2}$, $\mu_{3} = \num{11.8}$
\end{center}

Dla tego zestawu parametrów, udało nam się uzyskać następujące wskaźniki jakości:


\[
\begin{bmatrix}
    E_{\mathrm{1}} \\
    E_{\mathrm{2}} \\
    E_{\mathrm{3}} 
\end{bmatrix}
= 
\begin{bmatrix}
    \num{2.8373} \\
    \num{2.7830} \\
    \num{2.8297}
\end{bmatrix}
\]

Sumaryczny wskaźnik jakości wyniósł $\num{8.4501}$.\\

%% wyjscia 

\begin{figure}
    \centering
    \begin{subfigure}[b]{\textwidth}
        \centering
        \resizebox{\linewidth}{!}{
            \begin{tikzpicture}
                \begin{axis}[
                    width=0.98\textwidth,
                    height=7cm,
                    xmin=0.0,xmax=1000,ymin=-0.5,ymax=2,
                    xlabel={$k$},
                    ylabel={$y_{\mathrm{1}}[k]$},
                    legend pos=south east,
                    y tick label style={/pgf/number format/1000 sep=},
                    ] 
                    \addlegendentry{$y_{\mathrm{1}}[k]$},
                    \addlegendentry{$y_{\mathrm{1}}^{\mathrm{zad}}[k]$},
                
                    \addlegendimage{no markers,red}
                    \addlegendimage{no markers,blue}
                
                    \addplot[red, semithick] file{../data/project/zad4/dmc/najlepsze_mu/output_1.csv};  
                    \addplot[blue, semithick] file{../data/project/zad4/dmc/najlepsze_mu/stpt_1.csv};
                    
                    \end{axis}
            \end{tikzpicture}
        }
        \end{subfigure}

        \begin{subfigure}[b]{\textwidth}
            \centering
            \resizebox{\linewidth}{!}{
                \begin{tikzpicture}
                    \begin{axis}[
                        width=0.98\textwidth,
                        height=7cm,
                        xmin=0.0,xmax=1000,ymin=-0.5,ymax=2,
                        xlabel={$k$},
                        ylabel={$y_{\mathrm{2}}[k]$},
                        legend pos=south east,
                        y tick label style={/pgf/number format/1000 sep=},
                        ] 
                        \addlegendentry{$y_{\mathrm{2}}[k]$},
                        \addlegendentry{$y_{\mathrm{2}}^{\mathrm{zad}}[k]$},
                    
                        \addlegendimage{no markers,red}
                        \addlegendimage{no markers,blue}
                    
                        \addplot[red, semithick] file{../data/project/zad4/dmc/najlepsze_mu/output_2.csv};  
                        \addplot[blue, semithick] file{../data/project/zad4/dmc/najlepsze_mu/stpt_2.csv};
                        
                        \end{axis}
                \end{tikzpicture}
            }
        \end{subfigure}

        \begin{subfigure}[b]{\textwidth}
            \centering
            \resizebox{\linewidth}{!}{
                \begin{tikzpicture}
                    \begin{axis}[
                        width=0.98\textwidth,
                        height=7cm,
                        xmin=0.0,xmax=1000,ymin=-0.5,ymax=2,
                        xlabel={$k$},
                        ylabel={$y_{\mathrm{3}}[k]$},
                        legend pos=south east,
                        y tick label style={/pgf/number format/1000 sep=},
                        ] 
                        \addlegendentry{$y_{\mathrm{3}}[k]$},
                        \addlegendentry{$y_{\mathrm{3}}^{\mathrm{zad}}[k]$},
                    
                        \addlegendimage{no markers,red}
                        \addlegendimage{no markers,blue}
                    
                        \addplot[red, semithick] file{../data/project/zad4/dmc/najlepsze_mu/output_3.csv};  
                        \addplot[blue, semithick] file{../data/project/zad4/dmc/najlepsze_mu/stpt_3.csv};
                        
                        \end{axis}
                \end{tikzpicture}
            }
        \end{subfigure}
        \caption{Wyjścia procesu wielowymiarowego sterowanego regulatorem DMC}
        \label{pro_dmc_2_out}
\end{figure}

%% wejscia 

\begin{figure}
    \centering
    \begin{subfigure}[b]{\textwidth}
        \centering
        \resizebox{\linewidth}{!}{
            \begin{tikzpicture}
                \begin{axis}[
                    width=0.98\textwidth,
                    height=6cm,
                    xmin=0.0,xmax=1000,ymin=-2,ymax=2,
                    xlabel={$k$},
                    ylabel={$u_{\mathrm{1}}[k]$},
                    legend pos=south east,
                    y tick label style={/pgf/number format/1000 sep=},
                    ] 
                    \addlegendentry{$u_{\mathrm{1}}[k]$},
                    \addlegendimage{no markers,blue}
                    \addplot[const plot, blue, semithick] file{../data/project/zad4/dmc/najlepsze_mu/input_1.csv};  
                \end{axis}
            \end{tikzpicture}
        }
    \end{subfigure}

    \begin{subfigure}[b]{\textwidth}
        \centering
        \resizebox{\linewidth}{!}{
            \begin{tikzpicture}
                \begin{axis}[
                    width=0.98\textwidth,
                    height=6cm,
                    xmin=0.0,xmax=1000,ymin=-2,ymax=2,
                    xlabel={$k$},
                    ylabel={$u_{\mathrm{2}}[k]$},
                    legend pos=south east,
                    y tick label style={/pgf/number format/1000 sep=},
                    ] 
                    \addlegendentry{$u_{\mathrm{2}}[k]$},
                    \addlegendimage{no markers,blue}
                    \addplot[const plot, blue, semithick] file{../data/project/zad4/dmc/najlepsze_mu/input_2.csv};  
                \end{axis}
            \end{tikzpicture}
        }
    \end{subfigure}

    \begin{subfigure}[b]{\textwidth}
        \centering
        \resizebox{\linewidth}{!}{
            \begin{tikzpicture}
                \begin{axis}[
                    width=0.98\textwidth,
                    height=6cm,
                    xmin=0.0,xmax=1000,ymin=-2,ymax=2,
                    xlabel={$k$},
                    ylabel={$u_{\mathrm{3}}[k]$},
                    legend pos=south east,
                    y tick label style={/pgf/number format/1000 sep=},
                    ] 
                    \addlegendentry{$u_{\mathrm{3}}[k]$},
                    \addlegendimage{no markers,blue}
                    \addplot[const plot, blue, semithick] file{../data/project/zad4/dmc/najlepsze_mu/input_3.csv};  
                \end{axis}
            \end{tikzpicture}
        }
    \end{subfigure}

    \begin{subfigure}[b]{\textwidth}
        \centering
        \resizebox{\linewidth}{!}{
            \begin{tikzpicture}
                \begin{axis}[
                    width=0.98\textwidth,
                    height=6cm,
                    xmin=0.0,xmax=1000,ymin=-2,ymax=2,
                    xlabel={$k$},
                    ylabel={$u_{\mathrm{4}}[k]$},
                    legend pos=south east,
                    y tick label style={/pgf/number format/1000 sep=},
                    ] 
                    \addlegendentry{$u_{\mathrm{4}}[k]$},
                    \addlegendimage{no markers,blue}
                    \addplot[const plot, blue, semithick] file{../data/project/zad4/dmc/najlepsze_mu/input_4.csv};  
                \end{axis}
            \end{tikzpicture}
        }
    \end{subfigure}
    \caption{Wejścia procesu wielowymiarowego sterowanego regulatorem DMC}
    \label{pro_dmc_2_in}
\end{figure}
\FloatBarrier

Przebiegi wyjściowe oraz wynikające z nich błędy zostały w naszym mniemaniu,
doprowadzone do stanu niezwykle zadowalającego. Niestety, tak dobre przebiegi wyjściowe 
musiały powstać kosztem przebiegów wejściowych. Szczególnie wejście pierwsze i drugie cechują się
zbyt dużą szybkością zmian sterowania. Dodatkowo pierwsze wejście przy skokach wartości zadanej
zaczyna lekko oscylować. Dlatego też w ostatnim kroku dokonaliśmy lekkich modyfikacji 
parametrów $\lambda$. Zdecydowaliśmy się na zwiększenie wartości parametrów $\lambda_{1}$ oraz $\lambda_{2}$.

Ostateczny zestaw wszystkich parametrów regulatora DMC wygląda następująco:


\begin{center}
    $D = \num{200}$, $N = \num{20}$, $N_{\mathrm{u}} = \num{5}$,
    $\mu_{1} = \num{3.1}$, $\mu_{2} = \num{1.2}$, $\mu_{3} = \num{11.8}$,
    $\lambda{1} = \num{10}$, $\lambda{2} = \num{5}$, $\lambda{3} = \num{1}$, $\lambda{4} = \num{1}$
\end{center}

Dla tego zestawu parametrów uzyskaliśmy następujące wskaźniki jakości:

\[
\begin{bmatrix}
    E_{\mathrm{1}} \\
    E_{\mathrm{2}} \\
    E_{\mathrm{3}} 
\end{bmatrix}
= 
\begin{bmatrix}
    \num{3.5663} \\
    \num{2.8869} \\
    \num{3.3173}
\end{bmatrix}
\]

Sumaryczny wskaźnik jakości wyniósł $\num{9.7705}$.\\

Wartości wskaźników jakości zwiększyły się nieznacznie, jednak przebiegi omawianych sygnałów 
wejściowych wygładziły się. W naszym uznaniu jest to dobry kompromis pomiędzy jakością regulacji a 
postacią przebiegów sterowań. 

%% wyjscia 

\begin{figure}
    \centering
    \begin{subfigure}[b]{\textwidth}
        \centering
        \resizebox{\linewidth}{!}{
            \begin{tikzpicture}
                \begin{axis}[
                    width=0.98\textwidth,
                    height=7cm,
                    xmin=0.0,xmax=1000,ymin=-0.5,ymax=2,
                    xlabel={$k$},
                    ylabel={$y_{\mathrm{1}}[k]$},
                    legend pos=south east,
                    y tick label style={/pgf/number format/1000 sep=},
                    ] 
                    \addlegendentry{$y_{\mathrm{1}}[k]$},
                    \addlegendentry{$y_{\mathrm{1}}^{\mathrm{zad}}[k]$},
                
                    \addlegendimage{no markers,red}
                    \addlegendimage{no markers,blue}
                
                    \addplot[red, semithick] file{../data/project/zad4/dmc/ostateczny/output_1.csv};  
                    \addplot[blue, semithick] file{../data/project/zad4/dmc/ostateczny/stpt_1.csv};
                    
                    \end{axis}
            \end{tikzpicture}
        }
        \end{subfigure}

        \begin{subfigure}[b]{\textwidth}
            \centering
            \resizebox{\linewidth}{!}{
                \begin{tikzpicture}
                    \begin{axis}[
                        width=0.98\textwidth,
                        height=7cm,
                        xmin=0.0,xmax=1000,ymin=-0.5,ymax=2,
                        xlabel={$k$},
                        ylabel={$y_{\mathrm{2}}[k]$},
                        legend pos=south east,
                        y tick label style={/pgf/number format/1000 sep=},
                        ] 
                        \addlegendentry{$y_{\mathrm{2}}[k]$},
                        \addlegendentry{$y_{\mathrm{2}}^{\mathrm{zad}}[k]$},
                    
                        \addlegendimage{no markers,red}
                        \addlegendimage{no markers,blue}
                    
                        \addplot[red, semithick] file{../data/project/zad4/dmc/ostateczny/output_2.csv};  
                        \addplot[blue, semithick] file{../data/project/zad4/dmc/ostateczny/stpt_2.csv};
                        
                        \end{axis}
                \end{tikzpicture}
            }
        \end{subfigure}

        \begin{subfigure}[b]{\textwidth}
            \centering
            \resizebox{\linewidth}{!}{
                \begin{tikzpicture}
                    \begin{axis}[
                        width=0.98\textwidth,
                        height=7cm,
                        xmin=0.0,xmax=1000,ymin=-0.5,ymax=2,
                        xlabel={$k$},
                        ylabel={$y_{\mathrm{3}}[k]$},
                        legend pos=south east,
                        y tick label style={/pgf/number format/1000 sep=},
                        ] 
                        \addlegendentry{$y_{\mathrm{3}}[k]$},
                        \addlegendentry{$y_{\mathrm{3}}^{\mathrm{zad}}[k]$},
                    
                        \addlegendimage{no markers,red}
                        \addlegendimage{no markers,blue}
                    
                        \addplot[red, semithick] file{../data/project/zad4/dmc/ostateczny/output_3.csv};  
                        \addplot[blue, semithick] file{../data/project/zad4/dmc/ostateczny/stpt_3.csv};
                        
                        \end{axis}
                \end{tikzpicture}
            }
        \end{subfigure}
        \caption{Wyjścia procesu wielowymiarowego sterowanego regulatorem DMC}
        \label{pro_dmc_2_out}
\end{figure}

%% wejscia 

\begin{figure}
    \centering
    \begin{subfigure}[b]{\textwidth}
        \centering
        \resizebox{\linewidth}{!}{
            \begin{tikzpicture}
                \begin{axis}[
                    width=0.98\textwidth,
                    height=6cm,
                    xmin=0.0,xmax=1000,ymin=-2,ymax=2,
                    xlabel={$k$},
                    ylabel={$u_{\mathrm{1}}[k]$},
                    legend pos=south east,
                    y tick label style={/pgf/number format/1000 sep=},
                    ] 
                    \addlegendentry{$u_{\mathrm{1}}[k]$},
                    \addlegendimage{no markers,blue}
                    \addplot[const plot, blue, semithick] file{../data/project/zad4/dmc/ostateczny/input_1.csv};  
                \end{axis}
            \end{tikzpicture}
        }
    \end{subfigure}

    \begin{subfigure}[b]{\textwidth}
        \centering
        \resizebox{\linewidth}{!}{
            \begin{tikzpicture}
                \begin{axis}[
                    width=0.98\textwidth,
                    height=6cm,
                    xmin=0.0,xmax=1000,ymin=-2,ymax=2,
                    xlabel={$k$},
                    ylabel={$u_{\mathrm{2}}[k]$},
                    legend pos=south east,
                    y tick label style={/pgf/number format/1000 sep=},
                    ] 
                    \addlegendentry{$u_{\mathrm{2}}[k]$},
                    \addlegendimage{no markers,blue}
                    \addplot[const plot, blue, semithick] file{../data/project/zad4/dmc/ostateczny/input_2.csv};  
                \end{axis}
            \end{tikzpicture}
        }
    \end{subfigure}

    \begin{subfigure}[b]{\textwidth}
        \centering
        \resizebox{\linewidth}{!}{
            \begin{tikzpicture}
                \begin{axis}[
                    width=0.98\textwidth,
                    height=6cm,
                    xmin=0.0,xmax=1000,ymin=-2,ymax=2,
                    xlabel={$k$},
                    ylabel={$u_{\mathrm{3}}[k]$},
                    legend pos=south east,
                    y tick label style={/pgf/number format/1000 sep=},
                    ] 
                    \addlegendentry{$u_{\mathrm{3}}[k]$},
                    \addlegendimage{no markers,blue}
                    \addplot[const plot, blue, semithick] file{../data/project/zad4/dmc/ostateczny/input_3.csv};  
                \end{axis}
            \end{tikzpicture}
        }
    \end{subfigure}

    \begin{subfigure}[b]{\textwidth}
        \centering
        \resizebox{\linewidth}{!}{
            \begin{tikzpicture}
                \begin{axis}[
                    width=0.98\textwidth,
                    height=6cm,
                    xmin=0.0,xmax=1000,ymin=-2,ymax=2,
                    xlabel={$k$},
                    ylabel={$u_{\mathrm{4}}[k]$},
                    legend pos=south east,
                    y tick label style={/pgf/number format/1000 sep=},
                    ] 
                    \addlegendentry{$u_{\mathrm{4}}[k]$},
                    \addlegendimage{no markers,blue}
                    \addplot[const plot, blue, semithick] file{../data/project/zad4/dmc/ostateczny/input_4.csv};  
                \end{axis}
            \end{tikzpicture}
        }
    \end{subfigure}
    \caption{Wejścia procesu wielowymiarowego sterowanego regulatorem DMC}
    \label{pro_dmc_2_in}
\end{figure}
\FloatBarrier

\section{Strojenie wielowymiarowego regulatora DMC metodą optymalizacji parametrów}
\label{pro_dmc_optymalizacja}

Oprócz ręcznego dobierania parametrów wielowymiarowego regulatora DMC, przygotowaliśmy również 
skrypt \verb+dmc_optim.m+, który dobierał wartości parametrów metodą optymalizacji. Długości horyzontów
założyliśmy stałe, takie jak przy ręcznym strojeniu tj. $D = \num{200}$, $N = \num{20}$ i $N_{\mathrm{u}} = \num{5}$.

W wyniku działania skryptu otrzymaliśmy następujące parametry:

\begin{center}
    $\mu_{1} = \num{6.1905}$, $\mu_{2} = \num{1.8593}$, $\mu_{3} = \num{2.1408}$,
    $\lambda{1} = \num{4.8655}$, $\lambda{2} = \num{0.18078}$, $\lambda{3} = \num{1.4161}$, $\lambda{4} = \num{0.21492}$
\end{center}

Po przeprowadzeniu symulacji otrzymaliśmy następujące wskaźniki jakości:

\[
\begin{bmatrix}
    E_{\mathrm{1}} \\
    E_{\mathrm{2}} \\
    E_{\mathrm{3}} 
\end{bmatrix}
= 
\begin{bmatrix}
    \num{2.9391} \\
    \num{2.9265} \\
    \num{5.2855}
\end{bmatrix}
\]

Sumaryczny błąd wyniósł $\num{11.151}$.\\

Uzyskane wyniki są porównywalne z tymi uzyskanymi w wyniku ręcznego doboru parametrów.

%% wyjscia 

\begin{figure}
    \centering
    \begin{subfigure}[b]{\textwidth}
        \centering
        \resizebox{\linewidth}{!}{
            \begin{tikzpicture}
                \begin{axis}[
                    width=0.98\textwidth,
                    height=7cm,
                    xmin=0.0,xmax=1000,ymin=-0.5,ymax=2,
                    xlabel={$k$},
                    ylabel={$y_{\mathrm{1}}[k]$},
                    legend pos=south east,
                    y tick label style={/pgf/number format/1000 sep=},
                    ] 
                    \addlegendentry{$y_{\mathrm{1}}[k]$},
                    \addlegendentry{$y_{\mathrm{1}}^{\mathrm{zad}}[k]$},
                
                    \addlegendimage{no markers,red}
                    \addlegendimage{no markers,blue}
                
                    \addplot[red, semithick] file{../data/project/zad5/dmc/dmc_optim/output_1.csv};  
                    \addplot[blue, semithick] file{../data/project/zad5/dmc/dmc_optim/stpt_1.csv};
                    
                    \end{axis}
            \end{tikzpicture}
        }
        \end{subfigure}

        \begin{subfigure}[b]{\textwidth}
            \centering
            \resizebox{\linewidth}{!}{
                \begin{tikzpicture}
                    \begin{axis}[
                        width=0.98\textwidth,
                        height=7cm,
                        xmin=0.0,xmax=1000,ymin=-0.5,ymax=2,
                        xlabel={$k$},
                        ylabel={$y_{\mathrm{2}}[k]$},
                        legend pos=south east,
                        y tick label style={/pgf/number format/1000 sep=},
                        ] 
                        \addlegendentry{$y_{\mathrm{2}}[k]$},
                        \addlegendentry{$y_{\mathrm{2}}^{\mathrm{zad}}[k]$},
                    
                        \addlegendimage{no markers,red}
                        \addlegendimage{no markers,blue}
                    
                        \addplot[red, semithick] file{../data/project/zad5/dmc/dmc_optim/output_2.csv};  
                        \addplot[blue, semithick] file{../data/project/zad5/dmc/dmc_optim/stpt_2.csv};
                        
                        \end{axis}
                \end{tikzpicture}
            }
        \end{subfigure}

        \begin{subfigure}[b]{\textwidth}
            \centering
            \resizebox{\linewidth}{!}{
                \begin{tikzpicture}
                    \begin{axis}[
                        width=0.98\textwidth,
                        height=7cm,
                        xmin=0.0,xmax=1000,ymin=-0.5,ymax=2,
                        xlabel={$k$},
                        ylabel={$y_{\mathrm{3}}[k]$},
                        legend pos=south east,
                        y tick label style={/pgf/number format/1000 sep=},
                        ] 
                        \addlegendentry{$y_{\mathrm{3}}[k]$},
                        \addlegendentry{$y_{\mathrm{3}}^{\mathrm{zad}}[k]$},
                    
                        \addlegendimage{no markers,red}
                        \addlegendimage{no markers,blue}
                    
                        \addplot[red, semithick] file{../data/project/zad5/dmc/dmc_optim/output_3.csv};  
                        \addplot[blue, semithick] file{../data/project/zad5/dmc/dmc_optim/stpt_3.csv};
                        
                        \end{axis}
                \end{tikzpicture}
            }
        \end{subfigure}
        \caption{Wyjścia procesu wielowymiarowego sterowanego regulatorem DMC}
        \label{pro_dmc_2_out}
\end{figure}

%% wejscia 

\begin{figure}
    \centering
    \begin{subfigure}[b]{\textwidth}
        \centering
        \resizebox{\linewidth}{!}{
            \begin{tikzpicture}
                \begin{axis}[
                    width=0.98\textwidth,
                    height=6cm,
                    xmin=0.0,xmax=1000,ymin=-2,ymax=2,
                    xlabel={$k$},
                    ylabel={$u_{\mathrm{1}}[k]$},
                    legend pos=south east,
                    y tick label style={/pgf/number format/1000 sep=},
                    ] 
                    \addlegendentry{$u_{\mathrm{1}}[k]$},
                    \addlegendimage{no markers,blue}
                    \addplot[const plot, blue, semithick] file{../data/project/zad5/dmc/dmc_optim/input_1.csv};  
                \end{axis}
            \end{tikzpicture}
        }
    \end{subfigure}

    \begin{subfigure}[b]{\textwidth}
        \centering
        \resizebox{\linewidth}{!}{
            \begin{tikzpicture}
                \begin{axis}[
                    width=0.98\textwidth,
                    height=6cm,
                    xmin=0.0,xmax=1000,ymin=-2,ymax=2,
                    xlabel={$k$},
                    ylabel={$u_{\mathrm{2}}[k]$},
                    legend pos=south east,
                    y tick label style={/pgf/number format/1000 sep=},
                    ] 
                    \addlegendentry{$u_{\mathrm{2}}[k]$},
                    \addlegendimage{no markers,blue}
                    \addplot[const plot, blue, semithick] file{../data/project/zad5/dmc/dmc_optim/input_2.csv};  
                \end{axis}
            \end{tikzpicture}
        }
    \end{subfigure}

    \begin{subfigure}[b]{\textwidth}
        \centering
        \resizebox{\linewidth}{!}{
            \begin{tikzpicture}
                \begin{axis}[
                    width=0.98\textwidth,
                    height=6cm,
                    xmin=0.0,xmax=1000,ymin=-2,ymax=2,
                    xlabel={$k$},
                    ylabel={$u_{\mathrm{3}}[k]$},
                    legend pos=south east,
                    y tick label style={/pgf/number format/1000 sep=},
                    ] 
                    \addlegendentry{$u_{\mathrm{3}}[k]$},
                    \addlegendimage{no markers,blue}
                    \addplot[const plot, blue, semithick] file{../data/project/zad5/dmc/dmc_optim/input_3.csv};  
                \end{axis}
            \end{tikzpicture}
        }
    \end{subfigure}

    \begin{subfigure}[b]{\textwidth}
        \centering
        \resizebox{\linewidth}{!}{
            \begin{tikzpicture}
                \begin{axis}[
                    width=0.98\textwidth,
                    height=6cm,
                    xmin=0.0,xmax=1000,ymin=-2,ymax=2,
                    xlabel={$k$},
                    ylabel={$u_{\mathrm{4}}[k]$},
                    legend pos=south east,
                    y tick label style={/pgf/number format/1000 sep=},
                    ] 
                    \addlegendentry{$u_{\mathrm{4}}[k]$},
                    \addlegendimage{no markers,blue}
                    \addplot[const plot, blue, semithick] file{../data/project/zad5/dmc/dmc_optim/input_4.csv};  
                \end{axis}
            \end{tikzpicture}
        }
    \end{subfigure}
    \caption{Wejścia procesu wielowymiarowego sterowanego regulatorem DMC}
    \label{pro_dmc_2_in}
\end{figure}
\FloatBarrier

\section{Wnioski}
\label{pro_dmc_wnioski}

Regulator DMC bardzo dobrze radzi sobie z regulacją obiektów wielowymiarowych, nawet 
tych w których liczba wejść jest większa od liczby wyjść. Istotną zaletą jest fakt, że 
steruje wszystkimi wejściami procesu. Nawet bez strojenia parametrów,
regulator bardzo dobrze radzi sobie z nadążaniem za wartością zadaną. Problemem w przypadku
regulacji obiektów rzeczywistych tą metodą może być brak możliwości zebrania poprawnej 
odpowiedzi skokowej. Należy pamiętać że model w postaci odpowiedzi skokowej, używany w tym regulatorze
do predykcji, nadaje się tylko do obiektów stabilnych, w przypadku gdy jeden tor obiektu jest
niestabilny, tracimy możliwość wykorzystania regulatora DMC.
