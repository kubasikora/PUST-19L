\chapter{Porównanie działania algorytmu DMC w wersji klasycznej i w wersji oszczędnej}
\label{dmc_porownanie}

W ostatnim już zadaniu projektowym, zaimplementowaliśmy algorytm DMC w wersji
klasycznej, w której wyznaczane są sterowania na całym horyzoncie sterowania, a 
następnie porównaliśmy wyniki symulacji oraz czasy wykonania.

\section{Implementacja algorytmu DMC w wersji klasycznej}
\label{dmc_klasyk}
Algorytm DMC w wersji klasycznej zaimplementowaliśmy w pliku \verb+dmc_orig.m+.
Inicjalizacja symulacji nie różni się niczym od wersji oszczędnej.
W pierwszej kolejności, przy użyciu tych samych funkcji, na bazie odpowiedzi skokowej,
obliczane są macierze $\bf{} M \/$, $\bf{} M^{P} \/$,  $\bf{} \Lambda \/$ oraz 
$\bf{} \Psi \/$, aby na ich podstawie obliczyć macierz $\bf{} K \/$.\\

Zasadnicza różnica między wersją klasyczną a oszczędną tkwi w sposobie obliczania
nowego przyrostu sterowania. W wersji oszczędnej obliczane były tylko przyrosty w chwili 
następnej. W przypadku wersji klasycznej, zmiany wartości sygnałów sterujących obliczane są na całym
horyzoncie sterowania. W tym celu skorzystaliśmy ze wzoru:

\begin{center}
    $\Delta \bf{} U \/$$[k] = \bf{} K \/$$(Y^{zad} - Y - \bf{} M^{P} \/$$ \Delta \bf{}U^{P}\/)$
\end{center}

Wektory $Y^{zad}$ oraz $Y$ są specjalnie przygotowanymi wektorami o długości $n_{y}N$, w której
wartość zadana/aktualna wartość wyjścia wszystkich $n_{y}$ wyjść jest powtórzona $N$ razy.\\
W Matlabie, do generacji tych wektorów, przygotowaliśmy specjalną funkcję \verb+duplicateVector+, 
która powtarza podany wektor, zadaną liczbę razy.\\

Po wyznaczeniu całego wektora przyrostów sterowań, należy wziąć pierwsze $n_{u}$ elementów i dodać 
je do aktualnych wartości wyjść. W Matlabie, cała procedura została przedstawiona na poniższym listingu.

\begin{lstlisting}[style=custommatlab,frame=single,label={pro_dmc_class},
    caption={Obliczenie nowego sterowania w klasycznej implementacji regulatora DMC},captionpos=b]
outputs(k, :) = [y1 y2 y3];                        
errors(k, :)=setpoints(k, :)-outputs(k, :); % obliczanie uchybow
Y_ZAD = duplicateVector(setpoints(k, :)', N*ny);
Y = duplicateVector(outputs(k, :)', N*ny);
dUK = K*(Y_ZAD - Y - Mp*dUp); % obliczanie przyrostow
    
inputs(k, :) = (inputs(k-1, :))' +  dUK(1:nu); 
\end{lstlisting}

\section{Porównanie przebiegów}
\label{dmc_przebiegi}
W celu porównania działania obu implementacji, przygotowaliśmy skrypt \verb+dmc_comp.m+,
który symuluje działanie obu algorytmów a następnie porównuje przebiegi. Porównanie przebiegów
polega na policzeniu sumy wartości bezwzględnych różnic pomiędzy przebiegami każdego z wyjść
oraz wejść. Po jego uruchomieniu otrzymaliśmy następujące wyniki:

\begin{lstlisting}[frame=single,label={pro_dmc_comp_result},
    caption={Wynik porównania działania obu algorytmów},captionpos=b]
>> dmc_comp
Output error = 0.0000000000911403
Input error = 0.0000000007286367
\end{lstlisting}

Róznice pomiędzy przebiegami obu implementacji tego samego algorytmu są rzędu $10^{-11}$.
Są to róznice wynikające z błędów numerycznych a dokładniej z utraty dokładności przy mnożeniu
liczb zmiennoprzecinkowych. Błąd na takim poziomie wskazuje że algorytm w wersji klasycznej
działa poprawnie.

%% wyjscia 

\begin{figure}
    \centering
    \begin{subfigure}[b]{\textwidth}
        \centering
        \resizebox{\linewidth}{!}{
            \begin{tikzpicture}
                \begin{axis}[
                    width=0.98\textwidth,
                    height=7cm,
                    xmin=0.0,xmax=1000,ymin=-0.5,ymax=2,
                    xlabel={$k$},
                    ylabel={$y_{\mathrm{1}}[k]$},
                    legend pos=south east,
                    y tick label style={/pgf/number format/1000 sep=},
                    ] 
                    \addlegendentry{$y_{\mathrm{1}}[k]$},
                    \addlegendentry{$y_{\mathrm{1}}^{\mathrm{zad}}[k]$},
                
                    \addlegendimage{no markers,red}
                    \addlegendimage{no markers,blue}
                
                    \addplot[red, semithick] file{../data/project/zad4/dmc/najprostszy_dmc/output_1.csv};  
                    \addplot[blue, semithick] file{../data/project/zad4/dmc/najprostszy_dmc/stpt_1.csv};
                    
                    \end{axis}
            \end{tikzpicture}
        }
        \end{subfigure}

        \begin{subfigure}[b]{\textwidth}
            \centering
            \resizebox{\linewidth}{!}{
                \begin{tikzpicture}
                    \begin{axis}[
                        width=0.98\textwidth,
                        height=7cm,
                        xmin=0.0,xmax=1000,ymin=-0.5,ymax=2,
                        xlabel={$k$},
                        ylabel={$y_{\mathrm{2}}[k]$},
                        legend pos=south east,
                        y tick label style={/pgf/number format/1000 sep=},
                        ] 
                        \addlegendentry{$y_{\mathrm{2}}[k]$},
                        \addlegendentry{$y_{\mathrm{2}}^{\mathrm{zad}}[k]$},
                    
                        \addlegendimage{no markers,red}
                        \addlegendimage{no markers,blue}
                    
                        \addplot[red, semithick] file{../data/project/zad4/dmc/najprostszy_dmc/output_2.csv};  
                        \addplot[blue, semithick] file{../data/project/zad4/dmc/najprostszy_dmc/stpt_2.csv};
                        
                        \end{axis}
                \end{tikzpicture}
            }
        \end{subfigure}

        \begin{subfigure}[b]{\textwidth}
            \centering
            \resizebox{\linewidth}{!}{
                \begin{tikzpicture}
                    \begin{axis}[
                        width=0.98\textwidth,
                        height=7cm,
                        xmin=0.0,xmax=1000,ymin=-0.5,ymax=2,
                        xlabel={$k$},
                        ylabel={$y_{\mathrm{3}}[k]$},
                        legend pos=south east,
                        y tick label style={/pgf/number format/1000 sep=},
                        ] 
                        \addlegendentry{$y_{\mathrm{3}}[k]$},
                        \addlegendentry{$y_{\mathrm{3}}^{\mathrm{zad}}[k]$},
                    
                        \addlegendimage{no markers,red}
                        \addlegendimage{no markers,blue}
                    
                        \addplot[red, semithick] file{../data/project/zad4/dmc/najprostszy_dmc/output_3.csv};  
                        \addplot[blue, semithick] file{../data/project/zad4/dmc/najprostszy_dmc/stpt_3.csv};
                        
                        \end{axis}
                \end{tikzpicture}
            }
        \end{subfigure}
        \caption{Wyjścia procesu wielowymiarowego sterowanego regulatorem DMC DMC w wersji klasycznej}
        \label{pro_dmc_1_out}
\end{figure}

%% wejscia 

\begin{figure}
    \centering
    \begin{subfigure}[b]{\textwidth}
        \centering
        \resizebox{\linewidth}{!}{
            \begin{tikzpicture}
                \begin{axis}[
                    width=0.98\textwidth,
                    height=6cm,
                    xmin=0.0,xmax=1000,ymin=-2,ymax=2,
                    xlabel={$k$},
                    ylabel={$u_{\mathrm{1}}[k]$},
                    legend pos=south east,
                    y tick label style={/pgf/number format/1000 sep=},
                    ] 
                    \addlegendentry{$u_{\mathrm{1}}[k]$},
                    \addlegendimage{no markers,blue}
                    \addplot[const plot, blue, semithick] file{../data/project/zad4/dmc/najprostszy_dmc/input_1.csv};  
                \end{axis}
            \end{tikzpicture}
        }
    \end{subfigure}

    \begin{subfigure}[b]{\textwidth}
        \centering
        \resizebox{\linewidth}{!}{
            \begin{tikzpicture}
                \begin{axis}[
                    width=0.98\textwidth,
                    height=6cm,
                    xmin=0.0,xmax=1000,ymin=-2,ymax=2,
                    xlabel={$k$},
                    ylabel={$u_{\mathrm{2}}[k]$},
                    legend pos=south east,
                    y tick label style={/pgf/number format/1000 sep=},
                    ] 
                    \addlegendentry{$u_{\mathrm{2}}[k]$},
                    \addlegendimage{no markers,blue}
                    \addplot[const plot, blue, semithick] file{../data/project/zad4/dmc/najprostszy_dmc/input_2.csv};  
                \end{axis}
            \end{tikzpicture}
        }
    \end{subfigure}

    \begin{subfigure}[b]{\textwidth}
        \centering
        \resizebox{\linewidth}{!}{
            \begin{tikzpicture}
                \begin{axis}[
                    width=0.98\textwidth,
                    height=6cm,
                    xmin=0.0,xmax=1000,ymin=-2,ymax=2,
                    xlabel={$k$},
                    ylabel={$u_{\mathrm{3}}[k]$},
                    legend pos=south east,
                    y tick label style={/pgf/number format/1000 sep=},
                    ] 
                    \addlegendentry{$u_{\mathrm{3}}[k]$},
                    \addlegendimage{no markers,blue}
                    \addplot[const plot, blue, semithick] file{../data/project/zad4/dmc/najprostszy_dmc/input_3.csv};  
                \end{axis}
            \end{tikzpicture}
        }
    \end{subfigure}

    \begin{subfigure}[b]{\textwidth}
        \centering
        \resizebox{\linewidth}{!}{
            \begin{tikzpicture}
                \begin{axis}[
                    width=0.98\textwidth,
                    height=6cm,
                    xmin=0.0,xmax=1000,ymin=-2,ymax=2,
                    xlabel={$k$},
                    ylabel={$u_{\mathrm{4}}[k]$},
                    legend pos=south east,
                    y tick label style={/pgf/number format/1000 sep=},
                    ] 
                    \addlegendentry{$u_{\mathrm{4}}[k]$},
                    \addlegendimage{no markers,blue}
                    \addplot[const plot, blue, semithick] file{../data/project/zad4/dmc/najprostszy_dmc/input_4.csv};  
                \end{axis}
            \end{tikzpicture}
        }
    \end{subfigure}
    \caption{Wejścia procesu wielowymiarowego sterowane regulatorem DMC w wersji klasycznej}
    \label{pro_dmc_1_in}
\end{figure}
\FloatBarrier