\chapter{Wyznaczenie punktów pracy stanowiska chłodząco-grzejącego}
\label{thermal_punkty_pracy}
Na początku pracy z stanowiskiem sprawdziliśmy czy komunikacja z stanowiskiem przebiega prawidłowo. Nasza uwaga skupiła się głównie na sygnałach $W1, W2, G1, G2, T1$ i $T2$. Następnie przeszliśmy do wyznaczenia temperatur $T1$ oraz $T3$ dla punktów pracy równych $G1 = 26\%$, $G2 =31\%$ i $W1 = 50\%$, $W2 = 50\%$. Wartość temperatur w punkcie pracy wynoszą $T1 = \num{35.3}^{\circ} C$ i $T3 = \num{37}^{\circ} C$.

%\section{Opis stanowiska}
%\label{thermal_opis_stanowiska}

%\section{Wyznaczanie punktów pracy}
%\label{thermal_punkty_pracy}
% ZROBIONE POWYZEJ