\chapter{Automat sterujący}
\label{thermal_automat}
W programie GX Works 3 zaimplementowaliśmy prosty automat stanu, który miał na celu modyfikować co pewien czas wartości zadane. Czas ten inżynier może dowolnie manipulować w programie zmieniając wartość zmiennej \textit{AUTO\_COUNTER}, która ma zostać osiągnięta, aby automat znalazł się w konkretnym stanie.

\begin{lstlisting}[caption={Implementacja automatu sterującego w programie GX Works 3}, language=C]
AUTO_COUNTER := AUTO_COUNTER + 1;

IF (AUTO_COUNTER < 30) THEN
	Zadana_PID1 := 4500.0;
	Zadana_DMC1 := 4500.0;
	
	Zadana_PID2 := 4500.0;
	Zadana_DMC2 := 4500.0;
END_IF;

IF (AUTO_COUNTER > 30 AND AUTO_COUNTER < 60) THEN
	Zadana_PID1 := 3500.0;
	Zadana_DMC1 := 3500.0;
	
	Zadana_PID2 := 4500.0;
	Zadana_DMC2 := 4500.0;
END_IF;

IF (AUTO_COUNTER > 60 AND AUTO_COUNTER < 90) THEN
	Zadana_PID1 := 3500.0;
	Zadana_DMC1 := 3500.0;
	
	Zadana_PID2 := 5000.0;
	Zadana_DMC2 := 5000.0;
END_IF;


IF (AUTO_COUNTER > 90 AND AUTO_COUNTER < 120) THEN
	Zadana_PID1 := 4000.0;
	Zadana_DMC1 := 4000.0;
	
	Zadana_PID2 := 4000.0;
	Zadana_DMC2 := 4000.0;
END_IF;

IF AUTO_COUNTER > 120 THEN
	AUTO_COUNTER := 0;
	END_IF;
\end{lstlisting}
