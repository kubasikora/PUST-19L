\chapter{Dwuwymiarowy regulator DMC}
\label{thermal_dmc}


\section{Odpowiedzi skokowe}
\label{thermal_dmc_odp_skok}


\section{Implementacja algorytmu regulacji DMC na sterowniku PLC}
\label{thermal_dmc_impl}
TU OPISAĆ KOD Z MATLABA
Część algorytmu odpowiadającą za wyliczanie sterowań (wejść procesu) napisaliśmy w języku ST w programie GX Works 3 i umieściliśmy w części Fixed Scan. Plik w którym ten kod się znajduje nosi nazwę \textit{DMC}.
\begin{lstlisting}[caption={Kod obliczający sterownanie dla regulatora dwuwymiarowego DMC (GX Works)}, language=C]
DMC_K11 := 0.0935;
DMC_K12 := 0.004;
DMC_K21 := 0.0038;
DMC_K22 := 0.0952;

dU1 := DMC_K11 *( Zadana_DMC1 - D100 ) + DMC_K12 *( Zadana_DMC2 - D102 ) ;
dU2 := DMC_K21 *( Zadana_DMC1 - D100 ) + DMC_K22 *( Zadana_DMC2 - D102 ) ;
dU1 := dU1 /100.0;
dU2 := dU2 /100.0;
temp_1 := dU1 ;
temp_2 := dU2 ;

FOR licznik := 0 TO 98 BY 1 DO
	dU1 := dU1 - tablica_MP[0].wiersz[ licznik ]* tablica_MP[2].wiersz[ licznik ];
	dU2 := dU2 - tablica_MP[1].wiersz[ licznik ]* tablica_MP[2].wiersz[ licznik ];
END_FOR ;

FOR licznik := 98 TO 2 BY -1 DO
	tablica_MP[2].wiersz[ licznik ]:= tablica_MP[2].wiersz[ licznik-2];
END_FOR ;

tablica_MP[2].wiersz[0]:= dU1 ;
tablica_MP[2].wiersz[1]:= dU2 ;
U_DMC1 := ( U_DMC1 / 10 + dU1 ) * 10; 
U_DMC2 := ( U_DMC2 / 10 + dU2 ) * 10; 



IF ( D100 > 15000) THEN
	U_DMC1 := 0;
END_IF;

IF ( D102 > 15000) THEN
	U_DMC2 := 0;
END_IF;
\end{lstlisting}

\section{Wyniki działania algorytmu regulacji}
\label{thermal_dmc_wyniki}
%% wyjscia 

\begin{figure}
    \centering
    \begin{subfigure}[t]{\textwidth}
        \centering
        \resizebox{\linewidth}{!}{
            \begin{tikzpicture}
                \begin{axis}[
                    width=0.98\textwidth,
                    height=7cm,
                    xmin=0.0,xmax=284,%ymin=25,ymax=41,
                    xlabel={$k$},
                    ylabel={$y_{\mathrm{1}}[k]$},
                    legend pos=north east,
                    y tick label style={/pgf/number format/1000 sep=},
                    ] 
                    \addlegendentry{$y_{\mathrm{1}}[k]$},
                    \addlegendentry{$y_{\mathrm{1}}^{\mathrm{zad}}[k]$},
                
                    \addlegendimage{no markers,red}
                    \addlegendimage{no markers,blue}
                
                    \addplot[red, semithick] file{../lab/thermal_object/DMC/PROBA1output_1.csv};  
                    \addplot[blue, semithick] file{../lab/thermal_object/DMC/PROBA1stpt_1.csv};
                    
                    \end{axis}
            \end{tikzpicture}
        }
        \end{subfigure}

        \begin{subfigure}[t]{\textwidth}
            \centering
            \resizebox{\linewidth}{!}{
                \begin{tikzpicture}
                    \begin{axis}[
                        width=0.98\textwidth,
                        height=7cm,
                        xmin=0.0,xmax=284,%ymin=25,ymax=41,
                        xlabel={$k$},
                        ylabel={$y_{\mathrm{2}}[k]$},
                        legend pos=south east,
                        y tick label style={/pgf/number format/1000 sep=},
                        ] 
                        \addlegendentry{$y_{\mathrm{2}}[k]$},
                        \addlegendentry{$y_{\mathrm{2}}^{\mathrm{zad}}[k]$},
                    
                        \addlegendimage{no markers,red}
                        \addlegendimage{no markers,blue}
                    
                        \addplot[red, semithick] file{../lab/thermal_object/DMC/PROBA1output_2.csv};  
                        \addplot[blue, semithick] file{../lab/thermal_object/DMC/PROBA1stpt_2.csv};
                        
                        \end{axis}
                \end{tikzpicture}
            }
        \end{subfigure}
        \caption{Wyjścia procesu wielowymiarowego}
        \label{lab_dmc_1_out}

\end{figure}

%% wejscia 

\begin{figure}
    \centering
    \begin{subfigure}[b]{\textwidth}
        \centering
        \resizebox{\linewidth}{!}{
            \begin{tikzpicture}
                \begin{axis}[
                    width=0.98\textwidth,
                    height=6cm,
                    xmin=0.0,xmax=284,%ymin=-3,ymax=3,
                    xlabel={$k$},
                    ylabel={$u_{\mathrm{1}}[k]$},
                    legend pos=north east,
                    y tick label style={/pgf/number format/1000 sep=},
                    ] 
                    \addlegendentry{$u_{\mathrm{1}}[k]$},
                    \addlegendimage{no markers,blue}
                    \addplot[const plot, blue, semithick] file{../lab/thermal_object/DMC/PROBA1input_1.csv};  
                \end{axis}
            \end{tikzpicture}
        }
    \end{subfigure}

    \begin{subfigure}[b]{\textwidth}
        \centering
        \resizebox{\linewidth}{!}{
            \begin{tikzpicture}
                \begin{axis}[
                    width=0.98\textwidth,
                    height=6cm,
                    xmin=0.0,xmax=284,%ymin=-3,ymax=3,
                    xlabel={$k$},
                    ylabel={$u_{\mathrm{2}}[k]$},
                    legend pos=south east,
                    y tick label style={/pgf/number format/1000 sep=},
                    ] 
                    \addlegendentry{$u_{\mathrm{2}}[k]$},
                    \addlegendimage{no markers,blue}
                    \addplot[const plot, blue, semithick] file{../lab/thermal_object/DMC/PROBA1input_2.csv};  
                \end{axis}
            \end{tikzpicture}
        }
    \end{subfigure}

    
    \caption{Wejścia procesu wielowymiarowego}
    \label{lab_dmc_1_in}
\end{figure}
\FloatBarrier

%%%%%%%%%%%%%%%%%%%%%%%%%%%%%%%%%%%%%%%%%%%%%%%%%%%%%%%%%%%%%%%%%%%%%%%%%%%%%%%%%%%%%%
%% wyjscia 

\begin{figure}
    \centering
    \begin{subfigure}[t]{\textwidth}
        \centering
        \resizebox{\linewidth}{!}{
            \begin{tikzpicture}
                \begin{axis}[
                    width=0.98\textwidth,
                    height=7cm,
                    xmin=0.0,xmax=500,%ymin=25,ymax=41,
                    xlabel={$k$},
                    ylabel={$y_{\mathrm{1}}[k]$},
                    legend pos=south east,
                    y tick label style={/pgf/number format/1000 sep=},
                    ] 
                    \addlegendentry{$y_{\mathrm{1}}[k]$},
                    \addlegendentry{$y_{\mathrm{1}}^{\mathrm{zad}}[k]$},
                
                    \addlegendimage{no markers,red}
                    \addlegendimage{no markers,blue}
                
                    \addplot[red, semithick] file{../lab/thermal_object/DMC/PROBA2output_1.csv};  
                    \addplot[blue, semithick] file{../lab/thermal_object/DMC/PROBA2stpt_1.csv};
                    
                    \end{axis}
            \end{tikzpicture}
        }
        \end{subfigure}

        \begin{subfigure}[t]{\textwidth}
            \centering
            \resizebox{\linewidth}{!}{
                \begin{tikzpicture}
                    \begin{axis}[
                        width=0.98\textwidth,
                        height=7cm,
                        xmin=0.0,xmax=500,%ymin=25,ymax=41,
                        xlabel={$k$},
                        ylabel={$y_{\mathrm{2}}[k]$},
                        legend pos=north east,
                        y tick label style={/pgf/number format/1000 sep=},
                        ] 
                        \addlegendentry{$y_{\mathrm{2}}[k]$},
                        \addlegendentry{$y_{\mathrm{2}}^{\mathrm{zad}}[k]$},
                    
                        \addlegendimage{no markers,red}
                        \addlegendimage{no markers,blue}
                    
                        \addplot[red, semithick] file{../lab/thermal_object/DMC/PROBA2output_2.csv};  
                        \addplot[blue, semithick] file{../lab/thermal_object/DMC/PROBA2stpt_2.csv};
                        
                        \end{axis}
                \end{tikzpicture}
            }
        \end{subfigure}
        \caption{Wyjścia procesu wielowymiarowego}
        \label{lab_dmc_2_out}

\end{figure}

%% wejscia 

\begin{figure}
    \centering
    \begin{subfigure}[b]{\textwidth}
        \centering
        \resizebox{\linewidth}{!}{
            \begin{tikzpicture}
                \begin{axis}[
                    width=0.98\textwidth,
                    height=6cm,
                    xmin=0.0,xmax=500,%ymin=-3,ymax=3,
                    xlabel={$k$},
                    ylabel={$u_{\mathrm{1}}[k]$},
                    legend pos=south east,
                    y tick label style={/pgf/number format/1000 sep=},
                    ] 
                    \addlegendentry{$u_{\mathrm{1}}[k]$},
                    \addlegendimage{no markers,blue}
                    \addplot[const plot, blue, semithick] file{../lab/thermal_object/DMC/PROBA2input_1.csv};  
                \end{axis}
            \end{tikzpicture}
        }
    \end{subfigure}

    \begin{subfigure}[b]{\textwidth}
        \centering
        \resizebox{\linewidth}{!}{
            \begin{tikzpicture}
                \begin{axis}[
                    width=0.98\textwidth,
                    height=6cm,
                    xmin=0.0,xmax=500,%ymin=-3,ymax=3,
                    xlabel={$k$},
                    ylabel={$u_{\mathrm{2}}[k]$},
                    legend pos=north east,
                    y tick label style={/pgf/number format/1000 sep=},
                    ] 
                    \addlegendentry{$u_{\mathrm{2}}[k]$},
                    \addlegendimage{no markers,blue}
                    \addplot[const plot, blue, semithick] file{../lab/thermal_object/DMC/PROBA2input_2.csv};  
                \end{axis}
            \end{tikzpicture}
        }
    \end{subfigure}

    
    \caption{Wejścia procesu wielowymiarowego}
    \label{lab_dmc_2_in}
\end{figure}
\FloatBarrier