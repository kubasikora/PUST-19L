\chapter{IMplementacja regulatora PID przy użyciu wbudowanej funkcji sterownika PLC}
\label{pid_wbudowany}
Aby zaimplementować regulator PID przy użyciu wbudowanej funkcji PID należy 
do sekcji programów SCAN dodać program realizujący zadanie wbudowanego w PLC regulatora PID, w 
odróżnieniu od zwykłej implementacji gdzie program obliczający sterowanie znajdował się w sekcji programów 
FIXED SCAN. \\

Dla wbudowanego regulatora PID tworzymy elementy SV, PV, 
MV do opisu procesu oraz $Control\_ON$ do włączania regulatora. Przyjęty przez nas typ danych dla 
SV, PV, MV to Word[Signed], a dla $Control\_ON$ Bit. Należy dodać również zmienne, 
reprezentujące nastawy regulatora PID. Najlepiej zadeklarować omówione zmienne jako globalne. 
Niniejszy kod uruchamia wbudowany regulator PID.

\begin{lstlisting}[caption={Użycie wbudowanego regulatora PID}]
MOV(PID1.Control_ON, 
    REAL_TO_INT( Proces1.y_k ),
    PID1.PV);

PID( PID1.Control_ON, 
     PID1.SV , 
     PID1.PV , 
     PID1.params[0] , 
     PID1.MV);

EMOV(PID1.Control_ON, 
     INT_TO_REAL(PID1.MV), 
     Proces1.u_k);
\end{lstlisting}

Struktura PID1 przechowuje informacje o podstawowych nastawach regulatora PID oraz bardziej zaawansowane 
parametry takie jak limity sterowania, kierunki działania oraz ustawienia mechanizmu \textit{anti-windup}.\\

Z powodu błędu w instrukcji laboratoryjnej opisanego wcześniej, nie mogliśmy zrealizować tego zadania, gdyż 
nie starczyło nam czasu, mimo iż wiedzieliśmy jak je wykonać. 
Użycie wbudowanego regulatora PID jest o wiele prostsze i szybsze niż implementacja własnego regulatora
i należy z tej możliwości korzystać w przypadku gdy jest ona dostępna.

