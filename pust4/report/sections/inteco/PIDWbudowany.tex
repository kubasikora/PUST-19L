\chapter{Porównanie działania wbudowanego PID z PID-em zaimplementowanym przez nas na laboratorium}
\label{pid_wbudowany}
Do sekcji programów SCAN dodajemy program realizujący zadanie wbudowanego w PLC regulatora PID. Tak jak wcześniej sekcja programów FIXED SCAN zawiera programy odpowiedzialne za cykliczne wykonywanie symulacji procesu. Dla wbudowanego regulatora PID tworzymy elementy SV, PV, MV i $Control\_ON$ - do włączania regulatora. Przyjęty przez nas typ danych dla SV, PV, MV to Word[Signed], a dla $Control\_ON$ Bit. Dodaliśmy również zmienne, które prezentowały nastawy regulatora PID. Wszystkie wymienione zmienne zostały zadeklarowane jako zmienne globalne. Poniższy kodu pozwolił na wykorzystanie wbudowanego w sterowniku PLC regulatora PID (umieszczony w sekcji SCAN). 
\begin{lstlisting}[caption={Użycie wbudowanego regulatora PID}, language=C]
MOV(PID1.Control_ON, REAL_TO_INT( Proces1.y_k ),PID1.PV);
PID( PID1.Control_ON, PID1.SV , PID1.PV , PID1.params[0] , PID1.MV);
EMOV(PID1.Control_ON, INT_TO_REAL(PID1.MV), Proces1.u_k);
\end{lstlisting}
Symulacji procesu dokonujemy w odpowiednich programach z grupie FIXED SCAN. Przed uruchomieniem całego programu na sterowniku PLC musieliśmy jeszcze dokonać prawidłowej inicjalizacji. W tym celu w grupie INIT stworzyliśmy kod realizujący to zadanie. 