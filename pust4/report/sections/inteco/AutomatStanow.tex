\chapter{Automat sterujący}
\label{inteco_automat}

\section{Listing programu}
\label{automat_lst}

\begin{lstlisting}
IF automat THEN
  SET(TRUE, Reguluj);
  AUTO_COUNTER := AUTO_COUNTER + 1;

  IF (AUTO_COUNTER < 30) THEN
	Zadana_PID1 := 4000.0;
	Zadana_PID2 := 300.0;	
  END_IF;

  IF (AUTO_COUNTER > 30 AND AUTO_COUNTER < 60) THEN
	Zadana_PID1 := 4000.0;
	Zadana_PID2 := 500.0;
  END_IF;

  IF (AUTO_COUNTER > 60 AND AUTO_COUNTER < 90) THEN
	Zadana_PID1 := 3000.0;
	Zadana_PID2 := 500.0;
  END_IF;

  IF (AUTO_COUNTER > 90 AND AUTO_COUNTER < 120) THEN
  	Zadana_PID1 := 3000.0;
	Zadana_PID2 := 400.0;
  END_IF;
	
  IF (AUTO_COUNTER > 120 AND AUTO_COUNTER < 150) THEN
	Zadana_PID1 := 3500.0;
	Zadana_PID2 := 400.0;
  END_IF;

  IF AUTO_COUNTER > 150 THEN
	AUTO_COUNTER := 0;
  END_IF;

ELSE
  RST(TRUE, Reguluj);
END_IF;

\end{lstlisting}

\section{Opis automatu}
\label{automat_opis}
Program zawierający automat stanów umieszczony został w sekcji \verb+Fixed Scan+. Przejście na pracę automatyczną realizuje się poprzez ustawienie flagi \verb+automat+. Po ustawieniu flagi następuje zliczanie cykli wywoływania sekcji \verb+Fixed Scan+ za pomocą zmiennej \verb+AUTO_COUNTER+, od której zależą zmiany wartości zadanych dla każdego toru regulacji. Zmiany następują co 30 cykli programu co daje 15 sekund na regulację do poszczególnych wartości zadanych przy okresie próbkowania równym \num{0,5}s.