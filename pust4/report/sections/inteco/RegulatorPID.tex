\chapter{Wielopętlowy regulator PID}
\label{inteco_pid}

\section{Implementacja wielopętlowego regulatora PID dla stanowiska INTECO CRANE}
\label{inteco_impl}
\newpage
\section{Proces strojenia regulatorów PID}
\label{inteco_strojenie_pid}
Parametry regulatorów PID dla stanowiska INTECO TCRANE dobraliśmy metodą eksperymentalną.
W każdej kolejnej iteracji, wykorzystując doświadczenie i intuicje zmienialiśmy poszczególne
parametry obu regulatorów.\\


\subsubsection{Próba 1}
W pierwszej próbie, sprawdziliśmy jak zachowa się obiekt przy sterowaniu regulatorami
typu P z jednostkowymi wzmocnieniami. Takie eksperyment pozwolił nam na rozeznanie się
jak poszczególne wyjścia reagują na sterowanie w pętli sprzężenia zwrotnego.\\

Wykorzystany zestaw parametrów:

\begin{center}
    $K^{\num{1}}_{\mathrm{p}} = \num{1.0}$, $T^{\num{1}}_{\mathrm{i}} = \num{99999.0}$, $T^{\num{1}}_{\mathrm{d}} = \num{0.0}$ \\
    $K^{\num{2}}_{\mathrm{p}} = \num{1.0}$, $T^{\num{2}}_{\mathrm{i}} = \num{99999.0}$, $T^{\num{2}}_{\mathrm{d}} = \num{0.0}$ \\
\end{center}
\vspace{1cm}

Wyniki eksperymentu nie były zadowalające. Zaobserwowaliśmy bardzo duże uchyby ustalone, a 
wyjście osi X cechowały się sporymi oscylacjami. Mimo złych wyników, eksperyment ten dał nam dużo 
informacji o obiekcie i pozwolił na sprawniejsze strojenie.

\begin{figure}
    \centering
    \begin{subfigure}[b]{\textwidth}
        \centering
        \resizebox{\linewidth}{!}{
            \begin{tikzpicture}
                \begin{axis}[
                    width=0.98\textwidth,
                    height=7cm,
                    xmin=1,xmax=74,ymin=0,ymax=5000,
                    xlabel={$k$},
                    ylabel={$y_{\mathrm{X}}[k]$},
                    legend pos=south east,
                    y tick label style={/pgf/number format/1000 sep=},
                    ] 
                \addlegendentry{$y_{\mathrm{X}}[k]$},
                \addlegendentry{$y_{\mathrm{X}}^{\mathrm{zad}}[k]$},
                
                \addlegendimage{no markers,red}
                \addlegendimage{no markers,blue}
                
                \addplot[red, semithick] file{../data/lab/inteco_object/zad10/pid1_output_x.csv};  
                \addplot[blue, semithick] file{../data/lab/inteco_object/zad10/pid1_stpt_x.csv};
                  
                \end{axis}
            \end{tikzpicture}
        }
        \end{subfigure}

        \begin{subfigure}[b]{\textwidth}
            \centering
            \resizebox{\linewidth}{!}{
                \begin{tikzpicture}
                    \begin{axis}[
                        width=0.98\textwidth,
                        height=7cm,
                        xmin=0.0,xmax=74,ymin=-60,ymax=60,
                        xlabel={$k$},
                        ylabel={$u_{\mathrm{X}}[k]$},
                        legend pos=south east,
                        y tick label style={/pgf/number format/1000 sep=},
                        ] 
                    \addlegendentry{$u_{\mathrm{X}}[k]$},
                    
                    \addlegendimage{no markers, blue}
                    
                    \addplot[const plot, blue, semithick] file{../data/lab/inteco_object/zad10/pid1_input_x.csv};  
                    \end{axis}
            \end{tikzpicture}
        }
    \end{subfigure}
    \caption{Przebiegi sygnałów wyjściowych i wejściowych dla osi X}
    \label{pid_1_x}
\end{figure}

\begin{figure}
    \centering
    \begin{subfigure}[b]{\textwidth}
        \centering
        \resizebox{\linewidth}{!}{
            \begin{tikzpicture}
                \begin{axis}[
                    width=0.98\textwidth,
                    height=7cm,
                    xmin=1,xmax=74,ymin=0,ymax=1000,
                    xlabel={$k$},
                    ylabel={$y_{\mathrm{Y}}[k]$},
                    legend pos=south east,
                    y tick label style={/pgf/number format/1000 sep=},
                    ] 
                \addlegendentry{$y_{\mathrm{Y}}[k]$},
                \addlegendentry{$y_{\mathrm{Y}}^{\mathrm{zad}}[k]$},
                
                \addlegendimage{no markers,red}
                \addlegendimage{no markers,blue}
                
                \addplot[red, semithick] file{../data/lab/inteco_object/zad10/pid1_output_y.csv};  
                \addplot[blue, semithick] file{../data/lab/inteco_object/zad10/pid1_stpt_y.csv};
                  
                \end{axis}
            \end{tikzpicture}
        }
        \end{subfigure}

        \begin{subfigure}[b]{\textwidth}
            \centering
            \resizebox{\linewidth}{!}{
                \begin{tikzpicture}
                    \begin{axis}[
                        width=0.98\textwidth,
                        height=7cm,
                        xmin=0.0,xmax=74,ymin=-60,ymax=60,
                        xlabel={$k$},
                        ylabel={$u_{\mathrm{Y}}[k]$},
                        legend pos=south east,
                        y tick label style={/pgf/number format/1000 sep=},
                        ] 
                    \addlegendentry{$u_{\mathrm{Y}}[k]$},
                    
                    \addlegendimage{no markers, blue}
                    
                    \addplot[const plot, blue, semithick] file{../data/lab/inteco_object/zad10/pid1_input_y.csv};  
                    \end{axis}
            \end{tikzpicture}
        }
    \end{subfigure}
    \caption{Przebiegi sygnałów wyjściowych i wejściowych dla osi Y}
    \label{pid_1_y}
\end{figure}
\FloatBarrier



\subsubsection{Próba 2}
W drugiej próbie uruchomiliśmy człony całkujący i różniczkujący obu regulatorów oraz dopasowaliśmy wzmocnienia 
do charakteru wyjść. W przypadku osi X wzmocnienie zmniejszyliśmy tak aby zlikwidować oscylacje. W przypadku osi Y, 
zwiększylismy wzmocnienie aby zmniejszyć uchyb ustalony.

Wykorzystany zestaw parametrów:

\begin{center}
    $K^{\num{1}}_{\mathrm{p}} = \num{0.5}$, $T^{\num{1}}_{\mathrm{i}} = \num{600.0}$, $T^{\num{1}}_{\mathrm{d}} = \num{1.0}$ \\
    $K^{\num{2}}_{\mathrm{p}} = \num{2.0}$, $T^{\num{2}}_{\mathrm{i}} = \num{700.0}$, $T^{\num{2}}_{\mathrm{d}} = \num{0.5}$ \\
\end{center}
\vspace{1cm}

Wyjście osi X nadal pozostawało dalekie od doskonałości. Uruchomienie członu całkującego pozwoliło na 
usunięcie uchybu ustalonego, jednak nie zlikwidowało oscylacji położenia. Sygnał sterujący
oscyluje pomiędzy wartościami granicznymi co utrudnia skuteczną regulację. W przypadku osi Y, sytuacja
wygląda o wiele lepiej. Pomimo oscylacji w początkowej fazie, sygnał wyjściowy ustala się na wartości zadanej
a sterowanie spada do zera.


\begin{figure}
    \centering
    \begin{subfigure}[b]{\textwidth}
        \centering
        \resizebox{\linewidth}{!}{
            \begin{tikzpicture}
                \begin{axis}[
                    width=0.98\textwidth,
                    height=7cm,
                    xmin=1,xmax=64,ymin=0,ymax=5000,
                    xlabel={$k$},
                    ylabel={$y_{\mathrm{X}}[k]$},
                    legend pos=south east,
                    y tick label style={/pgf/number format/1000 sep=},
                    ] 
                \addlegendentry{$y_{\mathrm{X}}[k]$},
                \addlegendentry{$y_{\mathrm{X}}^{\mathrm{zad}}[k]$},
                
                \addlegendimage{no markers,red}
                \addlegendimage{no markers,blue}
                
                \addplot[red, semithick] file{../data/lab/inteco_object/zad10/pid2_output_x.csv};  
                \addplot[blue, semithick] file{../data/lab/inteco_object/zad10/pid2_stpt_x.csv};
                  
                \end{axis}
            \end{tikzpicture}
        }
        \end{subfigure}

        \begin{subfigure}[b]{\textwidth}
            \centering
            \resizebox{\linewidth}{!}{
                \begin{tikzpicture}
                    \begin{axis}[
                        width=0.98\textwidth,
                        height=7cm,
                        xmin=0.0,xmax=64,ymin=-60,ymax=60,
                        xlabel={$k$},
                        ylabel={$u_{\mathrm{X}}[k]$},
                        legend pos=south east,
                        y tick label style={/pgf/number format/1000 sep=},
                        ] 
                    \addlegendentry{$u_{\mathrm{X}}[k]$},
                    
                    \addlegendimage{no markers, blue}
                    
                    \addplot[const plot, blue, semithick] file{../data/lab/inteco_object/zad10/pid2_input_x.csv};  
                    \end{axis}
            \end{tikzpicture}
        }
    \end{subfigure}
    \caption{Przebiegi sygnałów wyjściowych i wejściowych dla osi X}
    \label{pid_2_x}
\end{figure}

\begin{figure}
    \centering
    \begin{subfigure}[b]{\textwidth}
        \centering
        \resizebox{\linewidth}{!}{
            \begin{tikzpicture}
                \begin{axis}[
                    width=0.98\textwidth,
                    height=7cm,
                    xmin=1,xmax=64,ymin=0,ymax=1000,
                    xlabel={$k$},
                    ylabel={$y_{\mathrm{Y}}[k]$},
                    legend pos=south east,
                    y tick label style={/pgf/number format/1000 sep=},
                    ] 
                \addlegendentry{$y_{\mathrm{Y}}[k]$},
                \addlegendentry{$y_{\mathrm{Y}}^{\mathrm{zad}}[k]$},
                
                \addlegendimage{no markers,red}
                \addlegendimage{no markers,blue}
                
                \addplot[red, semithick] file{../data/lab/inteco_object/zad10/pid2_output_y.csv};  
                \addplot[blue, semithick] file{../data/lab/inteco_object/zad10/pid2_stpt_y.csv};
                  
                \end{axis}
            \end{tikzpicture}
        }
        \end{subfigure}

        \begin{subfigure}[b]{\textwidth}
            \centering
            \resizebox{\linewidth}{!}{
                \begin{tikzpicture}
                    \begin{axis}[
                        width=0.98\textwidth,
                        height=7cm,
                        xmin=0.0,xmax=64,ymin=-60,ymax=60,
                        xlabel={$k$},
                        ylabel={$u_{\mathrm{Y}}[k]$},
                        legend pos=south east,
                        y tick label style={/pgf/number format/1000 sep=},
                        ] 
                    \addlegendentry{$u_{\mathrm{Y}}[k]$},
                    
                    \addlegendimage{no markers, blue}
                    
                    \addplot[const plot, blue, semithick] file{../data/lab/inteco_object/zad10/pid2_input_y.csv};  
                    \end{axis}
            \end{tikzpicture}
        }
    \end{subfigure}
    \caption{Przebiegi sygnałów wyjściowych i wejściowych dla osi Y}
    \label{pid_2_y}
\end{figure}
\FloatBarrier

%%%%%%%% proba 3

\subsubsection{Próba 3}
W trzeciej próbie uznaliśmy że jeszcze bardziej zmniejszymy wzmocnienie regulatora sterującego osią X, tak 
aby niedoprowadzić do oscylacji. Zmniejszając wzmocnienie, zdecydowaliśmy się na zmniejszenie parametru 
$T^{\num{1}}_{\mathrm{i}}$ i zwiększenie $T^{\num{1}}_{\mathrm{d}}$ tak aby utrzymać rozsądny czas regulacji.
W przypadku osi Y, głownym celem było skrócenie czasu oscylacji lub ich ewentualne wyeliminowanie. Spróbowaliśmy 
zrealizować ten cel poprzez zmniejszenie wzmocnienia i zwiększenie $T^{\num{2}}_{\mathrm{i}}$.\\

Wykorzystany zestaw parametrów:

\begin{center}
    $K^{\num{1}}_{\mathrm{p}} = \num{0.05}$, $T^{\num{1}}_{\mathrm{i}} = \num{200.0}$, $T^{\num{1}}_{\mathrm{d}} = \num{1.0}$ \\
    $K^{\num{2}}_{\mathrm{p}} = \num{1.6}$, $T^{\num{2}}_{\mathrm{i}} = \num{800.0}$, $T^{\num{2}}_{\mathrm{d}} = \num{0.5}$ \\
\end{center}
\vspace{1cm}

Taki dobór parametrów pozwolił na całkiem dobrą regulację obu wyjść stanowiska. W przypadku osi X
pojawił się niewielki uchyb ustalony. Wynika on z faktu że dla małych sterowań, stanowisko nie reaguje
z powodu za dużego oporu silnika. W przypadku osi Y, dalej zauważalne są niewielkie oscylacje, jednak 
udało się znacznie zmniejszyć czas ich trwania.

\begin{figure}
    \centering
    \begin{subfigure}[b]{\textwidth}
        \centering
        \resizebox{\linewidth}{!}{
            \begin{tikzpicture}
                \begin{axis}[
                    width=0.98\textwidth,
                    height=7cm,
                    xmin=1,xmax=118,ymin=0,ymax=5000,
                    xlabel={$k$},
                    ylabel={$y_{\mathrm{X}}[k]$},
                    legend pos=south east,
                    y tick label style={/pgf/number format/1000 sep=},
                    ] 
                \addlegendentry{$y_{\mathrm{X}}[k]$},
                \addlegendentry{$y_{\mathrm{X}}^{\mathrm{zad}}[k]$},
                
                \addlegendimage{no markers,red}
                \addlegendimage{no markers,blue}
                
                \addplot[red, semithick] file{../data/lab/inteco_object/zad10/pid3_output_x.csv};  
                \addplot[blue, semithick] file{../data/lab/inteco_object/zad10/pid3_stpt_x.csv};
                  
                \end{axis}
            \end{tikzpicture}
        }
        \end{subfigure}

        \begin{subfigure}[b]{\textwidth}
            \centering
            \resizebox{\linewidth}{!}{
                \begin{tikzpicture}
                    \begin{axis}[
                        width=0.98\textwidth,
                        height=7cm,
                        xmin=1,xmax=118,ymin=-60,ymax=60,
                        xlabel={$k$},
                        ylabel={$u_{\mathrm{X}}[k]$},
                        legend pos=south east,
                        y tick label style={/pgf/number format/1000 sep=},
                        ] 
                    \addlegendentry{$u_{\mathrm{X}}[k]$},
                    
                    \addlegendimage{no markers, blue}
                    
                    \addplot[const plot, blue, semithick] file{../data/lab/inteco_object/zad10/pid3_input_x.csv};  
                    \end{axis}
            \end{tikzpicture}
        }
    \end{subfigure}
    \caption{Przebiegi sygnałów wyjściowych i wejściowych dla osi X}
    \label{pid_3_x}
\end{figure}

\begin{figure}
    \centering
    \begin{subfigure}[b]{\textwidth}
        \centering
        \resizebox{\linewidth}{!}{
            \begin{tikzpicture}
                \begin{axis}[
                    width=0.98\textwidth,
                    height=7cm,
                    xmin=1,xmax=118,ymin=0,ymax=1000,
                    xlabel={$k$},
                    ylabel={$y_{\mathrm{Y}}[k]$},
                    legend pos=south east,
                    y tick label style={/pgf/number format/1000 sep=},
                    ] 
                \addlegendentry{$y_{\mathrm{Y}}[k]$},
                \addlegendentry{$y_{\mathrm{Y}}^{\mathrm{zad}}[k]$},
                
                \addlegendimage{no markers,red}
                \addlegendimage{no markers,blue}
                
                \addplot[red, semithick] file{../data/lab/inteco_object/zad10/pid3_output_y.csv};  
                \addplot[blue, semithick] file{../data/lab/inteco_object/zad10/pid3_stpt_y.csv};
                  
                \end{axis}
            \end{tikzpicture}
        }
        \end{subfigure}

        \begin{subfigure}[b]{\textwidth}
            \centering
            \resizebox{\linewidth}{!}{
                \begin{tikzpicture}
                    \begin{axis}[
                        width=0.98\textwidth,
                        height=7cm,
                        xmin=1,xmax=118,ymin=-60,ymax=60,
                        xlabel={$k$},
                        ylabel={$u_{\mathrm{Y}}[k]$},
                        legend pos=south east,
                        y tick label style={/pgf/number format/1000 sep=},
                        ] 
                    \addlegendentry{$u_{\mathrm{Y}}[k]$},
                    
                    \addlegendimage{no markers, blue}
                    
                    \addplot[const plot, blue, semithick] file{../data/lab/inteco_object/zad10/pid3_input_y.csv};  
                    \end{axis}
            \end{tikzpicture}
        }
    \end{subfigure}
    \caption{Przebiegi sygnałów wyjściowych i wejściowych dla osi Y}
    \label{pid_3_y}
\end{figure}
\FloatBarrier


%%%%%%%% proba 4

\subsubsection{Próba 4}
W jak się okazało ostatniej próbie zdecydowaliśmy się na lekkie modyfikacje poszczególnych parametrów obu
regulatorów. Zmian dokonywalismy na podstawie intuicji inżynierskiej. \\

Wykorzystany zestaw parametrów:

\begin{center}
    $K^{\num{1}}_{\mathrm{p}} = \num{0.06}$, $T^{\num{1}}_{\mathrm{i}} = \num{200.0}$, $T^{\num{1}}_{\mathrm{d}} = \num{1.1}$ \\
    $K^{\num{2}}_{\mathrm{p}} = \num{1.2}$, $T^{\num{2}}_{\mathrm{i}} = \num{900.0}$, $T^{\num{2}}_{\mathrm{d}} = \num{0.6}$ \\
\end{center}
\vspace{1cm}

Ostateczny zestaw parametrów regulatorów pozwolił nam na uzyskanie bardzo dobrych rezultatów eksperymentu.
Sygnał wyjściowy bardzo szybko osiągał okolice wartości zadanej, a sterowanie bardzo szybko ustalało się
na danej wartości.

\begin{figure}
    \centering
    \begin{subfigure}[b]{\textwidth}
        \centering
        \resizebox{\linewidth}{!}{
            \begin{tikzpicture}
                \begin{axis}[
                    width=0.98\textwidth,
                    height=7cm,
                    xmin=1,xmax=52,ymin=0,ymax=5000,
                    xlabel={$k$},
                    ylabel={$y_{\mathrm{X}}[k]$},
                    legend pos=south east,
                    y tick label style={/pgf/number format/1000 sep=},
                    ] 
                \addlegendentry{$y_{\mathrm{X}}[k]$},
                \addlegendentry{$y_{\mathrm{X}}^{\mathrm{zad}}[k]$},
                
                \addlegendimage{no markers,red}
                \addlegendimage{no markers,blue}
                
                \addplot[red, semithick] file{../data/lab/inteco_object/zad10/pid5_output_x.csv};  
                \addplot[blue, semithick] file{../data/lab/inteco_object/zad10/pid5_stpt_x.csv};
                  
                \end{axis}
            \end{tikzpicture}
        }
        \end{subfigure}

        \begin{subfigure}[b]{\textwidth}
            \centering
            \resizebox{\linewidth}{!}{
                \begin{tikzpicture}
                    \begin{axis}[
                        width=0.98\textwidth,
                        height=7cm,
                        xmin=1,xmax=52,ymin=-60,ymax=60,
                        xlabel={$k$},
                        ylabel={$u_{\mathrm{X}}[k]$},
                        legend pos=south east,
                        y tick label style={/pgf/number format/1000 sep=},
                        ] 
                    \addlegendentry{$u_{\mathrm{X}}[k]$},
                    
                    \addlegendimage{no markers, blue}
                    
                    \addplot[const plot, blue, semithick] file{../data/lab/inteco_object/zad10/pid5_input_x.csv};  
                    \end{axis}
            \end{tikzpicture}
        }
    \end{subfigure}
    \caption{Przebiegi sygnałów wyjściowych i wejściowych dla osi X}
    \label{pid_4_x}
\end{figure}

\begin{figure}
    \centering
    \begin{subfigure}[b]{\textwidth}
        \centering
        \resizebox{\linewidth}{!}{
            \begin{tikzpicture}
                \begin{axis}[
                    width=0.98\textwidth,
                    height=7cm,
                    xmin=1,xmax=52,ymin=0,ymax=1000,
                    xlabel={$k$},
                    ylabel={$y_{\mathrm{Y}}[k]$},
                    legend pos=south east,
                    y tick label style={/pgf/number format/1000 sep=},
                    ] 
                \addlegendentry{$y_{\mathrm{Y}}[k]$},
                \addlegendentry{$y_{\mathrm{Y}}^{\mathrm{zad}}[k]$},
                
                \addlegendimage{no markers,red}
                \addlegendimage{no markers,blue}
                
                \addplot[red, semithick] file{../data/lab/inteco_object/zad10/pid5_output_y.csv};  
                \addplot[blue, semithick] file{../data/lab/inteco_object/zad10/pid5_stpt_y.csv};
                  
                \end{axis}
            \end{tikzpicture}
        }
        \end{subfigure}

        \begin{subfigure}[b]{\textwidth}
            \centering
            \resizebox{\linewidth}{!}{
                \begin{tikzpicture}
                    \begin{axis}[
                        width=0.98\textwidth,
                        height=7cm,
                        xmin=1,xmax=52,ymin=-60,ymax=60,
                        xlabel={$k$},
                        ylabel={$u_{\mathrm{Y}}[k]$},
                        legend pos=south east,
                        y tick label style={/pgf/number format/1000 sep=},
                        ] 
                    \addlegendentry{$u_{\mathrm{Y}}[k]$},
                    
                    \addlegendimage{no markers, blue}
                    
                    \addplot[const plot, blue, semithick] file{../data/lab/inteco_object/zad10/pid5_input_y.csv};  
                    \end{axis}
            \end{tikzpicture}
        }
    \end{subfigure}
    \caption{Przebiegi sygnałów wyjściowych i wejściowych dla osi Y}
    \label{pid_4_y}
\end{figure}
\FloatBarrier

\section{Prezentacja działania regulatorów PID przy cyklicznej zmianie wartości zadanych przez automat sterujący}
\label{automat_sterowanie_pid}
Po zaimplementowaniu automatu sterującego, przetestowaliśmy działanie nastrojonych regulatorów przy zmianie 
wartości zadanych przez ten automat.

Jak widać na poniżej przedstawionych wykresach, regulatory bardzo dobrze radzą sobie ze zmienną wartością 
zadaną położenia.

\begin{figure}
    \centering
    \begin{subfigure}[b]{\textwidth}
        \centering
        \resizebox{\linewidth}{!}{
            \begin{tikzpicture}
                \begin{axis}[
                    width=0.98\textwidth,
                    height=7cm,
                    xmin=1,xmax=110,ymin=0,ymax=5000,
                    xlabel={$k$},
                    ylabel={$y_{\mathrm{X}}[k]$},
                    legend pos=south east,
                    y tick label style={/pgf/number format/1000 sep=},
                    ] 
                \addlegendentry{$y_{\mathrm{X}}[k]$},
                \addlegendentry{$y_{\mathrm{X}}^{\mathrm{zad}}[k]$},
                
                \addlegendimage{no markers,red}
                \addlegendimage{no markers,blue}
                
                \addplot[red, semithick] file{../data/lab/inteco_object/zad10/PID_FINAL_output_x.csv};  
                \addplot[blue, semithick] file{../data/lab/inteco_object/zad10/PID_FINAL_stpt_x.csv};
                  
                \end{axis}
            \end{tikzpicture}
        }
        \end{subfigure}

        \begin{subfigure}[b]{\textwidth}
            \centering
            \resizebox{\linewidth}{!}{
                \begin{tikzpicture}
                    \begin{axis}[
                        width=0.98\textwidth,
                        height=7cm,
                        xmin=1,xmax=110,ymin=-60,ymax=60,
                        xlabel={$k$},
                        ylabel={$u_{\mathrm{X}}[k]$},
                        legend pos=south east,
                        y tick label style={/pgf/number format/1000 sep=},
                        ] 
                    \addlegendentry{$u_{\mathrm{X}}[k]$},
                    
                    \addlegendimage{no markers, blue}
                    
                    \addplot[const plot, blue, semithick] file{../data/lab/inteco_object/zad10/PID_FINAL_input_x.csv};  
                    \end{axis}
            \end{tikzpicture}
        }
    \end{subfigure}
    \caption{Przebiegi sygnałów wyjściowych i wejściowych dla osi X}
    \label{pid_4_x}
\end{figure}

\begin{figure}
    \centering
    \begin{subfigure}[b]{\textwidth}
        \centering
        \resizebox{\linewidth}{!}{
            \begin{tikzpicture}
                \begin{axis}[
                    width=0.98\textwidth,
                    height=7cm,
                    xmin=1,xmax=110,ymin=0,ymax=1000,
                    xlabel={$k$},
                    ylabel={$y_{\mathrm{Y}}[k]$},
                    legend pos=south east,
                    y tick label style={/pgf/number format/1000 sep=},
                    ] 
                \addlegendentry{$y_{\mathrm{Y}}[k]$},
                \addlegendentry{$y_{\mathrm{Y}}^{\mathrm{zad}}[k]$},
                
                \addlegendimage{no markers,red}
                \addlegendimage{no markers,blue}
                
                \addplot[red, semithick] file{../data/lab/inteco_object/zad10/PID_FINAL_output_y.csv};  
                \addplot[blue, semithick] file{../data/lab/inteco_object/zad10/PID_FINAL_stpt_y.csv};
                  
                \end{axis}
            \end{tikzpicture}
        }
        \end{subfigure}

        \begin{subfigure}[b]{\textwidth}
            \centering
            \resizebox{\linewidth}{!}{
                \begin{tikzpicture}
                    \begin{axis}[
                        width=0.98\textwidth,
                        height=7cm,
                        xmin=1,xmax=110,ymin=-60,ymax=60,
                        xlabel={$k$},
                        ylabel={$u_{\mathrm{Y}}[k]$},
                        legend pos=south east,
                        y tick label style={/pgf/number format/1000 sep=},
                        ] 
                    \addlegendentry{$u_{\mathrm{Y}}[k]$},
                    
                    \addlegendimage{no markers, blue}
                    
                    \addplot[const plot, blue, semithick] file{../data/lab/inteco_object/zad10/PID_FINAL_input_y.csv};  
                    \end{axis}
            \end{tikzpicture}
        }
    \end{subfigure}
    \caption{Przebiegi sygnałów wyjściowych i wejściowych dla osi Y}
    \label{pid_4_y}
\end{figure}
\FloatBarrier

