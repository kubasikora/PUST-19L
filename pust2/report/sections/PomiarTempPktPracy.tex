\chapter{Obiekt laboratoryjny -- stanowisko chłodząco-grzejące}
\label{lab1}

\section{Wyznaczenie punktu pracy}
\label{lab1_punkt_pracy}

Po sprawdzeniu możliwości sterowania i pomiaru w komunikacji ze stanowiskiem przystąpiliśmy do wyznaczenia punktu pracy.
Od prowadzącego otrzymaliśmy informację o wartości sygnału sterującego w punkcie pracy 
równej $U_{\mathrm{pp}} = 26\%$. W celu wyznaczenia wartości sygnału wyjściowego w 
punkcie pracy $Y_{\mathrm{pp}}$ na wejście obiektu podaliśmy stałe wejście o wartości 
$U_{\mathrm{pp}}$ i zaczekaliśmy aż wyjście ustabilizuje się. Po odczekaniu około 5 minut,
wyjście obiektu ustabilizowało się na wartości $Y_{\mathrm{pp}} = \num{32} \degree $C


\section{Wyznaczenie odpowiedzi skokowych toru zakłócenie-wyjście}
\label{lab1_odpowiedzi}

\begin{figure}[b]
    \centering
    \begin{tikzpicture}
    \begin{axis}[
    width=0.98\textwidth,
    xmin=0.0,xmax=600,ymin=29.5,ymax=36,
    xlabel={$k$},
    ylabel={$y[k]$},
    legend pos=north west,
    y tick label style={/pgf/number format/1000 sep=},
    ] 
    \addlegendentry{$\Delta z = \num{-10}$}
    \addlegendentry{$\Delta z = \num{20}$}
    \addlegendentry{$\Delta z = \num{-20}$},
	\addlegendimage{no markers,red}
	\addlegendimage{no markers,blue}
	\addlegendimage{no markers,green}
    \addplot[red, semithick, thick] file{../data/lab/zad2/zad2_skok_zakl_o_-10.csv};
    \addplot[blue, semithick, thick] file{../data/lab/zad2/zad2_skok_zakl_o_20.csv};
    \addplot[green, semithick, thick] file{../data/lab/zad2/zad2_skok_zakl_o_-20.csv};
    \end{axis}
    \end{tikzpicture}
    \caption{Odpowiedzi procesu na skokową zmianę sygnału sterującego}
    \label{lab1_porow_odp_skok_lab}
\end{figure}
\FloatBarrier

Zebraliśmy trzy odpowiedzi
skokowe z punktu pracy w celu zbadania właściwości statycznych obiektu. 
Zostały one przedstawione na rysunku \ref{lab1_porow_odp_skok_lab}. 
Wzmocnienie obiektu różni się w zależności od tego czy skok sterowania 
był dodatni lub ujemny, jednak różnice te są niewielkie, że obiekt 
można traktować jako liniowy. Korzystając z wzoru:

\begin{equation}
    K_{\mathrm{stat}} = \lim_{t \to \infty} \frac{y(t)- Y_{\mathrm{pp}}}{u_0 - U_{\mathrm{pp}}}
    \label{zad2_wzm_statyczne_wzor}
 \end{equation}

wyliczyliśmy efektywne wzmocnienie dla każdego badanego skoku i policzyliśmy ich średnią arytmetyczną ostatecznie
otrzymując $K_{\mathrm{stat}} = \num{0,1581}$.