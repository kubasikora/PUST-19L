\chapter{Wyznaczenie odpowiedzi skokowych obiektu}
\label{zad2}

\section{Odpowiedź skokowa toru wejście-wyjście}
Eksperymenty wykonywane były z punktu pracy zdefiniowanego w zadaniu, przy sygnale zakłócenia $Z_{\mathrm{pp}} = 0$. Na wykresie \ref{zad2_odp_skok_input} możemy zaobserwować, że wraz z wzrostem wartości skoku rośnie również wartość odpowiedzi procesu $y[k]$. Zachowanie procesu jest zgodne 
z typową charakterystyką obiektów dynamicznych liniowych, w których wartość ustalona zmienia się liniowo wraz ze zmianą sygnału sterującego. 

\begin{figure}[t]
    \centering
    \begin{tikzpicture}
    \begin{axis}[
    width=0.98\textwidth,
    xmin=0.0,xmax=200,ymin=-1.5,ymax=1.5,
    xlabel={$k$},
    ylabel={$y[k]$},
    legend pos=south east,
    y tick label style={/pgf/number format/1000 sep=},
    ] 
    \addlegendentry{$\Delta u = \num{0,1}$},
    \addlegendentry{$\Delta u = \num{0,2}$}
    \addlegendentry{$\Delta u = \num{0,35}$},
    \addlegendentry{$\Delta u = \num{0,5}$},
    \addlegendentry{$\Delta u = \num{-0,1}$},
    \addlegendentry{$\Delta u = \num{-0,2}$},
    \addlegendentry{$\Delta u = \num{-0,35}$},
    \addlegendentry{$\Delta u = \num{-0,5}$},
    \addlegendimage{no markers,green}
	\addlegendimage{no markers,red}
	\addlegendimage{no markers,yellow}
	\addlegendimage{no markers,blue}
	\addlegendimage{no markers,black}
	\addlegendimage{no markers,orange}
	\addlegendimage{no markers,brown}
	\addlegendimage{no markers,violet}
    \addplot[green, semithick, thick] file{../data/zad2/zad2_input_output0.1.csv};
    \addplot[red, semithick, thick] file{../data/zad2/zad2_input_output0.2.csv};
    \addplot[yellow, semithick, thick] file{../data/zad2/zad2_input_output0.35.csv};
    \addplot[blue, semithick, thick] file{../data/zad2/zad2_input_output0.5.csv};
    \addplot[black, semithick, thick] file{../data/zad2/zad2_input_output-0.1.csv};
    \addplot[orange, semithick, thick] file{../data/zad2/zad2_input_output-0.2.csv};
    \addplot[brown, semithick, thick] file{../data/zad2/zad2_input_output-0.35.csv};
    \addplot[violet, semithick, thick] file{../data/zad2/zad2_input_output-0.5.csv};    
    \end{axis}
    \end{tikzpicture}
    \caption{Odpowiedzi procesu na skokową zmianę sygnału sterującego}
    \label{zad2_odp_skok_input}
\end{figure}

\section{Odpowiedź skokowa toru zakłócenie-wyjście}

Odpowiedź skokową toru zakłócenie-wyjście otrzymywaliśmy poprzez skokową zmianę zakłócenia. Eksperymenty wykonywane były ze stanu punktu pracy, w którym wszystkie sygnały wynosiły 0. Sygnał sterowania był stały podczas całej symulacji. Na wykresie \ref{zad2_odp_skok_disturbance} obserwujemy reakcję na zmiany zakłóceń i możemy stwierdzić, że jest ona podobna do tej otrzymanej przy zmianach sterowania, co tylko potwierdza nasze przypuszczenia o liniowości obiektu. 

\begin{figure}[b]
    \centering
    \begin{tikzpicture}
    \begin{axis}[
    width=0.98\textwidth,
    xmin=0.0,xmax=200,ymin=-1,ymax=1,
    xlabel={$k$},
    ylabel={$y[k]$},
    legend pos=south east,
    y tick label style={/pgf/number format/1000 sep=},
    ] 
    \addlegendentry{$\Delta z = \num{0,05}$},
    \addlegendentry{$\Delta z = \num{0,2}$}
    \addlegendentry{$\Delta z = \num{0,35}$},
    \addlegendentry{$\Delta z = \num{0,5}$},
    \addlegendentry{$\Delta z = \num{-0,05}$},
    \addlegendentry{$\Delta z = \num{-0,2}$},
    \addlegendentry{$\Delta z = \num{-0,35}$},
    \addlegendentry{$\Delta z = \num{-0,5}$},
    \addlegendimage{no markers,green}
	\addlegendimage{no markers,red}
	\addlegendimage{no markers,yellow}
	\addlegendimage{no markers,blue}
	\addlegendimage{no markers,black}
	\addlegendimage{no markers,orange}
	\addlegendimage{no markers,brown}
	\addlegendimage{no markers,violet}
    \addplot[green, semithick, thick] file{../data/zad2/zad2_disturbance_output0.05.csv};
    \addplot[red, semithick, thick] file{../data/zad2/zad2_disturbance_output0.2.csv};
    \addplot[yellow, semithick, thick] file{../data/zad2/zad2_disturbance_output0.35.csv};
    \addplot[blue, semithick, thick] file{../data/zad2/zad2_disturbance_output0.5.csv};
    \addplot[black, semithick, thick] file{../data/zad2/zad2_disturbance_output-0.05.csv};
    \addplot[orange, semithick, thick] file{../data/zad2/zad2_disturbance_output-0.2.csv};
    \addplot[brown, semithick, thick] file{../data/zad2/zad2_disturbance_output-0.35.csv};
    \addplot[violet, semithick, thick] file{../data/zad2/zad2_disturbance_output-0.5.csv};    
    \end{axis}
    \end{tikzpicture}
    \caption{Odpowiedzi procesu na skokową zmianę sygnału zaklócenia}
    \label{zad2_odp_skok_disturbance}
\end{figure}

\section{Wyznaczenie charakterystyki statycznej $y(u,z)$}
Wyznaczenie charakterystyki $y(u,z)$ rozpoczęliśmy od wyznaczenia charakterystyk $y(u)$ i $y(z)$. 
Na dalszym etapie zadania pomogło to w określeniu wzmocnień statycznych torów wejście-wyjście, zakłócenie-wyjśćie. 

Aby otrzymać wykres charakterystyki statycznej obiektu w zależności od 
dwóch argumentów należało przeprowadzić symulacje dla każdej 
wartości sygnałów $u$ oraz $d$ i zapamiętaniu wartości 
nasycenia sygnału $y$. Program, który realizujący zadanie to 
\verb+zad2_static_surface.m+ wykonywał wspomniane działania, 
a wyniki zapisywał do tablicy z wartościami charakterystyki statycznej. 
Wynik można zobaczyć na rysunku \ref{zad2_static_surf}.\\
\indent{} Wykres przedstawia płaszczyznę $y(u,z)$. Skoro powstały wykres 
jest płaszczyzną to można wywnioskować, że mamy do czynienia z obiektem liniowym. 

