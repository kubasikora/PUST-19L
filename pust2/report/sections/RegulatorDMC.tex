\chapter{Dobór parametrów algorytmu DMC}
\label{zad4}

Parametry regulatora DMC dobraliśmy metodą eksperymentalną.
W niniejszym rozdziale udokumentowalismy ten proces.

\section{Dobór horyzontu predykcji $N$}
W trakcie doboru horyzontu predykcji $N$ założyliśmy że wartość horyzontu sterowania nie może być 
większa od $N$, dlatego równocześnie zmniejszaliśmy oba parametry, utrzymując zależność $N = N_{\mathrm{u}}$.

\subsubsection{Horyzont predykcji $N = \num{155}$}
W pierwszej kolejności przyjęliśmy wartość horyzontu predykcji równą wartości horyzontu dynamiki
symulowanego procesu. Regulator działał w sposób zadowalający, jednak cechował się pofałdowanym
przebiegiem wyjściowym. Wskaźnik jakości regulacji wyniósł $E = \num{15,9773}$ co jest wartością
zadowalającą, jednak świadomi możliwości poprawy, postanowiliśmy skrócić długość horyzontu
predykcji. Przebiegi wyjściowe i wejściowe procesu z zaznaczoną wartością zadaną zostały przedstawione
na rysunkach \ref{zad4_N_1_y} i \ref{zad4_N_1_u}.

\subsubsection{Horyzont predykcji $N = \num{100}$}
Dla skróconego horyzontu predykcji, wartość wskaźnika jakości nie zmniejszyła się i utrzymała się
na poziomie $E = \num{15,9773}$. Przebieg wyjścia nadal pozostał pofałdowany. Potwierdzają to rysunki
\ref{zad4_N_2_y} i \ref{zad4_N_2_u}. Inżynierska intuicja 
podpowiada że za mało skróciliśmy horyzont predykcji aby zauważyć jakiekolwiek zmiany. 

\subsubsection{Horyzont predykcji $N = \num{70}$}
Dopiero przy skróceniu horyzontu predykcji do $N = \num{70}$ zauważyliśmy zmianę wskaźnika jakości, 
który dla takiego zestawu parametrów wyniósł $E = \num{15,9763}$. Mała zmiana tego parametru, może nieznacznie
poprawiła wskaźnik jakości, jednak nie wpłynęła na pofałdowanie przebiegu wyjścia procesu. Efekty działania tego
regulatora przedstawiono na rysunkach \ref{zad4_N_3_y} i \ref{zad4_N_3_u}.

\subsubsection{Horyzont predykcji $N = \num{60}$}
Próbując jeszcze bardziej poprawić wskaźnik jakości, zmniejszyliśmy długość horyzontu predykcji do $N = \num{60}$.
Po przeprowadzeniu symulacji, której wyniki znajdują się na wykresach \ref{zad4_N_4_y} i \ref{zad4_N_4_u}, 
zauważyliśmy niespodziewane pogorszenie wskaźnika jakości. Jego wartość wzrosła do $E = \num{15,9874}$.
Biorąc pod uwagę ten fakt, postanowiliśmy porzucić dalsze manipulowanie parametrem $N$ i pozostać przy
wartości horyzontu predykcji $N = \num{70}$ aby przejść do badania parametru $N_{\mathrm{u}}$.

\begin{figure}[t]
    \centering
    \begin{tikzpicture}
    \begin{axis}[
    width=0.98\textwidth,
    xmin=0,xmax=500,ymin=0, ymax=2,
    xlabel={$k$},
    ylabel={$y[k]$},
    %xtick={0, 100, 200, 300, 400, 500},
    %ytick={3.9, 4, 4.1, 4.2, 4.3, 4.4},
    legend pos=south east,
    y tick label style={/pgf/number format/1000 sep=},
    ]
    \addplot[blue, semithick] file{../data/zad4/zad4_stpt_N_155_Nu_155_lambda_100.csv};
    \addplot[red, semithick] file{../data/zad4/zad4_output_N_155_Nu_155_lambda_100.csv};
    \addlegendentry{$y^{\mathrm{zad}}[k]$},
    \addlegendentry{$y[k]$},
    \addlegendimage{no markers, blue}
	\addlegendimage{no markers, red}
    \end{axis}
    \end{tikzpicture}
    \caption{Przebieg procesu sterowanego za pomocą regulatora z parametrami $D = 155$, $N = 155$, $N_{\mathrm{u}} = 155$, $\lambda = 100$}
    \label{zad4_N_1_y}
\end{figure}

\begin{figure}[b]
    \centering
    \begin{tikzpicture}
    \begin{axis}[
    width=0.98\textwidth,
    xmin=0,xmax=500,ymin=0, ymax=1,
    xlabel={$k$},
    ylabel={$u[k]$},
    %xtick={0, 0.4, 0.5, 0.6, 0.7},
    %ytick={3.5, 3.75, 4, 4.25, 4.5},
    legend pos=south east,
    y tick label style={/pgf/number format/1000 sep=},
    ]
    \addplot[const plot, blue, semithick] file{../data/zad4/zad4_input_N_155_Nu_155_lambda_100.csv};
    \legend{$u[k]$}
    \end{axis}
    \end{tikzpicture}
    \caption{Przebieg sygnału sterującego przy parametrach parametrach: $D = 155$, $N = 155$, $N_{\mathrm{u}} = 155$, $\lambda = 100$}
    \label{zad4_N_1_u}
\end{figure}
\FloatBarrier

%%%

\begin{figure}[t]
    \centering
    \begin{tikzpicture}
    \begin{axis}[
    width=0.98\textwidth,
    xmin=0,xmax=500,ymin=0, ymax=2,
    xlabel={$k$},
    ylabel={$y[k]$},
    %xtick={0, 100, 200, 300, 400, 500},
    %ytick={3.9, 4, 4.1, 4.2, 4.3, 4.4},
    legend pos=south east,
    y tick label style={/pgf/number format/1000 sep=},
    ]
    \addplot[blue, semithick] file{../data/zad4/zad4_stpt_N_100_Nu_100_lambda_100.csv};
    \addplot[red, semithick] file{../data/zad4/zad4_output_N_100_Nu_100_lambda_100.csv};
    \addlegendentry{$y^{\mathrm{zad}}[k]$},
    \addlegendentry{$y[k]$},
    \addlegendimage{no markers, blue}
	\addlegendimage{no markers, red}
    \end{axis}
    \end{tikzpicture}
    \caption{Przebieg procesu sterowanego za pomocą regulatora z parametrami $D = 155$, $N = 100$, $N_{\mathrm{u}} = 100$, $\lambda = 100$}
    \label{zad4_N_2_y}
\end{figure}

\begin{figure}[b]
    \centering
    \begin{tikzpicture}
    \begin{axis}[
    width=0.98\textwidth,
    xmin=0,xmax=500,ymin=0, ymax=1,
    xlabel={$k$},
    ylabel={$u[k]$},
    %xtick={0, 0.4, 0.5, 0.6, 0.7},
    %ytick={3.5, 3.75, 4, 4.25, 4.5},
    legend pos=south east,
    y tick label style={/pgf/number format/1000 sep=},
    ]
    \addplot[const plot, blue, semithick] file{../data/zad4/zad4_input_N_100_Nu_100_lambda_100.csv};
    \legend{$u[k]$}
    \end{axis}
    \end{tikzpicture}
    \caption{Przebieg sygnału sterującego przy parametrach parametrach: $D = 155$, $N = 100$, $N_{\mathrm{u}} = 100$, $\lambda = 100$}
    \label{zad4_N_2_u}
\end{figure}
\FloatBarrier

%%%

\begin{figure}[t]
    \centering
    \begin{tikzpicture}
    \begin{axis}[
    width=0.98\textwidth,
    xmin=0,xmax=500,ymin=0, ymax=2,
    xlabel={$k$},
    ylabel={$y[k]$},
    %xtick={0, 100, 200, 300, 400, 500},
    %ytick={3.9, 4, 4.1, 4.2, 4.3, 4.4},
    legend pos=south east,
    y tick label style={/pgf/number format/1000 sep=},
    ]
    \addplot[blue, semithick] file{../data/zad4/zad4_stpt_N_70_Nu_70_lambda_100.csv};
    \addplot[red, semithick] file{../data/zad4/zad4_output_N_70_Nu_70_lambda_100.csv};
    \addlegendentry{$y^{\mathrm{zad}}[k]$},
    \addlegendentry{$y[k]$},
    \addlegendimage{no markers, blue}
	\addlegendimage{no markers, red}
    \end{axis}
    \end{tikzpicture}
    \caption{Przebieg procesu sterowanego za pomocą regulatora z parametrami $D = 155$, $N = 70$, $N_{\mathrm{u}} = 70$, $\lambda = 100$}
    \label{zad4_N_3_y}
\end{figure}

\begin{figure}[b]
    \centering
    \begin{tikzpicture}
    \begin{axis}[
    width=0.98\textwidth,
    xmin=0,xmax=500,ymin=0, ymax=1,
    xlabel={$k$},
    ylabel={$u[k]$},
    %xtick={0, 0.4, 0.5, 0.6, 0.7},
    %ytick={3.5, 3.75, 4, 4.25, 4.5},
    legend pos=south east,
    y tick label style={/pgf/number format/1000 sep=},
    ]
    \addplot[const plot, blue, semithick] file{../data/zad4/zad4_input_N_70_Nu_70_lambda_100.csv};
    \legend{$u[k]$}
    \end{axis}
    \end{tikzpicture}
    \caption{Przebieg sygnału sterującego przy parametrach parametrach: $D = 155$, $N = 70$, $N_{\mathrm{u}} = 70$, $\lambda = 100$}
    \label{zad4_N_3_u}
\end{figure}
\FloatBarrier

%%%

\begin{figure}[t]
    \centering
    \begin{tikzpicture}
    \begin{axis}[
    width=0.98\textwidth,
    xmin=0,xmax=500,ymin=0, ymax=2,
    xlabel={$k$},
    ylabel={$y[k]$},
    %xtick={0, 100, 200, 300, 400, 500},
    %ytick={3.9, 4, 4.1, 4.2, 4.3, 4.4},
    legend pos=south east,
    y tick label style={/pgf/number format/1000 sep=},
    ]
    \addplot[blue, semithick] file{../data/zad4/zad4_stpt_N_60_Nu_60_lambda_100.csv};
    \addplot[red, semithick] file{../data/zad4/zad4_output_N_60_Nu_60_lambda_100.csv};
    \addlegendentry{$y^{\mathrm{zad}}[k]$},
    \addlegendentry{$y[k]$},
    \addlegendimage{no markers, blue}
	\addlegendimage{no markers, red}
    \end{axis}
    \end{tikzpicture}
    \caption{Przebieg procesu sterowanego za pomocą regulatora z parametrami $D = 155$, $N = 60$, $N_{\mathrm{u}} = 60$, $\lambda = 100$}
    \label{zad4_N_4_y}
\end{figure}

\begin{figure}[b]
    \centering
    \begin{tikzpicture}
    \begin{axis}[
    width=0.98\textwidth,
    xmin=0,xmax=500,ymin=0, ymax=1,
    xlabel={$k$},
    ylabel={$u[k]$},
    %xtick={0, 0.4, 0.5, 0.6, 0.7},
    %ytick={3.5, 3.75, 4, 4.25, 4.5},
    legend pos=south east,
    y tick label style={/pgf/number format/1000 sep=},
    ]
    \addplot[const plot, blue, semithick] file{../data/zad4/zad4_input_N_60_Nu_60_lambda_100.csv};
    \legend{$u[k]$}
    \end{axis}
    \end{tikzpicture}
    \caption{Przebieg sygnału sterującego przy parametrach parametrach: $D = 155$, $N = 60$, $N_{\mathrm{u}} = 60$, $\lambda = 100$}
    \label{zad4_N_4_u}
\end{figure}
\FloatBarrier


\section{Dobór horyzontu sterowania $N_{\mathrm{u}}$}
W kolejnym kroku zmniejszaliśmy horyzont sterowania $N_{\mathrm{u}}$, utrzymując wartość
horyzontu predykcji na stałym poziomie równym $N = \num{70}$.

\subsubsection{Wartość horyzontu sterowania $N_{\mathrm{u}} = 35$}
W pierwszej kolejności zmniejszyliśmy parametr $N_{\mathrm{u}}$ aby rozpoznać jak jego zmiana
wpływa na jakość regulacji. Zmniejszenie wartości horyzontu sterowania pozwoliło na
delikatne wygładzenie przebiegu sygnału wyjściowego procesu i zmniejszenie wskaźnika jakości
do wartości $E=\num{15,9583}$. Przebiegi wejścia i wyjścia symulowanego obiektu przedstawiliśmy 
na rysunkach \ref{zad4_Nu_1_y} i \ref{zad4_Nu_1_u}. Idąc tym tropem postanowiliśmy jeszcze bardziej zmniejszyć
wartość tego parametru.

\subsubsection{Wartość horyzontu sterowania $N_{\mathrm{u}} = 2$}
Początkowy entuzjazm spowodowany poprawieniem jakości regulacji skłonił nas do przeprowadzania
próby w której wartość tego parametru jest równa $N_{\mathrm{u}} = 2$. Przebiegi sygnału 
wyjściowego znacznie bardziej wygładziły się, jednak kosztem pogorszenia wskaźnika jakości, 
który dla tej symulacji wyniósł $E = \num{16,3116}$. Wyniki symulacji zostały przedstawione 
na wykresach \ref{zad4_Nu_2_y} i \ref{zad4_Nu_2_u}

\subsection{Wartość horyzontu sterowania $N_{\mathrm{u}} = 1$}
Kierowani intuicją inżynierską, nie zważając na niepowodzenie poprzedniej symulacji, postanowiliśmy
jeszcze bardziej zmniejszyć wartość tego parametru, osiągając najniższą możliwą wartość równą 
$N_{\mathrm{u}} = 1$. Był to strzał w dziesiątkę, ponieważ taki zabieg pozwolił nam utrzymanie
gładkiego przebiegu sygnału wyjściowego oraz na zmniejszenie wartości wskaźnika jakości do poziomu
$E = \num{15,7448}$. Nie mogąc już dalej manipulować wartością horyzontu sterowania, przeszliśmy do 
ostatniego etapu strojenia parametrów algorytmu.

\begin{figure}[t]
    \centering
    \begin{tikzpicture}
    \begin{axis}[
    width=0.98\textwidth,
    xmin=0,xmax=500,ymin=0, ymax=2,
    xlabel={$k$},
    ylabel={$y[k]$},
    %xtick={0, 100, 200, 300, 400, 500},
    %ytick={3.9, 4, 4.1, 4.2, 4.3, 4.4},
    legend pos=south east,
    y tick label style={/pgf/number format/1000 sep=},
    ]
    \addplot[blue, semithick] file{../data/zad4/zad4_stpt_N_70_Nu_35_lambda_100.csv};
    \addplot[red, semithick] file{../data/zad4/zad4_output_N_70_Nu_35_lambda_100.csv};
    \addlegendentry{$y^{\mathrm{zad}}[k]$},
    \addlegendentry{$y[k]$},
    \addlegendimage{no markers, blue}
	\addlegendimage{no markers, red}
    \end{axis}
    \end{tikzpicture}
    \caption{Przebieg procesu sterowanego za pomocą regulatora z parametrami $D = 155$, $N = 70$, $N_{\mathrm{u}} = 35$, $\lambda = 100$}
    \label{zad4_Nu_1_y}
\end{figure}

\begin{figure}[b]
    \centering
    \begin{tikzpicture}
    \begin{axis}[
    width=0.98\textwidth,
    xmin=0,xmax=500,ymin=0, ymax=1,
    xlabel={$k$},
    ylabel={$u[k]$},
    %xtick={0, 0.4, 0.5, 0.6, 0.7},
    %ytick={3.5, 3.75, 4, 4.25, 4.5},
    legend pos=south east,
    y tick label style={/pgf/number format/1000 sep=},
    ]
    \addplot[const plot, blue, semithick] file{../data/zad4/zad4_input_N_70_Nu_35_lambda_100.csv};
    \legend{$u[k]$}
    \end{axis}
    \end{tikzpicture}
    \caption{Przebieg sygnału sterującego przy parametrach parametrach: $D = 155$, $N = 70$, $N_{\mathrm{u}} = 35$, $\lambda = 100$}
    \label{zad4_Nu_1_u}
\end{figure}
\FloatBarrier

%%%

\begin{figure}[t]
    \centering
    \begin{tikzpicture}
    \begin{axis}[
    width=0.98\textwidth,
    xmin=0,xmax=500,ymin=0, ymax=2,
    xlabel={$k$},
    ylabel={$y[k]$},
    %xtick={0, 100, 200, 300, 400, 500},
    %ytick={3.9, 4, 4.1, 4.2, 4.3, 4.4},
    legend pos=south east,
    y tick label style={/pgf/number format/1000 sep=},
    ]
    \addplot[blue, semithick] file{../data/zad4/zad4_stpt_N_70_Nu_2_lambda_100.csv};
    \addplot[red, semithick] file{../data/zad4/zad4_output_N_70_Nu_2_lambda_100.csv};
    \addlegendentry{$y^{\mathrm{zad}}[k]$},
    \addlegendentry{$y[k]$},
    \addlegendimage{no markers, blue}
	\addlegendimage{no markers, red}
    \end{axis}
    \end{tikzpicture}
    \caption{Przebieg procesu sterowanego za pomocą regulatora z parametrami $D = 155$, $N = 70$, $N_{\mathrm{u}} = 2$, $\lambda = 100$}
    \label{zad4_Nu_2_y}
\end{figure}

\begin{figure}[b]
    \centering
    \begin{tikzpicture}
    \begin{axis}[
    width=0.98\textwidth,
    xmin=0,xmax=500,ymin=0, ymax=1,
    xlabel={$k$},
    ylabel={$u[k]$},
    %xtick={0, 0.4, 0.5, 0.6, 0.7},
    %ytick={3.5, 3.75, 4, 4.25, 4.5},
    legend pos=south east,
    y tick label style={/pgf/number format/1000 sep=},
    ]
    \addplot[const plot, blue, semithick] file{../data/zad4/zad4_input_N_70_Nu_2_lambda_100.csv};
    \legend{$u[k]$}
    \end{axis}
    \end{tikzpicture}
    \caption{Przebieg sygnału sterującego przy parametrach parametrach: $D = 155$, $N = 70$, $N_{\mathrm{u}} = 2$, $\lambda = 100$}
    \label{zad4_Nu_2_u}
\end{figure}
\FloatBarrier

%%%

\begin{figure}[t]
    \centering
    \begin{tikzpicture}
    \begin{axis}[
    width=0.98\textwidth,
    xmin=0,xmax=500,ymin=0, ymax=2,
    xlabel={$k$},
    ylabel={$y[k]$},
    %xtick={0, 100, 200, 300, 400, 500},
    %ytick={3.9, 4, 4.1, 4.2, 4.3, 4.4},
    legend pos=south east,
    y tick label style={/pgf/number format/1000 sep=},
    ]
    \addplot[blue, semithick] file{../data/zad4/zad4_stpt_N_70_Nu_1_lambda_100.csv};
    \addplot[red, semithick] file{../data/zad4/zad4_output_N_70_Nu_1_lambda_100.csv};
    \addlegendentry{$y^{\mathrm{zad}}[k]$},
    \addlegendentry{$y[k]$},
    \addlegendimage{no markers, blue}
	\addlegendimage{no markers, red}
    \end{axis}
    \end{tikzpicture}
    \caption{Przebieg procesu sterowanego za pomocą regulatora z parametrami $D = 155$, $N = 70$, $N_{\mathrm{u}} = 1$, $\lambda = 100$}
    \label{zad4_Nu_3_y}
\end{figure}

\begin{figure}[b]
    \centering
    \begin{tikzpicture}
    \begin{axis}[
    width=0.98\textwidth,
    xmin=0,xmax=500,ymin=0, ymax=1,
    xlabel={$k$},
    ylabel={$u[k]$},
    %xtick={0, 0.4, 0.5, 0.6, 0.7},
    %ytick={3.5, 3.75, 4, 4.25, 4.5},
    legend pos=south east,
    y tick label style={/pgf/number format/1000 sep=},
    ]
    \addplot[const plot, blue, semithick] file{../data/zad4/zad4_input_N_70_Nu_1_lambda_100.csv};
    \legend{$u[k]$}
    \end{axis}
    \end{tikzpicture}
    \caption{Przebieg sygnału sterującego przy parametrach parametrach: $D = 155$, $N = 70$, $N_{\mathrm{u}} = 1$, $\lambda = 100$}
    \label{zad4_Nu_3_u}
\end{figure}
\FloatBarrier

\section{Dobór parametru $\lambda$}
Symulowany proces w bardzo spokojny sposób osiąga wartość zadaną, jednak wydaje się że czas regulacji
tego obiektu mógłby być krótszy. Taki efekt możemy osiągnąć poprzez zwiększenie zmienności
sygnału sterującego. Aby tego dokonać, sukcesywnie zmniejszaliśmy parametr kary $\lambda$, 
badając wskaźnik jakości.

\subsubsection{Wartość parametru $\lambda = \num{50}$}
Zmniejszenie parametru $\lambda$ o połowę nie spowodowało drastycznej zmiany przebiegów sygnałów
wejściowych i wyjściowych. Wyniki symulacji zostały przedstawione poniżej na rysunkach \ref{zad4_lambda_1_y}
i \ref{zad4_lambda_1_u}. Mimo że przebiegi na pierwszy rzut oka wyglądają tak samo, to wartość 
wskaźnika jakości zmalała do wartości $E = \num{15,509}$ co skłoniło nas do dalszego zmniejszania
tego parametru.

\subsubsection{Wartość parametru $\lambda = \num{20}$}
Kolejną symulacje przeprowadzilismy dla regulatora DMC z parametrem $\lambda = \num{20}$. Wyniki tej
symulacji zostały przedstawione na rysunkach \ref{zad4_lambda_2_y} i \ref{zad4_lambda_2_u}. Dla tej 
próby wskaźnik jakości jeszcze bardziej zmalał do wartości $E = \num{15,336}$.

\subsubsection{Wartość parametru $\lambda = \num{10}$}
Idąc dalej, zmniejszyliśmy aktualną wartość parametru o połowę do wartości $\lambda = \num{10}$. Symulacja
tego regulatora została zaprezentowana na wykresach \ref{zad4_lambda_3_y} i \ref{zad4_lambda_3_u}. Wartość wskaźnika
jakości wyniosła $E = \num{15,318}$, co można uznać za nieznaczną poprawę.

\subsubsection{Wartość parametru $\lambda = \num{0,01}$}
Aby przetestować jak bardzo jeszcze możemy zmniejszyć wartość parametru $\lambda$, postanowiliśmy
zmniejszyć ją o dwa rzędy wielkości do wartości $\lambda = \num{0,01}$. Przyniosło to korzyść w postaci
zmniejszenia wskaźnika jakości $E = \num{15,27}$, jednak spowodowało to że sygnał sterujący zmienia się 
w bardzo gwałtowny sposób. Idąc na inżynierski kompromis, mając na uwadze że ten regulator będzie wystawiany
na oddziaływanie zakłóceń postanowiliśmy wrócić do poprzedniej wartości tego parametru i zakończyć proces strojenia.
Przebieg wyjścia procesu został przedstawiony na rysunku \ref{zad4_lambda_4_y}. Przebieg sterowania, z widocznym 
większym pikiem sterowania został pokazany na wykresie \ref{zad4_lambda_4_u}.

\section{Ostateczny zestaw parametrów}
Ostatecznie zdecydowaliśmy się na następujący zestaw parametrów algorytmu regulacji DMC.

\begin{center}
    $N = \num{70}$ \hspace{1cm} $N_{\mathrm{u}} = \num{1}$ \hspace{1cm} $\lambda = \num{10}$ 
\end{center}



\begin{figure}[t]
    \centering
    \begin{tikzpicture}
    \begin{axis}[
    width=0.98\textwidth,
    xmin=0,xmax=500,ymin=0, ymax=2,
    xlabel={$k$},
    ylabel={$y[k]$},
    %xtick={0, 100, 200, 300, 400, 500},
    %ytick={3.9, 4, 4.1, 4.2, 4.3, 4.4},
    legend pos=south east,
    y tick label style={/pgf/number format/1000 sep=},
    ]
    \addplot[blue, semithick] file{../data/zad4/zad4_stpt_N_70_Nu_1_lambda_50.csv};
    \addplot[red, semithick] file{../data/zad4/zad4_output_N_70_Nu_1_lambda_50.csv};
    \addlegendentry{$y^{\mathrm{zad}}[k]$},
    \addlegendentry{$y[k]$},
    \addlegendimage{no markers, blue}
	\addlegendimage{no markers, red}
    \end{axis}
    \end{tikzpicture}
    \caption{Przebieg procesu sterowanego za pomocą regulatora z parametrami $D = 155$, $N = 70$, $N_{\mathrm{u}} = 1$, $\lambda = 50$}
    \label{zad4_lambda_1_y}
\end{figure}

\begin{figure}[b]
    \centering
    \begin{tikzpicture}
    \begin{axis}[
    width=0.98\textwidth,
    xmin=0,xmax=500,ymin=0, ymax=1,
    xlabel={$k$},
    ylabel={$u[k]$},
    %xtick={0, 0.4, 0.5, 0.6, 0.7},
    %ytick={3.5, 3.75, 4, 4.25, 4.5},
    legend pos=south east,
    y tick label style={/pgf/number format/1000 sep=},
    ]
    \addplot[const plot, blue, semithick] file{../data/zad4/zad4_input_N_70_Nu_1_lambda_50.csv};
    \legend{$u[k]$}
    \end{axis}
    \end{tikzpicture}
    \caption{Przebieg sygnału sterującego przy parametrach parametrach: $D = 155$, $N = 70$, $N_{\mathrm{u}} = 1$, $\lambda = 50$}
    \label{zad4_lambda_1_u}
\end{figure}
\FloatBarrier

%%%

\begin{figure}[t]
    \centering
    \begin{tikzpicture}
    \begin{axis}[
    width=0.98\textwidth,
    xmin=0,xmax=500,ymin=0, ymax=2,
    xlabel={$k$},
    ylabel={$y[k]$},
    %xtick={0, 100, 200, 300, 400, 500},
    %ytick={3.9, 4, 4.1, 4.2, 4.3, 4.4},
    legend pos=south east,
    y tick label style={/pgf/number format/1000 sep=},
    ]
    \addplot[blue, semithick] file{../data/zad4/zad4_stpt_N_70_Nu_1_lambda_20.csv};
    \addplot[red, semithick] file{../data/zad4/zad4_output_N_70_Nu_1_lambda_20.csv};
    \addlegendentry{$y^{\mathrm{zad}}[k]$},
    \addlegendentry{$y[k]$},
    \addlegendimage{no markers, blue}
	\addlegendimage{no markers, red}
    \end{axis}
    \end{tikzpicture}
    \caption{Przebieg procesu sterowanego za pomocą regulatora z parametrami $D = 155$, $N = 70$, $N_{\mathrm{u}} = 1$, $\lambda = 20$}
    \label{zad4_lambda_2_y}
\end{figure}

\begin{figure}[b]
    \centering
    \begin{tikzpicture}
    \begin{axis}[
    width=0.98\textwidth,
    xmin=0,xmax=500,ymin=0, ymax=1,
    xlabel={$k$},
    ylabel={$u[k]$},
    %xtick={0, 0.4, 0.5, 0.6, 0.7},
    %ytick={3.5, 3.75, 4, 4.25, 4.5},
    legend pos=south east,
    y tick label style={/pgf/number format/1000 sep=},
    ]
    \addplot[const plot, blue, semithick] file{../data/zad4/zad4_input_N_70_Nu_1_lambda_20.csv};
    \legend{$u[k]$}
    \end{axis}
    \end{tikzpicture}
    \caption{Przebieg sygnału sterującego przy parametrach parametrach: $D = 155$, $N = 70$, $N_{\mathrm{u}} = 1$, $\lambda = 20$}
    \label{zad4_lambda_2_u}
\end{figure}
\FloatBarrier

%%%

\begin{figure}[t]
    \centering
    \begin{tikzpicture}
    \begin{axis}[
    width=0.98\textwidth,
    xmin=0,xmax=500,ymin=0, ymax=2,
    xlabel={$k$},
    ylabel={$y[k]$},
    %xtick={0, 100, 200, 300, 400, 500},
    %ytick={3.9, 4, 4.1, 4.2, 4.3, 4.4},
    legend pos=south east,
    y tick label style={/pgf/number format/1000 sep=},
    ]
    \addplot[blue, semithick] file{../data/zad4/zad4_stpt_N_70_Nu_1_lambda_10.csv};
    \addplot[red, semithick] file{../data/zad4/zad4_output_N_70_Nu_1_lambda_10.csv};
    \addlegendentry{$y^{\mathrm{zad}}[k]$},
    \addlegendentry{$y[k]$},
    \addlegendimage{no markers, blue}
	\addlegendimage{no markers, red}
    \end{axis}
    \end{tikzpicture}
    \caption{Przebieg procesu sterowanego za pomocą regulatora z parametrami $D = 155$, $N = 70$, $N_{\mathrm{u}} = 1$, $\lambda = 10$}
    \label{zad4_lambda_3_y}
\end{figure}

\begin{figure}[b]
    \centering
    \begin{tikzpicture}
    \begin{axis}[
    width=0.98\textwidth,
    xmin=0,xmax=500,ymin=0, ymax=1,
    xlabel={$k$},
    ylabel={$u[k]$},
    %xtick={0, 0.4, 0.5, 0.6, 0.7},
    %ytick={3.5, 3.75, 4, 4.25, 4.5},
    legend pos=south east,
    y tick label style={/pgf/number format/1000 sep=},
    ]
    \addplot[const plot, blue, semithick] file{../data/zad4/zad4_input_N_70_Nu_1_lambda_10.csv};
    \legend{$u[k]$}
    \end{axis}
    \end{tikzpicture}
    \caption{Przebieg sygnału sterującego przy parametrach parametrach: $D = 155$, $N = 70$, $N_{\mathrm{u}} = 1$, $\lambda = 10$}
    \label{zad4_lambda_3_u}
\end{figure}
\FloatBarrier

%%%

\begin{figure}[t]
    \centering
    \begin{tikzpicture}
    \begin{axis}[
    width=0.98\textwidth,
    xmin=0,xmax=500,ymin=0, ymax=2,
    xlabel={$k$},
    ylabel={$y[k]$},
    %xtick={0, 100, 200, 300, 400, 500},
    %ytick={3.9, 4, 4.1, 4.2, 4.3, 4.4},
    legend pos=south east,
    y tick label style={/pgf/number format/1000 sep=},
    ]
    \addplot[blue, semithick] file{../data/zad4/zad4_stpt_N_70_Nu_1_lambda_0.01.csv};
    \addplot[red, semithick] file{../data/zad4/zad4_output_N_70_Nu_1_lambda_0.01.csv};
    \addlegendentry{$y^{\mathrm{zad}}[k]$},
    \addlegendentry{$y[k]$},
    \addlegendimage{no markers, blue}
	\addlegendimage{no markers, red}
    \end{axis}
    \end{tikzpicture}
    \caption{Przebieg procesu sterowanego za pomocą regulatora z parametrami $D = 155$, $N = 70$, $N_{\mathrm{u}} = 1$, $\lambda = \num{0.01}$}
    \label{zad4_lambda_4_y}
\end{figure}

\begin{figure}[b]
    \centering
    \begin{tikzpicture}
    \begin{axis}[
    width=0.98\textwidth,
    xmin=0,xmax=500,ymin=0, ymax=1,
    xlabel={$k$},
    ylabel={$u[k]$},
    %xtick={0, 0.4, 0.5, 0.6, 0.7},
    %ytick={3.5, 3.75, 4, 4.25, 4.5},
    legend pos=south east,
    y tick label style={/pgf/number format/1000 sep=},
    ]
    \addplot[const plot, blue, semithick] file{../data/zad4/zad4_input_N_70_Nu_1_lambda_0.01.csv};
    \legend{$u[k]$}
    \end{axis}
    \end{tikzpicture}
    \caption{Przebieg sygnału sterującego przy parametrach parametrach: $D = 155$, $N = 70$, $N_{\mathrm{u}} = 1$, $\lambda = \num{0.01}$}
    \label{zad4_lambda_4_u}
\end{figure}
\FloatBarrier