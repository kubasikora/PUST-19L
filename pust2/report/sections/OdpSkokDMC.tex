\section{Aproksymacja odpowiedzi skokowych}
\label{zad3_lab_opis}
\subsection{Zestaw liczb \texorpdfstring{$s_{1}, s_{2}, ...$ }{TEXT} }
Rysunek \ref{lab3_odp_skok_s} prezentuje przebieg znormalizowanej odpowiedzi skokowej dla zestawu liczb $s_{1}, s_{2}, ...$ . Postawionym przed nami zadaniem było przeprowadzenie aproksymacji odpowiedzi skokowej dowolnym narzędziem. W kolejnych punktach opisujemy jak tego dokonaliśmy i jakie uzyskaliśmy rezultaty. 
\begin{figure}[b]
    \centering
    \begin{tikzpicture}
    \begin{axis}[
    width=0.98\textwidth,
    xmin=8,xmax=500,ymin=0.0,ymax=0.35,
    xlabel={$k$},
    ylabel={$s_{k}$},
    legend pos=south west,
    y tick label style={/pgf/number format/1000 sep=},
    ] 
	\addlegendimage{no markers,red}
	\addlegendimage{no markers,blue}
	\addlegendimage{no markers,green}
    \addplot[red, semithick, thick] file{../data/lab/zad3/zad3_norm_s.csv};
    \end{axis}
    \end{tikzpicture}
    \caption{Postać znormalizowanej odpowiedzi skokowej rzeczywistego procesu}
    \label{lab3_odp_skok_s}
\end{figure}
\FloatBarrier

\begin{figure}[t]
    \centering
    \begin{tikzpicture}
    \begin{axis}[
    width=0.98\textwidth,
    xmin=8,xmax=500,ymin=0.0,ymax=0.35,
    xlabel={$k$},
    ylabel={$s_{k}$},
    legend pos=south east,
    y tick label style={/pgf/number format/1000 sep=},
    ] 
    \addlegendentry{Model}
    \addlegendentry{Obiekt}
	\addlegendimage{no markers,red}
	\addlegendimage{no markers,blue}
	\addlegendimage{no markers,green}
    \addplot[red, semithick, thick] file{../data/lab/zad3/zad3_norm_s.csv};
    \addplot[blue, semithick, thick] file{../data/lab/zad3/apro_1_s.csv};
    \end{axis}
    \end{tikzpicture}
    \caption{Porównanie modelu obiektu otrzymanego przy pomocy pakietu System Identifiaction
z rzeczywista odpowiedzia skokowa}
    \label{lab3_apro_1}
\end{figure}

\begin{figure}[b]
    \centering
    \begin{tikzpicture}
    
    \begin{axis}[
    width=0.98\textwidth,
    xmin=8,xmax=500,ymin=0.0,ymax=0.35,
    xlabel={$k$},
    ylabel={$s_{k}$},
    legend pos=south east,
    y tick label style={/pgf/number format/1000 sep=},
    ]     
    \addlegendentry{Model}
    \addlegendentry{Obiekt}
	\addlegendimage{no markers,red}
	\addlegendimage{no markers,blue}
	\addlegendimage{no markers,green}
    \addplot[red, semithick, thick] file{../data/lab/zad3/zad3_norm_s.csv};
    \addplot[blue, semithick, thick] file{../data/lab/zad3/apro_2_s.csv};
    \end{axis}
    \end{tikzpicture}
    \caption{Porównanie modelu obiektu otrzymanego przy pomocy pakietu System Identifiaction
z rzeczywista odpowiedzia skokowa}
    \label{lab3__apro_2}
\end{figure}

\begin{figure}[t]
    \centering
    \begin{tikzpicture}
    
    \begin{axis}[
    width=0.98\textwidth,
    xmin=8,xmax=500,ymin=0.0,ymax=0.35,
    xlabel={$k$},
    ylabel={$s_{k}$},
    legend pos=south east,
    y tick label style={/pgf/number format/1000 sep=},
    ]     
    \addlegendentry{Model}
    \addlegendentry{Obiekt}
	\addlegendimage{no markers,red}
	\addlegendimage{no markers,blue}
	\addlegendimage{no markers,green}
    \addplot[red, semithick, thick] file{../data/lab/zad3/zad3_norm_s.csv};
    \addplot[blue, semithick, thick] file{../data/lab/zad3/apro_3_s.csv};
    \end{axis}
    \end{tikzpicture}
    \caption{Porównanie modelu obiektu otrzymanego przy pomocy pakietu System Identifiaction
z rzeczywista odpowiedzia skokowa}
    \label{lab3__apro_3}
\end{figure}
\FloatBarrier