\section{Aproksymacja odpowiedzi skokowych}
\label{zad3_lab_opis}
Rysunek \ref{lab3_odp_skok_s} prezentuje przebieg znormalizowanej odpowiedzi skokowej dla zestawu liczb wejście - wyjście. Postawionym przed nami zadaniem było przeprowadzenie aproksymacji odpowiedzi skokowej dowolnym narzędziem. W kolejnych punktach opisujemy jak tego dokonaliśmy i jakie uzyskaliśmy rezultaty. 
\begin{figure}[b]
    \centering
    \begin{tikzpicture}
    \begin{axis}[
    width=0.98\textwidth,
    xmin=8,xmax=500,ymin=0.0,ymax=0.35,
    xlabel={$k$},
    ylabel={$s_{k}$},
    legend pos=south west,
    y tick label style={/pgf/number format/1000 sep=},
    ] 
	\addlegendimage{no markers,red}
	\addlegendimage{no markers,blue}
	\addlegendimage{no markers,green}
    \addplot[red, semithick, thick] file{../data/lab/zad3/zad3_norm_s.csv};
    \end{axis}
    \end{tikzpicture}
    \caption{Postać znormalizowanej odpowiedzi skokowej rzeczywistego procesu}
    \label{lab3_odp_skok_s}
\end{figure}
\FloatBarrier

\subsection{Wykorzystanie zaawansowanych narzędzi pakietu MATLAB do aproksymacji odpowiedzi skokowych dla zestawu wejście - wyjście}
W celu dokładnego zamodelowania badanego obiektu posłużylismy
się zaawansowanym toolboxem pakietu MATLAB System Identification.
Pakiet ten pozwala na znalezienie parametrów różnych modeli. Po załadowaniu
odpowiedzi skokowej, wybraliśmy Transfer Function Model jako rodzaj zadanego
modelu a następnie po chwili mieliśmy gotowy obiekt typu idtf. Kryterium, które braliśmy pod uwagę podczas oceny była wartość dopasowania modelu do obiektu wyrażona w procentach (fit to estimation data). \\
\indent Parametry, którymi manipulowaliśmy to ilość zakładanych biegunów i zer w transmitancji. Podczas eksperymentów w trakcie laboratorium zaobserwowaliśmy, że obiekt charakteryzuje się małym opóźnieniem, dlatego ustawiliśmy wartość czasu opóźnienia na $T_{\mathrm{d}}=5$ dla wszystkich aproksymacji. 
\\
\subsubsection{1. }
Dla pierwszego przypadku wybraliśmy ilość biegunów oraz zer równą dwa. Parametry te pozwoliły na uzyskanie dopasowania o wartości $ \num{95.91} $ \text{\%}. Na rysunku \ref{lab3_apro_1} możemy zobaczyć jak przebiega dopasowanie modelu do obiektu. Widzimy, że charakterystyki praktycznie się pokrywają. 

\subsubsection{2. }
Szukając jeszcze lepszego dopasowania postanowiliśmy zmniejszyć ilość zer do jednego, a liczbę biegunów pozostawić na tym samym poziomie. Dzięki temu udało nam sie uzyskać minimalnie lepszy model ponieważ wielkość dopasowania wzrosła o $\num{0.03}$ przez co po sumowaniu wartość dopasowania wynosi $\num{95.94}$. Wyniki w postaci wykresu możemy zaobserwować możemy na rysunku \ref{lab3__apro_2}. Po dłuższej analizie, możemy zauważyć że model jest minimalnie lepiej dopasowany. 
\subsubsection{3. }
DO SPRAWDZENIA 
\begin{figure}[t]
    \centering
    \begin{tikzpicture}
    \begin{axis}[
    width=0.98\textwidth,
    xmin=8,xmax=500,ymin=0.0,ymax=0.35,
    xlabel={$k$},
    ylabel={$s_{k}$},
    legend pos=south east,
    y tick label style={/pgf/number format/1000 sep=},
    ] 
    \addlegendentry{Obiekt}
    \addlegendentry{Model}
	\addlegendimage{no markers,red}
	\addlegendimage{no markers,blue}
	\addlegendimage{no markers,green}
    \addplot[red, semithick, thick] file{../data/lab/zad3/zad3_norm_s.csv};
    \addplot[blue, semithick, thick] file{../data/lab/zad3/apro_1_s.csv};
    \end{axis}
    \end{tikzpicture}
    \caption{Porównanie modelu obiektu otrzymanego przy pomocy pakietu System Identifiaction
z rzeczywista odpowiedzia skokowa}
    \label{lab3_apro_1}
\end{figure}

\begin{figure}[b]
    \centering
    \begin{tikzpicture}
    
    \begin{axis}[
    width=0.98\textwidth,
    xmin=8,xmax=500,ymin=0.0,ymax=0.35,
    xlabel={$k$},
    ylabel={$s_{k}$},
    legend pos=south east,
    y tick label style={/pgf/number format/1000 sep=},
    ]     
    \addlegendentry{Obiekt}
    \addlegendentry{Model}
	\addlegendimage{no markers,red}
	\addlegendimage{no markers,blue}
	\addlegendimage{no markers,green}
    \addplot[red, semithick, thick] file{../data/lab/zad3/zad3_norm_s.csv};
    \addplot[blue, semithick, thick] file{../data/lab/zad3/apro_2_s.csv};
    \end{axis}
    \end{tikzpicture}
    \caption{Porównanie modelu obiektu otrzymanego przy pomocy pakietu System Identifiaction
z rzeczywista odpowiedzia skokowa}
    \label{lab3__apro_2}
\end{figure}

\begin{figure}[t]
    \centering
    \begin{tikzpicture}
    
    \begin{axis}[
    width=0.98\textwidth,
    xmin=8,xmax=500,ymin=0.0,ymax=0.35,
    xlabel={$k$},
    ylabel={$s_{k}$},
    legend pos=south east,
    y tick label style={/pgf/number format/1000 sep=},
    ]     
    \addlegendentry{Obiekt}
    \addlegendentry{Model}
	\addlegendimage{no markers,red}
	\addlegendimage{no markers,blue}
	\addlegendimage{no markers,green}
    \addplot[red, semithick, thick] file{../data/lab/zad3/zad3_norm_s.csv};
    \addplot[blue, semithick, thick] file{../data/lab/zad3/apro_9_s.csv};
    \end{axis}
    \end{tikzpicture}
    \caption{Porównanie modelu obiektu otrzymanego przy pomocy pakietu System Identifiaction
z rzeczywista odpowiedzia skokowa}
    \label{lab3__apro_3}
\end{figure}
\FloatBarrier

\subsection{Wykorzystanie zaawansowanych narzędzi pakietu MATLAB do aproksymacji odpowiedzi skokowych dla zestawu zakłócenie - wyjście}
!!!FRAGMENT DO OPISANIA!!!
\\I rysunek to znormalizowana odpowiedź skokowa
\\ II rysunek to 2 bieguny, jedno zero (P:2 i Z:1) - tf1
\\ III rysunek to 2 bieguny, 2 zera (P:2 i Z:2) - tf6
\\ IV rysunek to 3 bieguny, 2 zera (P:2 i Z:1) - tf11

\\ tf3 to dodatkowe dane


\begin{figure}[b]
    \centering
    \begin{tikzpicture}
    \begin{axis}[
    width=0.98\textwidth,
    xmin=8,xmax=500,ymin=-3,ymax=0.2,
    xlabel={$k$},
    ylabel={$s_{k}$},
    legend pos=south west,
    y tick label style={/pgf/number format/1000 sep=},
    ] 
	\addlegendimage{no markers,red}
	\addlegendimage{no markers,blue}
	\addlegendimage{no markers,green}
    \addplot[red, semithick, thick] file{../data/lab/zad3/norm_zakl_sz.csv};
    \end{axis}
    \end{tikzpicture}
    \caption{Postać znormalizowanej odpowiedzi skokowej rzeczywistego procesu}
    \label{lab3_odp_skok_s}
\end{figure}
\FloatBarrier

\begin{figure}[t]
    \centering
    \begin{tikzpicture}
    \begin{axis}[
    width=0.98\textwidth,
    xmin=8,xmax=500,ymin=-3,ymax=0.2,
    xlabel={$k$},
    ylabel={$s_{k}$},
    legend pos=north east,
    y tick label style={/pgf/number format/1000 sep=},
    ] 
    \addlegendentry{Obiekt}
    \addlegendentry{Model}
	\addlegendimage{no markers,red}
	\addlegendimage{no markers,blue}
	\addlegendimage{no markers,green}
    \addplot[red, semithick, thick] file{../data/lab/zad3/norm_zakl_sz.csv};
    \addplot[blue, semithick, thick] file{../data/lab/zad3/apro_1_sz.csv};
    \end{axis}
    \end{tikzpicture}
    \caption{Porównanie modelu obiektu otrzymanego przy pomocy pakietu System Identifiaction
z rzeczywista odpowiedzia skokowa}
    \label{lab3_apro_1}
\end{figure}

\begin{figure}[b]
    \centering
    \begin{tikzpicture}
    
    \begin{axis}[
    width=0.98\textwidth,
    xmin=8,xmax=500,ymin=-22,ymax=0.2,
    xlabel={$k$},
    ylabel={$s_{k}$},
    legend pos=south west,
    y tick label style={/pgf/number format/1000 sep=},
    ]     
    \addlegendentry{Obiekt}
    \addlegendentry{Model}
	\addlegendimage{no markers,red}
	\addlegendimage{no markers,blue}
	\addlegendimage{no markers,green}
    \addplot[red, semithick, thick] file{../data/lab/zad3/norm_zakl_sz.csv};
    \addplot[blue, semithick, thick] file{../data/lab/zad3/apro_2_sz.csv};
    \end{axis}
    \end{tikzpicture}
    \caption{Porównanie modelu obiektu otrzymanego przy pomocy pakietu System Identifiaction
z rzeczywista odpowiedzia skokowa}
    \label{lab3__apro_2}
\end{figure}

\begin{figure}[t]
    \centering
    \begin{tikzpicture}
    
    \begin{axis}[
    width=0.98\textwidth,
    xmin=8,xmax=500,ymin=-3,ymax=0.2,
    xlabel={$k$},
    ylabel={$s_{k}$},
    legend pos=north east,
    y tick label style={/pgf/number format/1000 sep=},
    ]     
    \addlegendentry{Obiekt}
    \addlegendentry{Model}
	\addlegendimage{no markers,red}
	\addlegendimage{no markers,blue}
	\addlegendimage{no markers,green}
    \addplot[red, semithick, thick] file{../data/lab/zad3/norm_zakl_sz.csv};
    \addplot[blue, semithick, thick] file{../data/lab/zad3/apro_3_sz.csv};
    \end{axis}
    \end{tikzpicture}
    \caption{Porównanie modelu obiektu otrzymanego przy pomocy pakietu System Identifiaction
z rzeczywista odpowiedzia skokowa}
    \label{lab3__apro_3}
\end{figure}
\FloatBarrier