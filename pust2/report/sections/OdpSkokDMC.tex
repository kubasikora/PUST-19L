\section{Aproksymacja odpowiedzi skokowych}
\label{zad3_lab_opis}
\subsection{Normalizacja odpowiedzi skokowej dla zestawu wejście - wyjście}
Rysunek \ref{lab3_odp_skok_s} prezentuje przebieg znormalizowanej odpowiedzi skokowej dla zestawu liczb wejście - wyjście. Postawionym przed nami zadaniem było przeprowadzenie aproksymacji odpowiedzi skokowej dowolnym narzędziem. W kolejnych punktach opisujemy jak tego dokonaliśmy i jakie uzyskaliśmy rezultaty. 
\begin{figure}[b]
    \centering
    \begin{tikzpicture}
    \begin{axis}[
    width=0.98\textwidth,
    xmin=8,xmax=500,ymin=0.0,ymax=0.35,
    xlabel={$k$},
    ylabel={$s_{k}$},
    legend pos=south west,
    y tick label style={/pgf/number format/1000 sep=},
    ] 
	\addlegendimage{no markers,red}
	\addlegendimage{no markers,blue}
	\addlegendimage{no markers,green}
    \addplot[red, semithick, thick] file{../data/lab/zad3/zad3_norm_s.csv};
    \end{axis}
    \end{tikzpicture}
    \caption{Postać znormalizowanej odpowiedzi skokowej rzeczywistego procesu}
    \label{lab3_odp_skok_s}
\end{figure}
\FloatBarrier

\subsection{Wykorzystanie zaawansowanych narzędzi pakietu MATLAB do aproksymacji odpowiedzi skokowych dla zestawu wejście - wyjście}
W celu dokładnego zamodelowania badanego obiektu posłużyliśmy
się zaawansowanym toolboxem pakietu MATLAB System Identification.
Pakiet ten pozwala na znalezienie parametrów różnych modeli. Po załadowaniu
odpowiedzi skokowej, wybraliśmy Transfer Function Model jako rodzaj zadanego
modelu a następnie po chwili mieliśmy gotowy obiekt typu idtf. Kryterium, które braliśmy pod uwagę podczas oceny była wartość dopasowania modelu do obiektu wyrażona w procentach (fit to estimation data). \\
\indent Parametry, którymi manipulowaliśmy to ilość zakładanych biegunów i zer w transmitancji. Podczas eksperymentów w trakcie laboratorium zaobserwowaliśmy, że obiekt charakteryzuje się małym opóźnieniem, dlatego ustawiliśmy wartość czasu opóźnienia na $T_{\mathrm{d}}=5$ dla wszystkich aproksymacji. 
\\
\subsubsection{Wyniki symulacji dla modelu z dwoma biegunami i dwoma zerami}
Dla pierwszego przypadku wybraliśmy ilość biegunów oraz zer równą dwa. Parametry te pozwoliły na uzyskanie dopasowania o wartości $ \num{95.91} $ \text{\%}. Na rysunku \ref{lab3_apro_1} możemy zobaczyć jak przebiega dopasowanie modelu do obiektu. Widzimy, że charakterystyki praktycznie się pokrywają. 

\subsubsection{Wyniki symulacji dla modelu z dwoma biegunami i jednym zerem}
Szukając jeszcze lepszego dopasowania postanowiliśmy zmniejszyć ilość zer do jednego, a liczbę biegunów pozostawić na tym samym poziomie. Dzięki temu udało nam sie uzyskać minimalnie lepszy model ponieważ wielkość dopasowania wzrosła o $\num{0.03}$ \text{\%} przez co po sumowaniu wartość dopasowania wynosi $\num{95.94}$ \text{\%}. Wyniki w postaci wykresu możemy zaobserwować możemy na rysunku \ref{lab3__apro_2}. Po dłuższej analizie, możemy zauważyć że model jest minimalnie lepiej dopasowany. 

\begin{figure}[t]
    \centering
    \begin{tikzpicture}
    \begin{axis}[
    width=0.98\textwidth,
    xmin=8,xmax=500,ymin=0.0,ymax=0.35,
    xlabel={$k$},
    ylabel={$s_{k}$},
    legend pos=south east,
    y tick label style={/pgf/number format/1000 sep=},
    ] 
    \addlegendentry{Obiekt}
    \addlegendentry{Model}
	\addlegendimage{no markers,red}
	\addlegendimage{no markers,blue}
	\addlegendimage{no markers,green}
    \addplot[red, semithick, thick] file{../data/lab/zad3/zad3_norm_s.csv};
    \addplot[blue, semithick, thick] file{../data/lab/zad3/apro_1_s.csv};
    \end{axis}
    \end{tikzpicture}
    \caption{Porównanie modelu obiektu otrzymanego przy pomocy pakietu System Identifiaction
z rzeczywista odpowiedzia skokowa}
    \label{lab3_apro_1}
\end{figure}

\begin{figure}[b]
    \centering
    \begin{tikzpicture}
    
    \begin{axis}[
    width=0.98\textwidth,
    xmin=8,xmax=500,ymin=0.0,ymax=0.35,
    xlabel={$k$},
    ylabel={$s_{k}$},
    legend pos=south east,
    y tick label style={/pgf/number format/1000 sep=},
    ]     
    \addlegendentry{Obiekt}
    \addlegendentry{Model}
	\addlegendimage{no markers,red}
	\addlegendimage{no markers,blue}
	\addlegendimage{no markers,green}
    \addplot[red, semithick, thick] file{../data/lab/zad3/zad3_norm_s.csv};
    \addplot[blue, semithick, thick] file{../data/lab/zad3/apro_2_s.csv};
    \end{axis}
    \end{tikzpicture}
    \caption{Porównanie modelu obiektu otrzymanego przy pomocy pakietu System Identifiaction
z rzeczywista odpowiedzia skokowa}
    \label{lab3__apro_2}
\end{figure}
\FloatBarrier

\subsection{Normalizacja odpowiedzi skokowej dla zestawu zakłócenie - wyjście}
Na rysunku \ref{lab3_odp_skok_sz} możemy zobaczyć przebieg znormalizowanej odpowiedzi skokowej dla zestawu zakłócenie - wyjście. Tak samo jak dla zestawu wejścia - wyjście dokonaliśmy aproksymacji odpowiedzi skokowej narzędziem Transfer Function Model z programu MATLAB. Wyniki naszej pracy prezentujemy w dalszych punktach. 

\begin{figure}[b]
    \centering
    \begin{tikzpicture}
    \begin{axis}[
    width=0.98\textwidth,
    xmin=0,xmax=400,ymin=-0.01,ymax=0.15,
    xlabel={$k$},
    ylabel={$s_{k}$},
    legend pos=south west,
    y tick label style={/pgf/number format/1000 sep=},
    ] 
	\addlegendimage{no markers,red}
	\addlegendimage{no markers,blue}
	\addlegendimage{no markers,green}
    \addplot[red, semithick, thick] file{../data/lab/zad3/normDoWykresuSZ.csv};
    \end{axis}
    \end{tikzpicture}
    \caption{Postać znormalizowanej odpowiedzi skokowej rzeczywistego procesu}
    \label{lab3_odp_skok_sz}
\end{figure}
\FloatBarrier

\subsection{Wykorzystanie zaawansowanych narzędzi pakietu MATLAB do aproksymacji odpowiedzi skokowych dla zestawu zakłócenie - wyjście}
Tak jak dla poprzedniego przykładu parametrami, którymi manipulowaliśmy to ilość zakładanych biegunów i zer w transmitancji. Wartość opóźnienia była ustawiona na poziomie $T_{\mathrm{d}}=29$ dla wszystkich aproksymacji. 

\subsubsection{Wyniki symulacji dla modelu z dwoma biegunami i jednym zerem}
Rysunek \ref{lab3_apro_1_sz} prezentuje aproksymację dla przypadku z dwoma biegunami i jednym zerem. Możemy zauważyć, że przebieg modelu poprawnie naśladuje zachowanie rzeczywistego obiektu. Prawidłowo reaguje na zachowanie grzałki, czyli od razu po zmianie wartości zadanej (próbka 30) zaczyna rosnąć. Wartość dopasowania dla tego przypadku wyniosła $\num{91,52}$ \text{\%}.

\subsubsection{Wyniki symulacji dla modelu z dwoma biegunami i dwoma zerami}
Najlepszy wskaźnik dopasowania, udało nam się uzyskać dla przypadku z dwoma biegunami i dwoma zerami. Jego wartość ostatecznie wyniosła $\num{92,92}$ \text{\%}. Model obiektu praktycznie identycznie odwzorowuje zachowanie obiektu rzeczywistego. Przebiegi zostały zaprezentowane na rysunku \ref{lab3__apro_2_sz}. Widać na nich jak momentami model niczym się nie różni od obiektu. Niestety model zachował się nie do końca poprawnie na początku pomiaru. Gdy zakładamy zwiększenie temperatury na grzałce, tak jak to się dzieje w tej symulacji, jego wartość powinna wzrastać, nawet gdy obiekt chwilowo maleje. Z tego powodu nie możemy uznać tego modelu za najlepszy. 

\subsubsection{Wyniki symulacji dla modelu z jednym biegunem i dwoma zerami}
Kolejny wykres \ref{lab3__apro_3_sz} prezentuje model w którym zastosowaliśmy jeden biegu i dwa zera. Wartość dopasowania osiągnęła poziom $\num{91,34}$ \text{\%}, ale nie możemy opierać się w pełni na tym wskaźniku, bo jak widzimy na rysunku \ref{lab3__apro_3_sz} przebieg modelu obiektu nie jest poprawny. Na początku reaguje on zbyt gwałtownie na zmniejszenie wartość rzeczywistego obiektu. W tym przypadku powtórzyło się niepoprawne zachowanie modelu podobne do tego z poprzedniej symulacji. 

~\\\\Za najlepszy model zestawu zakłócenie - wyjście uznaliśmy model z pierwszej symulacji, który posiadał dwa bieguny i jedno zero. 

\begin{figure}[t]
    \centering
    \begin{tikzpicture}
    \begin{axis}[
    width=0.98\textwidth,
    xmin=0,xmax=400,ymin=-0.01,ymax=0.15,
    xlabel={$k$},
    ylabel={$s_{k}$},
    legend pos=south east,
    y tick label style={/pgf/number format/1000 sep=},
    ] 
    \addlegendentry{Obiekt}
    \addlegendentry{Model}
	\addlegendimage{no markers,red}
	\addlegendimage{no markers,blue}
	\addlegendimage{no markers,green}
    \addplot[red, semithick, thick] file{../data/lab/zad3/normDoWykresuSZ.csv};
    \addplot[blue, semithick, thick] file{../data/lab/zad3/apro_1_sz.csv};
    \end{axis}
    \end{tikzpicture}
    \caption{Porównanie modelu obiektu otrzymanego przy pomocy pakietu System Identifiaction
z rzeczywista odpowiedzia skokowa}
    \label{lab3_apro_1_sz}
\end{figure}

\begin{figure}[b]
    \centering
    \begin{tikzpicture}
    
    \begin{axis}[
    width=0.98\textwidth,
    xmin=0,xmax=400,ymin=-0.01,ymax=0.15,
    xlabel={$k$},
    ylabel={$s_{k}$},
    legend pos=south east,
    y tick label style={/pgf/number format/1000 sep=},
    ]     
    \addlegendentry{Obiekt}
    \addlegendentry{Model}
	\addlegendimage{no markers,red}
	\addlegendimage{no markers,blue}
	\addlegendimage{no markers,green}
    \addplot[red, semithick, thick] file{../data/lab/zad3/normDoWykresuSZ.csv};
    \addplot[blue, semithick, thick] file{../data/lab/zad3/apro_2_sz.csv};
    \end{axis}
    \end{tikzpicture}
    \caption{Porównanie modelu obiektu otrzymanego przy pomocy pakietu System Identifiaction
z rzeczywista odpowiedzia skokowa}
    \label{lab3__apro_2_sz}
\end{figure}

\begin{figure}[t]
    \centering
    \begin{tikzpicture}
    
    \begin{axis}[
    width=0.98\textwidth,
    xmin=0,xmax=400,ymin=-0.05,ymax=0.15,
    xlabel={$k$},
    ylabel={$s_{k}$},
    legend pos=south east,
    y tick label style={/pgf/number format/1000 sep=},
    ]     
    \addlegendentry{Obiekt}
    \addlegendentry{Model}
	\addlegendimage{no markers,red}
	\addlegendimage{no markers,blue}
	\addlegendimage{no markers,green}
    \addplot[red, semithick, thick] file{../data/lab/zad3/normDoWykresuSZ.csv};
    \addplot[blue, semithick, thick] file{../data/lab/zad3/apro_3_sz.csv};
    \end{axis}
    \end{tikzpicture}
    \caption{Porównanie modelu obiektu otrzymanego przy pomocy pakietu System Identifiaction
z rzeczywista odpowiedzia skokowa}
    \label{lab3__apro_3_sz}
\end{figure}
\FloatBarrier