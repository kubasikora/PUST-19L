\chapter{Regulacja procesem rzeczywistym przy użyciu alogrytmu DMC z uwzględnieniem zakłóceń mierzalnych}
\label{lab5}

\section{Dobór parametru $D_{\mathrm{z}}$}
\label{lab5_dobor_Dz}
Aby jak najlepiej dobrać parametr $D_{\mathrm{z}}$, zdecydowaliśmy się eksperymentalne 
zbadanie dziedziny parametru, rozpoczynając od bazowej wartości parametru równej
horyzontowi dynamiki toru zakłócenie-wyjście $D_{\mathrm{z}} = \num{400}$. 

\subsection{Opis eksperymentu}
\label{lab5_opis} 
Dla każdej przeprowadzonej próby, zmierzyliśmy 1000 próbek sygnału wejściowego. Proces był
regulowany w otoczeniu skokowo zmiennych zakłóceń. W $\num{250}$ próbce eksperymentu, następowała
skokowa zmiana zakłócenia z $z_{0} = \num{0}$ na $z_{1} = \num{30}$. Drugi skok zakłócenia następował
w $\num{600}$ próbce, z wartości $z_{1} = \num{30}$ do wartości $z_{2} = \num{10}$. Dodatni i ujemny skok
zakłócenia pozwolił nam na lepsze zbadanie charakterystyk algorytmu regulacji. Chwile skoków dobraliśmy
analizując przebiegi regulatora strojonego na laboratorium pierwszym oraz wcześniej przeprowadzanych prób.

\begin{figure}[b]
    \centering
    \begin{tikzpicture}
    \begin{axis}[
    width=\textwidth,
    xmin=1,xmax=1000,ymin=0,ymax=35,
    xlabel={$k$},
    ylabel={$z[k]$},
    %xtick={0, 50, 100, 150, 200},
    %ytick={0, 2, 4, 6, 8},
    legend pos=south east,
    y tick label style={/pgf/number format/1000 sep=},
    ]
    \addplot[red, semithick] file{../data/lab/zad5/disturb_Dz_400.csv};
    \legend{$z[k]$, }
    \end{axis} 
    \end{tikzpicture}
    \caption{Jednolity przebieg sygnału zakłócenia w przeprowadzanych przez nas eksperymentach}
    \label{zad5_lab_zaklocenia}
\end{figure}


\subsection{Wyniki eksperymentów}
\label{zad5_wyniki}

\subsubsection{Parametr $D_{\mathrm{z}} = \num{400}$}
Dobranie bazowej wartości parametru, pozwoliło nam na zaskakująco
zadowalającą regulację procesem przy skokowo zmieniającym się 
mierzalnym zakłóceniu. Skokowa zmiana zakłócenia jest szybko kompensowana 
przez gwałtowną zmianę sygnału sterującego, dzięki czemu wyjście procesu zmienia się
maksymalnie o $\num{1}$ stopień Celcjusza. Drugi skok zakłócenia, z racji mniejszej 
wartości bezwgzlędnej skoku, jest jeszcze lepiej skompensowany. Uzyskane przebiegi zostały przedstawione
na rysunkach \ref{lab5_dz_400_y} i \ref{lab5_dz_400_u}. Liczony na bieżąco w trakcie 
eksperymentu wskaźnik jakości opisany jako suma kwadratów uchybu uzyskał wartość 
$E = \num{4668.2}$. Aby ocenić czy jest to wartość duża czy mała, porównamy ją z wartościami
uzyskanymi w ramach kolejnych czterech eksperymentów z innym parametrem $D_{\mathrm{z}}$.


\subsubsection{Parametr $D_{\mathrm{z}} = \num{350}$}
W kolejnym eksperymencie sprawdziliśmy zachowanie regulatora z parametrem $D_{\mathrm{z}} = \num{350}$.
Jakościowo, przebiegi przedstawione na rysunkach \ref{lab5_dz_350_y} i \ref{lab5_dz_350_u} nie różnią
się znacznie od tych przedstawionych w poprzednim punkcie. Ilościowo, wskaźnik jakości poprawił się 
znacznie i wyniósł $E = \num{3583.7}$. 


\subsubsection{Parametr $D_{\mathrm{z}} = \num{300}$}
Następną próbę wykonaliśmy dla algorytmu DMC z paramtrem $D_{\mathrm{z}} = \num{300}$. Oczekując
poprawy, okazało się że wskaźnik jakości nieznacznie pogorszył się do wartości 
$E = \num{3707.5}$. Przebiegi jakościowo nie odbiegają od pożądanych, może martwić jedynie 
niewielki uchyb po kompensacji pierwszego zakłócenia ukazany na rysunku \ref{lab5_dz_300_y}. Przebieg
sterujący generowany przez ten regulator został przedstawiony na rysunku \ref{lab5_dz_300_u}.

\subsubsection{Parametr $D_{\mathrm{z}} = \num{250}$}
Dla parametru $D_{\mathrm{z}} = \num{250}$ udało nam się uzyskać najlepszy jak dotąd wskaźnik jakości
wynoszący $E = \num{3515.3}$. Skoki zakłócenia są szybko kompensowane przez regulator, który szybko
dopasowywuje sygnał sterujący ukazany na rysunku \ref{lab5_dz_250_u}. Wynikowy sygnał wyjściowy procesu
został przedstawiony na rysunku \ref{lab5_dz_250_y}.

\subsubsection{Parametr $D_{\mathrm{z}} = \num{200}$}
Przy jeszcze bardziej zmniejszonym parametrze $D_{\mathrm{z}}$ uwydatniły się problemy związane z za małą
jego wartością. Na przebiegu \ref{lab5_dz_200_y} widać jak po początkowej kompensacji zakłócenia, proces zostaje
wytrącony z wartości zadanej. Przebieg sterujący \ref{lab5_dz_200_u} oo pierwszym skoku, zamiast ustalić się na 
pewnej wartości zaczyna rosnąć, co powoduje wzrost uchybu. Spowodowane jest to za krótkim horyzontem dynamiki zakłóceń.
Niepełny model toru zakłócenie-wyjście sprawia że regulator źle reaguje na zmiany zakłócenia mierzalnego i sam wybija się 
z położenia równowagi. Dalsze zmniejszanie parametru $D_{\mathrm{z}}$ spowoduje tylko i wyłącznie pogłębienie tego efektu.


\subsection{Ostateczna wartość parametru $D_{\mathrm{z}}$}
Biorąc pod uwagę wyniki poprzednich eksperymentów, zdecydowaliśmy się na wartość parametru $D_{\mathrm{z}} = \num{250}$.
Dla tej wartości uzyskaliśmy najniższy wskaźnik jakości, jak i również przebiegi sygnałów procesowych wyglądaja 
w naszych oczach na bardzo zadowalające.









\begin{figure}[t]    
    \centering
    \begin{tikzpicture}
    \begin{axis}[
    width=\textwidth,
    xmin=0,xmax=1000,ymin=30,ymax=43,
    xlabel={$k$},
    ylabel={$y[k]$},
    %xtick={0, 50, 100, 150, 200},
    %ytick={0, 2, 4, 6, 8},
    legend pos=south east,
    y tick label style={/pgf/number format/1000 sep=},
    ]
    \addplot[red, semithick] file{../data/lab/zad5/output_Dz_400.csv};
	\addplot[blue, semithick] file{../data/lab/zad5/stpt_Dz_400.csv};    
    \legend{$y[k]$, $y^{\mathrm{zad}}[k]$}
    \end{axis} 
    \end{tikzpicture}
    \caption{Przebieg procesu sterowanego za pomocą dostrojonego regulatora DMC z parametrem $D_{\mathrm{z}} = \num{400}$}
    \label{lab5_dz_400_y}
\end{figure}

\begin{figure}[b]    
    \centering
    \begin{tikzpicture}
    \begin{axis}[
    width=\textwidth,
    xmin=0,xmax=1000,ymin=0,ymax=35,
    xlabel={$k$},
    ylabel={$u[k]$},
    legend pos=south east,
    y tick label style={/pgf/number format/1000 sep=},
    ]
    \addplot[red, semithick] file{../data/lab/zad5/input_Dz_400.csv};   
    \legend{$u[k]$}
    \end{axis} 
    \end{tikzpicture}
    \caption{Przebieg generowanego sterowania z dostrojonego regulatora DMC z parametrem $D_{\mathrm{z}} = \num{400}$}
    \label{lab5_dz_400_u}
\end{figure}





\begin{figure}[t]    
    \centering
    \begin{tikzpicture}
    \begin{axis}[
    width=\textwidth,
    xmin=0,xmax=1000,ymin=30,ymax=43,
    xlabel={$k$},
    ylabel={$y[k]$},
    %xtick={0, 50, 100, 150, 200},
    %ytick={0, 2, 4, 6, 8},
    legend pos=south east,
    y tick label style={/pgf/number format/1000 sep=},
    ]
    \addplot[red, semithick] file{../data/lab/zad5/output_Dz_350.csv};
	\addplot[blue, semithick] file{../data/lab/zad5/stpt_Dz_350.csv};    
    \legend{$y[k]$, $y^{\mathrm{zad}}[k]$}
    \end{axis} 
    \end{tikzpicture}
    \caption{Przebieg procesu sterowanego za pomocą dostrojonego regulatora DMC z parametrem $D_{\mathrm{z}} = \num{350}$}
    \label{lab5_dz_350_y}
\end{figure}

\begin{figure}[b]    
    \centering
    \begin{tikzpicture}
    \begin{axis}[
    width=\textwidth,
    xmin=0,xmax=1000,ymin=0,ymax=35,
    xlabel={$k$},
    ylabel={$u[k]$},
    legend pos=south east,
    y tick label style={/pgf/number format/1000 sep=},
    ]
    \addplot[red, semithick] file{../data/lab/zad5/input_Dz_350.csv};   
    \legend{$u[k]$}
    \end{axis} 
    \end{tikzpicture}
    \caption{Przebieg generowanego sterowania z dostrojonego regulatora DMC z parametrem $D_{\mathrm{z}} = \num{350}$}
    \label{lab5_dz_350_u}
\end{figure}





\begin{figure}[t]    
    \centering
    \begin{tikzpicture}
    \begin{axis}[
    width=\textwidth,
    xmin=0,xmax=1000,ymin=30,ymax=43,
    xlabel={$k$},
    ylabel={$y[k]$},
    %xtick={0, 50, 100, 150, 200},
    %ytick={0, 2, 4, 6, 8},
    legend pos=south east,
    y tick label style={/pgf/number format/1000 sep=},
    ]
    \addplot[red, semithick] file{../data/lab/zad5/output_Dz_300.csv};
	\addplot[blue, semithick] file{../data/lab/zad5/stpt_Dz_300.csv};    
    \legend{$y[k]$, $y^{\mathrm{zad}}[k]$}
    \end{axis} 
    \end{tikzpicture}
    \caption{Przebieg procesu sterowanego za pomocą dostrojonego regulatora DMC z parametrem $D_{\mathrm{z}} = \num{300}$}
    \label{lab5_dz_300_y}
\end{figure}

\begin{figure}[b]    
    \centering
    \begin{tikzpicture}
    \begin{axis}[
    width=\textwidth,
    xmin=0,xmax=1000,ymin=0,ymax=35,
    xlabel={$k$},
    ylabel={$u[k]$},
    legend pos=south east,
    y tick label style={/pgf/number format/1000 sep=},
    ]
    \addplot[red, semithick] file{../data/lab/zad5/input_Dz_300.csv};   
    \legend{$u[k]$}
    \end{axis} 
    \end{tikzpicture}
    \caption{Przebieg generowanego sterowania z dostrojonego regulatora DMC z parametrem $D_{\mathrm{z}} = \num{300}$}
    \label{lab5_dz_300_u}
\end{figure}





\begin{figure}[t]    
    \centering
    \begin{tikzpicture}
    \begin{axis}[
    width=\textwidth,
    xmin=0,xmax=1000,ymin=30,ymax=43,
    xlabel={$k$},
    ylabel={$y[k]$},
    %xtick={0, 50, 100, 150, 200},
    %ytick={0, 2, 4, 6, 8},
    legend pos=south east,
    y tick label style={/pgf/number format/1000 sep=},
    ]
    \addplot[red, semithick] file{../data/lab/zad5/output_Dz_250.csv};
	\addplot[blue, semithick] file{../data/lab/zad5/stpt_Dz_250.csv};    
    \legend{$y[k]$, $y^{\mathrm{zad}}[k]$}
    \end{axis} 
    \end{tikzpicture}
    \caption{Przebieg procesu sterowanego za pomocą dostrojonego regulatora DMC z parametrem $D_{\mathrm{z}} = \num{250}$}
    \label{lab5_dz_250_y}
\end{figure}

\begin{figure}[b]    
    \centering
    \begin{tikzpicture}
    \begin{axis}[
    width=\textwidth,
    xmin=0,xmax=1000,ymin=0,ymax=35,
    xlabel={$k$},
    ylabel={$u[k]$},
    legend pos=south east,
    y tick label style={/pgf/number format/1000 sep=},
    ]
    \addplot[red, semithick] file{../data/lab/zad5/input_Dz_250.csv};   
    \legend{$u[k]$}
    \end{axis} 
    \end{tikzpicture}
    \caption{Przebieg generowanego sterowania z dostrojonego regulatora DMC z parametrem $D_{\mathrm{z}} = \num{250}$}
    \label{lab5_dz_250_u}
\end{figure}


\begin{figure}[t]    
    \centering
    \begin{tikzpicture}
    \begin{axis}[
    width=\textwidth,
    xmin=0,xmax=1000,ymin=30,ymax=43,
    xlabel={$k$},
    ylabel={$y[k]$},
    %xtick={0, 50, 100, 150, 200},
    %ytick={0, 2, 4, 6, 8},
    legend pos=south east,
    y tick label style={/pgf/number format/1000 sep=},
    ]
    \addplot[red, semithick] file{../data/lab/zad5/output_Dz_200.csv};
	\addplot[blue, semithick] file{../data/lab/zad5/stpt_Dz_200.csv};    
    \legend{$y[k]$, $y^{\mathrm{zad}}[k]$}
    \end{axis} 
    \end{tikzpicture}
    \caption{Przebieg procesu sterowanego za pomocą dostrojonego regulatora DMC z parametrem $D_{\mathrm{z}} = \num{200}$}
    \label{lab5_dz_200_y}
\end{figure}

\begin{figure}[b]    
    \centering
    \begin{tikzpicture}
    \begin{axis}[
    width=\textwidth,
    xmin=0,xmax=1000,ymin=0,ymax=35,
    xlabel={$k$},
    ylabel={$u[k]$},
    legend pos=south east,
    y tick label style={/pgf/number format/1000 sep=},
    ]
    \addplot[red, semithick] file{../data/lab/zad5/input_Dz_200.csv};   
    \legend{$u[k]$}
    \end{axis} 
    \end{tikzpicture}
    \caption{Przebieg generowanego sterowania z dostrojonego regulatora DMC z parametrem $D_{\mathrm{z}} = \num{200}$}
    \label{lab5_dz_200_u}
\end{figure}

\FloatBarrier

\section{Porównanie działania algorytmu DMC bez i z uwzględnieniem zakłóceń mierzalnych}
\label{lab5_porownanie}

\subsection{Porównanie przebiegów wyjściowych}
Na rysunku \ref{lab5_porownanie_y} porównano wyjście procesu sterowanego przez regulator z i bez 
uwzględnienia zakłócenia. Jasno widać że przebieg wyjściowy przy uwzględnieniu zakłóceń, lepiej 
trzyma się wartości zadanej. Brak uwzględnienia zakłoceń powoduje że uchyb przy skokowej 
zmianie sygnału zakłócajacego jest prawie dwa razy większy niż w przypadku regulacji z uwzględnieniem zakłóceń.
Proces sterowany w klasyczny sposób potrzebuje również więcej czasu aby zniwelować wpływ zakłóceń na obiekt.

\begin{figure}[b]    
    \centering
    \begin{tikzpicture}
    \begin{axis}[
    width=\textwidth,
    xmin=0,xmax=1000,ymin=32,ymax=44,
    xlabel={$k$},
    ylabel={$y[k]$},
    %xtick={0, 50, 100, 150, 200},
    %ytick={0, 2, 4, 6, 8},
    legend pos=south east,
    y tick label style={/pgf/number format/1000 sep=},
    ]
    \addplot[red, semithick] file{../data/lab/zad5/output_Dz_250.csv};
    \addplot[blue, semithick] file{../data/lab/zad5/output_Dz_0.csv}; 
    \addplot[green, dashed] file{../data/lab/zad5/stpt_Dz_0.csv};     
    \legend{$y^{\mathrm{z}}[k]$, $y^{\mathrm{bez}}[k]$, $y^{\mathrm{zad}}[k]$}
    \end{axis} 
    \end{tikzpicture}
    \caption{Porównanie przebiegów wyjściowych procesu sterowanego przez regulator z i bez uwzględnienia zakłóceń}
    \label{lab5_porownanie_y}
\end{figure}
\FloatBarrier

\subsection{Porównanie przebiegów sygnału sterującego}

Uwzględnienie zakłoceń pozwala na szybszą reakcje w momencie skokowej zmiany zakłócenia. Na rysunku \ref{lab5_porownanie_u}
przedstawiono przebiegi sygnału sterującego generowanego przez regulator. Do momentu pierwszego skoku,
przebiegi powinny być porównywalne w wartościach ale jednak nie są. Spowodowane jest to faktem tygodniowej różnicy
w czasie pomiędzy zebraniem obu sygnałów, sygnał bez uwzględniania zakłóceń zebraliśmy na pierwszym spotkaniu 
laboratoryjnym a sygnał z uwzględnieniem zakłocenia na drugim, na którym w laboratorium panowały inne warunki.
Bardziej interesujące wydają się być przebiegi sterowania po pierwszym skoku zakłócenia. Widać jak regulator
z uwzględnieniem zakłóceń szybko zmienia sterowanie aby nadążyć za zakłóceniem i po około stu próbkach
ustalaswoją wartośc na stałym poziomie. Regulator bez uwzględnienia zakłóceń nie ma informacji jak powinien
się zachować w obliczu zakłócenia dlatego też zdecydowanie wolniej zmienia się generowany przez niego sygnał
sterujący. Sytuacja powtarza się przy drugim skoku.

\subsection{Wnioski}
\label{lab5_wnioski}
Na podstawie przeprowadzonego eksperymentu możemy jednoznacznie stwierdzić że uwzględnienie zakłóceń w regulatorze DMC,
prowadzi do znacznej poprawy jakości regulacji w otoczeniu zakłóceń które można zmierzyć. Nie jest to jednak tak proste,
ponieważ zwykle nie mamy możliwości tak dokładnego wyznaczenia modelu zakłócenia, które jest niezwykle potrzebne do 
poprawnej regulacji procesem rzeczywistym.

\begin{figure}[b]    
    \centering
    \begin{tikzpicture}
    \begin{axis}[
    width=\textwidth,
    xmin=0,xmax=1000,ymin=0 ,ymax=30,
    xlabel={$k$},
    ylabel={$u[k]$},
    %xtick={0, 50, 100, 150, 200},
    %ytick={0, 2, 4, 6, 8},
    legend pos=south east,
    y tick label style={/pgf/number format/1000 sep=},
    ]
    \addplot[red, semithick] file{../data/lab/zad5/input_Dz_250.csv};
    \addplot[blue, semithick] file{../data/lab/zad5/input_Dz_0.csv};     
    \legend{$u^{\mathrm{z}}[k]$, $u^{\mathrm{bez}}[k]$}
    \end{axis} 
    \end{tikzpicture}
    \caption{Porównanie przebiegów sygnału sterującego generowanego przez regulator z i bez uwzględnienia zakłóceń}
    \label{lab5_porownanie_u}
\end{figure}