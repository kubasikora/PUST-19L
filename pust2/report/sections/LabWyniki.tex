\chapter{Regulacja procesem rzeczywistym przy użyciu alogrytmu DMC z uwzględnieniem zakłóceń mierzalnych}
\label{lab5}

\section{Dobór parametru $Dz$}
\label{lab5_dobor_Dz}
Aby jak najlepiej dobrać parametr $Dz$, zdecydowaliśmy się eksperymentalne 
zbadanie dziedziny parametru, rozpoczynając od bazowej wartości parametru równej
horyzontowi dynamiki toru zakłócenie-wyjście $Dz = \num{400}$. 

\subsection{Opis eksperymentu}
\label{lab5_opis}
Dla każdej przeprowadzonej próby, zmierzyliśmy 1000 próbek sygnału wejściowego. Proces był
regulowany w otoczeniu skokowo zmiennych zakłóceń. W $\num{250}$ próbce eksperymentu, następowała
skokowa zmiana zakłócenia z $z_{0} = \num{0}$ na $z_{1} = \num{30}$. Drugi skok zakłócenia następował
w $\num{600}$ próbce, z wartości $z_{1} = \num{30}$ do wartości $z_{2} = \num{10}$. Dodatni i ujemny skok
zakłócenia pozwolił nam na lepsze zbadanie charakterystyk algorytmu regulacji.

\begin{figure}[b]
    \centering
    \begin{tikzpicture}
    \begin{axis}[
    width=\textwidth,
    xmin=1,xmax=1000,ymin=0,ymax=35,
    xlabel={$k$},
    ylabel={$z[k]$},
    %xtick={0, 50, 100, 150, 200},
    %ytick={0, 2, 4, 6, 8},
    legend pos=south east,
    y tick label style={/pgf/number format/1000 sep=},
    ]
    \addplot[red, semithick] file{../data/lab/zad5/disturb_Dz_400.csv};
    \legend{$z[k]$, }
    \end{axis} 
    \end{tikzpicture}
    \caption{Jednolity przebieg sygnału zakłócenia w przeprowadzanych przez nas eksperymentach}
    \label{zad5_lab_zaklocenia}
\end{figure}

\subsection{Wyniki eksperymentów}
\label{zad5_wyniki}
\subsubsection{Parametr $D_{z} = \num{400}$}
Dobranie bazowej wartości parametru, pozwoliło nam na zaskakująco
zadowalającą regulację procesem przy skokowo zmieniającym się 
mierzalnym zakłóceniu.  

\begin{figure}[t]    
    \centering
    \begin{tikzpicture}
    \begin{axis}[
    width=\textwidth,
    xmin=0,xmax=1000,ymin=30,ymax=43,
    xlabel={$k$},
    ylabel={$y[k]$},
    %xtick={0, 50, 100, 150, 200},
    %ytick={0, 2, 4, 6, 8},
    legend pos=south east,
    y tick label style={/pgf/number format/1000 sep=},
    ]
    \addplot[red, semithick] file{../data/lab/zad5/output_Dz_400.csv};
	\addplot[blue, semithick] file{../data/lab/zad5/stpt_Dz_400.csv};    
    \legend{$y[k]$, $y^{\mathrm{zad}}[k]$}
    \end{axis} 
    \end{tikzpicture}
    \caption{Przebieg procesu sterowanego za pomocą dostrojonego regulatora DMC z parametrem $Dz = \num{400}$}
    \label{lab_zad4_lab_proces_wykres}
\end{figure}

\begin{figure}[b]    
    \centering
    \begin{tikzpicture}
    \begin{axis}[
    width=\textwidth,
    xmin=0,xmax=1000,ymin=0,ymax=35,
    xlabel={$k$},
    ylabel={$u[k]$},
    legend pos=south east,
    y tick label style={/pgf/number format/1000 sep=},
    ]
    \addplot[red, semithick] file{../data/lab/zad5/input_Dz_400.csv};   
    \legend{$u[k]$}
    \end{axis} 
    \end{tikzpicture}
    \caption{Przebieg generowanego sterowania z dostrojonego regulatora DMC z parametrem $Dz = \num{400}$}
    \label{lab_zad4_lab_proces_wykres}
\end{figure}