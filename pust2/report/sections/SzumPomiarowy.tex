\chapter{Badanie wpływu błędu pomiarowego zakłócenia na algorytm regulacji}
\label{zad7}

Ostatnie zadanie projektowe polegało na zbadaniu wpływu błędu pomiarowego 
zakłócenia (zwanego szumem pomiarowym) na algorytm regulacji DMC.

\section{Pomiar a niepewność pomiarowa}
\label{zad7_pomiar}
\textbf{Pomiarem} nazywamy czynności, po wykonaniu których możemy 
stwierdzić, że w chwili pomiaru dokonanego w określonych warunkach, 
przy zastosowaniu określonych środków i wykonaniu odpowiednich czynności 
wielkość mierzona $x$ miała wartość $a \leq x \leq b$.
\\

\textbf{Niepewnością pomiarową} nazywamy parametr związany z wartościami (serią) 
pomiaru danej wielkości fizycznej w stałych warunkach, które można w uzasadniony 
sposób przypisać wartości mierzonej i charakteryzujący ich rozrzut w przedziale, 
wewnątrz którego można z zadowalającym prawdopodobieństwem usytuować wartość 
wielkości mierzonej.\\

Żródłami niepewności pomiarowej a co za tym idzie błędów pomiaru są:
\begin{itemize}
    \item niepełna znajomość wpływu otoczenia
    \item niedoskonały pomiar warunków otoczenia 
    \item klasa przyrządów pomiarowych (dokładność) 
    \item niedokładne wartości otrzymywane z zewnątrz, przykładowo stałe przyjmowane do obliczeń
\end{itemize}

Szum pomiarowy można zamodelować addytywnym gaussowskim szumem białym, w skrócie nazywanym AWGN.
Jest to rodzaj sygnału o całkowicie płaskim widmie, czyli jest mieszaniną wszystkich możliwych
częstotliwości. Pozwala to na zamodelowanie losowego błędu pomiaru w każdej chwili dyskretnej.

\section{Realizacja szumu pomiarowego w projekcie}
\label{zad7_szum_w_matlabie}

W środowisku \textsc{Matlab} istnieje polecenie \verb+awgn+, które pozwala na naniesienie AWGN 
na zadany sygnał, specyfikując przy tym stosunek sygnał-szum $SNR$. Im większy jest ten stosunek, tym 
mocniejszy jest sygnał względem szumu. Wartość $SNR$ podaje się w $dB$. Przykład działania 
polecenia \verb+awgn+ został przedstawiony na rysunku \ref{zad7_awgn_example}.

\begin{figure}[t]
    \begin{tikzpicture}
        \begin{groupplot}[
        group style={
        group name=my plots,
        group size=2 by 2,
        xlabels at=edge bottom,
        ylabels at=edge left,
        horizontal sep=1cm, vertical sep=1cm,
        },
        legend style={at={(.5,0.9)},anchor=north east},
        xmin=0, xmax=100,
        width=0.55\linewidth
        ]
        \nextgroupplot
        \addlegendentry{$y(x)$}
        \addplot[semithick,blue] file{../data/zad7/awgn20.csv};

        \nextgroupplot
        \addlegendentry{$y(x)$}
        \addplot[semithick,blue]file{../data/zad7/awgn10.csv};
        
        \nextgroupplot
        \addlegendentry{$y(x)$}
        \addplot[semithick,blue] file{../data/zad7/awgn0.csv};

        \nextgroupplot
        \addlegendentry{$y(x)$}
        \addplot[semithick,blue]file{../data/zad7/awgn-10.csv};
        
        \end{groupplot}
        \end{tikzpicture}
        \caption{Porównanie przebiegu funkcji zaszumionej funkcji $y = \num{0,002}x^{2}$ dla czterech wartości SNR  
        \mbox{\emph{a)} $SNR = \num{20}dB$, \emph{b)} $SNR = \num{10}dB$, 
        \emph{c)} $SNR = \num{0}dB$, \emph{d)} $SNR = \num{-10}dB$}}
        \label{zad7_awgn_example}
\end{figure}

\section{Badanie wpływu szumu na algorytm regulacji}
\label{zad7_realizacja}

Badanie wpływu szumu pomiarowego na algorytm regulacji zrealizowaliśmy 
za pomocą skryptu \verb+zad7_test.m+. Za pomocą parametru \verb+SNR+ manipulowaliśmy
wartościami błędów. Im większy parametr \verb+SNR+ tym mniejsze wartości błędów pomiarowych. Rysunki 
\ref{zad7_step_y}, \ref{zad7_step_u} i \ref{zad7_step_z} pokazują zachowanie przy skokowej zmianie zakłócenia
a \ref{zad7_sin_y}, \ref{zad7_sin_u} i \ref{zad7_sin_z} ukazują zachowanie układu przy zakłóceniu sinusoidalnym.

\begin{figure}[t]
    \begin{tikzpicture}
        \begin{groupplot}[
        group style={
        group name=my plots,
        group size=1 by 4,
        xlabels at=edge bottom,
        ylabels at=edge left,
        horizontal sep=1cm, vertical sep=1cm,
        },
        xmin=0, xmax=500, ymin=0, ymax=2,
        width=1.05\linewidth,
        height=0.5\linewidth
        ]
        \nextgroupplot
        \addlegendentry{$y[k]$}
        \addplot[semithick,blue] file{../data/zad7/step_output_snr_40.csv};

        \nextgroupplot
        \addlegendentry{$y[k]$}
        \addplot[semithick,blue]file{../data/zad7/step_output_snr_30.csv};
        
        \nextgroupplot
        \addlegendentry{$y[k]$}
        \addplot[semithick,blue] file{../data/zad7/step_output_snr_20.csv};

        \nextgroupplot
        \addlegendentry{$y[k]$}
        \addplot[semithick,blue] file{../data/zad7/step_output_snr_10.csv};
        
        \end{groupplot}
        \end{tikzpicture}
        \caption{Porównanie przebiegów wyjściowych procesu sterowanego przez regulator DMC przy:  
        \mbox{\emph{a)} $SNR = \num{40}dB$, \emph{b)} $SNR = \num{30}dB$, 
        \emph{c)} $SNR = \num{20}dB$, \emph{d)} $SNR = \num{10}dB$}}
        \label{zad7_step_y}
\end{figure}


\begin{figure}[t]
    \begin{tikzpicture}
        \begin{groupplot}[
        group style={
        group name=my plots,
        group size=1 by 4,
        xlabels at=edge bottom,
        ylabels at=edge left,
        horizontal sep=1cm, vertical sep=1cm,
        },
        xmin=0, xmax=500, ymin=-1, ymax=1,
        width=1.05\linewidth,
        height=0.5\linewidth
        ]
        \nextgroupplot
        \addlegendentry{$u[k]$}
        \addplot[semithick,blue] file{../data/zad7/step_input_snr_40.csv};

        \nextgroupplot
        \addlegendentry{$u[k]$}
        \addplot[semithick,blue]file{../data/zad7/step_input_snr_30.csv};
        
        \nextgroupplot
        \addlegendentry{$u[k]$}
        \addplot[semithick,blue] file{../data/zad7/step_input_snr_20.csv};

        \nextgroupplot
        \addlegendentry{$u[k]$}
        \addplot[semithick,blue] file{../data/zad7/step_input_snr_10.csv};
        
        \end{groupplot}
        \end{tikzpicture}
        \caption{Porównanie przebiegów wejściowych procesu sterowanego przez regulator DMC przy:  
        \mbox{\emph{a)} $SNR = \num{40}dB$, \emph{b)} $SNR = \num{30}dB$, 
        \emph{c)} $SNR = \num{20}dB$, \emph{d)} $SNR = \num{10}dB$}}
        \label{zad7_step_u}
\end{figure}


\begin{figure}[t]
    \begin{tikzpicture}
        \begin{groupplot}[
        group style={
        group name=my plots,
        group size=1 by 4,
        xlabels at=edge bottom,
        ylabels at=edge left,
        horizontal sep=1cm, vertical sep=1cm,
        },
        xmin=0, xmax=500, ymin=0, ymax=2,
        width=1.05\linewidth,
        height=0.5\linewidth
        ]
        \nextgroupplot
        \addlegendentry{$z[k]$}
        \addplot[semithick,blue] file{../data/zad7/step_disturbance_snr_40.csv};

        \nextgroupplot
        \addlegendentry{$z[k]$}
        \addplot[semithick,blue] file{../data/zad7/step_disturbance_snr_30.csv};
        
        \nextgroupplot
        \addlegendentry{$z[k]$}
        \addplot[semithick,blue] file{../data/zad7/step_disturbance_snr_20.csv};

        \nextgroupplot
        \addlegendentry{$z[k]$}
        \addplot[semithick,blue] file{../data/zad7/step_disturbance_snr_10.csv};
        
        \end{groupplot}
        \end{tikzpicture}
        \caption{Porównanie przebiegów mierzonego zakłócenia przy sterowaniu przez regulator DMC przy:  
        \mbox{\emph{a)} $SNR = \num{40}dB$, \emph{b)} $SNR = \num{30}dB$, 
        \emph{c)} $SNR = \num{20}dB$, \emph{d)} $SNR = \num{10}dB$}}
        \label{zad7_step_z}
\end{figure}

%%% 

\begin{figure}[t]
    \begin{tikzpicture}
        \begin{groupplot}[
        group style={
        group name=my plots,
        group size=1 by 4,
        xlabels at=edge bottom,
        ylabels at=edge left,
        horizontal sep=1cm, vertical sep=1cm,
        },
        xmin=0, xmax=500, ymin=0, ymax=2,
        width=1.05\linewidth,
        height=0.5\linewidth
        ]
        \nextgroupplot
        \addlegendentry{$y[k]$}
        \addplot[semithick,blue] file{../data/zad7/sin_output_snr_40.csv};

        \nextgroupplot
        \addlegendentry{$y[k]$}
        \addplot[semithick,blue]file{../data/zad7/sin_output_snr_30.csv};
        
        \nextgroupplot
        \addlegendentry{$y[k]$}
        \addplot[semithick,blue] file{../data/zad7/sin_output_snr_20.csv};

        \nextgroupplot
        \addlegendentry{$y[k]$}
        \addplot[semithick,blue] file{../data/zad7/sin_output_snr_10.csv};
        
        \end{groupplot}
        \end{tikzpicture}
        \caption{Porównanie przebiegów wyjściowych procesu sterowanego przez regulator DMC przy:  
        \mbox{\emph{a)} $SNR = \num{40}dB$, \emph{b)} $SNR = \num{30}dB$, 
        \emph{c)} $SNR = \num{20}dB$, \emph{d)} $SNR = \num{10}dB$}}
        \label{zad7_sin_y}
\end{figure}


\begin{figure}[t]
    \begin{tikzpicture}
        \begin{groupplot}[
        group style={
        group name=my plots,
        group size=1 by 4,
        xlabels at=edge bottom,
        ylabels at=edge left,
        horizontal sep=1cm, vertical sep=1cm,
        },
        xmin=0, xmax=500, ymin=-1, ymax=1.5,
        width=1.05\linewidth,
        height=0.5\linewidth
        ]
        \nextgroupplot
        \addlegendentry{$u[k]$}
        \addplot[semithick,blue] file{../data/zad7/sin_input_snr_40.csv};

        \nextgroupplot
        \addlegendentry{$u[k]$}
        \addplot[semithick,blue]file{../data/zad7/sin_input_snr_30.csv};
        
        \nextgroupplot
        \addlegendentry{$u[k]$}
        \addplot[semithick,blue] file{../data/zad7/sin_input_snr_20.csv};

        \nextgroupplot
        \addlegendentry{$u[k]$}
        \addplot[semithick,blue] file{../data/zad7/sin_input_snr_10.csv};
        
        \end{groupplot}
        \end{tikzpicture}
        \caption{Porównanie przebiegów wejściowych procesu sterowanego przez regulator DMC przy:  
        \mbox{\emph{a)} $SNR = \num{40}dB$, \emph{b)} $SNR = \num{30}dB$, 
        \emph{c)} $SNR = \num{20}dB$, \emph{d)} $SNR = \num{10}dB$}}
        \label{zad7_sin_u}
\end{figure}


\begin{figure}[t]
    \begin{tikzpicture}
        \begin{groupplot}[
        group style={
        group name=my plots,
        group size=1 by 4,
        xlabels at=edge bottom,
        ylabels at=edge left,
        horizontal sep=1cm, vertical sep=1cm,
        },
        xmin=0, xmax=500, ymin=-1.5, ymax=1.5,
        width=1.05\linewidth,
        height=0.5\linewidth
        ]
        \nextgroupplot
        \addlegendentry{$z[k]$}
        \addplot[semithick,blue] file{../data/zad7/sin_disturbance_snr_40.csv};

        \nextgroupplot
        \addlegendentry{$z[k]$}
        \addplot[semithick,blue] file{../data/zad7/sin_disturbance_snr_30.csv};
        
        \nextgroupplot
        \addlegendentry{$z[k]$}
        \addplot[semithick,blue] file{../data/zad7/sin_disturbance_snr_20.csv};

        \nextgroupplot
        \addlegendentry{$z[k]$}
        \addplot[semithick,blue] file{../data/zad7/sin_disturbance_snr_10.csv};
        
        \end{groupplot}
        \end{tikzpicture}
        \caption{Porównanie przebiegów mierzonego zakłócenia przy sterowaniu przez regulator DMC przy:  
        \mbox{\emph{a)} $SNR = \num{40}dB$, \emph{b)} $SNR = \num{30}dB$, 
        \emph{c)} $SNR = \num{20}dB$, \emph{d)} $SNR = \num{10}dB$}}
        \label{zad7_sin_z}
\end{figure}
\FloatBarrier

Powyżej przedstawione rysunki dowodzą że im mniejszy jest stosunek użytecznego sygnału do szumu, tym 
gorsza jest jakość regulacji. Przy braku możliwości dokładnego zmierzenia zakłócenia, warto rozważyć
zignorowanie tego sygnału i potraktowanie go jako zakłócenia niemierzalnego.\\

Dla zaprojektowanego przez nas regulatora, stosunek sygnał-szum powinien wynosić co najmniej $SNR = \num{30}dB$. 
W przypadku bardziej zaszumionego sygnału zakłócenia, jakość regulacji znacznie pogarsza się. 



