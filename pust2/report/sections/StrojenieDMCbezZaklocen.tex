\section{Strojenie regulatora DMC na stanowisku laboratoryjnym bez zakłóceń}
\label{zad4_lab_opis}
Strojąc algorytm DMC na obiekcie laboratoryjnym bez zakłóceń skorzystaliśmy z doświadczenia zebranego na poprzednim laboratorium. Dla naszego regulatora wybraliśmy wartości parametrów D, N, $N_\mathrm{u}$, $\Lambda$ dla których regulacja na poprzednim laboratorium przebiegała najlepiej. Po dokonaniu eksperymentu dla $D = 500$, $N = 140$, $N_\mathrm{u} = 1$, $\Lambda = 100$ otrzymaliśmy przebiegi zaprezentowane na rysunkach \ref{lab_zad4_lab_proces} i \ref{lab_zad4_lab_ster}. Możemy zauważyć, że nie pojawiają się przeregulowania, a wartość zadana jest osiągana bardzo szybko. Przebiegi procesu sterowanego i sygnału sterującego są pozbawione oscylacji. Warto zauważyć, że sygnał sterujący charakteryzuje się łagodnymi zmianami, co jest bardzo ważne dla elementów wykonawczych sterowanych przez regulator. Tak dobrze nastrojony regulator posłużył nam do wykonania następnych zadań podczas laboratorium. OPIS MATLAB 
\begin{figure}[t]
    \label{lab_zad4_lab_proces}
    \centering
    \begin{tikzpicture}
    \begin{axis}[
    width=\textwidth,
    xmin=4,xmax=450,ymin=29,ymax=41,
    xlabel={$k$},
    ylabel={$y[k]$},
    %xtick={0, 50, 100, 150, 200},
    %ytick={0, 2, 4, 6, 8},
    legend pos=south east,
    y tick label style={/pgf/number format/1000 sep=},
    ]
    \addplot[red, semithick] file{../data/lab/zad4/output_ts.csv};
	\addplot[blue, semithick] file{../data/lab/zad4/setpoint_ts.csv};    
    \legend{$y[k]$}
    \end{axis} 
    \end{tikzpicture}
    \caption{Przebieg procesu sterowanego za pomocą dostrojonego regulatora DMC}
\end{figure}

\begin{figure}[b]
    \label{lab_zad4_lab_ster}
    \centering
    \begin{tikzpicture}
    \begin{axis}[
    width=\textwidth,
    xmin=4,xmax=450,ymin=25,ymax=64,
    xlabel={$k$},
    ylabel={$y[k]$},
    %xtick={0, 50, 100, 150, 200},
    %ytick={0, 2, 4, 6, 8},
    legend pos=south east,
    y tick label style={/pgf/number format/1000 sep=},
    ]
    \addplot[blue, semithick] file{../data/lab/zad4/input_ts.csv};    
    \legend{$y[k]$}
    \end{axis} 
    \end{tikzpicture}
    \caption{Przebieg sygnału sterującego dostrojonego regulatora DMC}
\end{figure}
\FloatBarrier