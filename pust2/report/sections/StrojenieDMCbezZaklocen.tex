\definecolor{mylilas}{RGB}{170,55,241}

\lstset{language=Matlab,%
    %basicstyle=\color{red},
    breaklines=true,%
    morekeywords={matlab2tikz},
    keywordstyle=\color{blue},%
    morekeywords=[2]{1}, keywordstyle=[2]{\color{black}},
    identifierstyle=\color{black},%
    stringstyle=\color{mylilas},
    stringstyle=\color{magenta},
    commentstyle=\color{green},%
    showstringspaces=false,%without this there will be a symbol in the places where there is a space
    numbers=left,%
    numberstyle={\tiny \color{black}},% size of the numbers
    numbersep=9pt, % this defines how far the numbers are from the text
    emph=[1]{for,end,break},emphstyle=[1]\color{red}, %some words to emphasise
    %emph=[2]{word1,word2}, emphstyle=[2]{style},    
}

\chapter{Strojenie regulatora DMC na stanowisku laboratoryjnym bez zakłóceń}
\label{zad4_lab_opis}
\section{Omówienie implementacji algorytmu DMC}
Na początku skryptu następowało wczytanie modelu (znormalizowanej odpowiedzi skokowej obiektu) wykorzystywanego do regulacji algorytmem DMC.\\
\indent{} Następnie, tak jak poprzednio, zdefiniowane zostały stałe wykorzystywane w programie jak i parametry wykorzystywanego algorytmu. Horyzont dynamiki równał się długości wektora z odpowiedzią skokową. Horyzonty predykcji oraz sterowania, a także współczynnik kary $\lambda$ były odpowiednio modyfikowane w późniejszym procesie strojenia regulatora. \\
\indent{} Kolejnym krokiem, również powtórzonym była deklaracja wektorów wykorzystywanych do przechowywania wartości sygnałów.
%[language=Matlab]

\begin{lstlisting}[style=custommatlab,frame=single,label={zad4_vecDMC_lst},caption={Inizjalizacja wektorów używanych do przechowywania sygnałów.},captionpos=b]
% czas symulacji
sim_time = 1:sim_len; % do plotowania
sim_time = sim_time';

% wartosc zadana
stpt_value = 4.05;
setpoint = stpt_value*ones(sim_len,1);
setpoint(1:11) = Ypp;

% wektor sygnalu sterujacego
input = Upp*ones(sim_len, 1);

% wektor wyjscia
output = Ypp*ones(sim_len, 1);

rescaled_output = 0;
rescaled_input = zeros(sim_len, 1);

% wektor uchybu
error = zeros(sim_len, 1);

\end{lstlisting}

Pracę nad algorytmem należało rozpocząć od inicjalizacji macierzy algorytmu, używanych później do symulacji. Ważnym aspektem tej czynności jest poprawne określenie wymiarów każdej macierzy. Mając zdefiniowane w ten sposób zmienne mogliśmy kontrolować, czy poszczególne obliczenia, tworzące poniższe elementy algorytmu, przebiegały poprawnie.
\\

\begin{lstlisting}[style=custommatlab,frame=single,label={zad4_sim_lst},caption={Definicja macierzy algorytmu DMC},captionpos=b]
dU = zeros(Nu, 1);
dUp = zeros(D-1, 1);
M = zeros(N, Nu);
Mp = zeros(N, D-1);
K = zeros(Nu, N);
\end{lstlisting}

Następnie przystąpiliśmy do wyliczenia macierzy, których postać była niezmienna podczas działania programu. Kolejno obliczona została macierz $M$, $M_\mathrm{p}$ oraz $K$ zgodnie ze wzorami:
\\

\begin{equation}
M=\left[
\begin{array}
{cccc}
s_{1} & 0 & \ldots & 0\\
s_{2} & s_{1} & \ldots & 0\\
\vdots & \vdots & \ddots & \vdots\\
s_{N} & s_{N-1} & \ldots &  s_{N-N_{\mathrm{u}}+1}
\end{array}
\right]
\end{equation}
\\
\begin{equation}
M^{\mathrm{p}}=\left[
\begin{array}
{cccc}
s_{\mathrm{2}} - s_{\mathrm{1}} & s_{\mathrm{3}} - s_{\mathrm{2}} & \ldots & s_{\mathrm{D}} - s_{\mathrm{D-1}}\\
s_{\mathrm{3}} - s_{\mathrm{1}} & s_{\mathrm{4}} - s_{\mathrm{2}} & \ldots & s_{\mathrm{D+1}} - s_{\mathrm{D-1}}\\
\vdots & \vdots & \ddots & \vdots\\
s_{\mathrm{N+1}} - s_{1} & s_{\mathrm{N+2}} - s_{\mathrm{2}} & \ldots &  s_{N + \mathrm{D} - 1} - s_{{\mathrm{D-1}}}
\end{array}
\right]
\end{equation}
\\
\begin{equation}
K = (M^{\mathrm{T}} M + \lambda I_{\mathrm{N_{\mathrm{u}}} \times \mathrm{N_{\mathrm{u}}}})^{\mathrm{-1}} M^{\mathrm{T}}
\end{equation}
\\
Implementację powyższych wzorów przedstawiono na listingu \ref{zad4_matrices_lst}.\\

\begin{lstlisting}[style=custommatlab,frame=single,label={zad4_matrices_lst},caption={Wyliczenie macierzy stałych algorytmu DMC},captionpos=b]
% macierz M
for i = 1:Nu
    k = 1;
    for j = 1:N
        if j < i
            M(j, i) = 0;
        else
            M(j, i) = step(k);
            k = k + 1;
        end
    end
end

% macierz Mp
for i = 1:N
    for j = 1:(D-1)
        k = i + j;
        if k > D
            Mp(i,j) = step(D) - step(j);
        else
            Mp(i,j) = step(k) - step(j);
        end
    end
end

% macierz K
K = inv((M' * M + lambda * eye(Nu)))*M';
%\end{lstlisting}
Wersja algorytmu DMC użyta w programie jest wersją oszczędną to znaczy podczas symulacji nie jest obliczany cały wektor przyrostów sterowania $\Delta u(k)$, a jedynie pierwszy jego element faktycznie wykorzystywany do sterowania obiektem. Dana wartość obliczana jest ze wzoru:

\begin{equation}
\label{du(k)}
\Delta u(k) = K_{\mathrm{e}} e(k) - K_{\mathrm{u}} \Delta U^{\mathrm{P}}(k)
\end{equation}

gdzie $K_{\mathrm{e}}$ jest wyrażonym wzorem (\ref{eq_Ke}), $K_{\mathrm{u}}$ jest wektorem powstałym z pomnożenia pierwszego wiersza macierzy $K$ oraz macierzy $M^{\mathrm{P}}$ co przedstawia wzór (\ref{eq_Ku}). Wektor $\Delta U^{\mathrm{P}}(k)$ jest wektorem różnic sterowania między kolejnymi jego wartościami z przeszłości danym wzorem:

\begin{equation}
\Delta U^{\mathrm{P}}(k)=\left[
\begin{array}{c}
u(k-1) - u(k-2)\\
u(k-2) - u(k-3)\\
\vdots\\
u(k-\mathrm{D}+1) - u(k-\mathrm{D})
\end{array}
\right]
\end{equation}

\begin{equation}
\label{eq_Ke}
K_{\mathrm{e}} = \sum_{i=1}^{N} K_{1,i}
\end{equation}

\begin{equation}
\label{eq_Ku}
K_{\mathrm{u}} = \overline{K_{1}} M^{\mathrm{P}}
\end{equation}

Wektor $\Delta U^{\mathrm{P}}(k)$ jest obliczany w każdej iteracji pętli symulacyjnej w celu uniknięcia niepotrzebnych komplikacji związanych z przesuwaniem jego wartości. Przed wejściem do pętli obliczane są wartości $K_{\mathrm{e}}$ oraz $K_{\mathrm{e}}$. Symulacja działania algorytmu za pomocą pętli została przedstawiona na listingu \ref{zad4_dmc_sim_lst}.\\

\begin{lstlisting}[style=custommatlab,frame=single,label={zad4_dmc_sim_lst},caption={Pętla symulująca działanie regulatora DMC},captionpos=b]
Ke = sum(K(1,:));
Ku = K(1,:) * Mp;

for k=12:sim_len    
    % wektor dUp
    for i = 1:(D-1)
        if (k-i) <= 0
            du1 = 0;
        else
            du1 = rescaled_input(k - i);
        end
        if (k-i-1) <= 0
            du2 = 0;
        else
            du2 = rescaled_input(k - i - 1);
        end 
        dUp(i) = du1 - du2;
    end
    
    output(k) = symulacja_obiektu1Y(input(k-10), input(k-11),
    			output(k-1), output(k-2));    % pomiar wyjscia
    rescaled_output = output(k) - Ypp;  % skalowanie wyjscia   
    stpt = setpoint(k) - Ypp;   % przeskalowany setpoint
    error(k) = stpt - rescaled_output; % obliczenie uchyby   
    
    error_sum = error_sum + error(k)^2; % wskaznik jakosci
    
    rescaled_input(k) = Ke * error(k) - Ku * dUp;
    rescaled_input(k) = rescaled_input(k-1) + rescaled_input(k);
    
    % ograniczenia  
    if rescaled_input(k) - rescaled_input(k-1) >= dUmax
        rescaled_input(k) = dUmax + rescaled_input(k-1);
    elseif rescaled_input(k) - rescaled_input(k-1) <= -dUmax
        rescaled_input(k) = rescaled_input(k-1) - dUmax;
    end   
    
    input(k) = input(k) + rescaled_input(k);  
    
    if input(k) >= Umax
        input(k) = Umax;
    elseif input(k) <= Umin
        input(k) = Umin;
    end 
end
\end{lstlisting}


\section{Przebieg strojenia regulatora DMC}
Strojąc algorytm DMC na obiekcie laboratoryjnym bez zakłóceń skorzystaliśmy z doświadczenia zebranego na poprzednim laboratorium. Dla naszego regulatora wybraliśmy wartości parametrów D, N, $N_\mathrm{u}$, $\lambda$ dla których regulacja na poprzednim laboratorium przebiegała najlepiej.

Po dokonaniu eksperymentu dla $D = 500$, $N = 140$, $N_\mathrm{u} = 1$, 
$\lambda = 100$ otrzymaliśmy przebiegi zaprezentowane na rysunkach 
\ref{lab_zad4_lab_proces_wykres} i \ref{lab_zad4_lab_ster}. Możemy zauważyć, 
że nie pojawiają się przeregulowania, a wartość zadana jest osiągana bardzo szybko. 

\indent Przebiegi procesu sterowanego i sygnału sterującego są pozbawione oscylacji. 
Warto zauważyć, że sygnał sterujący charakteryzuje się łagodnymi zmianami, 
co jest bardzo ważne dla elementów wykonawczych sterowanych przez regulator. 
Tak dobrze nastrojony regulator posłużył nam do wykonania następnych 
zadań podczas laboratorium. 
\begin{figure}[t]
    
    \centering
    \begin{tikzpicture}
    \begin{axis}[
    width=\textwidth,
    xmin=4,xmax=450,ymin=29,ymax=41,
    xlabel={$k$},
    ylabel={$y[k]$},
    %xtick={0, 50, 100, 150, 200},
    %ytick={0, 2, 4, 6, 8},
    legend pos=south east,
    y tick label style={/pgf/number format/1000 sep=},
    ]
    \addplot[red, semithick] file{../data/lab/zad4/output_ts.csv};
	\addplot[blue, semithick] file{../data/lab/zad4/setpoint_ts.csv};    
    \legend{$y[k]$, $y^{\mathrm{zad}}[k]$}
    \end{axis} 
    \end{tikzpicture}
    \caption{Przebieg procesu sterowanego za pomocą dostrojonego regulatora DMC}
    \label{lab_zad4_lab_proces_wykres}
\end{figure}

\begin{figure}[b]
    
    \centering
    \begin{tikzpicture}
    \begin{axis}[
    width=\textwidth,
    xmin=4,xmax=450,ymin=25,ymax=64,
    xlabel={$k$},
    ylabel={$y[k]$},
    %xtick={0, 50, 100, 150, 200},
    %ytick={0, 2, 4, 6, 8},
    legend pos=south east,
    y tick label style={/pgf/number format/1000 sep=},
    ]
    \addplot[blue, semithick] file{../data/lab/zad4/input_ts.csv};    
    \legend{$y[k]$}
    \end{axis} 
    \end{tikzpicture}
    \caption{Przebieg sygnału sterującego dostrojonego regulatora DMC}
    \label{lab_zad4_lab_ster}
\end{figure}
\FloatBarrier