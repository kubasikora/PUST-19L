\chapter{Odpowiedź układu regulacji na zakłócenia}
\label{zad5}

\section{Dobór parametru $D_{\mathrm{z}}$}   
\label{zad5_dz} 
Aby jak najlepiej minimalizować wpływ zakłóceń mierzalnych na sterowany 
proces, postanowiliśmy zbadać wartość wskaźnika jakości dla każdej możliwej
wartości parametru z przedziału od $\num{1}$ do $D_{\mathrm{z}}^{max}$,
które jest równe horyzontowi dynamiki zakłóceń $D_{\mathrm{z}}^{max} = \num{147}$,
wyznaczonemu w \ref{zad3_wyznacznie_Dz}. Wyniki eksperymentu można nazwać co najmniej
zaskakującymi. Intuicja inżynierska podpowiadała nam że najlepsze efekty uzyskamy dla 
$D_{\mathrm{z}}$ równemu horyzontowi dynamiki. Symulacje pokazały jednak coś innego.

\subsection{Zakłócenia skokowe}
\label{zad5_dz_skok}
W pierwszej kolejności zbadaliśmy jak zmienia się wartość wskaźnika jakości regulacji
w funkcji parametru $D_{\mathrm{z}}$ przy regulacji procesu za pomocą algorytmu DMC z uwzględnieniem 
mierzalnego zakłócenia o skokowej zmianie z wartości $\num{0}$
do $\num{1}$ w chwili gdy proces osiągnie wartość zadaną wyjścia. Wynik eksperymentu przedstawiliśmy 
na rysunku \ref{zad5_dz_skok_wynik}. Okazało się że najmniejszą wartość wskaźnik jakości osiągnął 
dla $D_{\mathrm{z}}^{\mathrm{step}} = \num{55}$. 

\subsection{Zakłócenia sinusoidalne}
W kolejnym kroku zbadaliśmy wpływ parametru $D_{\mathrm{z}}$ na wskaźnik jakości regulacji przy zastosowaniu
regulatora DMC z uwzględnieniem mierzalnych zakłóceń o charakterze sinusoidalnym. Zakłócenie rozpoczynało się 
w tej samej chwili co w poprzedniej próbie, czyli po osiągnięciu przez proces wartości zadanej. Wynik tego eksperymentu
przedstawiliśmy na rysunku \ref{zad5_dz_sin_wynik}. W tym przypadku, najmniejszą wartość wskaźnika jakości uzyskaliśmy
dla $D_{\mathrm{z}}^{\mathrm{sine}} = \num{50}$.

\subsection{Ostateczny wybór}
Ostatecznie, aby jak najbardziej zminimalizować wpływ obu typów zakłóceń zdecydowaliśmy się na wartość średnią
z obu wyznaczonych parametrów $D_{\mathrm{z}}^{\mathrm{step}}$ i $D_{\mathrm{z}}^{\mathrm{sine}}$, ostatecznie
uzyskujące $D_{\mathrm{z}} = \num{53}$.

\begin{figure}[t]
    \centering
    \begin{tikzpicture}
    \begin{axis}[
    width=0.98\textwidth,
    xmin=1,xmax=147,ymin=20, ymax=45,
    xlabel={$D_{\mathrm{z}}$},
    ylabel={$E(D_{\mathrm{z}})$},
    legend pos=south east,
    y tick label style={/pgf/number format/1000 sep=},
    ]
    \addplot[blue, semithick] file{../data/zad5/find_Dz_step_min_idx_55.csv};
    \addlegendentry{$E(D_{\mathrm{z}})$},
    \addlegendimage{no markers, blue}
    \end{axis}
    \end{tikzpicture}
    \caption{Wskaźnik jakości regulacji $E$ w funkcji parametru $D_{\mathrm{z}}$ przy skokowym zakłóceniu}
    \label{zad5_dz_skok_wynik}
\end{figure}

\begin{figure}[b]
    \centering
    \begin{tikzpicture}
    \begin{axis}[
    width=0.98\textwidth,
    xmin=1,xmax=147,ymin=35, ymax=200,
    xlabel={$D_{\mathrm{z}}$},
    ylabel={$E(D_{\mathrm{z}})$},
    legend pos=south east,
    y tick label style={/pgf/number format/1000 sep=},
    ]
    \addplot[blue, semithick] file{../data/zad5/find_Dz_sine_min_idx_50.csv};
    \addlegendentry{$E(D_{\mathrm{z}})$},
    \addlegendimage{no markers, blue}
    \end{axis}
    \end{tikzpicture}
    \caption{Wskaźnik jakości regulacji $E$ w funkcji parametru $D_{\mathrm{z}}$ przy sinusoidalnym zakłóceniu}
    \label{zad5_dz_sin_wynik}
\end{figure}
\FloatBarrier

\section{Działanie algorytmu regulacji przy skokowej zmianie zakłócenia}
\label{zad5_step_dist}
Uwzględnienie zakłóceń mierzalnych w rozsądny sposób, powinno w każdym przypadku 
poprawić jakość regulacji. W kolejnych eksperymentach sprawdziliśmy zachowanie procesu
przy sterowaniu generowanym przez algorytm DMC w trzech przypadkach:

\begin{itemize}
    \item bez uwzględnienia mierzalnego zakłócenia
    \item z uwzględnieniem mierzalnego zakłócenia z nieoptymalizowanym parametrem $D_{\mathrm{z}} = \num{147}$
    \item z uwzględnieniem mierzalnego zakłócenia z optymalizowanym parametrem $D_{\mathrm{z}} = \num{53}$
\end{itemize}

\begin{figure}[b]
    \centering
    \begin{tikzpicture}
    \begin{axis}[
    width=0.98\textwidth,
    xmin=1,xmax=500,ymin=0, ymax=1.5,
    xlabel={$k$},
    ylabel={$z[k]$},
    legend pos=south east,
    y tick label style={/pgf/number format/1000 sep=},
    ]
    \addplot[blue, semithick] file{../data/zad5/zad5_disturbance_N_70_Nu_1_lambda_10_D_jump_184_Dz_53D_0.csv};
    \addlegendentry{$z[k]$},
    \addlegendimage{no markers, blue}
    \end{axis}
    \end{tikzpicture}
    \caption{Postać badanych zakłóceń - skokowa zmiana z $\num{0}$ do $\num{1}$}
    \label{zad5_step_disturbance_z}
\end{figure}
\FloatBarrier


\subsection{Regulacja bez uwzględnienia mierzalnego zakłócenia}
W przypadku regulacji bez pomiaru mierzalnego zakłócenia, uzyskaliśmy wskaźnik jakości regulacji
równy $E = \num{42,9247}$. Wynik symulacji znajduje się na rysunkach \ref{zad5_step_disturbance_without_y}
i \ref{zad5_step_disturbance_without_u}.

\subsection{Regulacja z uwzględnienia mierzalnego zakłócenia bez optymalizacji parametru $D_{\mathrm{z}}$}
W przypadku regulacji z pomiaru mierzalnego zakłócenia ale bez optymalizacji parametru $D_{\mathrm{z}}$, uzyskaliśmy wskaźnik jakości regulacji
równy $E = \num{23,5708}$. Wynik symulacji znajduje się na rysunkach \ref{zad5_step_disturbance_with_opt_without_y}
i \ref{zad5_step_disturbance_with_opt_without_u}.

\subsection{Regulacja z uwzględnienia mierzalnego zakłócenia Z optymalizacją parametru $D_{\mathrm{z}}$}
W przypadku regulacji z pomiaru mierzalnego zakłócenia z zoptymalizowanym parametrem $D_{\mathrm{z}}$, uzyskaliśmy wskaźnik jakości regulacji
równy $E = \num{22,5943}$. Wynik symulacji znajduje się na rysunkach \ref{zad5_step_disturbance_with_opt_with_y}
i \ref{zad5_step_disturbance_with_opt_with_u}.

\subsection{Wnioski}
Zgodnie z naszymi oczekiwaniami, najlepsze rezultaty uzyskaliśmy przy sterowaniu algorytmem DMC 
z uwzględnieniem zakłóceń mierzalnych i z zoptymalizowanym parametrem $D_{\mathrm{z}} = \num{53}$. Porównawcze przebiegi
wyjścia i wejścia wszystkich trzech symulacji zostały przedstawione na rysunkach \ref{zad5_step_disturbance_comp_y}
i \ref{zad5_step_disturbance_comp_u}.


\begin{figure}[t]
    \centering
    \begin{tikzpicture}
    \begin{axis}[
    width=0.98\textwidth,
    xmin=0,xmax=500,ymin=0, ymax=2.5,
    xlabel={$k$},
    ylabel={$y[k]$},
    %xtick={0, 100, 200, 300, 400, 500},
    %ytick={3.9, 4, 4.1, 4.2, 4.3, 4.4},
    legend pos=south east,
    y tick label style={/pgf/number format/1000 sep=},
    ]
    \addplot[blue, semithick] file{../data/zad5/zad5_stpt_N_70_Nu_1_lambda_10_D_jump_184D_0.csv};
    \addplot[red, semithick] file{../data/zad5/zad5_output_N_70_Nu_1_lambda_10_D_jump_184D_0.csv};
    \addlegendentry{$y^{\mathrm{zad}}[k]$},
    \addlegendentry{$y[k]$},
    \addlegendimage{no markers, blue}
	\addlegendimage{no markers, red}
    \end{axis}
    \end{tikzpicture}
    \caption{Przebieg procesu sterowanego za pomocą regulatora bez uwzględniania skokowego zakłócenia}
    \label{zad5_step_disturbance_without_y}
\end{figure}

\begin{figure}[b]
    \centering
    \begin{tikzpicture}
    \begin{axis}[
    width=0.98\textwidth,
    xmin=0,xmax=500,ymin=-0.5, ymax=1,
    xlabel={$k$},
    ylabel={$u[k]$},
    %xtick={0, 0.4, 0.5, 0.6, 0.7},
    %ytick={3.5, 3.75, 4, 4.25, 4.5},
    legend pos=south east,
    y tick label style={/pgf/number format/1000 sep=},
    ]
    \addplot[const plot, blue, semithick] file{../data/zad5/zad5_input_N_70_Nu_1_lambda_10_D_jump_184D_0.csv};
    \legend{$u[k]$}
    \end{axis}
    \end{tikzpicture}
    \caption{Przebieg sygnału sterującego bez uwzględniania skokowego zakłócenia}
    \label{zad5_step_disturbance_without_u}
\end{figure}

%%%%

\begin{figure}[t]
    \centering
    \begin{tikzpicture}
    \begin{axis}[
    width=0.98\textwidth,
    xmin=0,xmax=500,ymin=0, ymax=2.5,
    xlabel={$k$},
    ylabel={$y[k]$},
    %xtick={0, 100, 200, 300, 400, 500},
    %ytick={3.9, 4, 4.1, 4.2, 4.3, 4.4},
    legend pos=south east,
    y tick label style={/pgf/number format/1000 sep=},
    ]
    \addplot[blue, semithick] file{../data/zad5/zad5_stpt_N_70_Nu_1_lambda_10_D_jump_184D_1.csv};
    \addplot[red, semithick] file{../data/zad5/zad5_output_N_70_Nu_1_lambda_10_D_jump_184D_1.csv};
    \addlegendentry{$y^{\mathrm{zad}}[k]$},
    \addlegendentry{$y[k]$},
    \addlegendimage{no markers, blue}
	\addlegendimage{no markers, red}
    \end{axis}
    \end{tikzpicture}
    \caption{Przebieg procesu sterowanego za pomocą regulatora z uwzględnieiem skokowego zakłócenia 
    z niezoptymalizowanym parametrem $D_{\mathrm{z}} = \num{147}$}
    \label{zad5_step_disturbance_with_opt_without_y}
\end{figure}

\begin{figure}[b]
    \centering
    \begin{tikzpicture}
    \begin{axis}[
    width=0.98\textwidth,
    xmin=0,xmax=500,ymin=-1, ymax=1,
    xlabel={$k$},
    ylabel={$u[k]$},
    %xtick={0, 0.4, 0.5, 0.6, 0.7},
    %ytick={3.5, 3.75, 4, 4.25, 4.5},
    legend pos=south east,
    y tick label style={/pgf/number format/1000 sep=},
    ]
    \addplot[const plot, blue, semithick] file{../data/zad5/zad5_input_N_70_Nu_1_lambda_10_D_jump_184D_1.csv};
    \legend{$u[k]$}
    \end{axis}
    \end{tikzpicture}
    \caption{Przebieg sygnału sterującego z uwzględnieiem skokowego zakłócenia 
    z niezoptymalizowanym parametrem $D_{\mathrm{z}} = \num{147}$}
    \label{zad5_step_disturbance_with_opt_without_u}
\end{figure}

%%% 

\begin{figure}[t]
    \centering
    \begin{tikzpicture}
    \begin{axis}[
    width=0.98\textwidth,
    xmin=0,xmax=500,ymin=0, ymax=2.5,
    xlabel={$k$},
    ylabel={$y[k]$},
    %xtick={0, 100, 200, 300, 400, 500},
    %ytick={3.9, 4, 4.1, 4.2, 4.3, 4.4},
    legend pos=south east,
    y tick label style={/pgf/number format/1000 sep=},
    ]
    \addplot[blue, semithick] file{../data/zad5/zad5_stpt_N_70_Nu_1_lambda_10_D_jump_184_Dz_53D_1.csv};
    \addplot[red, semithick] file{../data/zad5/zad5_output_N_70_Nu_1_lambda_10_D_jump_184_Dz_53D_1.csv};
    \addlegendentry{$y^{\mathrm{zad}}[k]$},
    \addlegendentry{$y[k]$},
    \addlegendimage{no markers, blue}
	\addlegendimage{no markers, red}
    \end{axis}
    \end{tikzpicture}
    \caption{Przebieg procesu sterowanego za pomocą regulatora z uwzględnieiem skokowego zakłócenia 
    z zoptymalizowanym parametrem $D_{\mathrm{z}} = \num{147}$}
    \label{zad5_step_disturbance_with_opt_with_y}
\end{figure}

\begin{figure}[b]
    \centering
    \begin{tikzpicture}
    \begin{axis}[
    width=0.98\textwidth,
    xmin=0,xmax=500,ymin=-1, ymax=1,
    xlabel={$k$},
    ylabel={$u[k]$},
    %xtick={0, 0.4, 0.5, 0.6, 0.7},
    %ytick={3.5, 3.75, 4, 4.25, 4.5},
    legend pos=south east,
    y tick label style={/pgf/number format/1000 sep=},
    ]
    \addplot[const plot, blue, semithick] file{../data/zad5/zad5_input_N_70_Nu_1_lambda_10_D_jump_184_Dz_53D_1.csv};
    \legend{$u[k]$}
    \end{axis}
    \end{tikzpicture}
    \caption{Przebieg sygnału sterującego z uwzględnieiem skokowego zakłócenia 
    z zoptymalizowanym parametrem $D_{\mathrm{z}} = \num{147}$}
    \label{zad5_step_disturbance_with_opt_with_u}
\end{figure}

%%% 

\begin{figure}[t]
    \centering
    \begin{tikzpicture}
    \begin{axis}[
    width=0.98\textwidth,
    xmin=0,xmax=500,ymin=0, ymax=2.5,
    xlabel={$k$},
    ylabel={$y[k]$},
    %xtick={0, 100, 200, 300, 400, 500},
    %ytick={3.9, 4, 4.1, 4.2, 4.3, 4.4},
    legend pos=south east,
    y tick label style={/pgf/number format/1000 sep=},
    ]
    \addplot[blue, dotted] file{../data/zad5/zad5_output_N_70_Nu_1_lambda_10_D_jump_184D_0.csv};
    \addplot[red, semithick] file{../data/zad5/zad5_output_N_70_Nu_1_lambda_10_D_jump_184D_1.csv};
    \addplot[green, dash dot] file{../data/zad5/zad5_output_N_70_Nu_1_lambda_10_D_jump_184_Dz_53D_1.csv};
    \addlegendentry{$D_{\mathrm{z}} = \num{0}$},
    \addlegendentry{$D_{\mathrm{z}} = \num{55}$},
    \addlegendentry{$D_{\mathrm{z}} = \num{147}$},
    \addlegendimage{no markers, blue}
    \addlegendimage{no markers, red}
    \addlegendimage{no markers, green}
    \end{axis}
    \end{tikzpicture}
    \caption{Porównanie przebiegów wyjściowych}
    \label{zad5_step_disturbance_comp_y}
\end{figure}

\begin{figure}[b]
    \centering
    \begin{tikzpicture}
    \begin{axis}[
    width=0.98\textwidth,
    xmin=0,xmax=500,ymin=-0.8, ymax=0.8,
    xlabel={$k$},
    ylabel={$u[k]$},
    %xtick={0, 100, 200, 300, 400, 500},
    %ytick={3.9, 4, 4.1, 4.2, 4.3, 4.4},
    legend pos=south east,
    y tick label style={/pgf/number format/1000 sep=},
    ]
    \addplot[blue, dotted] file{../data/zad5/zad5_input_N_70_Nu_1_lambda_10_D_jump_184D_0.csv};
    \addplot[red, semithick] file{../data/zad5/zad5_input_N_70_Nu_1_lambda_10_D_jump_184D_1.csv};
    \addplot[green, dash dot] file{../data/zad5/zad5_input_N_70_Nu_1_lambda_10_D_jump_184_Dz_53D_1.csv};
    \addlegendentry{$D_{\mathrm{z}} = \num{0}$},
    \addlegendentry{$D_{\mathrm{z}} = \num{55}$},
    \addlegendentry{$D_{\mathrm{z}} = \num{147}$},
    \addlegendimage{no markers, blue}
    \addlegendimage{no markers, red}
    \addlegendimage{no markers, green}
    \end{axis}
    \end{tikzpicture}
    \caption{Porównanie generowanego sterowania}
    \label{zad5_step_disturbance_comp_u}
\end{figure}

\section{Działanie algorytmu regulacji przy zakłóceniu sinusoidalnym}
\label{zad5_sine_dist}
Wyżej przedstawione eksperymenty ponowiliśmy dla zakłócenia sinusoidalnego, aby zbadać
jak zachowa się obiekt wystawiony na działanie ciągle zmiennego zakłócenia. W przypadku
gdy algorytm bierze pod uwagę tylko przeszłe wartości zakłócenia, spodziewamy się że 
wyjście procesu będzie stale wytrącone z wartości zadanej. Tak jak poprzednio zdecydowaliśmy
się na przeprowadzenie trzech eksperymentów:

\begin{itemize}
    \item bez uwzględnienia mierzalnego zakłócenia
    \item z uwzględnieniem mierzalnego zakłócenia z nieoptymalizowanym parametrem $D_{\mathrm{z}} = \num{147}$
    \item z uwzględnieniem mierzalnego zakłócenia z optymalizowanym parametrem $D_{\mathrm{z}} = \num{53}$
\end{itemize}

\begin{figure}[b]
    \centering
    \begin{tikzpicture}
    \begin{axis}[
    width=0.98\textwidth,
    xmin=1,xmax=500,ymin=-1.2, ymax=1.2,
    xlabel={$k$},
    ylabel={$z[k]$},
    legend pos=south east,
    y tick label style={/pgf/number format/1000 sep=},
    ]
    \addplot[blue, semithick] file{../data/zad6/zad6_disturbance_N_70_Nu_1_lambda_10_D_jump_184_Dz_53D_1.csv};
    \addlegendentry{$z[k]$},
    \addlegendimage{no markers, blue}
    \end{axis}
    \end{tikzpicture}
    \caption{Postać badanych zakłóceń - sinusoidalny przebieg zakłóceń o amplitudzie $\num{1}$.}
    \label{zad5_sine_disturbance_z}
\end{figure}
\FloatBarrier


\subsection{Regulacja bez uwzględnienia mierzalnego zakłócenia}
W przypadku regulacji bez pomiaru mierzalnego zakłócenia, uzyskaliśmy wskaźnik jakości regulacji
równy $E = \num{185,2752}$. Wynik symulacji znajduje się na rysunkach \ref{zad5_sine_disturbance_without_y}
i \ref{zad5_sine_disturbance_without_u}.

\subsection{Regulacja z uwzględnienia mierzalnego zakłócenia bez optymalizacji parametru $D_{\mathrm{z}}$}
W przypadku regulacji z pomiaru mierzalnego zakłócenia ale bez optymalizacji parametru $D_{\mathrm{z}}$, uzyskaliśmy wskaźnik jakości regulacji
równy $E = \num{68,2297}$. Wynik symulacji znajduje się na rysunkach \ref{zad5_sine_disturbance_with_opt_without_y}
i \ref{zad5_sine_disturbance_with_opt_without_u}.

\subsection{Regulacja z uwzględnienia mierzalnego zakłócenia Z optymalizacją parametru $D_{\mathrm{z}}$}
W przypadku regulacji z pomiaru mierzalnego zakłócenia z zoptymalizowanym parametrem $D_{\mathrm{z}}$, uzyskaliśmy wskaźnik jakości regulacji
równy $E = \num{50,1943}$. Wynik symulacji znajduje się na rysunkach \ref{zad5_sine_disturbance_with_opt_with_y}
i \ref{zad5_sine_disturbance_with_opt_with_u}.

\subsection{Wnioski}
Zgodnie z naszymi oczekiwaniami, najlepsze rezultaty uzyskaliśmy przy sterowaniu algorytmem DMC 
z uwzględnieniem zakłóceń mierzalnych i z zoptymalizowanym parametrem $D_{\mathrm{z}} = \num{53}$. W przypadku zmiennego 
zakłócenia, optymalizacja parametru $D_{\mathrm{z}}$ przyniosła jeszcze większą poprawę niż w przypadku
stałego zakłócenia. Porównawcze przebiegi
wyjścia i wejścia wszystkich trzech symulacji zostały przedstawione na rysunkach \ref{zad5_sine_disturbance_comp_y}
i \ref{zad5_sine_disturbance_comp_u}. 

Aby jeszcze bardziej zminimalizować efekt zakłócenia, należałoby przewidywać
przyszłą wartość zakłócenia. W tym celu należy skonstruować model zakłócenia. Można zrobić to poprzez 
zaprojektowanie autoregresyjnego modelu AR niskiego rzędu. Niestety nie jest to tematem tego projektu.  

\begin{figure}[t]
    \centering
    \begin{tikzpicture}
    \begin{axis}[
    width=0.98\textwidth,
    xmin=0,xmax=500,ymin=-0.4, ymax=2.4,
    xlabel={$k$},
    ylabel={$y[k]$},
    %xtick={0, 100, 200, 300, 400, 500},
    %ytick={3.9, 4, 4.1, 4.2, 4.3, 4.4},
    legend pos=south east,
    y tick label style={/pgf/number format/1000 sep=},
    ]
    \addplot[blue, semithick] file{../data/zad6/zad6_stpt_N_70_Nu_1_lambda_10_D_jump_184D_0.csv};
    \addplot[red, semithick] file{../data/zad6/zad6_output_N_70_Nu_1_lambda_10_D_jump_184D_0.csv};
    \addlegendentry{$y^{\mathrm{zad}}[k]$},
    \addlegendentry{$y[k]$},
    \addlegendimage{no markers, blue}
	\addlegendimage{no markers, red}
    \end{axis}
    \end{tikzpicture}
    \caption{Przebieg procesu sterowanego za pomocą regulatora bez uwzględniania skokowego zakłócenia}
    \label{zad5_sine_disturbance_without_y}
\end{figure}

\begin{figure}[b]
    \centering
    \begin{tikzpicture}
    \begin{axis}[
    width=0.98\textwidth,
    xmin=0,xmax=500,ymin=-0.4, ymax=1.2,
    xlabel={$k$},
    ylabel={$u[k]$},
    %xtick={0, 0.4, 0.5, 0.6, 0.7},
    %ytick={3.5, 3.75, 4, 4.25, 4.5},
    legend pos=south east,
    y tick label style={/pgf/number format/1000 sep=},
    ]
    \addplot[const plot, blue, semithick] file{../data/zad6/zad6_input_N_70_Nu_1_lambda_10_D_jump_184D_0.csv};
    \legend{$u[k]$}
    \end{axis}
    \end{tikzpicture}
    \caption{Przebieg sygnału sterującego bez uwzględniania skokowego zakłócenia}
    \label{zad5_sine_disturbance_without_u}
\end{figure}

%%%%

\begin{figure}[t]
    \centering
    \begin{tikzpicture}
    \begin{axis}[
    width=0.98\textwidth,
    xmin=0,xmax=500,ymin=0, ymax=2,
    xlabel={$k$},
    ylabel={$y[k]$},
    %xtick={0, 100, 200, 300, 400, 500},
    %ytick={3.9, 4, 4.1, 4.2, 4.3, 4.4},
    legend pos=south east,
    y tick label style={/pgf/number format/1000 sep=},
    ]
    \addplot[blue, semithick] file{../data/zad6/zad6_stpt_N_70_Nu_1_lambda_10_D_jump_184D_1.csv};
    \addplot[red, semithick] file{../data/zad6/zad6_output_N_70_Nu_1_lambda_10_D_jump_184D_1.csv};
    \addlegendentry{$y^{\mathrm{zad}}[k]$},
    \addlegendentry{$y[k]$},
    \addlegendimage{no markers, blue}
	\addlegendimage{no markers, red}
    \end{axis}
    \end{tikzpicture}
    \caption{Przebieg procesu sterowanego za pomocą regulatora z uwzględnieiem skokowego zakłócenia 
    z niezoptymalizowanym parametrem $D_{\mathrm{z}} = \num{147}$}
    \label{zad5_sine_disturbance_with_opt_without_y}
\end{figure}

\begin{figure}[b]
    \centering
    \begin{tikzpicture}
    \begin{axis}[
    width=0.98\textwidth,
    xmin=0,xmax=500,ymin=-0.6, ymax=1.5,
    xlabel={$k$},
    ylabel={$u[k]$},
    %xtick={0, 0.4, 0.5, 0.6, 0.7},
    %ytick={3.5, 3.75, 4, 4.25, 4.5},
    legend pos=south east,
    y tick label style={/pgf/number format/1000 sep=},
    ]
    \addplot[const plot, blue, semithick] file{../data/zad6/zad6_input_N_70_Nu_1_lambda_10_D_jump_184D_1.csv};
    \legend{$u[k]$}
    \end{axis}
    \end{tikzpicture}
    \caption{Przebieg sygnału sterującego z uwzględnieiem skokowego zakłócenia 
    z niezoptymalizowanym parametrem $D_{\mathrm{z}} = \num{147}$}
    \label{zad5_sine_disturbance_with_opt_without_u}
\end{figure}

%%% 

\begin{figure}[t]
    \centering
    \begin{tikzpicture}
    \begin{axis}[
    width=0.98\textwidth,
    xmin=0,xmax=500,ymin=0, ymax=2,
    xlabel={$k$},
    ylabel={$y[k]$},
    %xtick={0, 100, 200, 300, 400, 500},
    %ytick={3.9, 4, 4.1, 4.2, 4.3, 4.4},
    legend pos=south east,
    y tick label style={/pgf/number format/1000 sep=},
    ]
    \addplot[blue, semithick] file{../data/zad6/zad6_stpt_N_70_Nu_1_lambda_10_D_jump_184_Dz_53D_1.csv};
    \addplot[red, semithick] file{../data/zad6/zad6_output_N_70_Nu_1_lambda_10_D_jump_184_Dz_53D_1.csv};
    \addlegendentry{$y^{\mathrm{zad}}[k]$},
    \addlegendentry{$y[k]$},
    \addlegendimage{no markers, blue}
	\addlegendimage{no markers, red}
    \end{axis}
    \end{tikzpicture}
    \caption{Przebieg procesu sterowanego za pomocą regulatora z uwzględnieiem skokowego zakłócenia 
    z zoptymalizowanym parametrem $D_{\mathrm{z}} = \num{147}$}
    \label{zad5_sine_disturbance_with_opt_with_y}
\end{figure}

\begin{figure}[b]
    \centering
    \begin{tikzpicture}
    \begin{axis}[
    width=0.98\textwidth,
    xmin=0,xmax=500,ymin=-0.5, ymax=1.5,
    xlabel={$k$},
    ylabel={$u[k]$},
    %xtick={0, 0.4, 0.5, 0.6, 0.7},
    %ytick={3.5, 3.75, 4, 4.25, 4.5},
    legend pos=south east,
    y tick label style={/pgf/number format/1000 sep=},
    ]
    \addplot[const plot, blue, semithick] file{../data/zad6/zad6_input_N_70_Nu_1_lambda_10_D_jump_184_Dz_53D_1.csv};
    \legend{$u[k]$}
    \end{axis}
    \end{tikzpicture}
    \caption{Przebieg sygnału sterującego z uwzględnieiem skokowego zakłócenia 
    z zoptymalizowanym parametrem $D_{\mathrm{z}} = \num{147}$}
    \label{zad5_sine_disturbance_with_opt_with_u}
\end{figure}

%%% 

\begin{figure}[t]
    \centering
    \begin{tikzpicture}
    \begin{axis}[
    width=0.98\textwidth,
    xmin=0,xmax=500,ymin=-0.5, ymax=2.5,
    xlabel={$k$},
    ylabel={$y[k]$},
    %xtick={0, 100, 200, 300, 400, 500},
    %ytick={3.9, 4, 4.1, 4.2, 4.3, 4.4},
    legend pos=south east,
    y tick label style={/pgf/number format/1000 sep=},
    ]
    \addplot[blue, dotted] file{../data/zad6/zad6_output_N_70_Nu_1_lambda_10_D_jump_184D_0.csv};
    \addplot[red, semithick] file{../data/zad6/zad6_output_N_70_Nu_1_lambda_10_D_jump_184D_1.csv};
    \addplot[green, dash dot] file{../data/zad6/zad6_output_N_70_Nu_1_lambda_10_D_jump_184_Dz_53D_1.csv};
    \addlegendentry{$D_{\mathrm{z}} = \num{0}$},
    \addlegendentry{$D_{\mathrm{z}} = \num{55}$},
    \addlegendentry{$D_{\mathrm{z}} = \num{147}$},
    \addlegendimage{no markers, blue}
    \addlegendimage{no markers, red}
    \addlegendimage{no markers, green}
    \end{axis}
    \end{tikzpicture}
    \caption{Porównanie przebiegów wyjściowych}
    \label{zad5_sine_disturbance_comp_y}
\end{figure}

\begin{figure}[b]
    \centering
    \begin{tikzpicture}
    \begin{axis}[
    width=0.98\textwidth,
    xmin=0,xmax=500,ymin=-1.2, ymax=1.5,
    xlabel={$k$},
    ylabel={$u[k]$},
    %xtick={0, 100, 200, 300, 400, 500},
    %ytick={3.9, 4, 4.1, 4.2, 4.3, 4.4},
    legend pos=south east,
    y tick label style={/pgf/number format/1000 sep=},
    ]
    \addplot[blue, dotted] file{../data/zad6/zad6_input_N_70_Nu_1_lambda_10_D_jump_184D_0.csv};
    \addplot[red, semithick] file{../data/zad6/zad6_input_N_70_Nu_1_lambda_10_D_jump_184D_1.csv};
    \addplot[green, dash dot] file{../data/zad6/zad6_input_N_70_Nu_1_lambda_10_D_jump_184_Dz_53D_1.csv};
    \addlegendentry{$D_{\mathrm{z}} = \num{0}$},
    \addlegendentry{$D_{\mathrm{z}} = \num{55}$},
    \addlegendentry{$D_{\mathrm{z}} = \num{147}$},
    \addlegendimage{no markers, blue}
    \addlegendimage{no markers, red}
    \addlegendimage{no markers, green}
    \end{axis}
    \end{tikzpicture}
    \caption{Porównanie generowanego sterowania}
    \label{zad5_sine_disturbance_comp_u}
\end{figure}

