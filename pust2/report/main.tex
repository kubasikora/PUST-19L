\documentclass{mwrep}

% Polskie znaki
\usepackage{polski}
\usepackage[utf8]{inputenc}
\usepackage[T1]{fontenc}
\usepackage{lmodern}
\usepackage{indentfirst}

% Strona tytułowa
\usepackage{pgfplots}
\usepackage{siunitx}
\usepackage{paracol}
\usepackage{gensymb}

% Pływające obrazki
\usepackage{float}
\usepackage{svg}
\usepackage{graphicx}

% table of contents refs
\usepackage{hyperref}
\usepackage{cleveref}
\usepackage{booktabs}
\usepackage{listings}
\usepackage{placeins}
\usepackage{xcolor}

\sisetup{detect-weight,exponent-product=\cdot,output-decimal-marker={,},per-mode=symbol,binary-units=true,range-phrase={-},range-units=single}
\definecolor{szary}{rgb}{0.95,0.95,0.95}
%konfiguracje pakietu listings
\lstset{
	backgroundcolor=\color{szary},
	frame=single,
	breaklines=true,
}
\lstdefinestyle{customlatex}{
	basicstyle=\footnotesize\ttfamily,
	%basicstyle=\small\ttfamily,
}
\lstdefinestyle{customc}{
	breaklines=true,
	frame=tb,
	language=C,
	xleftmargin=0pt,
	showstringspaces=false,
	basicstyle=\small\ttfamily,
	keywordstyle=\bfseries\color{green!40!black},
	commentstyle=\itshape\color{purple!40!black},
	identifierstyle=\color{blue},
	stringstyle=\color{orange},
}
\lstdefinestyle{custommatlab}{
	captionpos=t,
	breaklines=true,
	frame=tb,
	xleftmargin=0pt,
	language=matlab,
	showstringspaces=false,
	%basicstyle=\footnotesize\ttfamily,
	basicstyle=\scriptsize\ttfamily,
	keywordstyle=\bfseries\color{green!40!black},
	commentstyle=\itshape\color{purple!40!black},
	identifierstyle=\color{blue},
	stringstyle=\color{orange},
}

%wymiar tekstu
\def\figurename{Rys.}
\def\tablename{Tab.}

%konfiguracja liczby p�ywaj�cych element�w
\setcounter{topnumber}{0}%2
\setcounter{bottomnumber}{3}%1
\setcounter{totalnumber}{5}%3
\renewcommand{\textfraction}{0.01}%0.2
\renewcommand{\topfraction}{0.95}%0.7
\renewcommand{\bottomfraction}{0.95}%0.3
\renewcommand{\floatpagefraction}{0.35}%0.5

\SendSettingsToPgf
\title{\bf Sprawozdanie z projektu i ćwiczenia laboratoryjnego nr 2, zadanie nr 1 \vskip 0.1cm}
\author{Marcin Dolicher \\ Jakub Sikora \\ Robert Wojtaś}
\date{\today}
\pgfplotsset{compat=1.15}	
\begin{document}
\frenchspacing
\pagestyle{uheadings}

\makeatletter
\renewcommand{\maketitle}{\begin{titlepage}
		\begin{center}{
				\LARGE {\bf Politechnika Warszawska}}\\
            \vspace{0.4cm}
            \leftskip-0.9cm
            {\LARGE {\bf \mbox{Wydział Elektroniki i Technik Informacyjnych}}}\\
            \vspace{0.2cm}
            {\LARGE {\bf \mbox{Instytut Automatyki i Informatyki Stosowanej}}}\\
            
            \vspace{5cm}
            \leftskip-1.3cm
			{\bf \Huge \mbox{Projektowanie układów sterowania} \vskip 0.1cm}
		\end{center}
		\vspace{0.1cm}

		\begin{center}
			{\bf \LARGE \@title}
		\end{center}

		\vspace{9cm}
		\begin{paracol}{2}
			\addtocontents{toc}{\protect\setcounter{tocdepth}{1}}
			\subsection*{Zespół:}
			\bf{ \Large{ \noindent\@author \par}}
			\addtocontents{toc}{\protect\setcounter{tocdepth}{2}}

			\switchcolumn \addtocontents{toc}{\protect\setcounter{tocdepth}{1}}
			\subsection*{Prowadzący:}
			\bf{\Large{\noindent dr inż. Patryk Chaber}}
			\addtocontents{toc}{\protect\setcounter{tocdepth}{2}}

		\end{paracol}
		\vspace*{\stretch{6}}
		\begin{center}
			\bf{\large{Warszawa, \@date\vskip 0.1cm}}
		\end{center}
	\end{titlepage}
}
\makeatother
\maketitle

\tableofcontents
%\iffalse
\part{Projekt}
\label{PROJEKT}
%\chapter{Sprawdzenie poprawności wartości sygnałów $U_{\mathrm{pp}}$ i $Y_{\mathrm{pp}}$}
\label{zad1}

\section{Opis eksperymentu}
\label{zad1_opis}
Aby sprawdzić poprawność wartości sygnałów $U_{\mathrm{pp}}$ i $Y_{\mathrm{pp}}$ wykonaliśmy eksperyment polegający na
pobudzeniu wejścia obiektu stałym sygnałem o wartości $U_{\mathrm{pp}}$ i sprawdzeniu czy sygnał wyjściowy stabilizuje 
się na wartości $Y_{\mathrm{pp}}$. Symulację obiektu przeprowadziliśmy za pomocą funkcji 
\verb+symulacja_obiektu1Y+. Spodziewanym wyjściem procesu dla wejścia $U_{\mathrm{pp}} = \num{0.5}$ jest
wyjście równe $Y_{\mathrm{pp}} = \num{4}$.

\section{Realizacja badań}
\label{zad_realizacja}
Eksperyment przeprowadziliśmy za pomocą skryptu \verb+zad1.m+. Skrypt ten automatycznie przeprowadza symulację 
oraz zapisuje zebrane dane do pliku \verb+csv+. Zebrana odpowiedź obiektu została przedstawiona poniżej na rysunku \ref{zad1_output}. 
Obiekt przy stałym pobudzeniu $U_{\mathrm{pp}}=\num{0.5}$, stabilizuje sygnał wyjściowy na wartości $Y_{\mathrm{pp}}=\num{4}$.
Jednoznacznie potwierdza to poprawność zadanych wartości.

\begin{figure}[b]
    \label{zad1_output_sprawdzenie}
    \centering
    \begin{tikzpicture}
    \begin{axis}[
    width=\textwidth,
    xmin=0,xmax=200,ymin=0,ymax=8,
    xlabel={$k$},
    ylabel={$y[k]$},
    xtick={0, 50, 100, 150, 200},
    ytick={0, 2, 4, 6, 8},
    legend pos=south east,
    y tick label style={/pgf/number format/1000 sep=},
    ]
    \addplot[red, semithick] file{../data/zad1_output.csv};
    \legend{$y[k]$}
    \end{axis} 
    \end{tikzpicture}
    \caption{Odpowiedź symulowanego procesu na stałe wejście o wartości $U_{\mathrm{pp}}=\num{0.5}$}
\end{figure}
\chapter{Wyznaczenie odpowiedzi skokowych obiektu}
\label{zad2}

\section{Odpowiedź skokowa toru wejście-wyjście}
Eksperymenty wykonywane były z punktu pracy zdefiniowanego w zadaniu, przy sygnale zakłócenia $Z_{\mathrm{pp}} = 0$. Na wykresie \ref{zad2_odp_skok_input} możemy zaobserwować, że wraz z wzrostem wartości skoku rośnie również wartość odpowiedzi procesu $y[k]$. Zachowanie procesu jest zgodne 
z typową charakterystyką obiektów dynamicznych liniowych, w których wartość ustalona zmienia się liniowo wraz ze zmianą sygnału sterującego. 

\begin{figure}[t]
    \centering
    \begin{tikzpicture}
    \begin{axis}[
    width=0.98\textwidth,
    xmin=0.0,xmax=200,ymin=-1.5,ymax=1.5,
    xlabel={$k$},
    ylabel={$y[k]$},
    legend pos=south east,
    y tick label style={/pgf/number format/1000 sep=},
    ] 
    \addlegendentry{$\Delta u = \num{0,1}$},
    \addlegendentry{$\Delta u = \num{0,2}$}
    \addlegendentry{$\Delta u = \num{0,35}$},
    \addlegendentry{$\Delta u = \num{0,5}$},
    \addlegendentry{$\Delta u = \num{-0,1}$},
    \addlegendentry{$\Delta u = \num{-0,2}$},
    \addlegendentry{$\Delta u = \num{-0,35}$},
    \addlegendentry{$\Delta u = \num{-0,5}$},
    \addlegendimage{no markers,green}
	\addlegendimage{no markers,red}
	\addlegendimage{no markers,yellow}
	\addlegendimage{no markers,blue}
	\addlegendimage{no markers,black}
	\addlegendimage{no markers,orange}
	\addlegendimage{no markers,brown}
	\addlegendimage{no markers,violet}
    \addplot[green, semithick, thick] file{../data/zad2/zad2_input_output0.1.csv};
    \addplot[red, semithick, thick] file{../data/zad2/zad2_input_output0.2.csv};
    \addplot[yellow, semithick, thick] file{../data/zad2/zad2_input_output0.35.csv};
    \addplot[blue, semithick, thick] file{../data/zad2/zad2_input_output0.5.csv};
    \addplot[black, semithick, thick] file{../data/zad2/zad2_input_output-0.1.csv};
    \addplot[orange, semithick, thick] file{../data/zad2/zad2_input_output-0.2.csv};
    \addplot[brown, semithick, thick] file{../data/zad2/zad2_input_output-0.35.csv};
    \addplot[violet, semithick, thick] file{../data/zad2/zad2_input_output-0.5.csv};    
    \end{axis}
    \end{tikzpicture}
    \caption{Odpowiedzi procesu na skokową zmianę sygnału sterującego}
    \label{zad2_odp_skok_input}
\end{figure}

\section{Odpowiedź skokowa toru zakłócenie-wyjście}

Odpowiedź skokową toru zakłócenie-wyjście otrzymywaliśmy poprzez skokową zmianę zakłócenia. Eksperymenty wykonywane były ze stanu punktu pracy, w którym wszystkie sygnały wynosiły 0. Sygnał sterowania był stały podczas całej symulacji. Na wykresie \ref{zad2_odp_skok_disturbance} obserwujemy reakcję na zmiany zakłóceń i możemy stwierdzić, że jest ona podobna do tej otrzymanej przy zmianach sterowania, co tylko potwierdza nasze przypuszczenia o liniowości obiektu. 

\begin{figure}[b]
    \centering
    \begin{tikzpicture}
    \begin{axis}[
    width=0.98\textwidth,
    xmin=0.0,xmax=200,ymin=-1,ymax=1,
    xlabel={$k$},
    ylabel={$y[k]$},
    legend pos=south east,
    y tick label style={/pgf/number format/1000 sep=},
    ] 
    \addlegendentry{$\Delta z = \num{0,05}$},
    \addlegendentry{$\Delta z = \num{0,2}$}
    \addlegendentry{$\Delta z = \num{0,35}$},
    \addlegendentry{$\Delta z = \num{0,5}$},
    \addlegendentry{$\Delta z = \num{-0,05}$},
    \addlegendentry{$\Delta z = \num{-0,2}$},
    \addlegendentry{$\Delta z = \num{-0,35}$},
    \addlegendentry{$\Delta z = \num{-0,5}$},
    \addlegendimage{no markers,green}
	\addlegendimage{no markers,red}
	\addlegendimage{no markers,yellow}
	\addlegendimage{no markers,blue}
	\addlegendimage{no markers,black}
	\addlegendimage{no markers,orange}
	\addlegendimage{no markers,brown}
	\addlegendimage{no markers,violet}
    \addplot[green, semithick, thick] file{../data/zad2/zad2_disturbance_output0.05.csv};
    \addplot[red, semithick, thick] file{../data/zad2/zad2_disturbance_output0.2.csv};
    \addplot[yellow, semithick, thick] file{../data/zad2/zad2_disturbance_output0.35.csv};
    \addplot[blue, semithick, thick] file{../data/zad2/zad2_disturbance_output0.5.csv};
    \addplot[black, semithick, thick] file{../data/zad2/zad2_disturbance_output-0.05.csv};
    \addplot[orange, semithick, thick] file{../data/zad2/zad2_disturbance_output-0.2.csv};
    \addplot[brown, semithick, thick] file{../data/zad2/zad2_disturbance_output-0.35.csv};
    \addplot[violet, semithick, thick] file{../data/zad2/zad2_disturbance_output-0.5.csv};    
    \end{axis}
    \end{tikzpicture}
    \caption{Odpowiedzi procesu na skokową zmianę sygnału zaklócenia}
    \label{zad2_odp_skok_disturbance}
\end{figure}

\section{Wyznaczenie charakterystyki statycznej $y(u,z)$}
Wyznaczenie charakterystyki $y(u,z)$ rozpoczęliśmy od wyznaczenia charakterystyk $y(u)$ i $y(z)$. 
Na dalszym etapie zadania pomogło to w określeniu wzmocnień statycznych torów wejście-wyjście, zakłócenie-wyjśćie. 

Aby otrzymać wykres charakterystyki statycznej obiektu w zależności od 
dwóch argumentów należało przeprowadzić symulacje dla każdej 
wartości sygnałów $u$ oraz $d$ i zapamiętaniu wartości 
nasycenia sygnału $y$. Program, który realizujący zadanie to 
\verb+zad2_static_surface.m+ wykonywał wspomniane działania, 
a wyniki zapisywał do tablicy z wartościami charakterystyki statycznej. 
Wynik można zobaczyć na rysunku \ref{zad2_static_surf}.\\
\indent{} Wykres przedstawia płaszczyznę $y(u,z)$. Skoro powstały wykres 
jest płaszczyzną to można wywnioskować, że mamy do czynienia z obiektem liniowym. 


%\chapter{Przekształcenie odpowiedzi skokowych do postaci wykorzystywanej w algorytmie DMC}
\label{zad3}
Proces normalizacji odpowiedzi skokowych dla toru wejście-wyjście oraz dla toru zakłócenie-wyjście rozpoczęliśmy od przeprowadzenia dwóch eksperymentów oraz pobrania potrzebnych danych. Eksperymenty polegały na skokowej zmianie wartości sygnałów sterowania i zakłócenia. Skoki odbywały się z punktu pracy, w którym wartości sygnałów były równe 0. Zadanie realizuje skrypt \verb+zad3.m+.\\
\indent{} Po otrzymaniu odpowiedzi skokowych nieznormalizowanych następowała normalizacja otrzymanych danych, która np. dla skoku sterowania przyjmowała następującą postać:

\begin{equation}
\label{norm1}
s = \frac{Y - Y_{\mathrm{pp}}}{\Delta u}
\end{equation}

We wzorze \ref{norm1} $Y$ jest wektorem wyjścia obiektu po przeprowadzonym eksperymencie, wartość $Y_{\mathrm{pp}}$ wartością wyjścia w punkcie pracy, a $\Delta u$ zmianą wartości sterowania podczas skoku.

\section{Znormalizowana odpowiedź toru wejście-wyjście}
Na rysunku \ref{zad3_norm_odp_u} obserwujemy wynik normalizacji odpowiedzi skokowej obiektu na skokową zmianę sygnału sterującego. 
Wartość wyjścia w stanie ustalonym wynosi $y = \num{2,296}$. Wartość zgadza się z wynikiem otrzymanym na charakterystyce 
statycznej toru wejście-wyjście, gdzie możemy zaobserwować, że wartość sygnału $y$ przy pobudzeniu sygnałem $u = 1$ powinna wynosić $\num{2,296}$.
Wartość wyjścia w stanie ustalonym po pobudzeniu skokiem jednostkowym nazywana jest wzmocnieniem statycznym i dla tego toru wynosi ono właśnie 
$K_{\mathrm{stat}} = \num{2,296}$.  

\begin{figure}[t]
    \centering
    \begin{tikzpicture}
    \begin{axis}[
    width=0.98\textwidth,
    xmin=0.0,xmax=300,ymin=0,ymax=3,
    xlabel={$k$},
    ylabel={$y[k]$},
    legend pos=south east,
    y tick label style={/pgf/number format/1000 sep=},
    ] 
    \addlegendentry{$y[k]$},
	\addlegendimage{no markers,red}
    \addplot[red, semithick, thick] file{../data/zad3/zad3_input_norm_resp.csv};   
    \end{axis}
    \end{tikzpicture}
    \caption{Znormalizowana odpowiedź skokowa toru wejście-wyjście}
    \label{zad3_norm_odp_u}
\end{figure}


\section{Znormalizowana odpowiedź toru zakłócenie-wyjście}
Wynik normalizacji odpowiedzi skokowej toru zakłócenie-wyjście również znajduje potwierdzenie
 w charakterystyce statycznej tego toru. Przy pobudzeniu obiektu sygnałem zakłócenia o wartości 
 $\num{1}$ wyjście osiąga punkt pracy $y = \num{1,497}$, dlatego wzmocnienie statyczne toru zakłócenia
 wynosi $K_{\mathrm{stat}}^{\mathrm{z}} = \num{1,497}$.
 Znormalizowaną odpowiedź skokową przedstawia rysunek \ref{zad3_norm_odp_d}.

\begin{figure}[b]
    \centering
    \begin{tikzpicture}
    \begin{axis}[
    width=0.98\textwidth,
    xmin=0.0,xmax=300,ymin=0,ymax=2,
    xlabel={$k$},
    ylabel={$y[k]$},
    legend pos=south east,
    y tick label style={/pgf/number format/1000 sep=},
    ] 
    \addlegendentry{$y[k]$},
	\addlegendimage{no markers,red}
    \addplot[red, semithick, thick] file{../data/zad3/zad3_disturbance_norm_resp.csv};   
    \end{axis}
    \end{tikzpicture}
    \caption{Znormalizowana odpowiedź skokowa toru zakłócenie-wyjście}
    \label{zad3_norm_odp_d}
\end{figure}
\FloatBarrier

\section{Wyznaczenie horyzontu dynamiki $D$}
\label{zad3_wyznacznie_D}
Współczynniki odpowiedzi skokowej po nasyceniu można odrzucić ponieważ nie poprawiają one jakości regulacji a zwiększają narzut obliczeniowy. Należy więc dobrać horyzont dynamiki $D$ użytecznych próbek odpowiedzi skokowej, które będą użyte w algorytmie DMC.\\
\indent{} W skrypcie \verb+zad3.m+ dobór horyzontu jest realizowany automatycznie poprzez określenie dozwolonej odległości między kolejnymi próbkami odpowiedzi skokowej. Odległość ta wynosiła $eps = \num{0,0001}$, a wyznaczony horyzont dynamiki był długości $D = \num{155}$ próbek. Model uwzględniany w algorytmie DMC został przedstawia rysunek \ref{zad3_cut_resp}.

\begin{figure}[b]
    \centering
    \begin{tikzpicture}
    \begin{axis}[
    width=0.7\textwidth,
    xmin=0.0,xmax=155,ymin=0,ymax=3,
    xlabel={$k$},
    ylabel={$y[k]$},
    legend pos=south east,
    y tick label style={/pgf/number format/1000 sep=},
    ] 
    \addlegendentry{$y[k]$},
	\addlegendimage{no markers,red}
    \addplot[red, semithick, thick] file{../data/zad3/zad3_input_cut_response.csv};   
    \end{axis}
    \end{tikzpicture}
    \caption{Odpowiedź skokowa z uwzględnieniem horyzontu dynamiki równego $D = 155$}
    \label{zad3_cut_resp}
\end{figure}

\section{Wyznaczenie horyzontu dynamiki $D_{\mathrm{z}}$}
\label{zad3_wyznacznie_Dz}
W identyczny sposób wyznaczyliśmy horyzont dynamiki toru zakłocenia $D_{\mathrm{z}}$. 
Odpowiedź skokowa ustalała się po $D_{\mathrm{z}} = \num{147}$ próbkach.

\begin{figure}[b]
    \centering
    \begin{tikzpicture}
    \begin{axis}[
    width=0.7\textwidth,
    xmin=0.0,xmax=147,ymin=0,ymax=2,
    xlabel={$k$},
    ylabel={$y[k]$},
    legend pos=south east,
    y tick label style={/pgf/number format/1000 sep=},
    ] 
    \addlegendentry{$y[k]$},
	\addlegendimage{no markers,red}
    \addplot[red, semithick, thick] file{../data/zad3/zad3_disturbance_cut_response.csv};   
    \end{axis}
    \end{tikzpicture}
    \caption{Odpowiedź skokowa z uwzględnieniem horyzontu dynamiki zakłóceń równego $D_{\mathrm{z}} = 147$}
    \label{zad3_cut_resp_disturbance}
\end{figure}\fi
%\chapter{Dobieranie nastawów regulatorów DMC}
\label{dmc}

\section{Strojenie regulatora DMC}
\label{dmc_strojenie}

W celu doboru jak najlepszych parametrów regulatora DMC wykonaliśmy szereg testów związanych z poszczególnymi wartościami każdego z parametrów. Horyzont dynamiki podczas całego procesu strojenia wynosił 91. Punktem wyjściowym w procesie doboru nastaw był punkt, w którym wszystkie horyzonty były sobie równe, a współczynnik $\lambda$ był równy 1. Pracę regulatora w takim układzie przedstawiają rysunki \ref{dmc_start} i \ref{dmc_start_ster}. Podczas doboru odpowiednich parametrów pamiętaliśmy o tym, że długości poszczególnych horyzontów mają znaczący wpływ na złożoność obliczeniową algorytmu. Jakość regulacji ocenialiśmy na podstawie wykresów oraz wartości wskaźnika jakości E.

\begin{equation}
E=\sum_{k=0}^{K_{konc}}(y^{zad}[k]-y[k])^{2}
\end{equation}
\\

Strojenie odbywało się dla trzech różnych zmian wartości zadanej.
\\

\begin{figure}[t]
    \centering
    \begin{tikzpicture}
    \begin{axis}[
    width=0.98\textwidth,
    xmin=0,xmax=600,ymin=3.7, ymax=4.5,
    xlabel={$k$},
    ylabel={$y[k]$},
    %xtick={0, 100, 200, 300, 400, 500},
    %ytick={3.9, 4, 4.1, 4.2, 4.3, 4.4},
    legend pos=south east,
    y tick label style={/pgf/number format/1000 sep=},
    ]
    \addplot[blue, semithick] file{../data/zad5_multiplejumps/N/zad5_DMC_setpoint_N_91_Nu_91_lambda_1.csv};
    \addplot[red, semithick] file{../data/zad5_multiplejumps/N/zad5_DMC_output_N_91_Nu_91_lambda_1.csv};
    \addlegendentry{$y^{\mathrm{zad}}[k]$},
    \addlegendentry{$y[k]$},
    \addlegendimage{no markers, blue}
	\addlegendimage{no markers, red}
    \end{axis}
    \end{tikzpicture}
    \caption{Przebieg procesu sterowanego za pomocą regulatora z parametrami $D = 91$, $N = 91$, $N_{\mathrm{u}} = 91$, $\lambda = 1$}
    \label{dmc_start}
\end{figure}

\begin{figure}[b]
    \centering
    \begin{tikzpicture}
    \begin{axis}[
    width=0.98\textwidth,
    xmin=0,xmax=500,ymin=0.25, ymax=0.75,
    xlabel={$k$},
    ylabel={$u[k]$},
    %xtick={0, 0.4, 0.5, 0.6, 0.7},
    %ytick={3.5, 3.75, 4, 4.25, 4.5},
    legend pos=south east,
    y tick label style={/pgf/number format/1000 sep=},
    ]
    \addplot[const plot, blue, semithick] file{../data/zad5_multiplejumps/N/zad5_DMC_input_N_91_Nu_91_lambda_1.csv};
    \legend{$u[k]$}
    \end{axis}
    \end{tikzpicture}
    \caption{Przebieg sygnału sterującego przy parametrach parametrach: $D = 91$, $N = 91$, $N_{\mathrm{u}} = 91$, $\lambda = 1$}
    \label{dmc_start_ster}
\end{figure}
\FloatBarrier

\subsection{Dobór horyzontu predykcji N}
\label{zad_dobor_N}
Dobór horyzontu predykcji polegał na jego stopniowym skracaniu, wraz z nim skróceniu ulegał horyzont sterowania, który powinien być mniejszy lub równy horyzontowi sterowania. W procesie strojenia regulatora najważniejszymi aspektami były dla nas: zmniejszenie wartości sumy kwadratów uchybów oraz otrzymanie względnie łagodnych przebiegów sygnałów procesowych.

\subsubsection{Horyzont predykcji N = 40}
W przedziale od $D$ do $40$ nie zauważyliśmy różnicy w wykresach sygnałów procesowych. Co więcej, wskaźnik jakości regulacji również pozostawał stały. Jego wartość wynosiła $E = 3,918$. Występowały oscylacje sygnału wyjściowego i znaczne przeregulowanie. Sygnały stabilizowały się po około 95 próbkach od chwili skoku wartości $y^{\mathrm{zad}}$. Zjawiskiem niepożądanym było uderzanie sygnału sterującego w ograniczenia. Wyniki eksperymentu przedstawiają rysunki \ref{dmc_N_40_y} oraz \ref{dmc_N_40_u}.

\begin{figure}[t]
    \centering
    \begin{tikzpicture}
    \begin{axis}[
    width=0.98\textwidth,
    xmin=0,xmax=600,ymin=3.7, ymax=4.5,
    xlabel={$k$},
    ylabel={$y[k]$},
    %xtick={0, 100, 200, 300, 400, 500},
    %ytick={3.9, 4, 4.1, 4.2, 4.3, 4.4},
    legend pos=south east,
    y tick label style={/pgf/number format/1000 sep=},
    ]
    \addplot[blue, semithick] file{../data/zad5_multiplejumps/N/zad5_DMC_setpoint_N_40_Nu_40_lambda_1.csv};
    \addplot[red, semithick] file{../data/zad5_multiplejumps/N/zad5_DMC_output_N_40_Nu_40_lambda_1.csv};
    \addlegendentry{$y^{\mathrm{zad}}[k]$},
    \addlegendentry{$y[k]$},
    \addlegendimage{no markers, blue}
	\addlegendimage{no markers, red}
    \end{axis}
    \end{tikzpicture}
    \caption{Przebieg procesu sterowanego za pomocą regulatora z parametrami $D = 91$, $N = 40$, $N_{\mathrm{u}} = 40$, $\lambda = 1$}
    \label{dmc_N_40_y}
\end{figure}

\begin{figure}[b]
    \centering
    \begin{tikzpicture}
    \begin{axis}[
    width=0.98\textwidth,
    xmin=0,xmax=500,ymin=0.25, ymax=0.75,
    xlabel={$k$},
    ylabel={$u[k]$},
    %xtick={0, 0.4, 0.5, 0.6, 0.7},
    %ytick={3.5, 3.75, 4, 4.25, 4.5},
    legend pos=south east,
    y tick label style={/pgf/number format/1000 sep=},
    ]
    \addplot[const plot, blue, semithick] file{../data/zad5_multiplejumps/N/zad5_DMC_input_N_40_Nu_40_lambda_1.csv};
    \legend{$u[k]$}
    \end{axis}
    \end{tikzpicture}
    \caption{Przebieg sygnału sterującego przy parametrach parametrach: $D = 91$, $N = 40$, $N_{\mathrm{u}} = 40$, $\lambda = 1$}
    \label{dmc_N_40_u}
\end{figure}

\subsubsection{Horyzont predykcji N = 35}
Przy eksperymentach przyjęliśmy krok zmiejszania horyzontu predykcji równy 5. Regulator pracował tak samo jak w poprzednich przypadkach. Wykresy przebiegów zostały zamieszczone na rysunkach \ref{dmc_N_35_y} oraz \ref{dmc_N_35_u}.

\begin{figure}[t]
    \centering
    \begin{tikzpicture}
    \begin{axis}[
    width=0.98\textwidth,
    xmin=0,xmax=600,ymin=3.7, ymax=4.5,
    xlabel={$k$},
    ylabel={$y[k]$},
    %xtick={0, 100, 200, 300, 400, 500},
    %ytick={3.9, 4, 4.1, 4.2, 4.3, 4.4},
    legend pos=south east,
    y tick label style={/pgf/number format/1000 sep=},
    ]
    \addplot[blue, semithick] file{../data/zad5_multiplejumps/N/zad5_DMC_setpoint_N_35_Nu_35_lambda_1.csv};
    \addplot[red, semithick] file{../data/zad5_multiplejumps/N/zad5_DMC_output_N_35_Nu_35_lambda_1.csv};
    \addlegendentry{$y^{\mathrm{zad}}[k]$},
    \addlegendentry{$y[k]$},
    \addlegendimage{no markers, blue}
	\addlegendimage{no markers, red}
    \end{axis}
    \end{tikzpicture}
    \caption{Przebieg procesu sterowanego za pomocą regulatora z parametrami $D = 91$, $N = 35$, $N_{\mathrm{u}} = 35$, $\lambda = 1$}
    \label{dmc_N_35_y}
\end{figure}

\begin{figure}[b]
    \centering
    \begin{tikzpicture}
    \begin{axis}[
    width=0.98\textwidth,
    xmin=0,xmax=500,ymin=0.25, ymax=0.75,
    xlabel={$k$},
    ylabel={$u[k]$},
    %xtick={0, 0.4, 0.5, 0.6, 0.7},
    %ytick={3.5, 3.75, 4, 4.25, 4.5},
    legend pos=south east,
    y tick label style={/pgf/number format/1000 sep=},
    ]
    \addplot[const plot, blue, semithick] file{../data/zad5_multiplejumps/N/zad5_DMC_input_N_35_Nu_35_lambda_1.csv};
    \legend{$u[k]$}
    \end{axis}
    \end{tikzpicture}
    \caption{Przebieg sygnału sterującego przy parametrach parametrach: $D = 91$, $N = 35$, $N_{\mathrm{u}} = 35$, $\lambda = 1$}
    \label{dmc_N_35_u}
\end{figure}

\subsubsection{Horyzont predykcji N = 30}
W kolejnym kroku ponownie nie zaobserwowaliśmy żadnych zmian. Regulator wciąż działał dobrze, a wskaźnik jakości utrzymywał swoją początkową wartość $E = 3.918$. Wyniki przedstawiają rysunki \ref{dmc_N_30_y} oraz \ref{dmc_N_30_u}.

\begin{figure}[t]
    \centering
    \begin{tikzpicture}
    \begin{axis}[
    width=0.98\textwidth,
    xmin=0,xmax=600,ymin=3.7, ymax=4.5,
    xlabel={$k$},
    ylabel={$y[k]$},
    %xtick={0, 100, 200, 300, 400, 500},
    %ytick={3.9, 4, 4.1, 4.2, 4.3, 4.4},
    legend pos=south east,
    y tick label style={/pgf/number format/1000 sep=},
    ]
    \addplot[blue, semithick] file{../data/zad5_multiplejumps/N/zad5_DMC_setpoint_N_30_Nu_30_lambda_1.csv};
    \addplot[red, semithick] file{../data/zad5_multiplejumps/N/zad5_DMC_output_N_30_Nu_30_lambda_1.csv};
    \addlegendentry{$y^{\mathrm{zad}}[k]$},
    \addlegendentry{$y[k]$},
    \addlegendimage{no markers, blue}
	\addlegendimage{no markers, red}
    \end{axis}
    \end{tikzpicture}
    \caption{Przebieg procesu sterowanego za pomocą regulatora z parametrami $D = 91$, $N = 30$, $N_{\mathrm{u}} = 30$, $\lambda = 1$}
    \label{dmc_N_30_y}
\end{figure}

\begin{figure}[b]
    \centering
    \begin{tikzpicture}
    \begin{axis}[
    width=0.98\textwidth,
    xmin=0,xmax=500,ymin=0.25, ymax=0.75,
    xlabel={$k$},
    ylabel={$u[k]$},
    %xtick={0, 0.4, 0.5, 0.6, 0.7},
    %ytick={3.5, 3.75, 4, 4.25, 4.5},
    legend pos=south east,
    y tick label style={/pgf/number format/1000 sep=},
    ]
    \addplot[const plot, blue, semithick] file{../data/zad5_multiplejumps/N/zad5_DMC_input_N_30_Nu_30_lambda_1.csv};
    \legend{$u[k]$}
    \end{axis}
    \end{tikzpicture}
    \caption{Przebieg sygnału sterującego przy parametrach parametrach: $D = 91$, $N = 30$, $N_{\mathrm{u}} = 30$, $\lambda = 1$}
    \label{dmc_N_30_u}
\end{figure}

\subsubsection{Horyzont predykcji N = 25}
Po wykonanym eksperymencie przy $N = 25$, patrząc na wykresy nie zauważyliśmy żadnych zmian. Zmienił się jednak wskaźnik jakości, zwiększając swoją wartość do $E = 3.9188$. Mimo iż jest to mała zmiana do dalszego strojenia wybraliśmy wartość horyzontu predykcji $N = 30$. Wyniki dla horyzontu równego 25 ukazują rysunki \ref{dmc_N_25_y} i \ref{dmc_N_25_u}.

\begin{figure}[t]
    \centering
    \begin{tikzpicture}
    \begin{axis}[
    width=0.98\textwidth,
    xmin=0,xmax=600,ymin=3.7, ymax=4.5,
    xlabel={$k$},
    ylabel={$y[k]$},
    %xtick={0, 100, 200, 300, 400, 500},
    %ytick={3.9, 4, 4.1, 4.2, 4.3, 4.4},
    legend pos=south east,
    y tick label style={/pgf/number format/1000 sep=},
    ]
    \addplot[blue, semithick] file{../data/zad5_multiplejumps/N/zad5_DMC_setpoint_N_25_Nu_25_lambda_1.csv};
    \addplot[red, semithick] file{../data/zad5_multiplejumps/N/zad5_DMC_output_N_25_Nu_25_lambda_1.csv};
    \addlegendentry{$y^{\mathrm{zad}}[k]$},
    \addlegendentry{$y[k]$},
    \addlegendimage{no markers, blue}
	\addlegendimage{no markers, red}
    \end{axis}
    \end{tikzpicture}
    \caption{Przebieg procesu sterowanego za pomocą regulatora z parametrami $D = 91$, $N = 25$, $N_{\mathrm{u}} = 25$, $\lambda = 1$}
    \label{dmc_N_25_y}
\end{figure}

\begin{figure}[b]
    \centering
    \begin{tikzpicture}
    \begin{axis}[
    width=0.98\textwidth,
    xmin=0,xmax=500,ymin=0.25, ymax=0.75,
    xlabel={$k$},
    ylabel={$u[k]$},
    %xtick={0, 0.4, 0.5, 0.6, 0.7},
    %ytick={3.5, 3.75, 4, 4.25, 4.5},
    legend pos=south east,
    y tick label style={/pgf/number format/1000 sep=},
    ]
    \addplot[const plot, blue, semithick] file{../data/zad5_multiplejumps/N/zad5_DMC_input_N_25_Nu_25_lambda_1.csv};
    \legend{$u[k]$}
    \end{axis}
    \end{tikzpicture}
    \caption{Przebieg sygnału sterującego przy parametrach parametrach: $D = 91$, $N = 25$, $N_{\mathrm{u}} = 25$, $\lambda = 1$}
    \label{dmc_N_25_u}
\end{figure}
\FloatBarrier
\subsection{Dobór horyzontu sterowania $N_{\mathrm{u}}$}
\subsubsection{Horyzont sterowania $N_{\mathrm{u}} = 15$}
Zmniejszanie horyzontu sterowania nie przynosiło żadnych rezultatów aż do wartości $N_{\mathrm{u}} = 15$, przy której wskaźnik jakości zmniejszył się nieznacznie do wartości $E = 3,9179$. Jest to niezauważalna zmiana jednak wyznacza ona punkt początkowy do szukania optymalnego horyzontu sterowania. Praca regulatora nie różniła się znacząco od poprzednich przypadków - regulacja przebiegała szybko, ale wciąż nie udało się wyraźnie zmniejszyć przeregulowania oraz oscylacji sygnałów. Wynik ukazują rysunki \ref{dmc_Nu_15_y} i \ref{dmc_Nu_15_u}.
\\
\begin{figure}[t]
    \centering
    \begin{tikzpicture}
    \begin{axis}[
    width=0.98\textwidth,
    xmin=0,xmax=600,ymin=3.7, ymax=4.5,
    xlabel={$k$},
    ylabel={$y[k]$},
    %xtick={0, 100, 200, 300, 400, 500},
    %ytick={3.9, 4, 4.1, 4.2, 4.3, 4.4},
    legend pos=south east,
    y tick label style={/pgf/number format/1000 sep=},
    ]
    \addplot[blue, semithick] file{../data/zad5_multiplejumps/Nu/zad5_DMC_setpoint_N_30_Nu_15_lambda_1.csv};
    \addplot[red, semithick] file{../data/zad5_multiplejumps/Nu/zad5_DMC_output_N_30_Nu_15_lambda_1.csv};
    \addlegendentry{$y^{\mathrm{zad}}[k]$},
    \addlegendentry{$y[k]$},
    \addlegendimage{no markers, blue}
	\addlegendimage{no markers, red}
    \end{axis}
    \end{tikzpicture}
    \caption{Przebieg procesu sterowanego za pomocą regulatora z parametrami $D = 91$, $N = 30$, $N_{\mathrm{u}} = 15$, $\lambda = 1$}
    \label{dmc_Nu_15_y}
\end{figure}

\begin{figure}[b]
    \centering
    \begin{tikzpicture}
    \begin{axis}[
    width=0.98\textwidth,
    xmin=0,xmax=500,ymin=0.25, ymax=0.75,
    xlabel={$k$},
    ylabel={$u[k]$},
    %xtick={0, 0.4, 0.5, 0.6, 0.7},
    %ytick={3.5, 3.75, 4, 4.25, 4.5},
    legend pos=south east,
    y tick label style={/pgf/number format/1000 sep=},
    ]
    \addplot[const plot, blue, semithick] file{../data/zad5_multiplejumps/Nu/zad5_DMC_input_N_30_Nu_15_lambda_1.csv};
    \legend{$u[k]$}
    \end{axis}
    \end{tikzpicture}
    \caption{Przebieg sygnału sterującego przy parametrach parametrach: $D = 91$, $N = 30$, $N_{\mathrm{u}} = 15$, $\lambda = 1$}
    \label{dmc_Nu_15_u}
\end{figure}


\subsubsection{Horyzont sterowania $N_{\mathrm{u}} = 10$}
Po wykonaniu eksperymentu dla $N_{\mathrm{u}} = 10$ jedyną zmianą przez nas zaobserwowaną była zmiana wartości wskaźnika E, który był równy $E = 3,9171$. Wykresy zamieszczone zostały na rysunkach \ref{dmc_Nu_10_y_pro} i \ref{dmc_Nu_10_u_pro}.
\\
\begin{figure}[t]
    \centering
    \begin{tikzpicture}
    \begin{axis}[
    width=0.98\textwidth,
    xmin=0,xmax=600,ymin=3.7, ymax=4.5,
    xlabel={$k$},
    ylabel={$y[k]$},
    %xtick={0, 100, 200, 300, 400, 500},
    %ytick={3.9, 4, 4.1, 4.2, 4.3, 4.4},
    legend pos=south east,
    y tick label style={/pgf/number format/1000 sep=},
    ]
    \addplot[blue, semithick] file{../data/zad5_multiplejumps/Nu/zad5_DMC_setpoint_N_30_Nu_10_lambda_1.csv};
    \addplot[red, semithick] file{../data/zad5_multiplejumps/Nu/zad5_DMC_output_N_30_Nu_10_lambda_1.csv};
    \addlegendentry{$y^{\mathrm{zad}}[k]$},
    \addlegendentry{$y[k]$},
    \addlegendimage{no markers, blue}
	\addlegendimage{no markers, red}
    \end{axis}
    \end{tikzpicture}
    \caption{Przebieg procesu sterowanego za pomocą regulatora z parametrami $D = 91$, $N = 30$, $N_{\mathrm{u}} = 10$, $\lambda = 1$}
    \label{dmc_Nu_10_y_pro}
\end{figure}

\begin{figure}[b]
    \centering
    \begin{tikzpicture}
    \begin{axis}[
    width=0.98\textwidth,
    xmin=0,xmax=500,ymin=0.25, ymax=0.75,
    xlabel={$k$},
    ylabel={$u[k]$},
    %xtick={0, 0.4, 0.5, 0.6, 0.7},
    %ytick={3.5, 3.75, 4, 4.25, 4.5},
    legend pos=south east,
    y tick label style={/pgf/number format/1000 sep=},
    ]
    \addplot[const plot, blue, semithick] file{../data/zad5_multiplejumps/Nu/zad5_DMC_input_N_30_Nu_10_lambda_1.csv};
    \legend{$u[k]$}
    \end{axis}
    \end{tikzpicture}
    \caption{Przebieg sygnału sterującego przy parametrach parametrach: $D = 91$, $N = 30$, $N_{\mathrm{u}} = 10$, $\lambda = 1$}
    \label{dmc_Nu_10_u_pro}
\end{figure}

\subsubsection{Horyzont sterowania $N_{\mathrm{u}} = 5$}
Zmniejszając horyzont sterowania do 5 otrzymaliśmy przebieg sygnału wyjściowego o większym błędzie niż w przypadku poprzednim. Dla $N_{\mathrm{u}} = 5$ $E = 3,9198$. Mimo gorszego wyniku postalowiliśmy zbadać kilka mniejszych wartości. Wyniki dla $N_{\mathrm{u}} = 5$ przedstawiają rysunki \ref{dmc_Nu_5_y} oraz \ref{dmc_Nu_5_u}.
\\
\begin{figure}[t]
    \centering
    \begin{tikzpicture}
    \begin{axis}[
    width=0.98\textwidth,
    xmin=0,xmax=600,ymin=3.7, ymax=4.5,
    xlabel={$k$},
    ylabel={$y[k]$},
    %xtick={0, 100, 200, 300, 400, 500},
    %ytick={3.9, 4, 4.1, 4.2, 4.3, 4.4},
    legend pos=south east,
    y tick label style={/pgf/number format/1000 sep=},
    ]
    \addplot[blue, semithick] file{../data/zad5_multiplejumps/Nu/zad5_DMC_setpoint_N_30_Nu_5_lambda_1.csv};
    \addplot[red, semithick] file{../data/zad5_multiplejumps/Nu/zad5_DMC_output_N_30_Nu_5_lambda_1.csv};
    \addlegendentry{$y^{\mathrm{zad}}[k]$},
    \addlegendentry{$y[k]$},
    \addlegendimage{no markers, blue}
	\addlegendimage{no markers, red}
    \end{axis}
    \end{tikzpicture}
    \caption{Przebieg procesu sterowanego za pomocą regulatora z parametrami $D = 91$, $N = 30$, $N_{\mathrm{u}} = 5$, $\lambda = 1$}
    \label{dmc_Nu_5_y}
\end{figure}

\begin{figure}[b]
    \centering
    \begin{tikzpicture}
    \begin{axis}[
    width=0.98\textwidth,
    xmin=0,xmax=500,ymin=0.25, ymax=0.75,
    xlabel={$k$},
    ylabel={$u[k]$},
    %xtick={0, 0.4, 0.5, 0.6, 0.7},
    %ytick={3.5, 3.75, 4, 4.25, 4.5},
    legend pos=south east,
    y tick label style={/pgf/number format/1000 sep=},
    ]
    \addplot[const plot, blue, semithick] file{../data/zad5_multiplejumps/Nu/zad5_DMC_input_N_30_Nu_5_lambda_1.csv};
    \legend{$u[k]$}
    \end{axis}
    \end{tikzpicture}
    \caption{Przebieg sygnału sterującego przy parametrach parametrach: $D = 91$, $N = 30$, $N_{\mathrm{u}} = 5$, $\lambda = 1$}
    \label{dmc_Nu_5_u}
\end{figure}

\subsubsection{Horyzont sterowania $N_{\mathrm{u}} = 3$}
Ponowne zmniejszenie badanego parametru przyniosło zadowalający efekt. Wskaźnik jakości zmalał o względnie dużą różnicę w stosunku do poprzednich eksperymentów. Dla $N_{\mathrm{u}} = 3$ wynosił on $E = 3,9$ co było najlepszym wynikiem podczas wszystkich dotychczasowych testów. Niestety jednego celu nie udało nam się osiągnąć, mianowicie wykres sygnału sterującego wciąż charakteryzował się dosyć dużymi pikami oraz uderzał w ograniczenia dwukrotnie podczas ostatniego skoku wartości zadanej. Wynik przedstawiają rysunki \ref{dmc_Nu_3_y} i \ref{dmc_Nu_3_u}.

\begin{figure}[t]
    \centering
    \begin{tikzpicture}
    \begin{axis}[
    width=0.98\textwidth,
    xmin=0,xmax=600,ymin=3.7, ymax=4.5,
    xlabel={$k$},
    ylabel={$y[k]$},
    %xtick={0, 100, 200, 300, 400, 500},
    %ytick={3.9, 4, 4.1, 4.2, 4.3, 4.4},
    legend pos=south east,
    y tick label style={/pgf/number format/1000 sep=},
    ]
    \addplot[blue, semithick] file{../data/zad5_multiplejumps/Nu/zad5_DMC_setpoint_N_30_Nu_3_lambda_1.csv};
    \addplot[red, semithick] file{../data/zad5_multiplejumps/Nu/zad5_DMC_output_N_30_Nu_3_lambda_1.csv};
    \addlegendentry{$y^{\mathrm{zad}}[k]$},
    \addlegendentry{$y[k]$},
    \addlegendimage{no markers, blue}
	\addlegendimage{no markers, red}
    \end{axis}
    \end{tikzpicture}
    \caption{Przebieg procesu sterowanego za pomocą regulatora z parametrami $D = 91$, $N = 30$, $N_{\mathrm{u}} = 3$, $\lambda = 1$}
    \label{dmc_Nu_3_y}
\end{figure}

\begin{figure}[b]
    \centering
    \begin{tikzpicture}
    \begin{axis}[
    width=0.98\textwidth,
    xmin=0,xmax=500,ymin=0.25, ymax=0.75,
    xlabel={$k$},
    ylabel={$u[k]$},
    %xtick={0, 0.4, 0.5, 0.6, 0.7},
    %ytick={3.5, 3.75, 4, 4.25, 4.5},
    legend pos=south east,
    y tick label style={/pgf/number format/1000 sep=},
    ]
    \addplot[const plot, blue, semithick] file{../data/zad5_multiplejumps/Nu/zad5_DMC_input_N_30_Nu_3_lambda_1.csv};
    \legend{$u[k]$}
    \end{axis}
    \end{tikzpicture}
    \caption{Przebieg sygnału sterującego przy parametrach parametrach: $D = 91$, $N = 30$, $N_{\mathrm{u}} = 3$, $\lambda = 1$}
    \label{dmc_Nu_3_u}
\end{figure}

\subsubsection{Horyzont sterowania $N_{\mathrm{u}} = 2$}
Zmniejszenie horyzontu do 2 spowodowało nieznaczne powiększenie się sumy kwadratów uchybów, która wyniosła $E = 3,9034$. Uznaliśmy, że jest to niewielka strata do poprzedniego testu. W zamian za to udało nam się w końcu uzyskać łagodniejsze przebiegi sygnałów procesowych. Oscylacje sygnału wyjściowego zmniejszyły się. Podobnie było z sygnałem sterującym, który nie oscylował tak bardzo oraz uderzał w ograniczenia tylko jeden raz przy ostatniej zmianie wartości $y^{\mathrm{zad}}$. Wynik eksperymentu ukazują rysunki \ref{dmc_Nu_2_y} i \ref{dmc_Nu_2_u}.
\\  
\begin{figure}[t]
    \centering
    \begin{tikzpicture}
    \begin{axis}[
    width=0.98\textwidth,
    xmin=0,xmax=600,ymin=3.7, ymax=4.5,
    xlabel={$k$},
    ylabel={$y[k]$},
    %xtick={0, 100, 200, 300, 400, 500},
    %ytick={3.9, 4, 4.1, 4.2, 4.3, 4.4},
    legend pos=south east,
    y tick label style={/pgf/number format/1000 sep=},
    ]
    \addplot[blue, semithick] file{../data/zad5_multiplejumps/Nu/zad5_DMC_setpoint_N_30_Nu_2_lambda_1.csv};
    \addplot[red, semithick] file{../data/zad5_multiplejumps/Nu/zad5_DMC_output_N_30_Nu_2_lambda_1.csv};
    \addlegendentry{$y^{\mathrm{zad}}[k]$},
    \addlegendentry{$y[k]$},
    \addlegendimage{no markers, blue}
	\addlegendimage{no markers, red}
    \end{axis}
    \end{tikzpicture}
    \caption{Przebieg procesu sterowanego za pomocą regulatora z parametrami $D = 91$, $N = 30$, $N_{\mathrm{u}} = 2$, $\lambda = 1$}
    \label{dmc_Nu_2_y}
\end{figure}

\begin{figure}[b]
    \centering
    \begin{tikzpicture}
    \begin{axis}[
    width=0.98\textwidth,
    xmin=0,xmax=500,ymin=0.25, ymax=0.75,
    xlabel={$k$},
    ylabel={$u[k]$},
    %xtick={0, 0.4, 0.5, 0.6, 0.7},
    %ytick={3.5, 3.75, 4, 4.25, 4.5},
    legend pos=south east,
    y tick label style={/pgf/number format/1000 sep=},
    ]
    \addplot[const plot, blue, semithick] file{../data/zad5_multiplejumps/Nu/zad5_DMC_input_N_30_Nu_2_lambda_1.csv};
    \legend{$u[k]$}
    \end{axis}
    \end{tikzpicture}
    \caption{Przebieg sygnału sterującego przy parametrach parametrach: $D = 91$, $N = 30$, $N_{\mathrm{u}} = 2$, $\lambda = 1$}
    \label{dmc_Nu_2_u}
\end{figure}

\subsubsection{Horyzont sterowania $N_{\mathrm{u}} = 1$}
Zmniejszenie horyzontu sterowania do minimum skutkuje dotychczas najbardziej łagodnym przebiegiem. Sygnał sterowania również przebiega spokojnie, obserwujemy jedynie krótkie naruszenie ograniczeń. Niestety w porównaniu z poprzednimi przykładami błąd wzrasta o znaczną wartość i wynosi $E = 3,9774$. Przebieg ostatniego eksperymentu przedstawiony jest na rysunkach \ref{dmc_Nu_1_y} oraz \ref{dmc_Nu_1_u}.\\
\indent{} Za najlepszy wynik uznaliśmy horyzont sterowania $N_{\mathrm{u}} = 2$, który cechował się bardzo małym błędem przy jednoczesnym spokojnym przebiegu sygnału sterującego.
\\
\begin{figure}[t]
    \centering
    \begin{tikzpicture}
    \begin{axis}[
    width=0.98\textwidth,
    xmin=0,xmax=600,ymin=3.7, ymax=4.5,
    xlabel={$k$},
    ylabel={$y[k]$},
    %xtick={0, 100, 200, 300, 400, 500},
    %ytick={3.9, 4, 4.1, 4.2, 4.3, 4.4},
    legend pos=south east,
    y tick label style={/pgf/number format/1000 sep=},
    ]
    \addplot[blue, semithick] file{../data/zad5_multiplejumps/Nu/zad5_DMC_setpoint_N_30_Nu_1_lambda_1.csv};
    \addplot[red, semithick] file{../data/zad5_multiplejumps/Nu/zad5_DMC_output_N_30_Nu_1_lambda_1.csv};
    \addlegendentry{$y^{\mathrm{zad}}[k]$},
    \addlegendentry{$y[k]$},
    \addlegendimage{no markers, blue}
	\addlegendimage{no markers, red}
    \end{axis}
    \end{tikzpicture}
    \caption{Przebieg procesu sterowanego za pomocą regulatora z parametrami $D = 91$, $N = 30$, $N_{\mathrm{u}} = 1$, $\lambda = 1$}
    \label{dmc_Nu_1_y}
\end{figure}

\begin{figure}[b]
    \centering
    \begin{tikzpicture}
    \begin{axis}[
    width=0.98\textwidth,
    xmin=0,xmax=500,ymin=0.25, ymax=0.75,
    xlabel={$k$},
    ylabel={$u[k]$},
    %xtick={0, 0.4, 0.5, 0.6, 0.7},
    %ytick={3.5, 3.75, 4, 4.25, 4.5},
    legend pos=south east,
    y tick label style={/pgf/number format/1000 sep=},
    ]
    \addplot[const plot, blue, semithick] file{../data/zad5_multiplejumps/Nu/zad5_DMC_input_N_30_Nu_1_lambda_1.csv};
    \legend{$u[k]$}
    \end{axis}
    \end{tikzpicture}
    \caption{Przebieg sygnału sterującego przy parametrach parametrach: $D = 91$, $N = 30$, $N_{\mathrm{u}} = 1$, $\lambda = 1$}
    \label{dmc_Nu_1_u}
\end{figure}
\FloatBarrier
\subsection{Dobór współczynnika $\lambda$}
\subsubsection{Współczynnik $\lambda = 0.5$}
Zmniejszenie o połowę współczynnika kary skutkuje szybszym działaniem regulatora oraz zmniejszeniem błędu. Wskaźnik jakości wynosi $E = 3,8981$. Dzieje się tak ponieważ zmniejszanie wartości $\lambda$ powoduje przyspieszenie przebiegu wyjścia, poprzez stosowanie coraz większych amplitud sygnału sterowania. Można zauważyć, że przebiegi sygnałów stają się bardziej agresywne - kosztem szybszego działania jest większe przeregulowanie oraz oscylacje. Obrazują to rysunki \ref{dmc_lam_0_5_y} i \ref{dmc_lam_0_5_u}.

\begin{figure}[t]
    \centering
    \begin{tikzpicture}
    \begin{axis}[
    width=0.98\textwidth,
    xmin=0,xmax=600,ymin=3.7, ymax=4.5,
    xlabel={$k$},
    ylabel={$y[k]$},
    %xtick={0, 100, 200, 300, 400, 500},
    %ytick={3.9, 4, 4.1, 4.2, 4.3, 4.4},
    legend pos=south east,
    y tick label style={/pgf/number format/1000 sep=},
    ]
    \addplot[blue, semithick] file{../data/zad5_multiplejumps/lambda/zad5_DMC_setpoint_N_30_Nu_2_lambda_0.5.csv};
    \addplot[red, semithick] file{../data/zad5_multiplejumps/lambda/zad5_DMC_output_N_30_Nu_2_lambda_0.5.csv};
    \addlegendentry{$y^{\mathrm{zad}}[k]$},
    \addlegendentry{$y[k]$},
    \addlegendimage{no markers, blue}
	\addlegendimage{no markers, red}
    \end{axis}
    \end{tikzpicture}
    \caption{Przebieg procesu sterowanego za pomocą regulatora z parametrami $D = 91$, $N = 30$, $N_{\mathrm{u}} = 2$, $\lambda = 0.5$}
    \label{dmc_lam_0_5_y}
\end{figure}

\begin{figure}[b]
    \centering
    \begin{tikzpicture}
    \begin{axis}[
    width=0.98\textwidth,
    xmin=0,xmax=500,ymin=0.25, ymax=0.75,
    xlabel={$k$},
    ylabel={$u[k]$},
    %xtick={0, 0.4, 0.5, 0.6, 0.7},
    %ytick={3.5, 3.75, 4, 4.25, 4.5},
    legend pos=south east,
    y tick label style={/pgf/number format/1000 sep=},
    ]
    \addplot[const plot, blue, semithick] file{../data/zad5_multiplejumps/lambda/zad5_DMC_input_N_30_Nu_2_lambda_0.5.csv};
    \legend{$u[k]$}
    \end{axis}
    \end{tikzpicture}
    \caption{Przebieg sygnału sterującego przy parametrach parametrach: $D = 91$, $N = 30$, $N_{\mathrm{u}} = 2$, $\lambda = 0.5$}
    \label{dmc_lam_0_5_u}
\end{figure}

\subsubsection{Współczynnik $\lambda = 1.5$}
Zwiększanie współczynnika $\lambda$ powoduje z kolei uspokojenie sygnałów procesowych. Spokojny przebieg sterowania powoduje spowolnienie sygnału wyjścia. Skutkuje to mniejszym przeregulowaniem, jednak generalnie błąd regulacji rośnie. W eksperymencie z $\lambda = 1.5$ wskaźnik $E$ wynosił $\num{3,9124}$. Wynik eksperymentu przedstawiony został na rysunkach \ref{dmc_lam_1_5_y} oraz \ref{dmc_lam_1_5_u}

\begin{figure}[t]
    \centering
    \begin{tikzpicture}
    \begin{axis}[
    width=0.98\textwidth,
    xmin=0,xmax=600,ymin=3.7, ymax=4.5,
    xlabel={$k$},
    ylabel={$y[k]$},
    %xtick={0, 100, 200, 300, 400, 500},
    %ytick={3.9, 4, 4.1, 4.2, 4.3, 4.4},
    legend pos=south east,
    y tick label style={/pgf/number format/1000 sep=},
    ]
    \addplot[blue, semithick] file{../data/zad5_multiplejumps/lambda/zad5_DMC_setpoint_N_30_Nu_2_lambda_1.5.csv};
    \addplot[red, semithick] file{../data/zad5_multiplejumps/lambda/zad5_DMC_output_N_30_Nu_2_lambda_1.5.csv};
    \addlegendentry{$y^{\mathrm{zad}}[k]$},
    \addlegendentry{$y[k]$},
    \addlegendimage{no markers, blue}
	\addlegendimage{no markers, red}
    \end{axis}
    \end{tikzpicture}
    \caption{Przebieg procesu sterowanego za pomocą regulatora z parametrami $D = 91$, $N = 30$, $N_{\mathrm{u}} = 2$, $\lambda = 1.5$}
    \label{dmc_lam_1_5_y}
\end{figure}

\begin{figure}[b]
    \centering
    \begin{tikzpicture}
    \begin{axis}[
    width=0.98\textwidth,
    xmin=0,xmax=500,ymin=0.25, ymax=0.75,
    xlabel={$k$},
    ylabel={$u[k]$},
    %xtick={0, 0.4, 0.5, 0.6, 0.7},
    %ytick={3.5, 3.75, 4, 4.25, 4.5},
    legend pos=south east,
    y tick label style={/pgf/number format/1000 sep=},
    ]
    \addplot[const plot, blue, semithick] file{../data/zad5_multiplejumps/lambda/zad5_DMC_input_N_30_Nu_2_lambda_1.5.csv};
    \legend{$u[k]$}
    \end{axis}
    \end{tikzpicture}
    \caption{Przebieg sygnału sterującego przy parametrach parametrach: $D = 91$, $N = 30$, $N_{\mathrm{u}} = 2$, $\lambda = 1.5$}
    \label{dmc_lam_1_5_u}
\end{figure}

\subsubsection{Współczynnik $\lambda = 2$}
Zgodnie z naszymi przeczuciami, dalsze zwiększanie współczynnika $\lambda$ wygładza przebiegi sygnału wyjściowego oraz sterowania. Na tym etapie przeregulowanie oraz oscylacje sygnałów są już bardzo niewielkie, rośnie za to błąd spowodowany wolniejszym działaniem regulatora. Idąc tym tropem można stwierdzić, że dalsze zwiększanie $\lambda$ będzie skutkowało większym błędem oraz wolniejszym działaniem. Wynik testu zamieszczony został na rysunkach \ref{dmc_lam_2_y} i \ref{dmc_lam_2_u}.

\begin{figure}[t]
    \centering
    \begin{tikzpicture}
    \begin{axis}[
    width=0.98\textwidth,
    xmin=0,xmax=600,ymin=3.7, ymax=4.5,
    xlabel={$k$},
    ylabel={$y[k]$},
    %xtick={0, 100, 200, 300, 400, 500},
    %ytick={3.9, 4, 4.1, 4.2, 4.3, 4.4},
    legend pos=south east,
    y tick label style={/pgf/number format/1000 sep=},
    ]
    \addplot[blue, semithick] file{../data/zad5_multiplejumps/lambda/zad5_DMC_setpoint_N_30_Nu_2_lambda_2.csv};
    \addplot[red, semithick] file{../data/zad5_multiplejumps/lambda/zad5_DMC_output_N_30_Nu_2_lambda_2.csv};
    \addlegendentry{$y^{\mathrm{zad}}[k]$},
    \addlegendentry{$y[k]$},
    \addlegendimage{no markers, blue}
	\addlegendimage{no markers, red}
    \end{axis}
    \end{tikzpicture}
    \caption{Przebieg procesu sterowanego za pomocą regulatora z parametrami $D = 91$, $N = 30$, $N_{\mathrm{u}} = 2$, $\lambda = 2$}
    \label{dmc_lam_2_y}
\end{figure}

\begin{figure}[b]
    \centering
    \begin{tikzpicture}
    \begin{axis}[
    width=0.98\textwidth,
    xmin=0,xmax=500,ymin=0.25, ymax=0.75,
    xlabel={$k$},
    ylabel={$u[k]$},
    %xtick={0, 0.4, 0.5, 0.6, 0.7},
    %ytick={3.5, 3.75, 4, 4.25, 4.5},
    legend pos=south east,
    y tick label style={/pgf/number format/1000 sep=},
    ]
    \addplot[const plot, blue, semithick] file{../data/zad5_multiplejumps/lambda/zad5_DMC_input_N_30_Nu_2_lambda_2.csv};
    \legend{$u[k]$}
    \end{axis}
    \end{tikzpicture}
    \caption{Przebieg sygnału sterującego przy parametrach parametrach: $D = 91$, $N = 30$, $N_{\mathrm{u}} = 2$, $\lambda = 2$}
    \label{dmc_lam_2_u}
\end{figure}

\subsubsection{Współczynnik $\lambda = 10$}
Eksperyment przeprowadzony ze współczynnikiem $\lambda = 10$ służy jako potwierdzenie tezy postawionej w poprzednim podpunkcie. Wynikiem eksperymentu jest niemalże gładki sygnał wyjścia bez przeregulowania. Sygnał sterujący również przebiega bardzo spokojnie, nie udało się jednak wyeliminować uderzenia w ograniczenie górne sygnału. Wraz ze zwiększeniem wartości parametru wzrósł również błąd, a wskaźnik jakości regulacji wynosił $E = \num{3,9606}$ co jest znaczącą zmianą w porównaniu z poprzednimi wynikami. Przebiegi sygnałów można obejrzeć na rysunkach \ref{dmc_lam_10_y} oraz \ref{dmc_lam_10_u}.
	
\begin{figure}[t]
    \centering
    \begin{tikzpicture}
    \begin{axis}[
    width=0.98\textwidth,
    xmin=0,xmax=600,ymin=3.7, ymax=4.5,
    xlabel={$k$},
    ylabel={$y[k]$},
    %xtick={0, 100, 200, 300, 400, 500},
    %ytick={3.9, 4, 4.1, 4.2, 4.3, 4.4},
    legend pos=south east,
    y tick label style={/pgf/number format/1000 sep=},
    ]
    \addplot[blue, semithick] file{../data/zad5_multiplejumps/lambda/zad5_DMC_setpoint_N_30_Nu_2_lambda_10.csv};
    \addplot[red, semithick] file{../data/zad5_multiplejumps/lambda/zad5_DMC_output_N_30_Nu_2_lambda_10.csv};
    \addlegendentry{$y^{\mathrm{zad}}[k]$},
    \addlegendentry{$y[k]$},
    \addlegendimage{no markers, blue}
	\addlegendimage{no markers, red}
    \end{axis}
    \end{tikzpicture}
    \caption{Przebieg procesu sterowanego za pomocą regulatora z parametrami $D = 91$, $N = 30$, $N_{\mathrm{u}} = 2$, $\lambda = 10$}
    \label{dmc_lam_10_y}
\end{figure}

\begin{figure}[b]
    \centering
    \begin{tikzpicture}
    \begin{axis}[
    width=0.98\textwidth,
    xmin=0,xmax=500,ymin=0.25, ymax=0.75,
    xlabel={$k$},
    ylabel={$u[k]$},
    %xtick={0, 0.4, 0.5, 0.6, 0.7},
    %ytick={3.5, 3.75, 4, 4.25, 4.5},
    legend pos=south east,
    y tick label style={/pgf/number format/1000 sep=},
    ]
    \addplot[const plot, blue, semithick] file{../data/zad5_multiplejumps/lambda/zad5_DMC_input_N_30_Nu_2_lambda_10.csv};
    \legend{$u[k]$}
    \end{axis}
    \end{tikzpicture}
    \caption{Przebieg sygnału sterującego przy parametrach parametrach: $D = 91$, $N = 30$, $N_{\mathrm{u}} = 2$, $\lambda = 10$}
    \label{dmc_lam_10_u}
\end{figure}
\FloatBarrier
\section{Ostateczny zestaw parametrów}
Ostatecznie zdecydowaliśmy się na regulator o poniższych parametrach:\\

\begin{center}
    $N = \num{30} \mathrm{\hspace{1cm}}N_{\mathrm{u}} = \num{2} \mathrm{\hspace{1cm}} \lambda = \num{1}$
\end{center}
Wskaźnik jakości regulacji dla tego zestawu był jednym z najmniejszych uzyskanych podczas wszystkich eksperymentów i wyniósł $E = \num{3,9034}$. Warto zauważyć, że uzyskany wynik jest dużo lepszy od wyniku uzyskanego przez regulator PID. Co więcej, dostrojony regulator zapewniał pośrednią opcję między szybkością regulacji oraz jej dokładnością, a spokojnym przebiegiem sterowania. 
\label{dmc_result}

\part{Laboratorium}
\label{LAB}
\chapter{Obiekt laboratoryjny -- stanowisko chłodząco-grzejące}
\label{lab1}

\section{Wyznaczenie punktu pracy}
\label{lab1_punkt_pracy}

Po sprawdzeniu możliwości sterowania i pomiaru w komunikacji ze stanowiskiem przystąpiliśmy do wyznaczenia punktu pracy.
Od prowadzącego otrzymaliśmy informację o wartości sygnału sterującego w punkcie pracy 
równej $U_{\mathrm{pp}} = 26\%$. W celu wyznaczenia wartości sygnału wyjściowego w 
punkcie pracy $Y_{\mathrm{pp}}$ na wejście obiektu podaliśmy stałe wejście o wartości 
$U_{\mathrm{pp}}$ i zaczekaliśmy aż wyjście ustabilizuje się. Po odczekaniu około 5 minut,
wyjście obiektu ustabilizowało się na wartości $Y_{\mathrm{pp}} = \num{32} \degree $C


\section{Wyznaczenie odpowiedzi skokowych toru zakłócenie-wyjście}
\label{lab1_odpowiedzi}

\begin{figure}[b]
    \centering
    \begin{tikzpicture}
    \begin{axis}[
    width=0.98\textwidth,
    xmin=0.0,xmax=600,ymin=29.5,ymax=36,
    xlabel={$k$},
    ylabel={$y[k]$},
    legend pos=north west,
    y tick label style={/pgf/number format/1000 sep=},
    ] 
    \addlegendentry{Skok o 10},
    \addlegendentry{Skok o -10}
    \addlegendentry{Skok o 20}
    \addlegendentry{Skok o -20},
    \addlegendimage{no markers,green}
	\addlegendimage{no markers,red}
	\addlegendimage{no markers,blue}
	\addlegendimage{no markers, orange}
    \addplot[green, semithick, thick] file{../data/lab/zad2/zad2_skok_zakl_o_10.csv};
    \addplot[red, semithick, thick] file{../data/lab/zad2/zad2_skok_zakl_o_-10.csv};
    \addplot[blue, semithick, thick] file{../data/lab/zad2/zad2_skok_zakl_o_20.csv};
    \addplot[blue, semithick, thick] file{../data/lab/zad2/zad2_skok_zakl_o_-20.csv};
    \end{axis}
    \end{tikzpicture}
    \caption{Odpowiedzi procesu na skokową zmianę sygnału sterującego}
    \label{zad2_porow_odp_skok_lab}
\end{figure}
\FloatBarrier

Zebraliśmy cztery odpowiedzi
skokowe z punktu pracy w celu zbadania właściwości statycznych obiektu. Zostały one przedstawione na rysunku \ref{zad2_porow_odp_skok_lab}. Wzmocnienie obiektu różni się w zależności od tego czy skok sterowania był dodatni lub ujemny, jednak różnice te są niewielkie, że obiekt można traktować
jako liniowy. Korzystając z odpowiedniego  wzoru wyliczyliśmy efektywne wzmocnienie dla każdego badanego skoku i policzyliśmy ich średnia arytmetyczna, ostatecznie
otrzymując Kstat POLICZYC.
\section{Aproksymacja odpowiedzi skokowych}
\label{zad3_lab_opis}
\definecolor{mylilas}{RGB}{170,55,241}

\lstset{language=Matlab,%
    %basicstyle=\color{red},
    breaklines=true,%
    morekeywords={matlab2tikz},
    keywordstyle=\color{blue},%
    morekeywords=[2]{1}, keywordstyle=[2]{\color{black}},
    identifierstyle=\color{black},%
    stringstyle=\color{mylilas},
    stringstyle=\color{magenta},
    commentstyle=\color{green},%
    showstringspaces=false,%without this there will be a symbol in the places where there is a space
    numbers=left,%
    numberstyle={\tiny \color{black}},% size of the numbers
    numbersep=9pt, % this defines how far the numbers are from the text
    emph=[1]{for,end,break},emphstyle=[1]\color{red}, %some words to emphasise
    %emph=[2]{word1,word2}, emphstyle=[2]{style},    
}

\section{Strojenie regulatora DMC na stanowisku laboratoryjnym bez zakłóceń}
\label{zad4_lab_opis}
\subsection{Omówienie implementacji algorytmu DMC}
Na początku skryptu następowało wczytanie modelu (znormalizowanej odpowiedzi skokowej obiektu) wykorzystywanego do regulacji algorytmem DMC.\\
\indent{} Następnie, tak jak poprzednio, zdefiniowane zostały stałe wykorzystywane w programie jak i parametry wykorzystywanego algorytmu. Horyzont dynamiki równał się długości wektora z odpowiedzią skokową. Horyzonty predykcji oraz sterowania, a także współczynnik kary $\lambda$ były odpowiednio modyfikowane w późniejszym procesie strojenia regulatora. \\
\indent{} Kolejnym krokiem, również powtórzonym była deklaracja wektorów wykorzystywanych do przechowywania wartości sygnałów.
%[language=Matlab]

\begin{lstlisting}[style=custommatlab,frame=single,label={zad4_vecDMC_lst},caption={Inizjalizacja wektorów używanych do przechowywania sygnałów.},captionpos=b]
% czas symulacji
sim_time = 1:sim_len; % do plotowania
sim_time = sim_time';

% wartosc zadana
stpt_value = 4.05;
setpoint = stpt_value*ones(sim_len,1);
setpoint(1:11) = Ypp;

% wektor sygnalu sterujacego
input = Upp*ones(sim_len, 1);

% wektor wyjscia
output = Ypp*ones(sim_len, 1);

rescaled_output = 0;
rescaled_input = zeros(sim_len, 1);

% wektor uchybu
error = zeros(sim_len, 1);

\end{lstlisting}

Pracę nad algorytmem należało rozpocząć od inicjalizacji macierzy algorytmu, używanych później do symulacji. Ważnym aspektem tej czynności jest poprawne określenie wymiarów każdej macierzy. Mając zdefiniowane w ten sposób zmienne mogliśmy kontrolować, czy poszczególne obliczenia, tworzące poniższe elementy algorytmu, przebiegały poprawnie.
\\

\begin{lstlisting}[style=custommatlab,frame=single,label={zad4_sim_lst},caption={Definicja macierzy algorytmu DMC},captionpos=b]
dU = zeros(Nu, 1);
dUp = zeros(D-1, 1);
M = zeros(N, Nu);
Mp = zeros(N, D-1);
K = zeros(Nu, N);
\end{lstlisting}

Następnie przystąpiliśmy do wyliczenia macierzy, których postać była niezmienna podczas działania programu. Kolejno obliczona została macierz $M$, $M_\mathrm{p}$ oraz $K$ zgodnie ze wzorami:
\\

\begin{equation}
M=\left[
\begin{array}
{cccc}
s_{1} & 0 & \ldots & 0\\
s_{2} & s_{1} & \ldots & 0\\
\vdots & \vdots & \ddots & \vdots\\
s_{N} & s_{N-1} & \ldots &  s_{N-N_{\mathrm{u}}+1}
\end{array}
\right]
\end{equation}
\\
\begin{equation}
M^{\mathrm{p}}=\left[
\begin{array}
{cccc}
s_{\mathrm{2}} - s_{\mathrm{1}} & s_{\mathrm{3}} - s_{\mathrm{2}} & \ldots & s_{\mathrm{D}} - s_{\mathrm{D-1}}\\
s_{\mathrm{3}} - s_{\mathrm{1}} & s_{\mathrm{4}} - s_{\mathrm{2}} & \ldots & s_{\mathrm{D+1}} - s_{\mathrm{D-1}}\\
\vdots & \vdots & \ddots & \vdots\\
s_{\mathrm{N+1}} - s_{1} & s_{\mathrm{N+2}} - s_{\mathrm{2}} & \ldots &  s_{N + \mathrm{D} - 1} - s_{{\mathrm{D-1}}}
\end{array}
\right]
\end{equation}
\\
\begin{equation}
K = (M^{\mathrm{T}} M + \lambda I_{\mathrm{N_{\mathrm{u}}} \times \mathrm{N_{\mathrm{u}}}})^{\mathrm{-1}} M^{\mathrm{T}}
\end{equation}
\\
Implementację powyższych wzorów przedstawiono na listingu \ref{zad4_matrices_lst}.\\

\begin{lstlisting}[style=custommatlab,frame=single,label={zad4_matrices_lst},caption={Wyliczenie macierzy stałych algorytmu DMC},captionpos=b]
% macierz M
for i = 1:Nu
    k = 1;
    for j = 1:N
        if j < i
            M(j, i) = 0;
        else
            M(j, i) = step(k);
            k = k + 1;
        end
    end
end

% macierz Mp
for i = 1:N
    for j = 1:(D-1)
        k = i + j;
        if k > D
            Mp(i,j) = step(D) - step(j);
        else
            Mp(i,j) = step(k) - step(j);
        end
    end
end

% macierz K
K = inv((M' * M + lambda * eye(Nu)))*M';
%\end{lstlisting}
Wersja algorytmu DMC użyta w programie jest wersją oszczędną to znaczy podczas symulacji nie jest obliczany cały wektor przyrostów sterowania $\Delta u(k)$, a jedynie pierwszy jego element faktycznie wykorzystywany do sterowania obiektem. Dana wartość obliczana jest ze wzoru:

\begin{equation}
\label{du(k)}
\Delta u(k) = K_{\mathrm{e}} e(k) - K_{\mathrm{u}} \Delta U^{\mathrm{P}}(k)
\end{equation}

gdzie $K_{\mathrm{e}}$ jest wyrażonym wzorem (\ref{eq_Ke}), $K_{\mathrm{u}}$ jest wektorem powstałym z pomnożenia pierwszego wiersza macierzy $K$ oraz macierzy $M^{\mathrm{P}}$ co przedstawia wzór (\ref{eq_Ku}). Wektor $\Delta U^{\mathrm{P}}(k)$ jest wektorem różnic sterowania między kolejnymi jego wartościami z przeszłości danym wzorem:

\begin{equation}
\Delta U^{\mathrm{P}}(k)=\left[
\begin{array}{c}
u(k-1) - u(k-2)\\
u(k-2) - u(k-3)\\
\vdots\\
u(k-\mathrm{D}+1) - u(k-\mathrm{D})
\end{array}
\right]
\end{equation}

\begin{equation}
\label{eq_Ke}
K_{\mathrm{e}} = \sum_{i=1}^{N} K_{1,i}
\end{equation}

\begin{equation}
\label{eq_Ku}
K_{\mathrm{u}} = \overline{K_{1}} M^{\mathrm{P}}
\end{equation}

Wektor $\Delta U^{\mathrm{P}}(k)$ jest obliczany w każdej iteracji pętli symulacyjnej w celu uniknięcia niepotrzebnych komplikacji związanych z przesuwaniem jego wartości. Przed wejściem do pętli obliczane są wartości $K_{\mathrm{e}}$ oraz $K_{\mathrm{e}}$. Symulacja działania algorytmu za pomocą pętli została przedstawiona na listingu \ref{zad4_dmc_sim_lst}.\\

\begin{lstlisting}[style=custommatlab,frame=single,label={zad4_dmc_sim_lst},caption={Pętla symulująca działanie regulatora DMC},captionpos=b]
Ke = sum(K(1,:));
Ku = K(1,:) * Mp;

for k=12:sim_len    
    % wektor dUp
    for i = 1:(D-1)
        if (k-i) <= 0
            du1 = 0;
        else
            du1 = rescaled_input(k - i);
        end
        if (k-i-1) <= 0
            du2 = 0;
        else
            du2 = rescaled_input(k - i - 1);
        end 
        dUp(i) = du1 - du2;
    end
    
    output(k) = symulacja_obiektu1Y(input(k-10), input(k-11),
    			output(k-1), output(k-2));    % pomiar wyjscia
    rescaled_output = output(k) - Ypp;  % skalowanie wyjscia   
    stpt = setpoint(k) - Ypp;   % przeskalowany setpoint
    error(k) = stpt - rescaled_output; % obliczenie uchyby   
    
    error_sum = error_sum + error(k)^2; % wskaznik jakosci
    
    rescaled_input(k) = Ke * error(k) - Ku * dUp;
    rescaled_input(k) = rescaled_input(k-1) + rescaled_input(k);
    
    % ograniczenia  
    if rescaled_input(k) - rescaled_input(k-1) >= dUmax
        rescaled_input(k) = dUmax + rescaled_input(k-1);
    elseif rescaled_input(k) - rescaled_input(k-1) <= -dUmax
        rescaled_input(k) = rescaled_input(k-1) - dUmax;
    end   
    
    input(k) = input(k) + rescaled_input(k);  
    
    if input(k) >= Umax
        input(k) = Umax;
    elseif input(k) <= Umin
        input(k) = Umin;
    end 
end
\end{lstlisting}

Pętla jest zbudowana w taki podobny sposób co pętla do symulacji algorytmu PID. Różnicą na początku jest obliczenie wektora $\Delta U^{\mathrm{P}}(k)$. Następnie następuje pomiar wyjścia i przeskalowanie sygnałów: wyjściowego oraz wartości zadanej. Po obliczeniu uchybu oraz wskaźnika jakości następuje obliczenie wartości przyrostu sterowania zgodnie ze wzorem (\ref{du(k)}). Ostatnią częścią jest uwzględnienie ograniczeń sygnału sterującego, po którym następuje kolejna iteracja.\\
\indent{} W skrypcie \verb+zad4_DMC.m+ po zakończeniu symulacji również następuje wyświetlenie wyników, krótka obróbka danych oraz generowanie plików z danymi.


\subsection{Przebieg strojenia regulatora DMC}
Strojąc algorytm DMC na obiekcie laboratoryjnym bez zakłóceń skorzystaliśmy z doświadczenia zebranego na poprzednim laboratorium. Dla naszego regulatora wybraliśmy wartości parametrów D, N, $N_\mathrm{u}$, $\lambda$ dla których regulacja na poprzednim laboratorium przebiegała najlepiej.

Po dokonaniu eksperymentu dla $D = 500$, $N = 140$, $N_\mathrm{u} = 1$, 
$\lambda = 100$ otrzymaliśmy przebiegi zaprezentowane na rysunkach 
\ref{lab_zad4_lab_proces_wykres} i \ref{lab_zad4_lab_ster}. Możemy zauważyć, 
że nie pojawiają się przeregulowania, a wartość zadana jest osiągana bardzo szybko. 

\indent Przebiegi procesu sterowanego i sygnału sterującego są pozbawione oscylacji. 
Warto zauważyć, że sygnał sterujący charakteryzuje się łagodnymi zmianami, 
co jest bardzo ważne dla elementów wykonawczych sterowanych przez regulator. 
Tak dobrze nastrojony regulator posłużył nam do wykonania następnych 
zadań podczas laboratorium. 
\begin{figure}[t]
    
    \centering
    \begin{tikzpicture}
    \begin{axis}[
    width=\textwidth,
    xmin=4,xmax=450,ymin=29,ymax=41,
    xlabel={$k$},
    ylabel={$y[k]$},
    %xtick={0, 50, 100, 150, 200},
    %ytick={0, 2, 4, 6, 8},
    legend pos=south east,
    y tick label style={/pgf/number format/1000 sep=},
    ]
    \addplot[red, semithick] file{../data/lab/zad4/output_ts.csv};
	\addplot[blue, semithick] file{../data/lab/zad4/setpoint_ts.csv};    
    \legend{$y[k]$, $y^{\mathrm{zad}}[k]$}
    \end{axis} 
    \end{tikzpicture}
    \caption{Przebieg procesu sterowanego za pomocą dostrojonego regulatora DMC}
    \label{lab_zad4_lab_proces_wykres}
\end{figure}

\begin{figure}[b]
    
    \centering
    \begin{tikzpicture}
    \begin{axis}[
    width=\textwidth,
    xmin=4,xmax=450,ymin=25,ymax=64,
    xlabel={$k$},
    ylabel={$y[k]$},
    %xtick={0, 50, 100, 150, 200},
    %ytick={0, 2, 4, 6, 8},
    legend pos=south east,
    y tick label style={/pgf/number format/1000 sep=},
    ]
    \addplot[blue, semithick] file{../data/lab/zad4/input_ts.csv};    
    \legend{$y[k]$}
    \end{axis} 
    \end{tikzpicture}
    \caption{Przebieg sygnału sterującego dostrojonego regulatora DMC}
    \label{lab_zad4_lab_ster}
\end{figure}
\FloatBarrier
\section{Dobieranie parametru \texorpdfstring{$D^{z}$ }{TEXT} }
\label{zad5_lab_opis}

\end{document}