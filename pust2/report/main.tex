\documentclass{mwrep}

% Polskie znaki
\usepackage{polski}
\usepackage[utf8]{inputenc}
\usepackage[T1]{fontenc}
\usepackage{lmodern}
\usepackage{indentfirst}

% Strona tytułowa
\usepackage{pgfplots}
\usepackage{siunitx}
\usepackage{paracol}
\usepackage{gensymb}

% Pływające obrazki
\usepackage{float}
\usepackage{svg}
\usepackage{graphicx}

% table of contents refs
\usepackage{hyperref}
\usepackage{cleveref}
\usepackage{booktabs}
\usepackage{listings}
\usepackage{placeins}
\usepackage{xcolor}

\sisetup{detect-weight,exponent-product=\cdot,output-decimal-marker={,},per-mode=symbol,binary-units=true,range-phrase={-},range-units=single}
\definecolor{szary}{rgb}{0.95,0.95,0.95}
%konfiguracje pakietu listings
\lstset{
	backgroundcolor=\color{szary},
	frame=single,
	breaklines=true,
}
\lstdefinestyle{customlatex}{
	basicstyle=\footnotesize\ttfamily,
	%basicstyle=\small\ttfamily,
}
\lstdefinestyle{customc}{
	breaklines=true,
	frame=tb,
	language=C,
	xleftmargin=0pt,
	showstringspaces=false,
	basicstyle=\small\ttfamily,
	keywordstyle=\bfseries\color{green!40!black},
	commentstyle=\itshape\color{purple!40!black},
	identifierstyle=\color{blue},
	stringstyle=\color{orange},
}
\lstdefinestyle{custommatlab}{
	captionpos=t,
	breaklines=true,
	frame=tb,
	xleftmargin=0pt,
	language=matlab,
	showstringspaces=false,
	%basicstyle=\footnotesize\ttfamily,
	basicstyle=\scriptsize\ttfamily,
	keywordstyle=\bfseries\color{green!40!black},
	commentstyle=\itshape\color{purple!40!black},
	identifierstyle=\color{blue},
	stringstyle=\color{orange},
}

%wymiar tekstu
\def\figurename{Rys.}
\def\tablename{Tab.}

%konfiguracja liczby p�ywaj�cych element�w
\setcounter{topnumber}{0}%2
\setcounter{bottomnumber}{3}%1
\setcounter{totalnumber}{5}%3
\renewcommand{\textfraction}{0.01}%0.2
\renewcommand{\topfraction}{0.95}%0.7
\renewcommand{\bottomfraction}{0.95}%0.3
\renewcommand{\floatpagefraction}{0.35}%0.5

\SendSettingsToPgf
\title{\bf Sprawozdanie z projektu i ćwiczenia laboratoryjnego nr 2, zadanie nr 2 \vskip 0.1cm}
\author{Marcin Dolicher \\ Jakub Sikora \\ Robert Wojtaś}
\date{\today}
\pgfplotsset{compat=1.15}	
\begin{document}
\frenchspacing
\pagestyle{uheadings}

\makeatletter
\renewcommand{\maketitle}{\begin{titlepage}
		\begin{center}{
				\LARGE {\bf Politechnika Warszawska}}\\
            \vspace{0.4cm}
            \leftskip-0.9cm
            {\LARGE {\bf \mbox{Wydział Elektroniki i Technik Informacyjnych}}}\\
            \vspace{0.2cm}
            {\LARGE {\bf \mbox{Instytut Automatyki i Informatyki Stosowanej}}}\\
            
            \vspace{5cm}
            \leftskip-1.3cm
			{\bf \Huge \mbox{Projektowanie układów sterowania} \vskip 0.1cm}
		\end{center}
		\vspace{0.1cm}

		\begin{center}
			{\bf \LARGE \@title}
		\end{center}

		\vspace{9cm}
		\begin{paracol}{2}
			\addtocontents{toc}{\protect\setcounter{tocdepth}{1}}
			\subsection*{Zespół:}
			\bf{ \Large{ \noindent\@author \par}}
			\addtocontents{toc}{\protect\setcounter{tocdepth}{2}}

			\switchcolumn \addtocontents{toc}{\protect\setcounter{tocdepth}{1}}
			\subsection*{Prowadzący:}
			\bf{\Large{\noindent dr inż. Patryk Chaber}}
			\addtocontents{toc}{\protect\setcounter{tocdepth}{2}}

		\end{paracol}
		\vspace*{\stretch{6}}
		\begin{center}
			\bf{\large{Warszawa, \@date\vskip 0.1cm}}
		\end{center}
	\end{titlepage}
}
\makeatother
\maketitle

\tableofcontents

\part{Projekt}
\label{PROJEKT}
%\chapter{Sprawdzenie poprawności wartości sygnałów $U_{\mathrm{pp}}$ i $Y_{\mathrm{pp}}$}
\label{zad1}

\section{Opis eksperymentu}
\label{zad1_opis}
Aby sprawdzić poprawność wartości sygnałów $U_{\mathrm{pp}}$ i $Y_{\mathrm{pp}}$ wykonaliśmy eksperyment polegający na
pobudzeniu wejścia obiektu stałym sygnałem o wartości $U_{\mathrm{pp}}$ i sprawdzeniu czy sygnał wyjściowy stabilizuje 
się na wartości $Y_{\mathrm{pp}}$. Symulację obiektu przeprowadziliśmy za pomocą funkcji 
\verb+symulacja_obiektu1Y+. Spodziewanym wyjściem procesu dla wejścia $U_{\mathrm{pp}} = \num{0.5}$ jest
wyjście równe $Y_{\mathrm{pp}} = \num{4}$.

\section{Realizacja badań}
\label{zad_realizacja}
Eksperyment przeprowadziliśmy za pomocą skryptu \verb+zad1.m+. Skrypt ten automatycznie przeprowadza symulację 
oraz zapisuje zebrane dane do pliku \verb+csv+. Zebrana odpowiedź obiektu została przedstawiona poniżej na rysunku \ref{zad1_output}. 
Obiekt przy stałym pobudzeniu $U_{\mathrm{pp}}=\num{0.5}$, stabilizuje sygnał wyjściowy na wartości $Y_{\mathrm{pp}}=\num{4}$.
Jednoznacznie potwierdza to poprawność zadanych wartości.

\begin{figure}[b]
    \label{zad1_output_sprawdzenie}
    \centering
    \begin{tikzpicture}
    \begin{axis}[
    width=\textwidth,
    xmin=0,xmax=200,ymin=0,ymax=8,
    xlabel={$k$},
    ylabel={$y[k]$},
    xtick={0, 50, 100, 150, 200},
    ytick={0, 2, 4, 6, 8},
    legend pos=south east,
    y tick label style={/pgf/number format/1000 sep=},
    ]
    \addplot[red, semithick] file{../data/zad1_output.csv};
    \legend{$y[k]$}
    \end{axis} 
    \end{tikzpicture}
    \caption{Odpowiedź symulowanego procesu na stałe wejście o wartości $U_{\mathrm{pp}}=\num{0.5}$}
\end{figure}
\chapter{Wyznaczenie odpowiedzi skokowych obiektu}
\label{zad2}

\section{Odpowiedź skokowa toru wejście-wyjście}
Eksperymenty wykonywane były z punktu pracy zdefiniowanego w zadaniu, przy sygnale zakłócenia $Z_{\mathrm{pp}} = 0$. Na wykresie \ref{zad2_odp_skok_input} możemy zaobserwować, że wraz z wzrostem wartości skoku rośnie również wartość odpowiedzi procesu $y[k]$. Zachowanie procesu jest zgodne 
z typową charakterystyką obiektów dynamicznych liniowych, w których wartość ustalona zmienia się liniowo wraz ze zmianą sygnału sterującego. 

\begin{figure}[t]
    \centering
    \begin{tikzpicture}
    \begin{axis}[
    width=0.98\textwidth,
    xmin=0.0,xmax=200,ymin=-1.5,ymax=1.5,
    xlabel={$k$},
    ylabel={$y[k]$},
    legend pos=south east,
    y tick label style={/pgf/number format/1000 sep=},
    ] 
    \addlegendentry{$\Delta u = \num{0,1}$},
    \addlegendentry{$\Delta u = \num{0,2}$}
    \addlegendentry{$\Delta u = \num{0,35}$},
    \addlegendentry{$\Delta u = \num{0,5}$},
    \addlegendentry{$\Delta u = \num{-0,1}$},
    \addlegendentry{$\Delta u = \num{-0,2}$},
    \addlegendentry{$\Delta u = \num{-0,35}$},
    \addlegendentry{$\Delta u = \num{-0,5}$},
    \addlegendimage{no markers,green}
	\addlegendimage{no markers,red}
	\addlegendimage{no markers,yellow}
	\addlegendimage{no markers,blue}
	\addlegendimage{no markers,black}
	\addlegendimage{no markers,orange}
	\addlegendimage{no markers,brown}
	\addlegendimage{no markers,violet}
    \addplot[green, semithick, thick] file{../data/zad2/zad2_input_output0.1.csv};
    \addplot[red, semithick, thick] file{../data/zad2/zad2_input_output0.2.csv};
    \addplot[yellow, semithick, thick] file{../data/zad2/zad2_input_output0.35.csv};
    \addplot[blue, semithick, thick] file{../data/zad2/zad2_input_output0.5.csv};
    \addplot[black, semithick, thick] file{../data/zad2/zad2_input_output-0.1.csv};
    \addplot[orange, semithick, thick] file{../data/zad2/zad2_input_output-0.2.csv};
    \addplot[brown, semithick, thick] file{../data/zad2/zad2_input_output-0.35.csv};
    \addplot[violet, semithick, thick] file{../data/zad2/zad2_input_output-0.5.csv};    
    \end{axis}
    \end{tikzpicture}
    \caption{Odpowiedzi procesu na skokową zmianę sygnału sterującego}
    \label{zad2_odp_skok_input}
\end{figure}

\section{Odpowiedź skokowa toru zakłócenie-wyjście}

Odpowiedź skokową toru zakłócenie-wyjście otrzymywaliśmy poprzez skokową zmianę zakłócenia. Eksperymenty wykonywane były ze stanu punktu pracy, w którym wszystkie sygnały wynosiły 0. Sygnał sterowania był stały podczas całej symulacji. Na wykresie \ref{zad2_odp_skok_disturbance} obserwujemy reakcję na zmiany zakłóceń i możemy stwierdzić, że jest ona podobna do tej otrzymanej przy zmianach sterowania, co tylko potwierdza nasze przypuszczenia o liniowości obiektu. 

\begin{figure}[b]
    \centering
    \begin{tikzpicture}
    \begin{axis}[
    width=0.98\textwidth,
    xmin=0.0,xmax=200,ymin=-1,ymax=1,
    xlabel={$k$},
    ylabel={$y[k]$},
    legend pos=south east,
    y tick label style={/pgf/number format/1000 sep=},
    ] 
    \addlegendentry{$\Delta z = \num{0,05}$},
    \addlegendentry{$\Delta z = \num{0,2}$}
    \addlegendentry{$\Delta z = \num{0,35}$},
    \addlegendentry{$\Delta z = \num{0,5}$},
    \addlegendentry{$\Delta z = \num{-0,05}$},
    \addlegendentry{$\Delta z = \num{-0,2}$},
    \addlegendentry{$\Delta z = \num{-0,35}$},
    \addlegendentry{$\Delta z = \num{-0,5}$},
    \addlegendimage{no markers,green}
	\addlegendimage{no markers,red}
	\addlegendimage{no markers,yellow}
	\addlegendimage{no markers,blue}
	\addlegendimage{no markers,black}
	\addlegendimage{no markers,orange}
	\addlegendimage{no markers,brown}
	\addlegendimage{no markers,violet}
    \addplot[green, semithick, thick] file{../data/zad2/zad2_disturbance_output0.05.csv};
    \addplot[red, semithick, thick] file{../data/zad2/zad2_disturbance_output0.2.csv};
    \addplot[yellow, semithick, thick] file{../data/zad2/zad2_disturbance_output0.35.csv};
    \addplot[blue, semithick, thick] file{../data/zad2/zad2_disturbance_output0.5.csv};
    \addplot[black, semithick, thick] file{../data/zad2/zad2_disturbance_output-0.05.csv};
    \addplot[orange, semithick, thick] file{../data/zad2/zad2_disturbance_output-0.2.csv};
    \addplot[brown, semithick, thick] file{../data/zad2/zad2_disturbance_output-0.35.csv};
    \addplot[violet, semithick, thick] file{../data/zad2/zad2_disturbance_output-0.5.csv};    
    \end{axis}
    \end{tikzpicture}
    \caption{Odpowiedzi procesu na skokową zmianę sygnału zaklócenia}
    \label{zad2_odp_skok_disturbance}
\end{figure}

\section{Wyznaczenie charakterystyki statycznej $y(u,z)$}
Wyznaczenie charakterystyki $y(u,z)$ rozpoczęliśmy od wyznaczenia charakterystyk $y(u)$ i $y(z)$. 
Na dalszym etapie zadania pomogło to w określeniu wzmocnień statycznych torów wejście-wyjście, zakłócenie-wyjśćie. 

Aby otrzymać wykres charakterystyki statycznej obiektu w zależności od 
dwóch argumentów należało przeprowadzić symulacje dla każdej 
wartości sygnałów $u$ oraz $d$ i zapamiętaniu wartości 
nasycenia sygnału $y$. Program, który realizujący zadanie to 
\verb+zad2_static_surface.m+ wykonywał wspomniane działania, 
a wyniki zapisywał do tablicy z wartościami charakterystyki statycznej. 
Wynik można zobaczyć na rysunku \ref{zad2_static_surf}.\\
\indent{} Wykres przedstawia płaszczyznę $y(u,z)$. Skoro powstały wykres 
jest płaszczyzną to można wywnioskować, że mamy do czynienia z obiektem liniowym. 



\part{Laboratorium}
\label{LAB}

\end{document}