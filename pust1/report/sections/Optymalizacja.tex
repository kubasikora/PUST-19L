\chapter{Optymalizacjyne metody strojenia regulatorów PID i DMC}
\label{zad6}
Ostatnim krokiem, którym wykonaliśmy podczas strojenia regulatorów PID i DMC 
była optymalizacja parametrów tych regulatorów pod kątem wskaźnika jakości $E$. 
Eksperymenty przeprowadziliśmy przy użyciu  
funkcji \textit{fmincon} znajdującej minimum nieliniowej funkcji wielu zmiennych 
przy określonych ograniczeniach.

\section{Optymalizacjyne strojenie regulatora PID}
\label{zad6_pid}
Za pomocą funkcji \textit{fmincon} dobraliśmy współczynniki 
$K_{\mathrm{r}}, T_{\mathrm{i}}, T_{\mathrm{d}}$, które zapewniały 
jak najmniejszą wartość wskaźnika jakości $E$. Liczba iteracji  
\textit{fmincon} zależała zmiany obliczonego wskaźnika $E$. Algorytm zatrzymywał się 
w przypadku gdy zmiana $\Delta E$ jest mniejsza od $10^{\num{-6}}$. 
Jako punkt początkowy podaliśmy wektor złożony z $K, T_{\mathrm{i}}, 
T_{\mathrm{d}}$, który na początku eksperymentów dobieraliśmy całkowicie losowo. 
Na współczynniki $K_{\mathrm{r}}, T_{\mathrm{i}}, T_{\mathrm{d}}$ nałożyliśmy następujące
ograniczenia:

\begin{center}
$\num{0.01} \leq K_{\mathrm{r}} \leq \num{10}$\\
$\num{1} \leq T_{\mathrm{i}} \leq \num{20}$\\
$\num{0,001} \leq T_{\mathrm{d}} \leq \num{10}$
\end{center}

\subsection{Wyniki optymalizacji dla losowo wybranego punktu początkowego}
\label{zad6_losowo}
Jako punkt początkowy algorytmu optymalizacji wybraliśmy następujące współczynniki: 
\begin{center}
$K = 4, {T_{\mathrm{i}} = 8}, T_{\mathrm{d}} = \num{0.2}$. 
\end{center}
Po uruchomieniu skryptu \verb+zad6_PID_optimization.m+ otrzymaliśmy 
wyniki równe:
\begin{center}
$K_{\mathrm{r}} = \num{1.1496}, T_{\mathrm{i}} = \num{7.022}, T_{\mathrm{d}} = \num{2.1442}$
\end{center}

Wskaźnik jakości wyniósł $E=\num{4.8277}$ co jest
znaczną poprawą w porównaniu do ręcznie dobieranych parametrów regulatora. 
Niestety jakość regulacji tego regulatora jest nieakceptowalna. Jak widać na rysunkach 
\ref{zad6_losowo_1} i \ref{zad6_losowo_sterowanie_1} przebiegi charakteryzują 
się mocnymi narastającymi oscylacjami, które wykluczają ten regulator z użycia.\\

W kolejnej próbie spróbowaliśmy wystartować algorytm optymalizacyjny z mniejszego
punktu początkowego. Jako startowe wartości obraliśmy
\begin{center}
$K = 2, T_{\mathrm{i}} = 6, T_{\mathrm{d}} = 0.01$ 
\end{center} Po uruchomieniu solvera otrzymaliśmy 
wyniki równe:
\begin{center}
$K_{\mathrm{r}} = \num{0.659}, T_{\mathrm{i}} = \num{6.8394}, T_{\mathrm{d}} = \num{0.001}$. 
\end{center}
Testowe przebiegi zostały przedstawione na rysunkach \ref{zad6_losowo_2} i 
\ref{zad6_losowo_sterowanie_2}. Jakość regulacji tego regulatora jest znacznie lepsza,
uchyb zbiega do zera a sygnał sterujący nie zmienia się w bardzo gwałtowny sposób.
Dla danej próby udało się poprawić wartość wskaźnika jakości regulacji do $E = \num{4.5486}$.
Warto zauważyć że wartość parametru $T_{\mathrm{d}}$ osiągnęła swoje dolne ograniczenie, czyli 
w pewien sposób optymalizator zdecydował o wyłączeniu członu różniczkującego.

\subsubsection{Zmiana funkcji optymalizowanej}
Do tej pory w funkcji optymalizowanej, regulator był testowany na trzech skokach wartości zadanej.
Doszliśmy do wniosku że taki sposób symulacji może wpływać oscylacyjny charakter obu uzyskanych regulatorów.
Dlatego postanowiliśmy jeszcze raz spróbować znaleźć optymalne parametry, tym razem dla pojedynczego skoku wartości zadanej.
Okazało się że, istotnie budowa funkcji minimalizowanej ma duży wpływ ostateczny charakter regulatora.
Dla takich samych punktów początkowych uzyskaliśmy następujący zestaw parametrów:
\begin{center}
    $K_{\mathrm{r}} = \num{0.876}, T_{\mathrm{i}} = \num{5.6771}, T_{\mathrm{d}} = \num{2.7848}$. 
\end{center}
Na rysunkach \ref{zad6_jeden_skok} i \ref{zad6_jeden_skok_sterowanie} przedstawiony został test regulatora
dla kilku zmian wartości zadanej. Przebieg wyjścia obiektu jest najlepszy z wszystkich trzech prób, pomimo że wartość
wskaźnika regulacji pogorszyła się względem poprzedniego zestawu parametrów i wynosi $E = \num{4,8867}$.

\begin{figure}[t]
    \centering
    \begin{tikzpicture}
    \begin{axis}[
    width=0.98\textwidth,
    xmin=0,xmax=600,ymin=3.7, ymax=4.4,
    xlabel={$k$},
    ylabel={$y[k]$},
    %xtick={0.3, 0.4, 0.5, 0.6, 0.7},
    %ytick={3.5, 3.75, 4, 4.25, 4.5},
    legend pos=south east,
    y tick label style={/pgf/number format/1000 sep=},
    ]
    \addplot[blue, semithick] file{../data/zad6_multiplejumps/PID/zad6_PID_setpoint_exampleK_1.1496_Ti_7.022_Td_2.1442_E_4.4828.csv};
    \addplot[red, semithick] file{../data/zad6_multiplejumps/PID/zad6_PID_output_exampleK_1.1496_Ti_7.022_Td_2.1442_E_4.4828.csv};
    \addlegendentry{$y^{zad}[k]$},
    \addlegendentry{$y[k]$},
    \addlegendimage{no markers, blue}
	\addlegendimage{no markers, red}
    \end{axis}
    \end{tikzpicture}
    \caption{Przebieg procesu sterowanego za pomocą regulatora z parametrami \mbox{$K_{\mathrm{r}} = \num{1.1496}, T_{\mathrm{i}} = \num{7.022}, T_{\mathrm{d}} = \num{2.1442}$}}
    \label{zad6_losowo_1}
\end{figure}

\begin{figure}[b]
    \centering
    \begin{tikzpicture}
    \begin{axis}[
    width=0.98\textwidth,
    xmin=0.0,ymin=0.3, xmax= 600,
    xlabel={$k$},
    ylabel={$u[k]$},
    %xtick={0.3, 0.4, 0.5, 0.6, 0.7},
    %ytick={3.5, 3.75, 4, 4.25, 4.5},
    legend pos=south east,
    y tick label style={/pgf/number format/1000 sep=},
    ]
    \addplot[const plot, blue, semithick] file{../data/zad6_multiplejumps/PID/zad6_PID_input_exampleK_1.1496_Ti_7.022_Td_2.1442_E_4.4828.csv};
    \legend{$u[k]$}
    \end{axis}
    \end{tikzpicture}
    \caption{Przebieg sygnału sterującego regulatora z parametrami \mbox{$K_{\mathrm{r}} = \num{1.1496}, T_{\mathrm{i}} = \num{7.022}, T_{\mathrm{d}} = \num{2.1442}$}}
    \label{zad6_losowo_sterowanie_1}
\end{figure}
\FloatBarrier

\begin{figure}[t]
    \centering
    \begin{tikzpicture}
    \begin{axis}[
    width=0.98\textwidth,
    xmin=0,xmax=600,ymin=3.7, ymax=4.4,
    xlabel={$k$},
    ylabel={$y[k]$},
    %xtick={0.3, 0.4, 0.5, 0.6, 0.7},
    %ytick={3.5, 3.75, 4, 4.25, 4.5},
    legend pos=south east,
    y tick label style={/pgf/number format/1000 sep=},
    ]
    \addplot[blue, semithick] file{../data/zad6_multiplejumps/PID/zad6_PID_setpoint_exampleK_0.65915_Ti_6.8394_Td_0.001_E_4.5486.csv};
    \addplot[red, semithick] file{../data/zad6_multiplejumps/PID/zad6_PID_output_exampleK_0.65915_Ti_6.8394_Td_0.001_E_4.5486.csv};
    \addlegendentry{$y^{zad}[k]$},
    \addlegendentry{$y[k]$},
    \addlegendimage{no markers, blue}
	\addlegendimage{no markers, red}
    \end{axis}
    \end{tikzpicture}
    \caption{Przebieg procesu sterowanego za pomocą regulatora z parametrami \mbox{$K_{\mathrm{r}} = \num{0.65915}, T_{\mathrm{i}} = \num{6.8394}, T_{\mathrm{d}} = \num{0.001}$}}
    \label{zad6_losowo_2}
\end{figure}

\begin{figure}[b]
    \centering
    \begin{tikzpicture}
    \begin{axis}[
    width=0.98\textwidth,
    xmin=0.0,ymin=0.3, xmax= 600,
    xlabel={$k$},
    ylabel={$u[k]$},
    %xtick={0.3, 0.4, 0.5, 0.6, 0.7},
    %ytick={3.5, 3.75, 4, 4.25, 4.5},
    legend pos=south east,
    y tick label style={/pgf/number format/1000 sep=},
    ]
    \addplot[const plot, blue, semithick] file{../data/zad6_multiplejumps/PID/zad6_PID_input_exampleK_0.65915_Ti_6.8394_Td_0.001_E_4.5486.csv};
    \legend{$u[k]$}
    \end{axis}
    \end{tikzpicture}
    \caption{Przebieg sygnału sterującego regulatora z parametrami \mbox{$K_{\mathrm{r}} = \num{0.65915}, T_{\mathrm{i}} = \num{6.8394}, T_{\mathrm{d}} = \num{0.001}$}}
    \label{zad6_losowo_sterowanie_2}
\end{figure}
\FloatBarrier

\begin{figure}[t]
    \centering
    \begin{tikzpicture}
    \begin{axis}[
    width=0.98\textwidth,
    xmin=0,xmax=600,ymin=3.7, ymax=4.4,
    xlabel={$k$},
    ylabel={$y[k]$},
    %xtick={0.3, 0.4, 0.5, 0.6, 0.7},
    %ytick={3.5, 3.75, 4, 4.25, 4.5},
    legend pos=south east,
    y tick label style={/pgf/number format/1000 sep=},
    ]
    \addplot[blue, semithick] file{../data/zad6/zad6_PID_setpoint_exampleK_0.87623_Ti_5.6771_Td_2.7848_E_4.8867.csv};
    \addplot[red, semithick] file{../data/zad6/zad6_PID_output_exampleK_0.87623_Ti_5.6771_Td_2.7848_E_4.8867.csv};
    \addlegendentry{$y^{zad}[k]$},
    \addlegendentry{$y[k]$},
    \addlegendimage{no markers, blue}
	\addlegendimage{no markers, red}
    \end{axis}
    \end{tikzpicture}
    \caption{Przebieg procesu sterowanego za pomocą regulatora z parametrami \mbox{$K_{\mathrm{r}} = \num{0.87623}, T_{\mathrm{i}} = \num{5.6771}, T_{\mathrm{d}} = \num{2.7848}$}}
    \label{zad6_jeden_skok}
\end{figure}

\begin{figure}[b]
    \centering
    \begin{tikzpicture}
    \begin{axis}[
    width=0.98\textwidth,
    xmin=0.0,ymin=0.3, xmax= 600,
    xlabel={$k$},
    ylabel={$u[k]$},
    %xtick={0.3, 0.4, 0.5, 0.6, 0.7},
    %ytick={3.5, 3.75, 4, 4.25, 4.5},
    legend pos=south east,
    y tick label style={/pgf/number format/1000 sep=},
    ]
    \addplot[const plot, blue, semithick] file{../data/zad6/zad6_PID_input_exampleK_0.87623_Ti_5.6771_Td_2.7848_E_4.8867.csv};
    \legend{$u[k]$}
    \end{axis}
    \end{tikzpicture}
    \caption{Przebieg sygnału sterującego regulatora z parametrami \mbox{$K_{\mathrm{r}} = \num{0.87623}, T_{\mathrm{i}} = \num{5.6771}, T_{\mathrm{d}} = \num{2.7848}$}}
    \label{zad6_jeden_skok_sterowanie}
\end{figure}
\FloatBarrier

 
\section{Optymalizacyjne strojenie regulatora DMC}
\label{zad6_dmc}
