\chapter{Dobieranie nastawów regulatorów PID}
\label{zad5}

\section{Strojenie regulatora PID}
\label{zad5_PID_wykresy}
W celu dobrania optymalnych parametrów regulatora, wyszliśmy od parametrów uzyskanych metodą
Zieglera-Nicholsa a następnie ręcznie dostroiliśmy poszczególne parametry tak aby uzyskać
jak najlepszy efekty końcowy. Eksperyment ten polegał na znalezieniu takiego wzmocnienia $K_{r}$, 
zwanego wzmocnieniem krytycznym, dla którego obiekt wpadał w niegasnące oscylacje o okresie $T_{k}$. 

Cały proces dobierania parametrów składał się z następujących kroków:
\begin{center}
\begin{enumerate}
    \item Wyznaczenie wzmocnienia $K_{\mathrm{k}}$ i okresu oscylacji $T_{\mathrm{k}}$
    \item Wyliczenie parametrów regulatora
    \item Ręczna poprawa wartości wzmocnienia członu proporcjonalnego $K_{\mathrm{r}}$
    \item Ręczna poprawa wartości parametru członu całkującego $T_{\mathrm{i}}$
    \item Ręczna poprawa wartości parametru członu różniczkującego $T_{\mathrm{d}}$
\end{enumerate}
\end{center}

Chcąc znaleźć jeszcze lepsze nastawy regulatora, po wyznaczeniu parametrów za pomocą metody Zieglera-Nicholsa, zastosowaliśmy metodę eksperymentalną. W której jakość regulacji ocenialiśmy jakościowo (na podstawie przebiegów sygnałów na wykresach) oraz ilościowo, za pomocą wskaźnika jakości regulacji danego wzorem:
\begin{equation}
E=\sum_{k=0}^{K_{konc}}(y^{zad}[k]-y[k])^{2}
\end{equation}


\subsection{Wyznaczenie wzmocnienia $K_{\mathrm{k}}$ i okresu oscylacji $T_{\mathrm{k}}$}
\label{zad5_wyznaczanie_parametrow_zn}
Na początku doprowadziliśmy układ do granicy stabilności, czyli momentu
w którym zaczęły pojawiać się niegasnące oscylacje. Zrealizowaliśmy to
zwiększając współczynnik wzmocnienia $K_{\mathrm{r}}$ przy wyłączonych członach 
całkującym i różniczkującym. Iteracyjnie wyznaczaliśmy nowe wartości wzmocnienia,
zwiększając jego wartość w przypadku gasnących oscylacji oraz zmniejszając w przypadku
niestabilnej odpowiedzi obiektu.

Gdy ostatecznie zaobserwowaliśmy niegasnące oscylacje zapisaliśmy aktualne wzmocnienie 
jako wzmocnienie krytyczne $K_{k}$ i zmierzyliśmy okres oscylacji drgań 
krytycznych $T_{k}$ ($K_{k} = 1,8 \hspace{0.3 cm} T_{k}=18,5 $). Wiedząc że okres próbkowania procesu 
wynosi $\num{0,5}s$, a zmierzony okres oscylacji wynosił $\num{37}$ próbek, wyliczyliśmy że
okres drgań krytycznych równa się $T_{k}=18,5s$. 

\begin{figure}[t]
    \centering
    \begin{tikzpicture}
    \begin{axis}[
    width=0.98\textwidth,
    xmin=0,xmax=500,ymin=3.7, ymax=4.3,
    xlabel={$k$},
    ylabel={$y[k]$},
    %xtick={0, 100, 200, 300, 400, 500},
    %ytick={3.9, 4, 4.1, 4.2, 4.3, 4.4},
    legend pos=south east,
    y tick label style={/pgf/number format/1000 sep=},
    ]
    \addplot[blue, semithick] file{../data/Zad5_figure_data/zad5_PID_setpoint_exampleK_1.8_Ti_1000000000000_Td_0.csv};
    \addplot[red, semithick] file{../data/Zad5_figure_data/zad5_PID_output_exampleK_1.8_Ti_1000000000000_Td_0.csv};
    \addlegendentry{$y^{zad}[k]$},
    \addlegendentry{$y[k]$},
    \addlegendimage{no markers, blue}
	\addlegendimage{no markers, red}
    \end{axis}
    \end{tikzpicture}
    \caption{Niegasnące oscylacje wyjścia obiektu przy wzmocnieniu krytycznym}
    \label{zad5_niegasnace_oscylacje}
\end{figure}

\begin{figure}[b]
    \centering
    \begin{tikzpicture}
    \begin{axis}[
    width=0.98\textwidth,
    xmin=0,xmax=500,ymin=0.25, ymax=0.75,
    xlabel={$k$},
    ylabel={$u[k]$},
    %xtick={0, 0.4, 0.5, 0.6, 0.7},
    %ytick={3.5, 3.75, 4, 4.25, 4.5},
    legend pos=south east,
    y tick label style={/pgf/number format/1000 sep=},
    ]
    \addplot[const plot, blue, semithick] file{../data/Zad5_figure_data/zad5_PID_input_exampleK_1.8_Ti_1000000000000_Td_0.csv};
    \legend{$u[k]$}
    \end{axis}
    \end{tikzpicture}
    \caption{Przebieg sygnału sterującego }
    \label{zad5_niegasnace_oscylacje_ster}
\end{figure}
\FloatBarrier

\subsection{Wyliczenie wstępnych parametrów regulatora}
\label{zad5_regulator_zn}
Dobierając nastawy według metody Zieglera-Nicholsa, 
stosowaliśmy wzory do obliczenia parametrów regulatora PID: 

\begin{equation}
\label{zad5_zn_wzory}
K_{\mathrm{r}}=\num{0,6}K_{\mathrm{k}}; \hspace{0.5 cm} T_{\mathrm{i}}=\num{0,5} \hspace{0.5 cm} T_{\mathrm{k}}; T_{\mathrm{d}}=\num{0,12} T_{\mathrm{k}}
\end{equation}
gdzie:
\begin{center}
\begin{itemize}
    \item $K_{\mathrm{k}}$ - wzmocnienie krytyczne
    \item $T_{\mathrm{k}}$ - okres oscylacji
\end{itemize}
\end{center}

Wyliczone wartości liczbowe parametrów:
\begin{center}
$K_{r}=\num{1,08} \hspace{0.5 cm}
T_{i}=\num{9,25}   \hspace{0.5 cm}
T_{d}=\num{2,22}$
\end{center}

Regulator z parametrami wyznaczonymi przy pomocy metody opisanej przez panów Johna Zieglera
i Nathaniela Nicholsa, nadal charakteryzuje się niegasnącymi oscylacjami. Takie 
przebiegi sygnałów są nieakceptowalne, dlatego w dalszych krokach ręcznie modyfikowaliśmy 
wartości parametrów. 

Na rysunku \ref{zad5_zn_regulator} przedstawiono przebiegu wartości zadanej oraz
wyjścia obiektu z widocznie przedstawionymi oscylacjami wokół wartości zadanej.
Rysunek \ref{zad5_zn_sterowanie} przedstawia zmianę sygnału sterującego
generowanego przez regulator.

Tak nastrojony regulator PID bardzo słabo radzi sobie z utrzymywaniem sygnału wyjściowego
procesu na poziomie wartości zadanej. Niegasnące oscylacje nie pozwalają na zastosowanie
tego zestawu parametrów do regulacji procesem. Dla zadanej trajektorii testowej, obliczony 
wskaźnik jakości regulacji wyniósł $E = \num{5,8385}$. Pomimo swoich wad, regulator ten jest 
dobrym punktem wyjścia dla metody eksperymentalnej.

\begin{figure}[t]
    \centering
    \begin{tikzpicture}
    \begin{axis}[
    width=0.98\textwidth,
    xmin=0,xmax=600,ymin=3.7, ymax=4.4,
    xlabel={$k$},
    ylabel={$y[k]$},
    %xtick={0.3, 0.4, 0.5, 0.6, 0.7},
    %ytick={3.5, 3.75, 4, 4.25, 4.5},
    legend pos=south east,
    y tick label style={/pgf/number format/1000 sep=},
    ]
    \addplot[blue, semithick] file{../data/zad5_multiplejumps/zad5_PID_setpoint_exampleK_1.08_Ti_9.25_Td_2.22_E_5.8385.csv};
    \addplot[red, semithick] file{../data/zad5_multiplejumps/zad5_PID_output_exampleK_1.08_Ti_9.25_Td_2.22_E_5.8385.csv};
    \addlegendentry{$y^{zad}[k]$},
    \addlegendentry{$y[k]$},
    \addlegendimage{no markers, blue}
	\addlegendimage{no markers, red}
    \end{axis}
    \end{tikzpicture}
    \caption{Przebieg procesu sterowanego za pomocą regulatora z parametrami wyznaczonymi za pomocą metody Zieglera-Nicholsa}
    \label{zad5_zn_regulator}
\end{figure}

\begin{figure}[b]
    \centering
    \begin{tikzpicture}
    \begin{axis}[
    width=0.98\textwidth,
    xmin=0.0,ymin=0.3, xmax= 600,
    xlabel={$k$},
    ylabel={$u[k]$},
    %xtick={0.3, 0.4, 0.5, 0.6, 0.7},
    %ytick={3.5, 3.75, 4, 4.25, 4.5},
    legend pos=south east,
    y tick label style={/pgf/number format/1000 sep=},
    ]
    \addplot[const plot, blue, semithick] file{../data/zad5_multiplejumps/zad5_PID_input_exampleK_1.08_Ti_9.25_Td_2.22_E_5.8385.csv};
    \legend{$u[k]$}
    \end{axis}
    \end{tikzpicture}
    \caption{Przebieg sygnału sterującego regulatora z parametrami wyznaczonymi za pomocą metody Zieglera-Nicholsa}
    \label{zad5_zn_sterowanie}
\end{figure}
\FloatBarrier




\subsection{Ręczna poprawa wartości wzmocnienia członu proporcjonalnego $K_{\mathrm{r}}$ }
\label{zad5_poprawa_Kr}
\subsubsection{Wzmocnienie regulatora $\mathbf{K_{\mathrm{r}}}=\num{0,75}$}
W pierwszej kolejności skupiliśmy się na zmianie parametru $K_{\mathrm{r}}$.
Zmniejszenie $K_{\mathrm{r}}$ do wartości $\num{0.75}$ sprawiło że pozbyliśmy się oscylacji. Czas regulacji
wyniósł około 100 próbek. Dodatkowo, sygnał wyjściowy procesu nie cechuje się przeregulowaniem,
czyli przekroczenie wartości zadanej. Wyjście obiektu już nie oscyluje tak jak to było w przypadku regulatora z \ref{zad5_regulator_zn},
jak i została przywrócona stabilność procesu. Sygnał sterujący cechuje się oscylacjami przy zmianach
wartości zadanej, co powoduje nierówny przebieg wyjścia procesu.
Wartość wskaźnika jakości regulacji dla tego regulatora przy tej samej
testowej trajektorii zadanej wyniosła $E=\num{5,6914}$, co jest niewielką poprawą w stosunku 
do bazowego regulatora.
Przebiegi sygnału wyjściowego jaki i sygnału sterującego, odbiegały od pożądanych.

Wykres \ref{zad5_1_K_odp} przedstawia przebiegi wyjścia procesu oraz trajektorii zadanej.
Na rysunku \ref{zad5_1_zmiana_K} przedstawiliśmy przebieg sygnału sterującego generowanego przez regulator.

\subsubsection{Wzmocnienie regulatora $\mathbf{K_{\mathrm{r}}}=\num{0,6}$}
Zmniejszenie $K_{\mathrm{r}}$ do poziomu $\num{0,6}$ spowodowało wizualną poprawę odpowiedzi układu 
na zmianę wartości zadanej. Przebieg sygnału wyjściowego wygładził się, proces w łagodniejszy sposób
dociera do wartości zadanej. Czas regulacji niie zmienił się w porównaniu do poprzedniego regulatora.
Przebieg sygnału sterującego również uległ wygładzeniu, chociaż oscylacje nie zniknęły w całości.
Wartość wskaźnika jakości regulacji dla tego 
przypadku wyniosła $E=\num{6.1449}$, co zaskakująco, jest wartością większą w porównaniu do poprzedniego 
$K_{\mathrm{r}}$ regulatora. Łagodniejszy przebieg wyjścia powoduje zwiększenie sumy kwadratów uchybu.

Wykres \ref{zad5_2_K_odp} przedstawia przebiegi wyjścia procesu oraz trajektorii zadanej.
Na rysunku \ref{zad5_2_zmiana_K} przedstawiliśmy przebieg sygnału sterującego generowanego przez regulator.

\subsubsection{Wzmocnienie regulatora $\mathbf{K_{\mathrm{r}}}=\num{0,5}$}
Na wykresach \ref{zad5_3_K_odp} i \ref{zad5_3_zmiana_K} przedstawiono przebiegi powstałe 
dla regulatora z wzmocnieniem $ K_{\mathrm{r}} = \num{0,5}$. Sygnał sterujący praktycznie nie
oscyluje, natomiast sygnał wyjściowy ma jeszcze łagodniejszy przebieg co niestety sprawia że czas regulacji 
nieznacznie wzrósł. Niestety, wskaźnik jakości 
regulacji wzrósł do wartości $E=\num{6,5814}$. Zauważyliśmy, że jego wartość rośnie przy zmniejszaniu  
parametru $ K_{\mathrm{r}}$, dlatego postanowiliśmy 
ostatecznie przyjąć wartość $K_{\mathrm{r}} = \num{0,5}$ i przejść do manipulowania 
pozostałymi parametrami. Dalsze zmniejszanie nie spowoduje 
znaczącej poprawy przebiegów a jedynie pogorszy wskaźnik jakości regulacji. 

%%%%%%%%%%%%%%%%%%%%%%%%%%%%% wykresy dla k = 0.75

\begin{figure}[t]
    \centering
    \begin{tikzpicture}
    \begin{axis}[
    width=0.98\textwidth,
    xmin=0,xmax=600,ymin=3.8, ymax=4.4,
    xlabel={$k$},
    ylabel={$y[k]$},
    %xtick={0.3, 0.4, 0.5, 0.6, 0.7},
    %ytick={3.5, 3.75, 4, 4.25, 4.5},
    legend pos=south east,
    y tick label style={/pgf/number format/1000 sep=},
    ]
    \addplot[blue, semithick] file{../data/zad5_multiplejumps/P/zad5_PID_setpoint_exampleK_0.75_Ti_9.25_Td_2.22_E_5.6914.csv};
    \addplot[red, semithick] file{../data/zad5_multiplejumps/P/zad5_PID_output_exampleK_0.75_Ti_9.25_Td_2.22_E_5.6914.csv};
    \addlegendentry{$y^{zad}[k]$},
    \addlegendentry{$y[k]$},
    \addlegendimage{no markers, blue}
	\addlegendimage{no markers, red}
    \end{axis}
    \end{tikzpicture}
    \caption{Przebieg procesu sterowanego za pomocą regulatora z 
    parametrami \mbox{$K_{\mathrm{r}} = \num{0,75}$; $T_{\mathrm{i}} = \num{9,25}$; $T_{\mathrm{d}} = \num{2,22}$}}
    \label{zad5_1_K_odp}
\end{figure}

\begin{figure}[b]
    \centering
    \begin{tikzpicture}
    \begin{axis}[
    width=0.98\textwidth,
    xmin=0.0, ymin=0.44, xmax=600, ymax=0.68,
    xlabel={$k$},
    ylabel={$u[k]$},
    %xtick={0.3, 0.4, 0.5, 0.6, 0.7},
    %ytick={3.5, 3.75, 4, 4.25, 4.5},
    legend pos=south east,
    y tick label style={/pgf/number format/1000 sep=},
    ]
    \addplot[const plot, blue, semithick] file{../data/zad5_multiplejumps/P/zad5_PID_input_exampleK_0.75_Ti_9.25_Td_2.22_E_5.6914.csv};
    \legend{$u[k]$}
    \end{axis}
    \end{tikzpicture}
    \caption{Przebieg sygnału sterującego generowanego przez regulator z 
    parametrami \mbox{$K_{\mathrm{r}} = \num{0,75}$; $T_{\mathrm{i}} = \num{9,25}$; $T_{\mathrm{d}} = \num{2,22}$}}
    \label{zad5_1_zmiana_K}
\end{figure}
\FloatBarrier

%%%%%%%%%%%%%%%%%%%%%%%%%%%%% wykresy dla k = 0.6


\begin{figure}[t]
    \centering
    \begin{tikzpicture}
    \begin{axis}[
    width=0.98\textwidth,
    xmin=0,xmax=600,ymin=3.8, ymax=4.4,
    xlabel={$k$},
    ylabel={$y[k]$},
    %xtick={0.3, 0.4, 0.5, 0.6, 0.7},
    %ytick={3.5, 3.75, 4, 4.25, 4.5},
    legend pos=south east,
    y tick label style={/pgf/number format/1000 sep=},
    ]
    \addplot[blue, semithick] file{../data/zad5_multiplejumps/P/zad5_PID_setpoint_exampleK_0.6_Ti_9.25_Td_2.22_E_6.1449.csv};
    \addplot[red, semithick] file{../data/zad5_multiplejumps/P/zad5_PID_output_exampleK_0.6_Ti_9.25_Td_2.22_E_6.1449.csv};
    \addlegendentry{$y^{zad}[k]$},
    \addlegendentry{$y[k]$},
    \addlegendimage{no markers, blue}
	\addlegendimage{no markers, red}
    \end{axis}
    \end{tikzpicture}
    \caption{Przebieg procesu sterowanego za pomocą regulatora z 
    parametrami \mbox{$K_{\mathrm{r}} = \num{0,6}$; $T_{\mathrm{i}} = \num{9,25}$; $T_{\mathrm{d}} = \num{2,22}$}}
    \label{zad5_2_K_odp}
\end{figure}

\begin{figure}[b]
    \centering
    \begin{tikzpicture}
    \begin{axis}[
    width=0.98\textwidth,
    xmin=0.0,ymin=0.44, xmax=600, ymax=0.68,
    xlabel={$k$},
    ylabel={$u[k]$},
    %xtick={0.3, 0.4, 0.5, 0.6, 0.7},
    %ytick={3.5, 3.75, 4, 4.25, 4.5},
    legend pos=south east,
    y tick label style={/pgf/number format/1000 sep=},
    ]
    \addplot[const plot, blue, semithick] file{../data/zad5_multiplejumps/P/zad5_PID_input_exampleK_0.6_Ti_9.25_Td_2.22_E_6.1449.csv};
    \legend{$u[k]$}
    \end{axis}
    \end{tikzpicture}
    \caption{Przebieg sygnału sterującego generowanego przez regulator z 
    parametrami \mbox{$K_{\mathrm{r}} = \num{0,6}$; $T_{\mathrm{i}} = \num{9,25}$; $T_{\mathrm{d}} = \num{2,22}$}}
    \label{zad5_2_zmiana_K}
\end{figure}
\FloatBarrier

%%%%%%%%%%%%%%%%%%%%%% wykresy dla k = 0.5

\begin{figure}[t]
    \centering
    \begin{tikzpicture}
    \begin{axis}[
    width=0.98\textwidth,
    xmin=0,xmax=600, ymin=3.8, ymax=4.4,
    xlabel={$k$},
    ylabel={$y[k]$},
    %xtick={0.3, 0.4, 0.5, 0.6, 0.7},
    %ytick={3.5, 3.75, 4, 4.25, 4.5},
    legend pos=south east,
    y tick label style={/pgf/number format/1000 sep=},
    ]
    \addplot[blue, semithick] file{../data/zad5_multiplejumps/P/zad5_PID_setpoint_exampleK_0.5_Ti_9.25_Td_2.22_E_6.5814.csv};
    \addplot[red, semithick] file{../data/zad5_multiplejumps/P/zad5_PID_output_exampleK_0.5_Ti_9.25_Td_2.22_E_6.5814.csv};
    \addlegendentry{$y^{zad}[k]$},
    \addlegendentry{$y[k]$},
    \addlegendimage{no markers, blue}
	\addlegendimage{no markers, red}
    \end{axis}
    \end{tikzpicture}
    \caption{Przebieg procesu sterowanego za pomocą regulatora z 
    parametrami \mbox{$K_{\mathrm{r}} = \num{0,5}$; $T_{\mathrm{i}} = \num{9,25}$; $T_{\mathrm{d}} = \num{2,22}$}}
    \label{zad5_3_K_odp}
\end{figure}

\begin{figure}[b]
    \centering
    \begin{tikzpicture}
    \begin{axis}[
    width=0.98\textwidth,
    xmin=0.0,ymin=0.44, xmax=600, ymax=0.68,
    xlabel={$k$},
    ylabel={$u[k]$},
    %xtick={0.3, 0.4, 0.5, 0.6, 0.7},
    %ytick={3.5, 3.75, 4, 4.25, 4.5},
    legend pos=south east,
    y tick label style={/pgf/number format/1000 sep=},
    ]
    \addplot[const plot, blue, semithick] file{../data/zad5_multiplejumps/P/zad5_PID_input_exampleK_0.5_Ti_9.25_Td_2.22_E_6.5814.csv};
    \legend{$u[k]$}
    \end{axis}
    \end{tikzpicture}
    \caption{Przebieg sygnału sterującego generowanego przez regulator z 
    parametrami \mbox{$K_{\mathrm{r}} = \num{0,5}$; $T_{\mathrm{i}} = \num{9,25}$; $T_{\mathrm{d}} = \num{2,22}$}}
    \label{zad5_3_zmiana_K}
\end{figure}
\FloatBarrier

\subsection{Ręczna poprawa wartości czasu zdwojenia $T_{\mathrm{i}}$ }
\label{zad5_poprawa_Ti}

\subsubsection{Czas zdwojenia regulatora $\mathbf{T_{\mathrm{i}}}=\num{8,25}$}
W pierwszej kolejności postanowiliśmy minimalnie zmniejszyć wartość czasu zdwojenia
aby sprawdzić jak niewielkie zmiany tego parametru wpływają na końcowe przebiegi.
Okazuje się że zmniejszenie parametru $\mathbf{T_{\mathrm{i}}}=\num{8,25}$, bardzo
pozytywnie wpływa na przebiegi sygnałów wejściowego i wyjściowego. Nie występuje przeregulowanie
a czas regulacji został skrócony. Wartość wskaźnika jakości równa jest $E = \num{6,1901}$, co jest poprawą
w stosunku do poprzednio omawianego regulatora.

Przebiegi sygnału wyjściowego z zaznaczoną trajektorią zadaną oraz sygnału wejściowego znajdują się 
odpowiednio na rysunkach \ref{zad5_1_Ti_odp} i \ref{zad5_1_zmiana_Ti}.

\subsubsection{Czas zdwojenia regulatora $\mathbf{T_{\mathrm{i}}}=\num{7,25}$}
Idąc drogą obraną przy pierwszym regulatorze, postanowiliśmy jeszcze bardziej zmniejszyć 
czas zdwojenia $T_{\mathrm{i}}$. W kolejnej iteracji obraliśmy wartość parametru
$T_{\mathrm{i}}=\num{7,25}$. Zastosowanie takiej wartości spowodowało że przebieg 
sygnału wyjściowego jeszcze szybciej zbiegł do wartości zadanej. Większy wpływ członu całkującego
sprawia że regulator agresywniej podąża za trajektorią zadaną co objawia się pojawieniem 
niewielkiego przeregulowania i zwiększoną zrywnością sygnału sterującego. Nasze jakościowe 
rozważania potwierdza wskaźnik jakości regulacji którego wartość dla tego eksperymentu
wyniósł $E=\num{5,8298}$ co jest znacząco lepszym wynikiem niż przy $T_{\mathrm{i}}=\num{8,25}$.

Przebieg wyjścia procesu w porównaniu do wartości zadanej znajduje się na rysunku \ref{zad5_2_Ti_odp}.
Wyjście omawianego regulatora znajduje się na wykresie przedstawionym na rysunku \ref{zad5_2_zmiana_Ti}. 

\subsubsection{Czas zdwojenia regulatora $\mathbf{T_{\mathrm{i}}}=\num{6,25}$}
Kontynuując rozważania na temat czasu zdwojenia, zdecydowaliśmy się na jeszcze większe zmniejszenie 
parametru $T_{\mathrm{i}}$. Tym razem wybraliśmy wartość $T_{\mathrm{i}}=\num{6,25}$. Przebieg 
wyjścia procesu charakteryzuje się delikatnym przeregulowaniem, jednak dzięki temu udało się
jeszcze bardziej zredukować czas regulacji. Sygnał sterujący jest charakteryzuje się
większą agresywnością, co objawia się poprzez duże piki przy zmianie wartości zadanej.
Mimo wizualnie gorzej wyglądającego przebiegu wyjścia, wskaźnik jakości regulacji osiągnął najmniejszą
wartość $E=\num{5,5257}$. Przebieg wyjścia obiektu znajduje się na wykresie \ref{zad5_3_Ti_odp}. 
Przebieg sygnału sterującego znajduje się na rysunku \ref{zad5_3_zmiana_Ti}.

\subsubsection{Czas zdwojenia regulatora $\mathbf{T_{\mathrm{i}}}=\num{5,25}$}
Kolejną próbę wykonaliśmy dla regulatora z czasem zdwojenia $T_{\mathrm{i}}=\num{5,25}$. Pomimo
jeszcze lepszej wartości wskaźnika regulacji $E=\num{5,3567}$, to przebiegi sygnału wyjściowego i wejściowego
powodują że musimy odrzucić ten regulator. Bardzo duża zrywność regulatora powoduje że 
sygnał sterujący nasyca się na górnych ograniczeniach przy większych skokach wartości zadanej. Co więcej,
wyjście obiektu cechuje się zwiększonym przeregulowaniem i porównywalnym czasem regulacji. Omawiane przebiegi
znajdują się na rysunkach \ref{zad5_4_Ti_odp} i \ref{zad5_4_zmiana_Ti}. 

Biorąc pod uwagę powyższe rozważania, zdecydowaliśmy się na pozostanie przy wartości parametru $T_{\mathrm{i}} = \num{6,25}$.
Minimalnie przeregulowanie postaramy się zniwelować za pomocą parametru $T_{\mathrm{d}}$. 

%%%%%%%%%%% Ti = 8.25

\begin{figure}[t]
    \centering
    \begin{tikzpicture}
    \begin{axis}[
    width=0.98\textwidth,
    xmin=0,xmax=600, ymin=3.8, ymax=4.4,
    xlabel={$k$},
    ylabel={$y[k]$},
    %xtick={0.3, 0.4, 0.5, 0.6, 0.7},
    %ytick={3.5, 3.75, 4, 4.25, 4.5},
    legend pos=south east,
    y tick label style={/pgf/number format/1000 sep=},
    ]
    \addplot[blue, semithick] file{../data/zad5_multiplejumps/I/zad5_PID_setpoint_exampleK_0.5_Ti_8.25_Td_2.22_E_6.1901.csv};
    \addplot[red, semithick] file{../data/zad5_multiplejumps/I/zad5_PID_output_exampleK_0.5_Ti_8.25_Td_2.22_E_6.1901.csv};
    \addlegendentry{$y^{zad}[k]$},
    \addlegendentry{$y[k]$},
    \addlegendimage{no markers, blue}
	\addlegendimage{no markers, red}
    \end{axis}
    \end{tikzpicture}
    \caption{Przebieg procesu sterowanego za pomocą regulatora z 
    parametrami \mbox{$K_{\mathrm{r}} = \num{0,5}$; $T_{\mathrm{i}} = \num{8,25}$; $T_{\mathrm{d}} = \num{2,22}$}}
    \label{zad5_1_Ti_odp}
\end{figure}

\begin{figure}[b]
    \centering
    \begin{tikzpicture}
    \begin{axis}[
    width=0.98\textwidth,
    xmin=0.0,ymin=0.44, xmax=600, ymax=0.68,
    xlabel={$k$},
    ylabel={$u[k]$},
    %xtick={0.3, 0.4, 0.5, 0.6, 0.7},
    %ytick={3.5, 3.75, 4, 4.25, 4.5},
    legend pos=south east,
    y tick label style={/pgf/number format/1000 sep=},
    ]
    \addplot[const plot, blue, semithick] file{../data/zad5_multiplejumps/I/zad5_PID_input_exampleK_0.5_Ti_8.25_Td_2.22_E_6.1901.csv};
    \legend{$u[k]$}
    \end{axis}
    \end{tikzpicture}
    \caption{Przebieg sygnału sterującego generowanego przez regulator z 
    parametrami \mbox{$K_{\mathrm{r}} = \num{0,5}$; $T_{\mathrm{i}} = \num{8,25}$; $T_{\mathrm{d}} = \num{2,22}$}}
    \label{zad5_1_zmiana_Ti}
\end{figure}


%%%%%%%%%%% Ti = 7.25


\begin{figure}[t]
    \centering
    \begin{tikzpicture}
    \begin{axis}[
    width=0.98\textwidth,
    xmin=0,xmax=600, ymin=3.8, ymax=4.4,
    xlabel={$k$},
    ylabel={$y[k]$},
    %xtick={0.3, 0.4, 0.5, 0.6, 0.7},
    %ytick={3.5, 3.75, 4, 4.25, 4.5},
    legend pos=south east,
    y tick label style={/pgf/number format/1000 sep=},
    ]
    \addplot[blue, semithick] file{../data/zad5_multiplejumps/I/zad5_PID_setpoint_exampleK_0.5_Ti_7.25_Td_2.22_E_5.8298.csv};
    \addplot[red, semithick] file{../data/zad5_multiplejumps/I/zad5_PID_output_exampleK_0.5_Ti_7.25_Td_2.22_E_5.8298.csv};
    \addlegendentry{$y^{zad}[k]$},
    \addlegendentry{$y[k]$},
    \addlegendimage{no markers, blue}
	\addlegendimage{no markers, red}
    \end{axis}
    \end{tikzpicture}
    \caption{Przebieg procesu sterowanego za pomocą regulatora z 
    parametrami \mbox{$K_{\mathrm{r}} = \num{0,5}$; $T_{\mathrm{i}} = \num{7,25}$; $T_{\mathrm{d}} = \num{2,22}$}}
    \label{zad5_2_Ti_odp}
\end{figure}

\begin{figure}[b]
    \centering
    \begin{tikzpicture}
    \begin{axis}[
    width=0.98\textwidth,
    xmin=0.0,ymin=0.44, xmax=600, ymax=0.68,
    xlabel={$k$},
    ylabel={$u[k]$},
    %xtick={0.3, 0.4, 0.5, 0.6, 0.7},
    %ytick={3.5, 3.75, 4, 4.25, 4.5},
    legend pos=south east,
    y tick label style={/pgf/number format/1000 sep=},
    ]
    \addplot[const plot, blue, semithick] file{../data/zad5_multiplejumps/I/zad5_PID_input_exampleK_0.5_Ti_7.25_Td_2.22_E_5.8298.csv};
    \legend{$u[k]$}
    \end{axis}
    \end{tikzpicture}
    \caption{Przebieg sygnału sterującego generowanego przez regulator z 
    parametrami \mbox{$K_{\mathrm{r}} = \num{0,5}$; $T_{\mathrm{i}} = \num{7,25}$; $T_{\mathrm{d}} = \num{2,22}$}}
    \label{zad5_2_zmiana_Ti}
\end{figure}
\FloatBarrier


%%%%%%%%%%%% Ti = 6.25


\begin{figure}[t]
    \centering
    \begin{tikzpicture}
    \begin{axis}[
    width=0.98\textwidth,
    xmin=0,xmax=600, ymin=3.8, ymax=4.4,
    xlabel={$k$},
    ylabel={$y[k]$},
    %xtick={0.3, 0.4, 0.5, 0.6, 0.7},
    %ytick={3.5, 3.75, 4, 4.25, 4.5},
    legend pos=south east,
    y tick label style={/pgf/number format/1000 sep=},
    ]
    \addplot[blue, semithick] file{../data/zad5_multiplejumps/I/zad5_PID_setpoint_exampleK_0.5_Ti_6.25_Td_2.22_E_5.5257.csv};
    \addplot[red, semithick] file{../data/zad5_multiplejumps/I/zad5_PID_output_exampleK_0.5_Ti_6.25_Td_2.22_E_5.5257.csv};
    \addlegendentry{$y^{zad}[k]$},
    \addlegendentry{$y[k]$},
    \addlegendimage{no markers, blue}
	\addlegendimage{no markers, red}
    \end{axis}
    \end{tikzpicture}
    \caption{Przebieg procesu sterowanego za pomocą regulatora z 
    parametrami \mbox{$K_{\mathrm{r}} = \num{0,5}$; $T_{\mathrm{i}} = \num{6,25}$; $T_{\mathrm{d}} = \num{2,22}$}}
    \label{zad5_3_Ti_odp}
\end{figure}

\begin{figure}[b]
    \centering
    \begin{tikzpicture}
    \begin{axis}[
    width=0.98\textwidth,
    xmin=0.0,ymin=0.42, xmax=600, ymax=0.7,
    xlabel={$k$},
    ylabel={$u[k]$},
    %xtick={0.3, 0.4, 0.5, 0.6, 0.7},
    %ytick={3.5, 3.75, 4, 4.25, 4.5},
    legend pos=south east,
    y tick label style={/pgf/number format/1000 sep=},
    ]
    \addplot[const plot, blue, semithick] file{../data/zad5_multiplejumps/I/zad5_PID_input_exampleK_0.5_Ti_6.25_Td_2.22_E_5.5257.csv};
    \legend{$u[k]$}
    \end{axis}
    \end{tikzpicture}
    \caption{Przebieg sygnału sterującego generowanego przez regulator z 
    parametrami \mbox{$K_{\mathrm{r}} = \num{0,5}$; $T_{\mathrm{i}} = \num{6,25}$; $T_{\mathrm{d}} = \num{2,22}$}}
    \label{zad5_3_zmiana_Ti}
\end{figure}
\FloatBarrier

%%%%%%%%%%%%% Ti = 5.25


\begin{figure}[t]
    \centering
    \begin{tikzpicture}
    \begin{axis}[
    width=0.98\textwidth,
    xmin=0,xmax=600, ymin=3.8, ymax=4.4,
    xlabel={$k$},
    ylabel={$y[k]$},
    %xtick={0.3, 0.4, 0.5, 0.6, 0.7},
    %ytick={3.5, 3.75, 4, 4.25, 4.5},
    legend pos=south east,
    y tick label style={/pgf/number format/1000 sep=},
    ]
    \addplot[blue, semithick] file{../data/zad5_multiplejumps/I/zad5_PID_setpoint_exampleK_0.5_Ti_5.25_Td_2.22_E_5.3567.csv};
    \addplot[red, semithick] file{../data/zad5_multiplejumps/I/zad5_PID_output_exampleK_0.5_Ti_5.25_Td_2.22_E_5.3567.csv};
    \addlegendentry{$y^{zad}[k]$},
    \addlegendentry{$y[k]$},
    \addlegendimage{no markers, blue}
	\addlegendimage{no markers, red}
    \end{axis}
    \end{tikzpicture}
    \caption{Przebieg procesu sterowanego za pomocą regulatora z 
    parametrami \mbox{$K_{\mathrm{r}} = \num{0,5}$; $T_{\mathrm{i}} = \num{5,25}$; $T_{\mathrm{d}} = \num{2,22}$}}
    \label{zad5_4_Ti_odp}
\end{figure}

\begin{figure}[b]
    \centering
    \begin{tikzpicture}
    \begin{axis}[
    width=0.98\textwidth,
    xmin=0.0,ymin=0.38, xmax=600, ymax=0.72,
    xlabel={$k$},
    ylabel={$u[k]$},
    %xtick={0.3, 0.4, 0.5, 0.6, 0.7},
    %ytick={3.5, 3.75, 4, 4.25, 4.5},
    legend pos=south east,
    y tick label style={/pgf/number format/1000 sep=},
    ]
    \addplot[const plot, blue, semithick] file{../data/zad5_multiplejumps/I/zad5_PID_input_exampleK_0.5_Ti_5.25_Td_2.22_E_5.3567.csv};
    \legend{$u[k]$}
    \end{axis}
    \end{tikzpicture}
    \caption{Przebieg sygnału sterującego generowanego przez regulator z 
    parametrami \mbox{$K_{\mathrm{r}} = \num{0,5}$; $T_{\mathrm{i}} = \num{5,25}$; $T_{\mathrm{d}} = \num{2,22}$}}
    \label{zad5_4_zmiana_Ti}
\end{figure}
\FloatBarrier


\iffalse
\indent 3.
\subsection{Ręczna poprawa wartości wzmocnienia członu całkującego $T_{\mathrm{d}}$ }
\indent 1. Człon całkujący zminiejszyliśmy o 1 
\begin{figure}[b]
    \centering
    \begin{tikzpicture}
    \begin{axis}[d
    width=0.98\textwidth,
    xmin=0,xmax=500,ymin=3.9, ymax=4.3,
    xlabel={$k$},
    ylabel={$y[k]$},
    %xtick={0.3, 0.4, 0.5, 0.6, 0.7},
    %ytick={3.5, 3.75, 4, 4.25, 4.5},
    legend pos=south east,
    y tick label style={/pgf/number format/1000 sep=},
    ]
    \addplot[blue, semithick] file{../data/Zad5_figure_data/Zmiany_na_Td/zad5_PID_setpoint_exampleK_0.5_Ti_7.25_Td_1.86_E_1.1087.csv};
    \addplot[red, semithick] file{../data/Zad5_figure_data/Zmiany_na_Td/zad5_PID_output_exampleK_0.5_Ti_7.25_Td_1.86_E_1.1087.csv};
    \addlegendentry{$y^{zad}[k]$},
    \addlegendentry{$y[k]$},
    \addlegendimage{no markers, blue}
	\addlegendimage{no markers, red}
    \end{axis}
    \end{tikzpicture}
    \caption{Odpowiedź układu na skok wartości zadanej}
    \label{zad5_1_Td_odp}
\end{figure}

\begin{figure}[t]
    \centering
    \begin{tikzpicture}
    \begin{axis}[
    width=0.98\textwidth,
    xmin=0.0,ymin=0.45, xmax= 500,
    xlabel={$k$},
    ylabel={$u[k]$},
    %xtick={0.3, 0.4, 0.5, 0.6, 0.7},
    %ytick={3.5, 3.75, 4, 4.25, 4.5},
    legend pos=south east,
    y tick label style={/pgf/number format/1000 sep=},
    ]
    \addplot[const plot, blue, semithick] file{../data/Zad5_figure_data/Zmiany_na_Td/zad5_PID_input_exampleK_0.5_Ti_7.25_Td_1.86_E_1.1087.csv};
    \legend{$u[k]$}
    \end{axis}
    \end{tikzpicture}
    \caption{Przebieg sygnału sterującego regulatora}
    \label{zad5_1_zmiana_Td}
\end{figure}
\FloatBarrier

\indent 2.
\begin{figure}[b]
    \centering
    \begin{tikzpicture}
    \begin{axis}[
    width=0.98\textwidth,
    xmin=0,xmax=500,ymin=3.9, ymax=4.3,
    xlabel={$k$},
    ylabel={$y[k]$},
    %xtick={0.3, 0.4, 0.5, 0.6, 0.7},
    %ytick={3.5, 3.75, 4, 4.25, 4.5},
    legend pos=south east,
    y tick label style={/pgf/number format/1000 sep=},
    ]
    \addplot[blue, semithick] file{../data/Zad5_figure_data/Zmiany_na_Td/zad5_PID_setpoint_exampleK_0.5_Ti_7.25_Td_1.62_E_1.1056.csv};
    \addplot[red, semithick] file{../data/Zad5_figure_data/Zmiany_na_Td/zad5_PID_output_exampleK_0.5_Ti_7.25_Td_1.62_E_1.1056.csv};
    \addlegendentry{$y^{zad}[k]$},
    \addlegendentry{$y[k]$},
    \addlegendimage{no markers, blue}
	\addlegendimage{no markers, red}
    \end{axis}
    \end{tikzpicture}
    \caption{Odpowiedź układu na skok wartości zadanej}
    \label{zad5_2_Td_odp}
\end{figure}

\begin{figure}[t]
    \centering
    \begin{tikzpicture}
    \begin{axis}[
    width=0.98\textwidth,
    xmin=0.0,ymin=0.45, xmax= 500,
    xlabel={$k$},
    ylabel={$u[k]$},
    %xtick={0.3, 0.4, 0.5, 0.6, 0.7},
    %ytick={3.5, 3.75, 4, 4.25, 4.5},
    legend pos=south east,
    y tick label style={/pgf/number format/1000 sep=},
    ]
    \addplot[const plot, blue, semithick] file{../data/Zad5_figure_data/Zmiany_na_Td/zad5_PID_input_exampleK_0.5_Ti_7.25_Td_1.62_E_1.1056.csv};
    \legend{$u[k]$}
    \end{axis}
    \end{tikzpicture}
    \caption{Przebieg sygnału sterującego regulatora}
    \label{zad5_2_zmiana_Td}
\end{figure}
\FloatBarrier

\indent 3.
\begin{figure}[b]
    \centering
    \begin{tikzpicture}
    \begin{axis}[
    width=0.98\textwidth,
    xmin=0,xmax=500,ymin=3.9, ymax=4.3,
    xlabel={$k$},
    ylabel={$y[k]$},
    %xtick={0.3, 0.4, 0.5, 0.6, 0.7},
    %ytick={3.5, 3.75, 4, 4.25, 4.5},
    legend pos=south east,
    y tick label style={/pgf/number format/1000 sep=},
    ]
    \addplot[blue, semithick] file{../data/Zad5_figure_data/Zmiany_na_Td/zad5_PID_setpoint_exampleK_0.5_Ti_7.25_Td_1.26_E_1.1015.csv};
    \addplot[red, semithick] file{../data/Zad5_figure_data/Zmiany_na_Td/zad5_PID_output_exampleK_0.5_Ti_7.25_Td_1.26_E_1.1015.csv};
    \addlegendentry{$y^{zad}[k]$},
    \addlegendentry{$y[k]$},
    \addlegendimage{no markers, blue}
	\addlegendimage{no markers, red}
    \end{axis}
    \end{tikzpicture}
    \caption{Odpowiedź układu na skok wartości zadanej}
    \label{zad5_3_Td_odp}
\end{figure}

\begin{figure}[t]
    \centering
    \begin{tikzpicture}
    \begin{axis}[
    width=0.98\textwidth,
    xmin=0.0,ymin=0.45, xmax= 500,
    xlabel={$k$},
    ylabel={$u[k]$},
    %xtick={0.3, 0.4, 0.5, 0.6, 0.7},
    %ytick={3.5, 3.75, 4, 4.25, 4.5},
    legend pos=south east,
    y tick label style={/pgf/number format/1000 sep=},
    ]
    \addplot[const plot, blue, semithick] file{../data/Zad5_figure_data/Zmiany_na_Td/zad5_PID_input_exampleK_0.5_Ti_7.25_Td_1.26_E_1.1015.csv};
    \legend{$u[k]$}
    \end{axis}
    \end{tikzpicture}
    \caption{Przebieg sygnału sterującego regulatora}
    \label{zad5_3_zmiana_Td}
\end{figure}
\FloatBarrier
\fi