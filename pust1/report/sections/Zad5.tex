\chapter{Dobieranie nastawów regulatorów PID}
\label{zad5}

\section{Strojenie regulatora PID}
\label{zad5_PID_wykresy}
W celu dobrania optymalnych parametrów regulatora, wyszliśmy od parametrów uzyskanych metodą
Zieglera-Nicholsa a następnie ręcznie dostroiliśmy poszczególne parametry tak aby uzyskać
jak najlepszy efekty końcowy. Eksperyment ten polegał na znalezieniu takiego wzmocnienia $K_{r}$, 
zwanego wzmocnieniem krytycznym, dla którego obiekt wpadał w niegasnące oscylacje o okresie $T_{k}$. 

Cały proces dobierania parametrów składał się z następujących kroków:
\begin{center}
\begin{enumerate}
    \item Wyznaczenie wzmocnienia $K_{\mathrm{k}}$ i okresu oscylacji $T_{\mathrm{k}}$
    \item Wyliczenie parametrów regulatora
    \item Ręczna poprawa wartości wzmocnienia członu proporcjonalnego $K_{\mathrm{r}}$
    \item Ręczna poprawa wartości parametru członu całkującego $T_{\mathrm{i}}$
    \item Ręczna poprawa wartości parametru członu różniczkującego $T_{\mathrm{d}}$
\end{enumerate}
\end{center}

Chcąc znaleźć jeszcze lepsze nastawy regulatora, po wyznaczeniu parametrów za pomocą metody Zieglera-Nicholsa, zastosowaliśmy metodę eksperymentalną. W której jakość regulacji ocenialiśmy jakościowo (na podstawie przebiegów sygnałów na wykresach) oraz ilościowo, za pomocą wskaźnika jakości regulacji danego wzorem:
\begin{equation}
E=\sum_{k-1}^{K_{konc}}(y^{zad}[k]-y[k])^{2}
\end{equation}
\subsection{Wyznaczenie wzmocnienia $K_{\mathrm{k}}$ i okresu oscylacji $T_{\mathrm{k}}$}
Na początku doprowadziliśmy układ do granicy stabilności, czyli momentu
w którym zaczęły pojawiać się niegasnące oscylacje. Zrealizowaliśmy to
zwiększając współczynnik wzmocnienia $K_{\mathrm{r}}$ przy wyłączonych członach 
całkującym i różniczkującym. Iteracyjnie wyznaczaliśmy nowe wartości wzmocnienia,
zwiększając jego wartość w przypadku gasnących oscylacji oraz zmniejszając w przypadku
niestabilnej odpowiedzi obiektu.

Gdy ostatecznie zaobserwowaliśmy niegasnące oscylacje zapisaliśmy aktualne wzmocnienie 
jako wzmocnienie krytyczne $K_{k}$ i zmierzyliśmy okres oscylacji drgań 
krytycznych $T_{k}$ ($K_{k} = 1,8 \hspace{0.3 cm} T_{k}=18,5 $). Wiedząc że okres próbkowania procesu 
wynosi $\num{0,5}s$, a zmierzony okres oscylacji wynosił $\num{37}$ próbek, wyliczyliśmy że
okres drgań krytycznych równa się $T_{k}=18,5s$. 

\begin{figure}[t]
    \centering
    \begin{tikzpicture}
    \begin{axis}[
    width=0.98\textwidth,
    xmin=0,xmax=500,ymin=3.7, ymax=4.3,
    xlabel={$k$},
    ylabel={$y[k]$},
    %xtick={0, 100, 200, 300, 400, 500},
    %ytick={3.9, 4, 4.1, 4.2, 4.3, 4.4},
    legend pos=south east,
    y tick label style={/pgf/number format/1000 sep=},
    ]
    \addplot[blue, semithick] file{../data/Zad5_figure_data/zad5_PID_setpoint_exampleK_1.8_Ti_1000000000000_Td_0.csv};
    \addplot[red, semithick] file{../data/Zad5_figure_data/zad5_PID_output_exampleK_1.8_Ti_1000000000000_Td_0.csv};
    \addlegendentry{$y^{zad}[k]$},
    \addlegendentry{$y[k]$},
    \addlegendimage{no markers, blue}
	\addlegendimage{no markers, red}
    \end{axis}
    \end{tikzpicture}
    \caption{Niegasnące oscylacje wyjścia obiektu przy wzmocnieniu krytycznym}
    \label{zad5_niegasnace_oscylacje}
\end{figure}

\begin{figure}[b]
    \centering
    \begin{tikzpicture}
    \begin{axis}[
    width=0.98\textwidth,
    xmin=0,xmax=500,ymin=0.25, ymax=0.75,
    xlabel={$k$},
    ylabel={$u[k]$},
    %xtick={0, 0.4, 0.5, 0.6, 0.7},
    %ytick={3.5, 3.75, 4, 4.25, 4.5},
    legend pos=south east,
    y tick label style={/pgf/number format/1000 sep=},
    ]
    \addplot[const plot, blue, semithick] file{../data/Zad5_figure_data/zad5_PID_input_exampleK_1.8_Ti_1000000000000_Td_0.csv};
    \legend{$u[k]$}
    \end{axis}
    \end{tikzpicture}
    \caption{Przebieg sygnału sterującego }
    \label{zad5_niegasnace_oscylacje_ster}
\end{figure}
\FloatBarrier

\subsection{Wyliczenie parametrów regulatora}
Dobierając nastawy według metody Zieglera-Nicholsa, 
stosowaliśmy wzory do obliczenia parametrów regulatora PID: 

\begin{equation}
\label{zad5_zn_wzory}
K_{\mathrm{r}}=\num{0,6}K_{\mathrm{k}}; \hspace{0.5 cm} T_{\mathrm{i}}=\num{0,5} \hspace{0.5 cm} T_{\mathrm{k}}; T_{\mathrm{d}}=\num{0,12} T_{\mathrm{k}}
\end{equation}
gdzie:
\begin{center}
\begin{itemize}
    \item $K_{\mathrm{k}}$ - wzmocnienie krytyczne
    \item $T_{\mathrm{k}}$ - okres oscylacji
\end{itemize}
\end{center}

Wyliczone wartości liczbowe parametrów:
\begin{center}
$K_{r}=\num{1,08} \hspace{0.5 cm}
T_{i}=\num{9,25}   \hspace{0.5 cm}
T_{d}=\num{2,22}$
\end{center}

Regulator z parametrami wyznaczonymi przy pomocy metody opisanej przez panów Johna Zieglera i Nathaniela Nicholsa, nadal charakteryzuje się niegasnącymi oscylacjami. Oczywiście takie przebiegi sygnałów nie są dla nas satysfakcjonujące, z tego powodu użyliśmy metody eksperymentalnej w następnych krokach. Polegała ona na zmianie poszczególnych parametrów i krytycznej ocenie jakości regulacji po takich zmianach. 

~\\Na rysunku \ref{zad5_zn_regulator} przedstawiono przebiegu wartości zadanej oraz
wyjścia obiektu z widocznie przedstawionymi oscylacjami wokół wartości zadanej. Rysunek \ref{zad5_zn_sterowanie} przedstawia zmianę sygnału sterującego
generowanego przez regulator.
\begin{figure}[b]
    \centering
    \begin{tikzpicture}
    \begin{axis}[
    width=0.98\textwidth,
    xmin=0,xmax=500,ymin=3.9, ymax=4.4,
    xlabel={$k$},
    ylabel={$y[k]$},
    %xtick={0.3, 0.4, 0.5, 0.6, 0.7},
    %ytick={3.5, 3.75, 4, 4.25, 4.5},
    legend pos=south east,
    y tick label style={/pgf/number format/1000 sep=},
    ]
    \addplot[blue, semithick] file{../data/Zad5_figure_data/zad5_PID_setpoint_exampleK_1.08_Ti_9.25_Td_2.22.csv};
    \addplot[red, semithick] file{../data/Zad5_figure_data/zad5_PID_output_exampleK_1.08_Ti_9.25_Td_2.22.csv};
    \addlegendentry{$y^{zad}[k]$},
    \addlegendentry{$y[k]$},
    \addlegendimage{no markers, blue}
	\addlegendimage{no markers, red}
    \end{axis}
    \end{tikzpicture}
    \caption{Przebieg procesu sterowanego za pomocą regulatora regulatora z parametrami wyznaczonymi za pomocą metody Zieglera-Nicholsa}
    \label{zad5_zn_regulator}
\end{figure}

\begin{figure}[t]
    \centering
    \begin{tikzpicture}
    \begin{axis}[
    width=0.98\textwidth,
    xmin=0.0,ymin=0.45, xmax= 500,
    xlabel={$k$},
    ylabel={$u[k]$},
    %xtick={0.3, 0.4, 0.5, 0.6, 0.7},
    %ytick={3.5, 3.75, 4, 4.25, 4.5},
    legend pos=south east,
    y tick label style={/pgf/number format/1000 sep=},
    ]
    \addplot[const plot, blue, semithick] file{../data/Zad5_figure_data/zad5_PID_input_exampleK_1.08_Ti_9.25_Td_2.22.csv};
    \legend{$u[k]$}
    \end{axis}
    \end{tikzpicture}
    \caption{Przebieg sygnału sterującego regulatora z parametrami wyznaczonymi za pomocą metody Zieglera-Nicholsa}
    \label{zad5_zn_sterowanie}
\end{figure}
\FloatBarrier

\subsection{Ręczna poprawa wartości wzmocnienia członu proporcjonalnego $K_{\mathrm{r}}$ }
\indent 1. Pierwszą naszą czynnością było zmniejszenie $K_{\mathrm{r}}$ do wartości $\num{0.75}$. Dzięki temu pozbyliśmy się oscylacji. Wartość wskaźnika jakości regulacji dla tego przypadku wyniosła $E=\num{1.088}$, była to jedna z niższych wartości jakie udało nam się uzyskać, jednak przebiegi odpowiedzi skokowej jaki i sygnału sterującego odbiegały nadal od takich które charakteryzują dobre układy regulacji. Pojawiają się krótkie gasnące oscylacje na wykresie \ref{zad5_1_zmiana_K}, a wartość zadania jest osiągana krótkimi skokami zaprezentowanymi na \ref{zad5_1_K_odp}. Szukając lepszych nastawów dalej zmniejszaliśmy wartość wzmocnienia członu proporcjonalnego.
\begin{figure}[b]
    \centering
    \begin{tikzpicture}
    \begin{axis}[
    width=0.98\textwidth,
    xmin=0,xmax=500,ymin=3.9, ymax=4.3,
    xlabel={$k$},
    ylabel={$y[k]$},
    %xtick={0.3, 0.4, 0.5, 0.6, 0.7},
    %ytick={3.5, 3.75, 4, 4.25, 4.5},
    legend pos=south east,
    y tick label style={/pgf/number format/1000 sep=},
    ]
    \addplot[blue, semithick] file{../data/Zad5_figure_data/Zmiany_na_K/zad5_PID_setpoint_exampleK_0.75_Ti_9.25_Td_2.22_E_1.088.csv};
    \addplot[red, semithick] file{../data/Zad5_figure_data/Zmiany_na_K/zad5_PID_output_exampleK_0.75_Ti_9.25_Td_2.22_E_1.088.csv};
    \addlegendentry{$y^{zad}[k]$},
    \addlegendentry{$y[k]$},
    \addlegendimage{no markers, blue}
	\addlegendimage{no markers, red}
    \end{axis}
    \end{tikzpicture}
    \caption{Odpowiedź układu na skok wartości zadanej}
    \label{zad5_1_K_odp}
\end{figure}

\begin{figure}[t]
    \centering
    \begin{tikzpicture}
    \begin{axis}[
    width=0.98\textwidth,
    xmin=0.0,ymin=0.45, xmax= 500,
    xlabel={$k$},
    ylabel={$u[k]$},
    %xtick={0.3, 0.4, 0.5, 0.6, 0.7},
    %ytick={3.5, 3.75, 4, 4.25, 4.5},
    legend pos=south east,
    y tick label style={/pgf/number format/1000 sep=},
    ]
    \addplot[const plot, blue, semithick] file{../data/Zad5_figure_data/Zmiany_na_K/zad5_PID_input_exampleK_0.75_Ti_9.25_Td_2.22_E_1.088.csv};
    \legend{$u[k]$}
    \end{axis}
    \end{tikzpicture}
    \caption{Przebieg sygnału sterującego regulatora}
    \label{zad5_1_zmiana_K}
\end{figure}
\FloatBarrier

\indent 2. Zmniejszenie $K_{\mathrm{r}}$ do poziomu $\num{0,6}$ spowodowało poprawę odpowiedzi układu na skok wartości zadanej, ale przebieg sygnału sterującego ciągle odbiegał od oczekiwanego (nadal widoczne były krótkie oscylacje). Wartość wskaźnika jakości regulacji dla tego przypadku wyniosła $E=\num{1.175}$, jest to wartość większa w porównaniu do poprzedniego $K_{\mathrm{r}}$, jednak opłacało się ją trochę zwiększyć bo uzyskaliśmy dużo lepszy przebieg sygnału na wykresie \ref{zad5_2_K_odp}.

\begin{figure}[b]
    \centering
    \begin{tikzpicture}
    \begin{axis}[
    width=0.98\textwidth,
    xmin=0,xmax=500,ymin=3.9, ymax=4.3,
    xlabel={$k$},
    ylabel={$y[k]$},
    %xtick={0.3, 0.4, 0.5, 0.6, 0.7},
    %ytick={3.5, 3.75, 4, 4.25, 4.5},
    legend pos=south east,
    y tick label style={/pgf/number format/1000 sep=},
    ]
    \addplot[blue, semithick] file{../data/Zad5_figure_data/Zmiany_na_K/zad5_PID_setpoint_exampleK_0.6_Ti_9.25_Td_2.22_E_1.175.csv};
    \addplot[red, semithick] file{../data/Zad5_figure_data/Zmiany_na_K/zad5_PID_output_exampleK_0.6_Ti_9.25_Td_2.22_E_1.175.csv};
    \addlegendentry{$y^{zad}[k]$},
    \addlegendentry{$y[k]$},
    \addlegendimage{no markers, blue}
	\addlegendimage{no markers, red}
    \end{axis}
    \end{tikzpicture}
    \caption{Odpowiedź układu na skok wartości zadanej}
    \label{zad5_2_K_odp}
\end{figure}

\begin{figure}[t]
    \centering
    \begin{tikzpicture}
    \begin{axis}[
    width=0.98\textwidth,
    xmin=0.0,ymin=0.45, xmax= 500,
    xlabel={$k$},
    ylabel={$u[k]$},
    %xtick={0.3, 0.4, 0.5, 0.6, 0.7},
    %ytick={3.5, 3.75, 4, 4.25, 4.5},
    legend pos=south east,
    y tick label style={/pgf/number format/1000 sep=},
    ]
    \addplot[const plot, blue, semithick] file{../data/Zad5_figure_data/Zmiany_na_K/zad5_PID_input_exampleK_0.6_Ti_9.25_Td_2.22_E_1.175.csv};
    \legend{$u[k]$}
    \end{axis}
    \end{tikzpicture}
    \caption{Przebieg sygnału sterującego regulatora}
    \label{zad5_1_zmiana_K}
\end{figure}
\FloatBarrier

\indent 3. Prezentowane poniżej przebiegi powstały dla $ K_{\mathrm{r}} = \num{0,5}$. Nastąpiła wyrźna poprawa sygnału sterującego (nie ma już oscylacji), został niestety spory skok tego sygnału na początku symulacji, jednak go postaramy się wyeliminować za pomocą zmiany parametrów $T_{\mathrm{i}}$ i $T_{\mathrm{d}}$. Niestety znowu wzrósł nam znacznie wskaźnika jakości regulacji do wartości $E=\num{1.2857}$. Możemy zauważyć, że jest to skolerowane z zmniejszaniem parametru $ K_{\mathrm{r}}$, dlatego postanowiliśmy ostatecznie przyjąć wartość $ K_{\mathrm{r}} = \num{0,5}$, ponieważ dalsze zmniejszanie nie spowoduje znaczącej poprawy przebiegów sygnału w porównaniu do tych którymi teraz dysponujemy, a pogorszy jedynie wskaźnik jakości regulacji. 
\begin{figure}[b]
    \centering
    \begin{tikzpicture}
    \begin{axis}[
    width=0.98\textwidth,
    xmin=0,xmax=500,ymin=3.9, ymax=4.3,
    xlabel={$k$},
    ylabel={$y[k]$},
    %xtick={0.3, 0.4, 0.5, 0.6, 0.7},
    %ytick={3.5, 3.75, 4, 4.25, 4.5},
    legend pos=south east,
    y tick label style={/pgf/number format/1000 sep=},
    ]
    \addplot[blue, semithick] file{../data/Zad5_figure_data/Zmiany_na_K/zad5_PID_setpoint_exampleK_0.5_Ti_9.25_Td_2.22_E_1.2587.csv};
    \addplot[red, semithick] file{../data/Zad5_figure_data/Zmiany_na_K/zad5_PID_output_exampleK_0.5_Ti_9.25_Td_2.22_E_1.2587.csv};
    \addlegendentry{$y^{zad}[k]$},
    \addlegendentry{$y[k]$},
    \addlegendimage{no markers, blue}
	\addlegendimage{no markers, red}
    \end{axis}
    \end{tikzpicture}
    \caption{Odpowiedź układu na skok wartości zadanej}
    \label{zad5_1_K_odp}
\end{figure}

\begin{figure}[t]
    \centering
    \begin{tikzpicture}
    \begin{axis}[
    width=0.98\textwidth,
    xmin=0.0,ymin=0.45, xmax= 500,
    xlabel={$k$},
    ylabel={$u[k]$},
    %xtick={0.3, 0.4, 0.5, 0.6, 0.7},
    %ytick={3.5, 3.75, 4, 4.25, 4.5},
    legend pos=south east,
    y tick label style={/pgf/number format/1000 sep=},
    ]
    \addplot[const plot, blue, semithick] file{../data/Zad5_figure_data/Zmiany_na_K/zad5_PID_input_exampleK_0.5_Ti_9.25_Td_2.22_E_1.2587.csv};
    \legend{$u[k]$}
    \end{axis}
    \end{tikzpicture}
    \caption{Przebieg sygnału sterującego regulatora}
    \label{zad5_1_zmiana_K}
\end{figure}
\FloatBarrier

\subsection{Ręczna poprawa wartości wzmocnienia członu całkującego $T_{\mathrm{i}}$ }
\indent 1. Człon całkujący zminiejszyliśmy o 1 
\begin{figure}[b]
    \centering
    \begin{tikzpicture}
    \begin{axis}[
    width=0.98\textwidth,
    xmin=0,xmax=500,ymin=3.9, ymax=4.3,
    xlabel={$k$},
    ylabel={$y[k]$},
    %xtick={0.3, 0.4, 0.5, 0.6, 0.7},
    %ytick={3.5, 3.75, 4, 4.25, 4.5},
    legend pos=south east,
    y tick label style={/pgf/number format/1000 sep=},
    ]
    \addplot[blue, semithick] file{../data/Zad5_figure_data/Zmiany_na_Ti/zad5_PID_setpoint_exampleK_0.5_Ti_8.25_Td_2.22_E_1.1833.csv};
    \addplot[red, semithick] file{../data/Zad5_figure_data/Zmiany_na_Ti/zad5_PID_output_exampleK_0.5_Ti_8.25_Td_2.22_E_1.1833.csv};
    \addlegendentry{$y^{zad}[k]$},
    \addlegendentry{$y[k]$},
    \addlegendimage{no markers, blue}
	\addlegendimage{no markers, red}
    \end{axis}
    \end{tikzpicture}
    \caption{Odpowiedź układu na skok wartości zadanej}
    \label{zad5_1_K_odp}
\end{figure}

\begin{figure}[t]
    \centering
    \begin{tikzpicture}
    \begin{axis}[
    width=0.98\textwidth,
    xmin=0.0,ymin=0.45, xmax= 500,
    xlabel={$k$},
    ylabel={$u[k]$},
    %xtick={0.3, 0.4, 0.5, 0.6, 0.7},
    %ytick={3.5, 3.75, 4, 4.25, 4.5},
    legend pos=south east,
    y tick label style={/pgf/number format/1000 sep=},
    ]
    \addplot[const plot, blue, semithick] file{../data/Zad5_figure_data/Zmiany_na_Ti/zad5_PID_input_exampleK_0.5_Ti_8.25_Td_2.22_E_1.1833.csv};
    \legend{$u[k]$}
    \end{axis}
    \end{tikzpicture}
    \caption{Przebieg sygnału sterującego regulatora}
    \label{zad5_1_zmiana_K}
\end{figure}
\FloatBarrier

\indent 2.
\begin{figure}[b]
    \centering
    \begin{tikzpicture}
    \begin{axis}[
    width=0.98\textwidth,
    xmin=0,xmax=500,ymin=3.9, ymax=4.3,
    xlabel={$k$},
    ylabel={$y[k]$},
    %xtick={0.3, 0.4, 0.5, 0.6, 0.7},
    %ytick={3.5, 3.75, 4, 4.25, 4.5},
    legend pos=south east,
    y tick label style={/pgf/number format/1000 sep=},
    ]
    \addplot[blue, semithick] file{../data/Zad5_figure_data/Zmiany_na_Ti/zad5_PID_setpoint_exampleK_0.5_Ti_7.25_Td_2.22_E_1.1138.csv};
    \addplot[red, semithick] file{../data/Zad5_figure_data/Zmiany_na_Ti/zad5_PID_output_exampleK_0.5_Ti_7.25_Td_2.22_E_1.1138.csv};
    \addlegendentry{$y^{zad}[k]$},
    \addlegendentry{$y[k]$},
    \addlegendimage{no markers, blue}
	\addlegendimage{no markers, red}
    \end{axis}
    \end{tikzpicture}
    \caption{Odpowiedź układu na skok wartości zadanej}
    \label{zad5_1_K_odp}
\end{figure}

\begin{figure}[t]
    \centering
    \begin{tikzpicture}
    \begin{axis}[
    width=0.98\textwidth,
    xmin=0.0,ymin=0.45, xmax= 500,
    xlabel={$k$},
    ylabel={$u[k]$},
    %xtick={0.3, 0.4, 0.5, 0.6, 0.7},
    %ytick={3.5, 3.75, 4, 4.25, 4.5},
    legend pos=south east,
    y tick label style={/pgf/number format/1000 sep=},
    ]
    \addplot[const plot, blue, semithick] file{../data/Zad5_figure_data/Zmiany_na_Ti/zad5_PID_input_exampleK_0.5_Ti_7.25_Td_2.22_E_1.1138.csv};
    \legend{$u[k]$}
    \end{axis}
    \end{tikzpicture}
    \caption{Przebieg sygnału sterującego regulatora}
    \label{zad5_1_zmiana_K}
\end{figure}
\FloatBarrier

\indent 3.
\begin{figure}[b]
    \centering
    \begin{tikzpicture}
    \begin{axis}[
    width=0.98\textwidth,
    xmin=0,xmax=500,ymin=3.9, ymax=4.3,
    xlabel={$k$},
    ylabel={$y[k]$},
    %xtick={0.3, 0.4, 0.5, 0.6, 0.7},
    %ytick={3.5, 3.75, 4, 4.25, 4.5},
    legend pos=south east,
    y tick label style={/pgf/number format/1000 sep=},
    ]
    \addplot[blue, semithick] file{../data/Zad5_figure_data/Zmiany_na_Ti/zad5_PID_setpoint_exampleK_0.5_Ti_6.25_Td_2.22_E_1.0548.csv};
    \addplot[red, semithick] file{../data/Zad5_figure_data/Zmiany_na_Ti/zad5_PID_output_exampleK_0.5_Ti_6.25_Td_2.22_E_1.0548.csv};
    \addlegendentry{$y^{zad}[k]$},
    \addlegendentry{$y[k]$},
    \addlegendimage{no markers, blue}
	\addlegendimage{no markers, red}
    \end{axis}
    \end{tikzpicture}
    \caption{Odpowiedź układu na skok wartości zadanej}
    \label{zad5_1_K_odp}
\end{figure}

\begin{figure}[t]
    \centering
    \begin{tikzpicture}
    \begin{axis}[
    width=0.98\textwidth,
    xmin=0.0,ymin=0.45, xmax= 500,
    xlabel={$k$},
    ylabel={$u[k]$},
    %xtick={0.3, 0.4, 0.5, 0.6, 0.7},
    %ytick={3.5, 3.75, 4, 4.25, 4.5},
    legend pos=south east,
    y tick label style={/pgf/number format/1000 sep=},
    ]
    \addplot[const plot, blue, semithick] file{../data/Zad5_figure_data/Zmiany_na_Ti/zad5_PID_input_exampleK_0.5_Ti_6.25_Td_2.22_E_1.0548.csv};
    \legend{$u[k]$}
    \end{axis}
    \end{tikzpicture}
    \caption{Przebieg sygnału sterującego regulatora}
    \label{zad5_1_zmiana_K}
\end{figure}
\FloatBarrier

\subsection{Ręczna poprawa wartości wzmocnienia członu całkującego $T_{\mathrm{d}}$ }
\indent 1. Człon całkujący zminiejszyliśmy o 1 
\begin{figure}[b]
    \centering
    \begin{tikzpicture}
    \begin{axis}[
    width=0.98\textwidth,
    xmin=0,xmax=500,ymin=3.9, ymax=4.3,
    xlabel={$k$},
    ylabel={$y[k]$},
    %xtick={0.3, 0.4, 0.5, 0.6, 0.7},
    %ytick={3.5, 3.75, 4, 4.25, 4.5},
    legend pos=south east,
    y tick label style={/pgf/number format/1000 sep=},
    ]
    \addplot[blue, semithick] file{../data/Zad5_figure_data/Zmiany_na_Td/zad5_PID_setpoint_exampleK_0.5_Ti_7.25_Td_1.86_E_1.1087.csv};
    \addplot[red, semithick] file{../data/Zad5_figure_data/Zmiany_na_Td/zad5_PID_output_exampleK_0.5_Ti_7.25_Td_1.86_E_1.1087.csv};
    \addlegendentry{$y^{zad}[k]$},
    \addlegendentry{$y[k]$},
    \addlegendimage{no markers, blue}
	\addlegendimage{no markers, red}
    \end{axis}
    \end{tikzpicture}
    \caption{Odpowiedź układu na skok wartości zadanej}
    \label{zad5_1_K_odp}
\end{figure}

\begin{figure}[t]
    \centering
    \begin{tikzpicture}
    \begin{axis}[
    width=0.98\textwidth,
    xmin=0.0,ymin=0.45, xmax= 500,
    xlabel={$k$},
    ylabel={$u[k]$},
    %xtick={0.3, 0.4, 0.5, 0.6, 0.7},
    %ytick={3.5, 3.75, 4, 4.25, 4.5},
    legend pos=south east,
    y tick label style={/pgf/number format/1000 sep=},
    ]
    \addplot[const plot, blue, semithick] file{../data/Zad5_figure_data/Zmiany_na_Td/zad5_PID_input_exampleK_0.5_Ti_7.25_Td_1.86_E_1.1087.csv};
    \legend{$u[k]$}
    \end{axis}
    \end{tikzpicture}
    \caption{Przebieg sygnału sterującego regulatora}
    \label{zad5_1_zmiana_K}
\end{figure}
\FloatBarrier

\indent 2.
\begin{figure}[b]
    \centering
    \begin{tikzpicture}
    \begin{axis}[
    width=0.98\textwidth,
    xmin=0,xmax=500,ymin=3.9, ymax=4.3,
    xlabel={$k$},
    ylabel={$y[k]$},
    %xtick={0.3, 0.4, 0.5, 0.6, 0.7},
    %ytick={3.5, 3.75, 4, 4.25, 4.5},
    legend pos=south east,
    y tick label style={/pgf/number format/1000 sep=},
    ]
    \addplot[blue, semithick] file{../data/Zad5_figure_data/Zmiany_na_Td/zad5_PID_setpoint_exampleK_0.5_Ti_7.25_Td_1.62_E_1.1056.csv};
    \addplot[red, semithick] file{../data/Zad5_figure_data/Zmiany_na_Td/zad5_PID_output_exampleK_0.5_Ti_7.25_Td_1.62_E_1.1056.csv};
    \addlegendentry{$y^{zad}[k]$},
    \addlegendentry{$y[k]$},
    \addlegendimage{no markers, blue}
	\addlegendimage{no markers, red}
    \end{axis}
    \end{tikzpicture}
    \caption{Odpowiedź układu na skok wartości zadanej}
    \label{zad5_1_K_odp}
\end{figure}

\begin{figure}[t]
    \centering
    \begin{tikzpicture}
    \begin{axis}[
    width=0.98\textwidth,
    xmin=0.0,ymin=0.45, xmax= 500,
    xlabel={$k$},
    ylabel={$u[k]$},
    %xtick={0.3, 0.4, 0.5, 0.6, 0.7},
    %ytick={3.5, 3.75, 4, 4.25, 4.5},
    legend pos=south east,
    y tick label style={/pgf/number format/1000 sep=},
    ]
    \addplot[const plot, blue, semithick] file{../data/Zad5_figure_data/Zmiany_na_Td/zad5_PID_input_exampleK_0.5_Ti_7.25_Td_1.62_E_1.1056.csv};
    \legend{$u[k]$}
    \end{axis}
    \end{tikzpicture}
    \caption{Przebieg sygnału sterującego regulatora}
    \label{zad5_1_zmiana_K}
\end{figure}
\FloatBarrier

\indent 3.
\begin{figure}[b]
    \centering
    \begin{tikzpicture}
    \begin{axis}[
    width=0.98\textwidth,
    xmin=0,xmax=500,ymin=3.9, ymax=4.3,
    xlabel={$k$},
    ylabel={$y[k]$},
    %xtick={0.3, 0.4, 0.5, 0.6, 0.7},
    %ytick={3.5, 3.75, 4, 4.25, 4.5},
    legend pos=south east,
    y tick label style={/pgf/number format/1000 sep=},
    ]
    \addplot[blue, semithick] file{../data/Zad5_figure_data/Zmiany_na_Td/zad5_PID_setpoint_exampleK_0.5_Ti_7.25_Td_1.26_E_1.1015.csv};
    \addplot[red, semithick] file{../data/Zad5_figure_data/Zmiany_na_Td/zad5_PID_output_exampleK_0.5_Ti_7.25_Td_1.26_E_1.1015.csv};
    \addlegendentry{$y^{zad}[k]$},
    \addlegendentry{$y[k]$},
    \addlegendimage{no markers, blue}
	\addlegendimage{no markers, red}
    \end{axis}
    \end{tikzpicture}
    \caption{Odpowiedź układu na skok wartości zadanej}
    \label{zad5_1_K_odp}
\end{figure}

\begin{figure}[t]
    \centering
    \begin{tikzpicture}
    \begin{axis}[
    width=0.98\textwidth,
    xmin=0.0,ymin=0.45, xmax= 500,
    xlabel={$k$},
    ylabel={$u[k]$},
    %xtick={0.3, 0.4, 0.5, 0.6, 0.7},
    %ytick={3.5, 3.75, 4, 4.25, 4.5},
    legend pos=south east,
    y tick label style={/pgf/number format/1000 sep=},
    ]
    \addplot[const plot, blue, semithick] file{../data/Zad5_figure_data/Zmiany_na_Td/zad5_PID_input_exampleK_0.5_Ti_7.25_Td_1.26_E_1.1015.csv};
    \legend{$u[k]$}
    \end{axis}
    \end{tikzpicture}
    \caption{Przebieg sygnału sterującego regulatora}
    \label{zad5_1_zmiana_K}
\end{figure}
\FloatBarrier




































%% od tego momentu jest zle


\iffalse
\indent 3. Zmiany zaczęliśmy od członu proporcjonalnego. Postanowiliśmy go zmniejszyć, co spowodowało wygładzenie odpowiedzi skokowej. Pas ponad wartością zadaną stał się szerszy, ale przebieg sygnału był zdecydowanie łagodniejszy (nie było szpiczastych pików i zniknęły małe drgania). Nastąpiła zdecydowana zmiana przebiegu sygnału sterującego. Był on nadal ucinany przez ograniczenie, ale jego zmiany stały się bardziej regularne (łagodnie opadł do poziomu 0,6 co zapewnia utrzymanie wartości zadanej).

\begin{center}
Wartość wskaźnika regulacji wynosi $E=2,9694$.
\end{center}
\begin{figure}[h]
    \centering
    \begin{tikzpicture}
    \begin{axis}[
    width=0.98\textwidth,
    xmin=0,xmax=500,ymin=3.9, ymax=4.4,
    xlabel={$k$},
    ylabel={$y[k]$},
    %xtick={0.3, 0.4, 0.5, 0.6, 0.7},
    %ytick={3.5, 3.75, 4, 4.25, 4.5},
    legend pos=south east,
    y tick label style={/pgf/number format/1000 sep=},
    ]
    \addplot[blue, semithick] file{../data/zad5_PID_setpoint_exampleK_0.45_Ti_9.5_Td_2.28.csv};
    \addplot[red, semithick] file{../data/zad5_PID_output_exampleK_0.45_Ti_9.5_Td_2.28.csv};
    \addlegendentry{$y^{zad}[k]$},
    \addlegendentry{$y[k]$},
    \addlegendimage{no markers, blue}
	\addlegendimage{no markers, red}
    \end{axis}
    \end{tikzpicture}
    \caption{Odpowiedź skokowa układu regulacji}
    \label{zad5_niegasnące_oscylacje}
\end{figure}

\begin{figure}[h]
    \centering
    \begin{tikzpicture}
    \begin{axis}[
    width=0.98\textwidth,
    xmin=0.0,ymin=0.45, xmax= 500,
    xlabel={$k$},
    ylabel={$u[k]$},
    %xtick={0.3, 0.4, 0.5, 0.6, 0.7},
    %ytick={3.5, 3.75, 4, 4.25, 4.5},
    legend pos=south east,
    y tick label style={/pgf/number format/1000 sep=},
    ]
    \addplot[blue, semithick] file{../data/zad5_PID_input_exampleK_0.45_Ti_9.5_Td_2.28.csv};
    \legend{$u[k]$}
    \end{axis}
    \end{tikzpicture}
    \caption{Sygnał sterujący u}
    \label{zad2_stat_wykres}
\end{figure}
\FloatBarrier

\indent 4. W następnym kroku zmniejszyliśmy jeszcze bardziej wartość członu proporcjonalnego. W ten sposób chcieliśmy zmniejszyć wartość uchybu ustalonego. Niestety nie osiągneliśmy zadanego celu, dodatkowo zwiększył nam się czas regulacji (jest to zachowanie zgodne z matematycznymi wzorami dla regulatora PID), dlatego podjęliśmy decyzję o zmianach wartościami dla innych członów. 
\begin{center}
Wartość wskaźnika regulacji wynosi $E=5,7761$.
\end{center}
\begin{figure}[h]
    \centering
    \begin{tikzpicture}
    \begin{axis}[
    width=0.98\textwidth,
    xmin=0,xmax=500,ymin=3.9, ymax=4.4,
    xlabel={$k$},
    ylabel={$y[k]$},
    %xtick={0.3, 0.4, 0.5, 0.6, 0.7},
    %ytick={3.5, 3.75, 4, 4.25, 4.5},
    legend pos=south east,
    y tick label style={/pgf/number format/1000 sep=},
    ]
    \addplot[blue, semithick] file{../data/zad5_PID_setpoint_exampleK_0.225_Ti_9.5_Td_2.28.csv};
    \addplot[red, semithick] file{../data/zad5_PID_output_exampleK_0.225_Ti_9.5_Td_2.28.csv};
    \addlegendentry{$y^{zad}[k]$},
    \addlegendentry{$y[k]$},
    \addlegendimage{no markers, blue}
	\addlegendimage{no markers, red}
    \end{axis}
    \end{tikzpicture}
    \caption{Odpowiedź skokowa układu regulacji}
    \label{zad5_niegasnące_oscylacje}
\end{figure}

\begin{figure}[h]
	\label{zad5_sterowanie_ustalenie_k}
    \centering
    \begin{tikzpicture}
    \begin{axis}[
    width=0.98\textwidth,
    xmin=0.0,ymin=0.45, xmax= 500,
    xlabel={$k$},
    ylabel={$u[k]$},
    %xtick={0.3, 0.4, 0.5, 0.6, 0.7},
    %ytick={3.5, 3.75, 4, 4.25, 4.5},
    legend pos=south east,
    y tick label style={/pgf/number format/1000 sep=},
    ]
    \addplot[blue, semithick] file{../data/zad5_PID_input_exampleK_0.225_Ti_9.5_Td_2.28.csv};
    \legend{$u[k]$}
    \end{axis}
    \end{tikzpicture}
    \caption{Sygnał sterujący u}
\end{figure}
\FloatBarrier


\indent 5. Zmniejszenie wartości $T_{i}$ powoduj wzrost znaczenie członu całkującego. Najprościej mówiąc dodaje on stale sygnał uchybu. Gdy uchyb jest powyżej 0 wymusza on na sygnale sterującym coraz większą wartość. Efektem jego działania jest sprowadzenie uchybu do zera. Jego działanie możemy zaobserwować po przez szybszy czas regulacji (wartość z regulatora szybciej zbiega do wartości zadanej), a przy nie zerowym uchybie sterowanie również osiąga większe wartości (widzimy to porównując wykresy \ref{zad5_sterowanie_ustalenie_k} i \ref{zad5_sterowanie_ustalenie_ti}).
\begin{center}
Wartość wskaźnika regulacji wynosi $E=4,1742$.
\end{center}
\begin{figure}[h]
    \centering
    \begin{tikzpicture}
    \begin{axis}[
    width=0.98\textwidth,
    xmin=0,xmax=500,ymin=3.9, ymax=4.4,
    xlabel={$k$},
    ylabel={$y[k]$},
    %xtick={0.3, 0.4, 0.5, 0.6, 0.7},
    %ytick={3.5, 3.75, 4, 4.25, 4.5},
    legend pos=south east,
    y tick label style={/pgf/number format/1000 sep=},
    ]
    \addplot[blue, semithick] file{../data/zad5_PID_setpoint_exampleK_0.19152_Ti_4.9007_Td_0.1.csv};
    \addplot[red, semithick] file{../data/zad5_PID_output_exampleK_0.19152_Ti_4.9007_Td_0.1.csv};
    \addlegendentry{$y^{zad}[k]$},
    \addlegendentry{$y[k]$},
    \addlegendimage{no markers, blue}
	\addlegendimage{no markers, red}
    \end{axis}
    \end{tikzpicture}
    \caption{Odpowiedź skokowa układu regulacji}
    \label{zad5_niegasnące_oscylacje}
\end{figure}

\begin{figure}[h]
    \label{zad5_sterowanie_ustalenie_ti}
    \centering
    \begin{tikzpicture}
    \begin{axis}[
    width=0.98\textwidth,
    xmin=0.0,ymin=0.45, xmax= 500,
    xlabel={$k$},
    ylabel={$u[k]$},
    %xtick={0.3, 0.4, 0.5, 0.6, 0.7},
    %ytick={3.5, 3.75, 4, 4.25, 4.5},
    legend pos=south east,
    y tick label style={/pgf/number format/1000 sep=},
    ]
    \addplot[blue, semithick] file{../data/zad5_PID_input_exampleK_0.19152_Ti_4.9007_Td_0.1.csv};
    \legend{$u[k]$}
    \end{axis}
    \end{tikzpicture}
    \caption{Sygnał sterujący u}
\end{figure}
\FloatBarrier

\indent 6. W ostatnim kroku kluczowym etapem było obniżenie wartości członu różniczkującego. Jego działanie jest tym mocniejsze im szybciej zmienia się uchyb. Jego 10-krotne zmniejszenie spowodowało zanik przeregulowania, a także skróciło czas regulacji. Są to wielkości, którymi możemy manipulować za pomocą wartości $T_{d}$. Zmieniając kosmetycznie wartości K i $T_{i}$ uzyskaliśmy najmnniejszą wartość wskaźnika regulacji równą $E=1,016$. W tym przypadku przebieg sygnałów odpowiedzi regulatora i sterującego jest także najlepszy. 
\begin{figure}[h]
    \centering
    \begin{tikzpicture}
    \begin{axis}[
    width=0.98\textwidth,
    xmin=0,xmax=500,ymin=3.9, ymax=4.4,
    xlabel={$k$},
    ylabel={$y[k]$},
    %xtick={0.3, 0.4, 0.5, 0.6, 0.7},
    %ytick={3.5, 3.75, 4, 4.25, 4.5},
    legend pos=south east,
    y tick label style={/pgf/number format/1000 sep=},
    ]
    \addplot[blue, semithick] file{../data/zad5_PID_setpoint_exampleK_0.171_Ti_5.2007_Td_0.090366.csv};
    \addplot[red, semithick] file{../data/zad5_PID_output_exampleK_0.171_Ti_5.2007_Td_0.090366.csv};
    \addlegendentry{$y^{zad}[k]$},
    \addlegendentry{$y[k]$},
    \addlegendimage{no markers, blue}
	\addlegendimage{no markers, red}
    \end{axis}
    \end{tikzpicture}
    \caption{Odpowiedź skokowa układu regulacji}
    \label{zad5_niegasnące_oscylacje}
\end{figure}

\begin{figure}[h]
    \centering
    \begin{tikzpicture}
    \begin{axis}[
    width=0.98\textwidth,
    xmin=0.0,ymin=0.45, xmax= 500,
    xlabel={$k$},
    ylabel={$u[k]$},
    %xtick={0.3, 0.4, 0.5, 0.6, 0.7},
    %ytick={3.5, 3.75, 4, 4.25, 4.5},
    legend pos=south east,
    y tick label style={/pgf/number format/1000 sep=},
    ]
    \addplot[blue, semithick] file{../data/zad5_PID_input_exampleK_0.171_Ti_5.2007_Td_0.090366.csv};
    \legend{$u[k]$}
    \end{axis}
    \end{tikzpicture}
    \caption{Sygnał sterujący u}
    \label{zad2_stat_wykres}
\end{figure}
\FloatBarrier
\fi