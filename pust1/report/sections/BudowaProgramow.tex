\definecolor{mylilas}{RGB}{170,55,241}

\lstset{language=Matlab,%
    %basicstyle=\color{red},
    breaklines=true,%
    morekeywords={matlab2tikz},
    keywordstyle=\color{blue},%
    morekeywords=[2]{1}, keywordstyle=[2]{\color{black}},
    identifierstyle=\color{black},%
    stringstyle=\color{mylilas},
    stringstyle=\color{magenta},
    commentstyle=\color{green},%
    showstringspaces=false,%without this there will be a symbol in the places where there is a space
    numbers=left,%
    numberstyle={\tiny \color{black}},% size of the numbers
    numbersep=9pt, % this defines how far the numbers are from the text
    emph=[1]{for,end,break},emphstyle=[1]\color{red}, %some words to emphasise
    %emph=[2]{word1,word2}, emphstyle=[2]{style},    
}

\chapter{Omówienie implementacji algorytmów PID i DMC}
\label{zad4}

\section{Implementacja algorytmu PID}
\label{zad4_pid}
Tworzenie programu symulującego działanie regulatora PID należało rozpocząć od zdefiniowania stałych, podanych w poleceniu. Były to podane w poleceniu wartości w punkcie pracy sygnałów wyjściowego oraz sterującego oznaczonych jako $Y_{\mathrm{pp}}$ i $U_{\mathrm{pp}}$. Kolejno zdefiniowane zostały wartości ograniczeń sygnału sterującego oraz okres próbkowania i długość symulacji.
\\ 

\begin{lstlisting}[style=custommatlab,frame=single,label={zad4_const_lst},caption={Definicja stałych wartości używanych w programie},captionpos=b]
Upp = 0.5;
Ypp = 4;

Umin = 0.3;
Umax = 0.7;
dUmax = 0.05;

T = 0.5;   
sim_len = 600;
\end{lstlisting}


W następnej części programu miała miejsce rezerwacja pamięci na wektory danych programu. Pierwszym wektorem, który wykorzystywany jest przy symulacji jest wektor \verb+setpoint+. Aby możliwe było wykonanie zaplanowanej trajektorii zadanej dla kilku skoków wartości, zostały zdefiniowane pomocnicze wartości: \verb+stpt_value_1+, \verb+stpt_value_2+, \verb+stpt_value_3+.\\
\indent{} Wektorami przechowującymi najważniejsze dane są wektory \verb+input+ oraz \verb+output+, które są reprezentacją sygnałów wejściowych i wyjściowych podczas pracy skryptu. Zostały one zainicjalizowane wartościami wspomnianych sygnałów w punkcie pracy w celu uproszczenia przeskalowywania wyliczanych wartości.\\ 
\indent{} Skalar \verb+rescaled_output+ oraz wektor \verb+rescaled_input+ to zmienne używane do przechowywania tymczasowych wartości przeskalowanych sygnałów podczas symulacji. Na koniec został utworzony wektor \verb+error+ przechowujący uchyby podczas trwania symualcji.

\begin{lstlisting}[style=custommatlab,frame=single,label={zad4_wek_lst},caption={Inicjalizacja wektorów},captionpos=b]
%czas symulacji
sim_time = 1:sim_len; %wykorzystywany do tworzenia wykresow
sim_time = sim_time';

%wartosc zadana
stpt_value_1 = 4.15;
stpt_value_2 = 3.91;
stpt_value_3 = 4.3;
setpoint = [(stpt_value_1*ones(sim_len/3,1))'
			(stpt_value_2*ones(sim_len/3,1))'
			(stpt_value_3*ones(sim_len/3,1))']';
setpoint(1:11) = Ypp;
%wektor sygnalu sterujacego
input = Upp*ones(sim_len, 1);
% wektor wyjscia
output = Ypp*ones(sim_len, 1);
rescaled_output = 0;
rescaled_input = zeros(sim_len, 1);
% wektor uchybu
error = zeros(sim_len, 1);
\end{lstlisting}

Po wykonaniu niezbędnych przygotowań programu oraz inicjalizacji przeszliśmy do implementacji samego algorytmu, zaczynając od definicji parametrów ciągłego regulatora PID: $K$, $T_\mathrm{i}$, $T_\mathrm{d}$. Następnie zdefiniowane zostały współczynniki regulatora cyfrowego: $r_0$, $r_1$, $r_2$ oraz zaimplementowane zostały wzory (\ref{eq_r0}), (\ref{eq_r1}), (\ref{eq_r2}). Wartości współczynników wykorzystywane były później w równaniu służącym do obliczania sterowania.
\\
\begin{equation}
\label{eq_r0}
r_{0} = K (1 + \frac{T}{2 T_{\mathrm{i}}} + \frac{T_{\mathrm{d}}}{T})
\end{equation}

\begin{equation}
\label{eq_r1}
r_{1} = K (\frac{T}{2 T_{\mathrm{i}}} - \frac{2 T_{\mathrm{d}}}{T} - 1)
\end{equation}

\begin{equation}
\label{eq_r2}
r_{2} = \frac{K T_{\mathrm{d}}}{T}
\end{equation}

\begin{lstlisting}[style=custommatlab,frame=single,label={zad4_param_lst},caption={Definicja współczynników regulatora PID},captionpos=b]
K = 0.5;
Ti = 6.25;
Td = 1.62;

r0 = K*(1 + T/(2*Ti) + Td/T);
r1 = K*(T/(2*Ti) - (2*Td)/T - 1);
r2 = (K*Td)/T;
\end{lstlisting}

Ostatnią fazą było utworzenie pętli, w której odbywała się symulacja. Każdy przebieg rozpoczynał pomiar aktualnej wartości wyjścia obiektu otrzymywanej za pomocą skryptu \verb+symulacja_obiektu1Y.m+. Kolejnym krokiem było przeskalowanie zmierzonego wyjścia obiektu oraz wartości zadanej poprzez odjęcie od nich wartości $Y_{\mathrm{pp}}$. Po tej operacji obliczany był uchyb. Podczas symulacji w każdej chwili dyskretnej $k$ obliczany był wskaźnik jakości jako suma kwadratów uchybów ze wszystkich chwil. W programie zmienna przechowująca wartość wskaźnika została nazwana \verb+error_sum+. Ostatnia część iteracji związana była z wyliczeniem sterowania. Wyliczona wartość jest wartością przeskalowaną, którą należy dodać do wartości sterowania w punkcie pracy. Po nałożeniu ograniczeń na sygnał sterujący rozpoczynała się kolejna iteracja. Postać pętli symulacyjnej przedstawia listing \ref{zad4_sim_lst}.\\ 
\begin{lstlisting}[style=custommatlab,frame=single,label={zad4_sim_lst},caption={Petla symulujaca dzialanie cyfrowego algorytmu PID},captionpos=b]
error_sum = 0;
for k = 12:sim_len     
    output(k) = symulacja_obiektu1Y(input(k-10), input(k-11),
    			output(k-1), output(k-2));    % pomiar wyjscia
    rescaled_output = output(k) - Ypp;  % skalowanie wyjscia   
    stpt = setpoint(k) - Ypp;   % przeskalowany setpoint
    error(k) = stpt - rescaled_output;   % obliczenie uchyby   
    
    error_sum = error_sum + error(k)^2;
    
    rescaled_input(k) = r2 * error(k-2) + r1 * error(k-1) +
    		r0 * error(k) + rescaled_input(k-1);  % wyliczenie sterowania 
  
    %ograniczenia  
    if rescaled_input(k) - rescaled_input(k-1) >= dUmax
        rescaled_input(k) = dUmax + rescaled_input(k-1);
    elseif rescaled_input(k) - rescaled_input(k-1) <= -dUmax
        rescaled_input(k) = rescaled_input(k-1) - dUmax;
    end   
    input(k) = input(k) + rescaled_input(k);
    if input(k) >= Umax
        input(k) = Umax;
    elseif input(k) <= Umin
        input(k) = Umin;
    end 
end
\end{lstlisting}
Po zakończonej symulacji wyświetlane były wykresy z wynikami oraz po odpowiedniej, krótkiej obróbce danych generowane były pliki \verb+.csv+ wykorzystywane do tworzenia wykresów w systemie \LaTeX.


\section{Implementacja algorytmu DMC}
\label{zad4_dmc}

Implementacja drugiego algorytmu przebiegała w bardzo podobny sposób, jeśli chodzi o budowę programu. Różnicą pojawiającą się na początku skryptu było wczytanie modelu wykorzystywanego do regulacji z wykorzystaniem algorytmu DMC - znormalizowanej odpowiedzi skokowej obiektu.\\
\indent{} Następnie, tak jak poprzednio, zdefiniowane zostały stałe wykorzystywane w programie jak i parametry wykorzystywanego algorytmu. Horyzont dynamiki równał się długości wektora z odpowiedzią skokową. Horyzonty predykcji oraz sterowania, a także współczynnik kary $\lambda$ były odpowiednio modyfikowane w późniejszym procesie strojenia regulatora. \\
\indent{} Kolejnym krokiem, również powtórzonym była deklaracja wektorów wykorzystywanych do przechowywania wartości sygnałów.
%[language=Matlab]

\begin{lstlisting}[style=custommatlab,frame=single,label={zad4_vecDMC_lst},caption={Inizjalizacja wektorów używanych do przechowywania sygnałów.},captionpos=b]
% czas symulacji
sim_time = 1:sim_len; % do plotowania
sim_time = sim_time';

% wartosc zadana
stpt_value = 4.05;
setpoint = stpt_value*ones(sim_len,1);
setpoint(1:11) = Ypp;

% wektor sygnalu sterujacego
input = Upp*ones(sim_len, 1);

% wektor wyjscia
output = Ypp*ones(sim_len, 1);

rescaled_output = 0;
rescaled_input = zeros(sim_len, 1);

% wektor uchybu
error = zeros(sim_len, 1);

\end{lstlisting}

Pracę nad algorytmem należało rozpocząć od inicjalizacji macierzy algorytmu, używanych później do symulacji. Ważnym aspektem tej czynności jest poprawne określenie wymiarów każdej macierzy. Mając zdefiniowane w ten sposób zmienne mogliśmy kontrolować, czy poszczególne obliczenia, tworzące poniższe elementy algorytmu, przebiegały poprawnie.
\\

\begin{lstlisting}[style=custommatlab,frame=single,label={zad4_sim_lst},caption={Definicja macierzy algorytmu DMC},captionpos=b]
dU = zeros(Nu, 1);
dUp = zeros(D-1, 1);
M = zeros(N, Nu);
Mp = zeros(N, D-1);
K = zeros(Nu, N);
\end{lstlisting}

Następnie przystąpiliśmy do wyliczenia macierzy, których postać była niezmienna podczas działania programu. Kolejno obliczona została macierz $M$, $M_\mathrm{p}$ oraz $K$ zgodnie ze wzorami:
\\

\begin{equation}
M=\left[
\begin{array}
{cccc}
s_{1} & 0 & \ldots & 0\\
s_{2} & s_{1} & \ldots & 0\\
\vdots & \vdots & \ddots & \vdots\\
s_{N} & s_{N-1} & \ldots &  s_{N-N_{\mathrm{u}}+1}
\end{array}
\right]
\end{equation}
\\
\begin{equation}
M^{\mathrm{p}}=\left[
\begin{array}
{cccc}
s_{\mathrm{2}} - s_{\mathrm{1}} & s_{\mathrm{3}} - s_{\mathrm{2}} & \ldots & s_{\mathrm{D}} - s_{\mathrm{D-1}}\\
s_{\mathrm{3}} - s_{\mathrm{1}} & s_{\mathrm{4}} - s_{\mathrm{2}} & \ldots & s_{\mathrm{D+1}} - s_{\mathrm{D-1}}\\
\vdots & \vdots & \ddots & \vdots\\
s_{\mathrm{N+1}} - s_{1} & s_{\mathrm{N+2}} - s_{\mathrm{2}} & \ldots &  s_{N + \mathrm{D} - 1} - s_{{\mathrm{D-1}}}
\end{array}
\right]
\end{equation}
\\
\begin{equation}
K = (M^{\mathrm{T}} M + \lambda I_{\mathrm{N_{\mathrm{u}}} \times \mathrm{N_{\mathrm{u}}}})^{\mathrm{-1}} M^{\mathrm{T}}
\end{equation}
\\
Implementację powyższych wzorów przedstawiono na listingu \ref{zad4_matrices_lst}.\\

\begin{lstlisting}[style=custommatlab,frame=single,label={zad4_matrices_lst},caption={Wyliczenie macierzy stałych algorytmu DMC},captionpos=b]
% macierz M
for i = 1:Nu
    k = 1;
    for j = 1:N
        if j < i
            M(j, i) = 0;
        else
            M(j, i) = step(k);
            k = k + 1;
        end
    end
end

% macierz Mp
for i = 1:N
    for j = 1:(D-1)
        k = i + j;
        if k > D
            Mp(i,j) = step(D) - step(j);
        else
            Mp(i,j) = step(k) - step(j);
        end
    end
end

% macierz K
K = inv((M' * M + lambda * eye(Nu)))*M';
%\end{lstlisting}
Wersja algorytmu DMC użyta w programie jest wersją oszczędną to znaczy podczas symulacji nie jest obliczany cały wektor przyrostów sterowania $\Delta u(k)$, a jedynie pierwszy jego element faktycznie wykorzystywany do sterowania obiektem. Dana wartość obliczana jest ze wzoru:

\begin{equation}
\label{du(k)}
\Delta u(k) = K_{\mathrm{e}} e(k) - K_{\mathrm{u}} \Delta U^{\mathrm{P}}(k)
\end{equation}

gdzie $K_{\mathrm{e}}$ jest wyrażonym wzorem (\ref{eq_Ke}), $K_{\mathrm{u}}$ jest wektorem powstałym z pomnożenia pierwszego wiersza macierzy $K$ oraz macierzy $M^{\mathrm{P}}$ co przedstawia wzór (\ref{eq_Ku}). Wektor $\Delta U^{\mathrm{P}}(k)$ jest wektorem różnic sterowania między kolejnymi jego wartościami z przeszłości danym wzorem:

\begin{equation}
\Delta U^{\mathrm{P}}(k)=\left[
\begin{array}{c}
u(k-1) - u(k-2)\\
u(k-2) - u(k-3)\\
\vdots\\
u(k-\mathrm{D}+1) - u(k-\mathrm{D})
\end{array}
\right]
\end{equation}

\begin{equation}
\label{eq_Ke}
K_{\mathrm{e}} = \sum_{i=1}^{N} K_{1,i}
\end{equation}

\begin{equation}
\label{eq_Ku}
K_{\mathrm{u}} = \overline{K_{1}} M^{\mathrm{P}}
\end{equation}

Wektor $\Delta U^{\mathrm{P}}(k)$ jest obliczany w każdej iteracji pętli symulacyjnej w celu uniknięcia niepotrzebnych komplikacji związanych z przesuwaniem jego wartości. Przed wejściem do pętli obliczane są wartości $K_{\mathrm{e}}$ oraz $K_{\mathrm{e}}$. Symulacja działania algorytmu za pomocą pętli została przedstawiona na listingu \ref{zad4_dmc_sim_lst}.\\

\begin{lstlisting}[style=custommatlab,frame=single,label={zad4_dmc_sim_lst},caption={Pętla symulująca działanie regulatora DMC},captionpos=b]
Ke = sum(K(1,:));
Ku = K(1,:) * Mp;

for k=12:sim_len    
    % wektor dUp
    for i = 1:(D-1)
        if (k-i) <= 0
            du1 = 0;
        else
            du1 = rescaled_input(k - i);
        end
        if (k-i-1) <= 0
            du2 = 0;
        else
            du2 = rescaled_input(k - i - 1);
        end 
        dUp(i) = du1 - du2;
    end
    
    output(k) = symulacja_obiektu1Y(input(k-10), input(k-11),
    			output(k-1), output(k-2));    % pomiar wyjscia
    rescaled_output = output(k) - Ypp;  % skalowanie wyjscia   
    stpt = setpoint(k) - Ypp;   % przeskalowany setpoint
    error(k) = stpt - rescaled_output; % obliczenie uchyby   
    
    error_sum = error_sum + error(k)^2; % wskaznik jakosci
    
    rescaled_input(k) = Ke * error(k) - Ku * dUp;
    rescaled_input(k) = rescaled_input(k-1) + rescaled_input(k);
    
    % ograniczenia  
    if rescaled_input(k) - rescaled_input(k-1) >= dUmax
        rescaled_input(k) = dUmax + rescaled_input(k-1);
    elseif rescaled_input(k) - rescaled_input(k-1) <= -dUmax
        rescaled_input(k) = rescaled_input(k-1) - dUmax;
    end   
    
    input(k) = input(k) + rescaled_input(k);  
    
    if input(k) >= Umax
        input(k) = Umax;
    elseif input(k) <= Umin
        input(k) = Umin;
    end 
end
\end{lstlisting}

Pętla jest zbudowana w taki podobny sposób co pętla do symulacji algorytmu PID. Różnicą na początku jest obliczenie wektora $\Delta U^{\mathrm{P}}(k)$. Następnie następuje pomiar wyjścia i przeskalowanie sygnałów: wyjściowego oraz wartości zadanej. Po obliczeniu uchybu oraz wskaźnika jakości następuje obliczenie wartości przyrostu sterowania zgodnie ze wzorem (\ref{du(k)}). Ostatnią częścią jest uwzględnienie ograniczeń sygnału sterującego, po którym następuje kolejna iteracja.\\
\indent{} W skrypcie \verb+zad4_DMC.m+ po zakończeniu symulacji również następuje wyświetlenie wyników, krótka obróbka danych oraz generowanie plików z danymi.

