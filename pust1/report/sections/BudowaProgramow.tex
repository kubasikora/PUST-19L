\chapter{Omówienie implementacji algorytmów PID i DMC}
\label{zad4}

\section{Implementacja algorytmu PID}
\label{zad4_pid}
Tworzenie programu symulującego działanie regulatora PID należało rozpocząć od zdefiniowania stałych, podanych w poleceniu. Były to podane w poleceniu wartości w punkcie pracy sygnałów wyjściowego oraz sterującego oznaczonych jako $Y_{\mathrm{pp}}$ i $U_{\mathrm{pp}}$. Kolejno zdefiniowane zostały wartości ograniczeń sygnału sterującego oraz okres próbkowania i długość symulacji.
\\ 

\begin{lstlisting}[style=custommatlab,frame=single,label={zad4_const_lst},caption={Definicja stałych wartości używanych w programie},captionpos=b]
Upp = 0.5;
Ypp = 4;

Umin = 0.3;
Umax = 0.7;
dUmax = 0.05;

T = 0.5;   
sim_len = 600;
\end{lstlisting}


W następnej części programu miała miejsce rezerwacja pamięci na wektory danych programu. Pierwszym wektorem, który wykorzystywany jest przy symulacji jest wektor \verb+setpoint+. Aby możliwe było wykonanie zaplanowanej trajektorii zadanej dla kilku skoków wartości, zostały zdefiniowane pomocnicze wartości: \verb+stpt_value_1+, \verb+stpt_value_2+, \verb+stpt_value_3+.\\
\indent{} Wektorami przechowującymi najważniejsze dane są wektory \verb+input+ oraz \verb+output+, które są reprezentacją sygnałów wejściowych i wyjściowych podczas pracy skryptu. Zostały one zainicjalizowane wartościami wspomnianych sygnałów w punkcie pracy w celu uproszczenia przeskalowywania wyliczanych wartości.\\ 
\indent{} Skalar \verb+rescaled_output+ oraz wektor \verb+rescaled_input+ to zmienne używane do przechowywania tymczasowych wartości przeskalowanych sygnałów podczas symulacji. Na koniec został utworzony wektor \verb+error+ przechowujący uchyby podczas trwania symualcji.

\begin{lstlisting}[style=custommatlab,frame=single,label={zad4_wek_lst},caption={Inicjalizacja wektorów},captionpos=b]
%czas symulacji
sim_time = 1:sim_len; %wykorzystywany do tworzenia wykresow
sim_time = sim_time';

%wartosc zadana
stpt_value_1 = 4.15;
stpt_value_2 = 3.91;
stpt_value_3 = 4.3;
setpoint = [(stpt_value_1*ones(sim_len/3,1))' (stpt_value_2*ones(sim_len/3,1))' (stpt_value_3*ones(sim_len/3,1))']';
setpoint(1:11) = Ypp;

%wektor sygnalu sterujacego
input = Upp*ones(sim_len, 1);

% wektor wyjscia
output = Ypp*ones(sim_len, 1);

rescaled_output = 0;
rescaled_input = zeros(sim_len, 1);

% wektor uchybu
error = zeros(sim_len, 1);
\end{lstlisting}

Po wykonaniu niezbędnych przygotowań programu oraz inicjalizacji przeszliśmy do implementacji samego algorytmu, zaczynając od definicji parametrów ciągłego regulatora PID: $K$, $T_\mathrm{i}$, $T_\mathrm{d}$. Następnie zdefiniowane zostały współczynniki regulatora cyfrowego: $r_0$, $r_1$, $r_2$ wykorzystywane później w równaniu służącym do obliczania sterowania.
\\

\begin{lstlisting}[style=custommatlab,frame=single,label={zad4_param_lst},caption={Definicja współczynników regulatora PID},captionpos=b]
K = 0.5;
Ti = 6.25;
Td = 1.62;

r0 = K*(1 + T/(2*Ti) + Td/T);
r1 = K*(T/(2*Ti) - (2*Td)/T - 1);
r2 = (K*Td)/T;
\end{lstlisting}

Ostatnią fazą było utworzenie pętli, w której odbywała się symulacja. Każdy przebieg rozpoczynał pomiar aktualnej wartości wyjścia obiektu otrzymywanej za pomocą skryptu \verb+symulacja_obiektu1Y.m+. Kolejnym krokiem było przeskalowanie zmierzonego wyjścia obiektu oraz wartości zadanej poprzez odjęcie od nich wartości $Y_{\mathrm{pp}}$. Po tej operacji obliczany był uchyb. Podczas symulacji w każdej chwili dyskretnej $k$ obliczany był wskaźnik jakości jako suma kwadratów uchybów ze wszystkich chwil. W programie zmienna przechowująca wartość wskaźnika została nazwana \verb+error_sum+. Ostatnia część iteracji związana była z wyliczeniem sterowania. Wyliczona wartość jest wartością przeskalowaną, którą należy dodać do wartości sterowania w punkcie pracy. Po nałożeniu ograniczeń na sygnał sterujący rozpoczynała się kolejna iteracja. Postać pętli symulacyjnej przedstawia listing \ref{zad4_sim_lst}.\\ 
\begin{lstlisting}[style=custommatlab,frame=single,label={zad4_sim_lst},caption={Petla symulujaca dzialanie cyfrowego algorytmu PID},captionpos=b]
error_sum = 0;
for k = 12:sim_len     
    output(k) = symulacja_obiektu1Y(input(k-10), input(k-11), output(k-1), output(k-2));    % pomiar wyjscia
    rescaled_output = output(k) - Ypp;  % skalowanie wyjscia   
    stpt = setpoint(k) - Ypp;   % przeskalowany setpoint
    error(k) = stpt - rescaled_output;   % obliczenie uchyby   
    
    error_sum = error_sum + error(k)^2;
    
    rescaled_input(k) = r2 * error(k-2) + r1 * error(k-1) + r0 * error(k) + rescaled_input(k-1);  % wyliczenie sterowania 
  
    %ograniczenia  
    if rescaled_input(k) - rescaled_input(k-1) >= dUmax
        rescaled_input(k) = dUmax + rescaled_input(k-1);
    elseif rescaled_input(k) - rescaled_input(k-1) <= -dUmax
        rescaled_input(k) = rescaled_input(k-1) - dUmax;
    end   
    input(k) = input(k) + rescaled_input(k);
    if input(k) >= Umax
        input(k) = Umax;
    elseif input(k) <= Umin
        input(k) = Umin;
    end 
end
\end{lstlisting}

Po zakończonej symulacji wyświetlane były wykresy z wynikami oraz po odpowiedniej, krótkiej obróbce danych generowane były pliki \verb+.csv+ wykorzystywane do tworzenia wykresów w systemie \LaTeX.


\section{Implementacja algorytmu DMC}
\label{zad4_dmc}

Implementacja drugiego algorytmu przebiegała w bardzo podobny sposób, jeśli chodzi o budowę programu. Różnicą pojawiającą się na początku skryptu było wczytanie modelu wykorzystywanego do regulacji z wykorzystaniem algorytmu DMC - znormalizowanej odpowiedzi skokowej obiektu.

\begin{lstlisting}
[style=custommatlab,frame=single,label={zad4_step_lst},caption={Import znormalizowanej odpowiedzi skokowej},captionpos=b]
step = importdata('../data/zad3_cut_norm_odp.csv');
step = step(:, 2);
\end{lstlisting}

Następnie, tak jak poprzednio, zdefiniowane zostały stałe wykorzystywane w programie jak i parametry wykorzystywanego algorytmu. Horyzont dynamiki równał się długości wektora z odpowiedzią skokową. Horyzonty predykcji oraz sterowania, a także współczynnik kary $\lambda$ były odpowiednio modyfikowane w późniejszym procesie strojenia regulatora. \\
\indent{} Kolejnym krokiem, również powtórzonym była deklaracja wektorów wykorzystywanych do przechowywania wartości sygnałów.


\begin{lstlisting}
[style=custommatlab,frame=single,label={zad4_vecDMC_lst}, caption={Inicjalizacja wektorów}, captionpos=b]
% czas symulacji
sim_time = 1:sim_len; % do plotowania
sim_time = sim_time';

% wartosc zadana
stpt_value = 4.05;
setpoint = stpt_value*ones(sim_len,1);
setpoint(1:11) = Ypp;

% wektor sygnalu sterujacego
input = Upp*ones(sim_len, 1);

% wektor wyjscia
output = Ypp*ones(sim_len, 1);

rescaled_output = 0;
rescaled_input = zeros(sim_len, 1);

% wektor uchybu
error = zeros(sim_len, 1);

\end{lstlisting}

Pracę nad algorytmem należało rozpocząć od inicjalizacji macierzy algorytmu, używanych później do symulacji. Ważnym aspektem tej czynności jest poprawne określenie wymiarów każdej macierzy. Mając zdefiniowane w ten sposób zmienne mogliśmy kontrolować, czy poszczególne obliczenia, tworzące poniższe elementy algorytmu, przebiegały poprawnie.
\\

\begin{lstlisting}
[style=custommatlab,frame=single,label={zad4_sim_lst},caption={Definicja macierzy algorytmu DMC},captionpos=b]
dU = zeros(Nu, 1);
dUp = zeros(D-1, 1);
M = zeros(N, Nu);
Mp = zeros(N, D-1);
K = zeros(Nu, N);
\end{lstlisting}

Następnie przystąpiliśmy do wyliczenia macierzy, których postać była niezmienna podczas działania programu. Kolejno obliczona została macierz $M$, $M_\mathrm{p}$ oraz $K$ zgodnie ze wzorami:
\\

\begin{equation}
M=\left[
\begin{array}
{cccc}
s_{1} & 0 & \ldots & 0\\
s_{2} & s_{1} & \ldots & 0\\
\vdots & \vdots & \ddots & \vdots\\
s_{N} & s_{N-1} & \ldots &  s_{N-N_{\mathrm{u}}+1}
\end{array}
\right]
\end{equation}

\begin{equation}
M^{\mathrm{p}}=\left[
\begin{array}
{cccc}
s_{\mathrm{2}} - s_{\mathrm{1}} & s_{\mathrm{3}} - s_{\mathrm{2}} & \ldots & s_{\mathrm{D}} - s_{\mathrm{D-1}}\\
s_{\mathrm{3}} - s_{\mathrm{1}} & s_{\mathrm{4}} - s_{\mathrm{2}} & \ldots & s_{\mathrm{D+1}} - s_{\mathrm{D-1}}\\
\vdots & \vdots & \ddots & \vdots\\
s_{\mathrm{N+1}} - s_{1} & s_{\mathrm{N+2}} - s_{\mathrm{2}} & \ldots &  s_{N + \mathrm{D} - 1} - s_{{\mathrm{D-1}}}
\end{array}
\right]
\end{equation}
\\

%\begin{lstlisting}
%[style=custommatlab,frame=single,label={zad4_sim_lst},caption={Definicja macierzy algorytmu DMC},captionpos=b]
%\end{lstlisting}
%\begin{lstlisting}
%[style=custommatlab,frame=single,label={zad4_sim_lst},caption={Definicja macierzy algorytmu DMC},captionpos=b]
%\end{lstlisting}
%\begin{lstlisting}
%[style=custommatlab,frame=single,label={zad4_sim_lst},caption={Definicja macierzy algorytmu DMC},captionpos=b]
%\end{lstlisting}

