\chapter{Wyznaczanie odpowiedzi skokowej procesu}
\label{zad2}

\section{Odpowiedź procesu na skokową zmianę sygnału sterującego}
\label{zad2_skoki}



\section{Wyznaczanie charakterystyki statycznej procesu}
\label{zad2_char_stat}
Poprzednio przedstawione odpowiedzi skokowe symulowanego procesy pozwalają nam
na podejrzenie że proces jest liniowy. Aby dokładnie wykreślić charakterystykę statyczną 
obiektu i z większą pewnością stwierdzić jego liniowość napisaliśmy skrypt \verb+zad2_static.m+,
którego zadaniem było przeprowadzenie symulacji dla wejść zmieniających się w zakresie ograniczeń
sygnału sterującego od $U^{\mathrm{min}} = \num{0.3}$ do $U^{\mathrm{max}} = \num{0.7}$. Dla każdego 
badanego wejścia, program badał wartość na której proces nasycał się. Efektem działania skryptu
jest rysunek \ref{zad2_stat_wykres}. 

\begin{figure}[b]
    \centering
    \begin{tikzpicture}
    \begin{axis}[
    width=\textwidth,
    xmin=0.3,xmax=0.7,ymin=3.5,ymax=4.5,
    xlabel={$u$},
    ylabel={$y(u)$},
    xtick={0.3, 0.4, 0.5, 0.6, 0.7},
    ytick={3.5, 3.75, 4, 4.25, 4.5},
    legend pos=south east,
    y tick label style={/pgf/number format/1000 sep=},
    ]
    \addplot[red, semithick] file{../data/zad2_static_char.csv};
    \legend{$y(u)$}
    \end{axis}
    \end{tikzpicture}
    \caption{Charakterystka statyczna $y(u)$ symulowanego procesu}
    \label{zad2_stat_wykres}
\end{figure}

\section{Analiza właściwości statycznych i dynamicznych procesu}
