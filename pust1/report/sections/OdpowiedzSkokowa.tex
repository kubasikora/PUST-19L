\chapter{Wyznaczanie odpowiedzi skokowej procesu}
\label{zad2}

\section{Odpowiedź procesu na skokową zmianę sygnału sterującego}
\label{zad2_skoki}

\begin{figure}[t]
    \centering
    \begin{tikzpicture}
    \begin{axis}[
    width=0.98\textwidth,
    xmin=0.0,xmax=200,ymin=3.5,ymax=4.5,
    xlabel={$k$},
    ylabel={$y[k]$},
    legend pos=south east,
    y tick label style={/pgf/number format/1000 sep=},
    ] 
    \addlegendentry{$\Delta u = \num{0,05}$},
    \addlegendentry{$\Delta u = \num{0,10}$}
    \addlegendentry{$\Delta u = \num{0,15}$},
    \addlegendentry{$\Delta u = \num{0,20}$},
    \addlegendentry{$\Delta u = \num{-0,05}$},
    \addlegendentry{$\Delta u = \num{-0,10}$},
    \addlegendentry{$\Delta u = \num{-0,15}$},
    \addlegendentry{$\Delta u = \num{-0,20}$},
    \addlegendimage{no markers,green}
	\addlegendimage{no markers,red}
	\addlegendimage{no markers,yellow}
	\addlegendimage{no markers,blue}
	\addlegendimage{no markers,black}
	\addlegendimage{no markers,orange}
	\addlegendimage{no markers,brown}
	\addlegendimage{no markers,violet}
    \addplot[green, semithick, thick] file{../data/zad2_output0.05.csv};
    \addplot[red, semithick, thick] file{../data/zad2_output0.1.csv};
    \addplot[yellow, semithick, thick] file{../data/zad2_output0.15.csv};
    \addplot[blue, semithick, thick] file{../data/zad2_output0.2.csv};
    \addplot[black, semithick, thick] file{../data/zad2_output-0.05.csv};
    \addplot[orange, semithick, thick] file{../data/zad2_output-0.1.csv};
    \addplot[brown, semithick, thick] file{../data/zad2_output-0.15.csv};
    \addplot[violet, semithick, thick] file{../data/zad2_output-0.2.csv};

    
    \end{axis}
    \end{tikzpicture}
    \caption{Odpowiedzi procesu na skokową zmianę sygnału sterującego}
    \label{zad2_porow_odp_skok}
\end{figure}

Na wykresie \ref{zad2_porow_odp_skok} możemy zaobserwować, że wraz z wzrostem wartości skoku 
rośnie również wartość odpowiedzi procesu y[k]. Zachowanie procesu jest zgodne 
z typową charakterystyką obiektów dynamicznych liniowych, w których wartość ustalona 
zmienia się liniowo wraz ze zmianą sygnału sterującego. 

\section{Wyznaczanie charakterystyki statycznej procesu}
\label{zad2_char_stat}
Poprzednio przedstawione odpowiedzi skokowe symulowanego procesy pozwalają nam
na podejrzenie że proces jest liniowy. Aby dokładnie wykreślić charakterystykę statyczną 
obiektu i z większą pewnością stwierdzić jego liniowość napisaliśmy skrypt \verb+zad2_static.m+,
którego zadaniem było przeprowadzenie symulacji dla wejść zmieniających się w zakresie ograniczeń
sygnału sterującego od $U^{\mathrm{min}} = \num{0.3}$ do $U^{\mathrm{max}} = \num{0.7}$. Dla każdego 
badanego wejścia, program badał wartość na której proces nasycał się. Efektem działania skryptu
jest rysunek \ref{zad2_stat_wykres}. 

\begin{figure}[b]
    \centering
    \begin{tikzpicture}
    \begin{axis}[
    width=0.98\textwidth,
    xmin=0.3,xmax=0.7,ymin=3.5,ymax=4.5,
    xlabel={$u$},
    ylabel={$y(u)$},
    xtick={0.3, 0.4, 0.5, 0.6, 0.7},
    ytick={3.5, 3.75, 4, 4.25, 4.5},
    legend pos=south east,
    y tick label style={/pgf/number format/1000 sep=},
    ]
    \addplot[red, semithick] file{../data/zad2_static_char.csv};
    \legend{$y(u)$}
    \end{axis}
    \end{tikzpicture}
    \caption{Charakterystka statyczna $y(u)$ symulowanego procesu}
    \label{zad2_stat_wykres}
\end{figure}

\section{Wzmocnienie statyczne}
\label{zad2_wzmocnienie}
Procesy dynamiczne o liniowej dynamice charakteryzują się 
parametrem zwanym wzmocnieniem statycznym, 
standardowo oznaczanym przez $K_{\mathrm{stat}}$.
Opisuje on stosunek między róznicą wyjścia procesu w stanie ustalonym i wyjścia w punkcie pracy
od różnicy stałego wejścia procesu i wejścia w punkcie pracy. Innymi słowy, opisuje on zależność
wyjścia w stanie ustalonym od wejścia względem pewnego punktu pracy.

\begin{equation}
K_{\mathrm{stat}} = \lim_{t \to \infty} \frac{y(t)- Y_{\mathrm{pp}}}{u_0 - U_{\mathrm{pp}}}
\label{zad2_wzm_statyczne_wzor}
\end{equation}

W przypadku tak wykreślonej charakterystyki, wzmocnienie statyczne jest równe tangensowi kąta $\alpha$
pomiędzy prostą a osią $OX$. 
\begin{equation}
K_{\mathrm{stat}} = \tg{\alpha}
\label{zad2_wzm_statyczne_wzor_tg}
\end{equation}

Dla symulowanego procesu omawianego w tym projekcie wzmocnienie statyczne 
równe jest $K_{\mathrm{stat}} = 2$. 



