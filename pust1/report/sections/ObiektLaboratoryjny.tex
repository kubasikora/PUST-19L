\chapter{Obiekt laboratoryjny -- stanowisko chłodząco-grzejące}
\label{lab1}

\section{Wyznaczenie punktu pracy}
\label{lab1_punkt_pracy}

Po zapoznaniu się ze środowiskiem przystąpiliśmy do wyznaczenia punktu pracy.
Od prowadzącego otrzymaliśmy informację o wartości sygnału sterującego w punkcie pracy 
równej $U_{\mathrm{pp}} = 26\%$. W celu wyznaczenia wartości sygnału wyjściowego w 
punkcie pracy $Y_{\mathrm{pp}}$ na wejście obiektu podaliśmy stałe wejście o wartości 
$U_{\mathrm{pp}}$ i zaczekaliśmy aż wyjście ustabilizuje się. Po odczekaniu około 5 minut,
wyjście obiektu ustabilizowało się na wartości $Y_{\mathrm{pp}} = \num{34,5} \degree $C


\section{Wzmocnienie statyczne}
\label{lab1_odpowiedzi}
Aby zbadać właściwości statyczne obiektu, zebraliśmy cztery odpowiedzi skokowe z punktu pracy.
Zostały one przedstawione na rysunku \ref{lab1_odpowiedzi_rys}. Wzmocnienie obiektu różni się w 
zależności od tego czy skok sterowania był dodatni lub ujemny, jednak pomimo tego różnice są 
tak niewielkie, że obiekt można traktować jako liniowy. Korzystając ze wzoru \ref{zad2_wzm_statyczne_wzor}
wyliczyliśmy efektywne wzmocnienie dla każdego badanego skoku i policzyliśmy ich średnią arytmetyczną, 
ostatecznie otrzymując $K_{\mathrm{stat}} = \num{0.328}$.

\begin{figure}[b]
    \centering
    \begin{tikzpicture}
    \begin{axis}[
    width=0.98\textwidth,
    xmin=0.0,xmax=550,ymin=30,ymax=38,
    xlabel={$k$},
    ylabel={$y[k]$},
    legend pos=south east,
    y tick label style={/pgf/number format/1000 sep=},
    ] 
    \addlegendentry{$\Delta u = \num{4}\%$},
    \addlegendentry{$\Delta u = \num{10}\%$}
    \addlegendentry{$\Delta u = \num{-5}\%$},
    \addlegendentry{$\Delta u = \num{-10}\%$},
    \addlegendimage{no markers,green}
	\addlegendimage{no markers,red}
	\addlegendimage{no markers,yellow}
	\addlegendimage{no markers,blue}
    \addplot[green, semithick, thick] file{../data/lab/zad2_skok_o_4.csv};
    \addplot[red, semithick, thick] file{../data/lab/zad2_skok_o_10.csv};
    \addplot[yellow, semithick, thick] file{../data/lab/zad2_skok_o_-5.csv};
    \addplot[blue, semithick, thick] file{../data/lab/zad2_skok_o_-10.csv};
    
    \end{axis}
    \end{tikzpicture}
    \caption{Odpowiedzi procesu na skokową zmianę sygnału sterującego}
    \label{lab1_odpowiedzi_rys}
\end{figure}
