\chapter{Strojenie regulatora PID na obiekcie rzeczywistym}
\label{lab5_pid}

W ramach laboratorium, dobralismy nastawy regulatora PID metodą inżynierską.
Korzystając z modelu wyznaczonego w \ref{lab3}, w domu zebralismy wstępne nastawy
aby następnie przetestować je w trakcie zajęc. Korzystając z tych nastawów 
jako punktów startowych, ręcznie poprawialiśmy wartości aby uzyskać jak 
najlepszą jakość regulacji.

\section{Dobór wzmocnienia regulatora $K_{\mathrm{r}}$}

\subsubsection{Wzmocnienie $\mathbf{K_{\mathrm{r}}} = \num{1}$}
W pierwszej kolejności zdecydowaliśmy się na sprawdzenie jak będzie zachowywał się
obiekt sterowany regulatorem o wzmocnieniu $K_{\mathrm{r}}$. Pozwoliło to nam
na rozeznanie się jakie rzędy wartości parametru, potrzebne są do skutecznej
regulacji. Okazuje się że wzmocnienie jednostkowe bardzo słabo radzi sobie
z regulacją stanowiska chłodząco-grzejącego. 

\begin{figure}[t]
    \centering
    \begin{tikzpicture}
    \begin{axis}[
    width=0.98\textwidth,
    xmin=0,xmax=800,ymin=28, ymax=42,
    xlabel={$k$},
    ylabel={$y[k]$},
    %xtick={0, 100, 200, 300, 400, 500},
    %ytick={3.9, 4, 4.1, 4.2, 4.3, 4.4},
    legend pos=south east,
    y tick label style={/pgf/number format/1000 sep=},
    ]
    \addplot[red, semithick] file{../data/lab/pid/stpt_k=1.csv};
    \addplot[blue, semithick] file{../data/lab/pid/output_k=1.csv};
    \addlegendentry{$y^{zad}[k]$},
    \addlegendentry{$y[k]$},
    \addlegendimage{no markers, blue}
	\addlegendimage{no markers, red}
    \end{axis}
    \end{tikzpicture}
    \caption{Przebieg wyjścia obiektu sterowanego regulatora typu P o wzmocnieniu $K_{\mathrm{r}} = \num{1}$}
    \label{zad5_niegasnace_oscylacje}
\end{figure}

\begin{figure}[b]
    \centering
    \begin{tikzpicture}
    \begin{axis}[
    width=0.98\textwidth,
    xmin=0,xmax=800,ymin=20, ymax=35,
    xlabel={$k$},
    ylabel={$u[k]$},
    %xtick={0, 0.4, 0.5, 0.6, 0.7},
    %ytick={3.5, 3.75, 4, 4.25, 4.5},
    legend pos=south east,
    y tick label style={/pgf/number format/1000 sep=},
    ]
    \addplot[const plot, blue, semithick] file{../data/lab/pid/input_k=1.csv};
    \legend{$u[k]$}
    \end{axis}
    \end{tikzpicture}
    \caption{Przebieg sygnału sterującego generowanego przez regulator typu P o wzmocnieniu $K_{\mathrm{r}} = \num{1}$}
    \label{zad5_niegasnace_oscylacje_ster}
\end{figure}
\FloatBarrier

\begin{figure}[t]
    \centering
    \begin{tikzpicture}
    \begin{axis}[
    width=0.98\textwidth,
    xmin=0,xmax=600,ymin=28, ymax=42,
    xlabel={$k$},
    ylabel={$y[k]$},
    %xtick={0, 100, 200, 300, 400, 500},
    %ytick={3.9, 4, 4.1, 4.2, 4.3, 4.4},
    legend pos=south east,
    y tick label style={/pgf/number format/1000 sep=},
    ]
    \addplot[red, semithick] file{../data/lab/pid/stpt_k=20.csv};
    \addplot[blue, semithick] file{../data/lab/pid/output_k=20.csv};
    \addlegendentry{$y^{zad}[k]$},
    \addlegendentry{$y[k]$},
    \addlegendimage{no markers, blue}
	\addlegendimage{no markers, red}
    \end{axis}
    \end{tikzpicture}
    \caption{Przebieg wyjścia obiektu sterowanego regulatora typu P o wzmocnieniu $K_{\mathrm{r}} = \num{20}$}
    \label{zad5_niegasnace_oscylacje_20}
\end{figure}

\begin{figure}[b]
    \centering
    \begin{tikzpicture}
    \begin{axis}[
    width=0.98\textwidth,
    xmin=0,xmax=600,ymin=0, ymax=100,
    xlabel={$k$},
    ylabel={$u[k]$},
    %xtick={0, 0.4, 0.5, 0.6, 0.7},
    %ytick={3.5, 3.75, 4, 4.25, 4.5},
    legend pos=south east,
    y tick label style={/pgf/number format/1000 sep=},
    ]
    \addplot[const plot, blue, semithick] file{../data/lab/pid/input_k=20.csv};
    \legend{$u[k]$}
    \end{axis}
    \end{tikzpicture}
    \caption{Przebieg sygnału sterującego generowanego przez regulator typu P o wzmocnieniu $K_{\mathrm{r}} = \num{20}$}
    \label{zad5_niegasnace_oscylacje_ster_20}
\end{figure}
\FloatBarrier

\begin{figure}[t]
    \centering
    \begin{tikzpicture}
    \begin{axis}[
    width=0.98\textwidth,
    xmin=0,xmax=400,ymin=28, ymax=42,
    xlabel={$k$},
    ylabel={$y[k]$},
    %xtick={0, 100, 200, 300, 400, 500},
    %ytick={3.9, 4, 4.1, 4.2, 4.3, 4.4},
    legend pos=south east,
    y tick label style={/pgf/number format/1000 sep=},
    ]
    \addplot[red, semithick] file{../data/lab/pid/stpt_k=40.csv};
    \addplot[blue, semithick] file{../data/lab/pid/output_k=40.csv};
    \addlegendentry{$y^{zad}[k]$},
    \addlegendentry{$y[k]$},
    \addlegendimage{no markers, blue}
	\addlegendimage{no markers, red}
    \end{axis}
    \end{tikzpicture}
    \caption{Przebieg wyjścia obiektu sterowanego regulatora typu P o wzmocnieniu $K_{\mathrm{r}} = \num{40}$}
    \label{zad5_niegasnace_oscylacje_40}
\end{figure}

\begin{figure}[b]
    \centering
    \begin{tikzpicture}
    \begin{axis}[
    width=0.98\textwidth,
    xmin=0,xmax=400,ymin=0, ymax=100,
    xlabel={$k$},
    ylabel={$u[k]$},
    %xtick={0, 0.4, 0.5, 0.6, 0.7},
    %ytick={3.5, 3.75, 4, 4.25, 4.5},
    legend pos=south east,
    y tick label style={/pgf/number format/1000 sep=},
    ]
    \addplot[const plot, blue, semithick] file{../data/lab/pid/input_k=40.csv};
    \legend{$u[k]$}
    \end{axis}
    \end{tikzpicture}
    \caption{Przebieg sygnału sterującego generowanego przez regulator typu P o wzmocnieniu $K_{\mathrm{r}} = \num{40}$}
    \label{zad5_niegasnace_oscylacje_ster_40}
\end{figure}
\FloatBarrier

\begin{figure}[t]
    \centering
    \begin{tikzpicture}
    \begin{axis}[
    width=0.98\textwidth,
    xmin=0,xmax=400,ymin=28, ymax=42,
    xlabel={$k$},
    ylabel={$y[k]$},
    %xtick={0, 100, 200, 300, 400, 500},
    %ytick={3.9, 4, 4.1, 4.2, 4.3, 4.4},
    legend pos=south east,
    y tick label style={/pgf/number format/1000 sep=},
    ]
    \addplot[red, semithick] file{../data/lab/pid/stpt_k=20_ti=200.csv};
    \addplot[blue, semithick] file{../data/lab/pid/output_k=20_ti=200.csv};
    \addlegendentry{$y^{zad}[k]$},
    \addlegendentry{$y[k]$},
    \addlegendimage{no markers, blue}
	\addlegendimage{no markers, red}
    \end{axis}
    \end{tikzpicture}
    \caption{Przebieg wyjścia obiektu sterowanego regulatora typu PI o parametrach $K_{\mathrm{r}} = \num{40}, T_{\mathrm{i}} = 200$}
    \label{zad5_niegasnace_oscylacje_40_200}
\end{figure}

\begin{figure}[b]
    \centering
    \begin{tikzpicture}
    \begin{axis}[
    width=0.98\textwidth,
    xmin=0,xmax=400,ymin=0, ymax=100,
    xlabel={$k$},
    ylabel={$u[k]$},
    %xtick={0, 0.4, 0.5, 0.6, 0.7},
    %ytick={3.5, 3.75, 4, 4.25, 4.5},
    legend pos=south east,
    y tick label style={/pgf/number format/1000 sep=},
    ]
    \addplot[const plot, blue, semithick] file{../data/lab/pid/input_k=20_ti=200.csv};
    \legend{$u[k]$}
    \end{axis}
    \end{tikzpicture}
    \caption{Przebieg sygnału sterującego generowanego przez regulator typu PI o wzmocnieniu $K_{\mathrm{r}} = \num{40}, T_{\mathrm{i}} = 200$}
    \label{zad5_niegasnace_oscylacje_ster_40_200}
\end{figure}
\FloatBarrier

\begin{figure}[t]
    \centering
    \begin{tikzpicture}
    \begin{axis}[
    width=0.98\textwidth,
    xmin=0,xmax=400,ymin=28, ymax=42,
    xlabel={$k$},
    ylabel={$y[k]$},
    %xtick={0, 100, 200, 300, 400, 500},
    %ytick={3.9, 4, 4.1, 4.2, 4.3, 4.4},
    legend pos=south east,
    y tick label style={/pgf/number format/1000 sep=},
    ]
    \addplot[red, semithick] file{../data/lab/pid/stpt_k=20_ti=500.csv};
    \addplot[blue, semithick] file{../data/lab/pid/output_k=20_ti=500.csv};
    \addlegendentry{$y^{zad}[k]$},
    \addlegendentry{$y[k]$},
    \addlegendimage{no markers, blue}
	\addlegendimage{no markers, red}
    \end{axis}
    \end{tikzpicture}
    \caption{Przebieg wyjścia obiektu sterowanego regulatora typu PI o wzmocnieniu $K_{\mathrm{r}} = \num{40}, T_{\mathrm{i}} = 500$}
    \label{zad5_niegasnace_oscylacje_20_500}
\end{figure}

\begin{figure}[b]
    \centering
    \begin{tikzpicture}
    \begin{axis}[
    width=0.98\textwidth,
    xmin=0,xmax=400,ymin=0, ymax=100,
    xlabel={$k$},
    ylabel={$u[k]$},
    %xtick={0, 0.4, 0.5, 0.6, 0.7},
    %ytick={3.5, 3.75, 4, 4.25, 4.5},
    legend pos=south east,
    y tick label style={/pgf/number format/1000 sep=},
    ]
    \addplot[const plot, blue, semithick] file{../data/lab/pid/input_k=20_ti=500.csv};
    \legend{$u[k]$}
    \end{axis}
    \end{tikzpicture}
    \caption{Przebieg sygnału sterującego generowanego przez regulator typu PI o wzmocnieniu $K_{\mathrm{r}} = \num{40}, T_{\mathrm{i}} = 500$}
    \label{zad5_niegasnace_oscylacje_ster_20_500}
\end{figure}
\FloatBarrier


\begin{figure}[t]
    \centering
    \begin{tikzpicture}
    \begin{axis}[
    width=0.98\textwidth,
    xmin=0,xmax=400,ymin=28, ymax=42,
    xlabel={$k$},
    ylabel={$y[k]$},
    %xtick={0, 100, 200, 300, 400, 500},
    %ytick={3.9, 4, 4.1, 4.2, 4.3, 4.4},
    legend pos=south east,
    y tick label style={/pgf/number format/1000 sep=},
    ]
    \addplot[red, semithick] file{../data/lab/pid/stpt_k=20_ti=1000.csv};
    \addplot[blue, semithick] file{../data/lab/pid/output_k=20_ti=1000.csv};
    \addlegendentry{$y^{zad}[k]$},
    \addlegendentry{$y[k]$},
    \addlegendimage{no markers, blue}
	\addlegendimage{no markers, red}
    \end{axis}
    \end{tikzpicture}
    \caption{Przebieg wyjścia obiektu sterowanego regulatora typu PI o wzmocnieniu $K_{\mathrm{r}} = \num{20}, T_{\mathrm{i}} = 1000$}
    \label{zad5_niegasnace_oscylacje_20_1000}
\end{figure}

\begin{figure}[b]
    \centering
    \begin{tikzpicture}
    \begin{axis}[
    width=0.98\textwidth,
    xmin=0,xmax=400,ymin=0, ymax=100,
    xlabel={$k$},
    ylabel={$u[k]$},
    %xtick={0, 0.4, 0.5, 0.6, 0.7},
    %ytick={3.5, 3.75, 4, 4.25, 4.5},
    legend pos=south east,
    y tick label style={/pgf/number format/1000 sep=},
    ]
    \addplot[const plot, blue, semithick] file{../data/lab/pid/input_k=20_ti=1000.csv};
    \legend{$u[k]$}
    \end{axis}
    \end{tikzpicture}
    \caption{Przebieg sygnału sterującego generowanego przez regulator typu PI o wzmocnieniu $K_{\mathrm{r}} = \num{20}, T_{\mathrm{i}} = 1000$}
    \label{zad5_niegasnace_oscylacje_ster_20_1000}
\end{figure}
\FloatBarrier