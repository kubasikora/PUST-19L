\chapter{Normalizacja odpowiedzi skokowej}
\label{zad3}

\section{Postać ogólna znormalizowanej odpowiedzi skokowej}
Znormalizowana odpowiedź skokowa wykorzystywana jest jako model w predykcyjnym
regulatorze DMC. Jest to zestaw liczb $s_{1}, s_{2}, \ldots$, które opisują zachowanie
procesu przy skoku sterowania z $u_{\mathrm{0}}=\num{0}$ na $u_{\mathrm{1}}=\num{1}$ w chwili $k=\num{0}$. 
Aby znormalizować uzyskaną wcześniej odpowiedź, biorąc pod uwagę punkt pracy procesu należy
posłużyć się przekształceniem:

\begin{equation}
    s_{i} = \frac{s^{\mathrm{0}}_{i} - Y_{\mathrm{pp}}}{U_{\mathrm{s}} - U_{\mathrm{pp}}},  \hbox{ dla } i = 1,2,\ldots
    \label{zad3_norm_odp_skok_wzor}
\end{equation}

W celu wyznaczenia znormalizowanej odpowiedzi skokowej napisaliśmy skrypt \verb+zad3.m+,
który wykonał symulację skoku sterowania o $\Delta u = \num{0,1}$ a następnie za pomocą
wzoru \ref{zad3_norm_odp_skok_wzor} znormalizował odpowiedź. Efektem działania tego programu
jest rysunek \ref{zad3_norm_odp}.

\begin{figure}[b]
    \label{zad3_norm_odp}
    \centering
    \begin{tikzpicture}
    \begin{axis}[
    width=\textwidth,
    xmin=0,xmax=200,ymin=0,ymax=4,
    xlabel={$k$},
    ylabel={$y[k]$},
    xtick={0, 50, 100, 150, 200},
    ytick={0, 1, 2, 3, 4},
    legend pos=south east,
    y tick label style={/pgf/number format/1000 sep=},
    ]
    \addplot[red, semithick] file{../data/zad3_norm_odp.csv};
    \legend{$y[k]$}
    \end{axis}
    \end{tikzpicture}
    \caption{Postać znormalizowanej odpowiedzi skokowej symulowanego procesu}
\end{figure}
