\chapter{Normalizacja odpowiedzi skokowej}
\label{zad3}

\section{Postać ogólna znormalizowanej odpowiedzi skokowej}
\label{zad3_postac_ogolna}
Znormalizowana odpowiedź skokowa wykorzystywana jest jako model w predykcyjnym
regulatorze DMC. Jest to zestaw liczb $s_{1}, s_{2}, \ldots$, które opisują zachowanie
procesu przy skoku sterowania z $u_{\mathrm{0}}=\num{0}$ na $u_{\mathrm{1}}=\num{1}$ w chwili $k=\num{0}$. 
Aby znormalizować uzyskaną wcześniej odpowiedź, biorąc pod uwagę punkt pracy procesu należy
posłużyć się przekształceniem:

\begin{equation}
    s_{i} = \frac{s^{\mathrm{0}}_{i} - Y_{\mathrm{pp}}}{U_{\mathrm{s}} - U_{\mathrm{pp}}},  \hbox{ dla } i = 1,2,\ldots
    \label{zad3_norm_odp_skok_wzor}
\end{equation}

W celu wyznaczenia znormalizowanej odpowiedzi skokowej napisaliśmy skrypt \verb+zad3.m+,
który wykonał symulację skoku sterowania o $\Delta u = \num{0,1}$ a następnie za pomocą
wzoru \ref{zad3_norm_odp_skok_wzor} znormalizował odpowiedź. Efektem działania tego programu
jest rysunek \ref{zad3_norm_odp}.

\begin{figure}[b]
    \label{zad3_norm_odp}
    \centering
    \begin{tikzpicture}
    \begin{axis}[
    width=\textwidth,
    xmin=1,xmax=200,ymin=0,ymax=3,
    xlabel={$k$},
    ylabel={$s_{k}$},
    xtick={1, 50, 100, 150, 200},
    ytick={0, 1, 2, 3},
    legend pos=south east,
    y tick label style={/pgf/number format/1000 sep=},
    ]
    \addplot[red, semithick] file{../data/zad3_norm_odp.csv};
    \legend{$y[k]$}
    \end{axis}
    \end{tikzpicture}
    \caption{Postać znormalizowanej nieobciętej odpowiedzi skokowej symulowanego procesu}
\end{figure}
\FloatBarrier

\section{Stabilizacja odpowiedzi skokowej}
\label{zad3_stabilizacja}
Współczynniki odpowiedzi skokowej po nasyceniu można odrzucić ponieważ
nie poprawiają one jakości regulacji a zwiększają narzut obliczeniowy.
Długość użytecznej odpowiedzi skokowej oznaczamy przez $D$ i nazywamy
horyzontem dynamiki procesu. \\
\indent{} Druga część skryptu \verb+zad3.m+ automatycznie 
wyznacza ten parametr poprzez porównywanie róznic pomiędzy próbkami $y[i]$ a 
$y[i-1]$. Jeżeli wartości próbek są niezerowe oraz różnica jest mniejsza niż \verb+EPS+.
Dokładny opis algorytmu został przedstawiony na listingu \ref{zad3_lst}\\


\begin{lstlisting}[style=customc,frame=single,label={zad3_lst},caption={Opis automatycznego skracania odpowiedzi skokowej},captionpos=b] 
EPS = 0.0001;
cut_time = 0;
for i=2:SIM_LEN
    if (output(i)-output(i-1) < EPS && output(i) ~= 0)
        cut_time = i;
        break
    end
end
\end{lstlisting}


W przypadku rozważanego procesu horyzont dynamiki równy jest $D=\num{91}$. Ostateczna 
postać odpowiedzi skokowej została przedstawiona na rysunku \ref{zad3_obcieta_norm_odp}.
Informacja ta zostanie wykorzystana w dalszej części projektu, przy implementacji
algorytmu DMC.

\begin{figure}[b]
    \label{zad3_obcieta_norm_odp}
    \centering
    \begin{tikzpicture}
    \begin{axis}[
    width=\textwidth,
    xmin=1,xmax=91,ymin=0,ymax=3,
    xlabel={$k$},
    ylabel={$s_{k}$},
    xtick={1, 30, 60, 91},
    ytick={0, 1, 2, 3},
    legend pos=south east,
    y tick label style={/pgf/number format/1000 sep=},
    ]
    \addplot[red, semithick] file{../data/zad3_cut_norm_odp.csv};
    \legend{$y[k]$}
    \end{axis}
    \end{tikzpicture}
    \caption{Ostateczna postać znormalizowanej odpowiedzi skokowej symulowanego procesu}
\end{figure}