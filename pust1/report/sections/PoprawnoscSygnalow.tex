\chapter{Sprawdzenie poprawności wartości sygnałów $U_{\mathrm{pp}}$ i $Y_{\mathrm{pp}}$}
\label{zad1}

\section{Opis eksperymentu}
Aby sprawdzić poprawność wartości sygnałów $U_{\mathrm{pp}}$ i $Y_{\mathrm{pp}}$ wykonaliśmy eksperyment polegający na
pobudzeniu wejścia obiektu stałym sygnałem o wartości $U_{\mathrm{pp}}$ i sprawdzeniu czy sygnał wyjściowy stabilizuje 
się na wartości $Y_{\mathrm{pp}}$. Symulację obiektu przeprowadziliśmy za pomocą funkcji 
\verb+symulacja_obiektu1Y+. Spodziewanym wyjściem procesu dla wejścia $U_{\mathrm{pp}} = \num{0.5}$ jest
wyjście równe $Y_{\mathrm{pp}} = \num{4}$.

\section{Przeprowadzenie badań}
Eksperyment przeprowadziliśmy za pomocą skryptu \verb+zad1.m+, którego zadaniem było przeprowadzenie symulacji 
oraz zapis danych do pliku. Zebrana odpowiedź obiektu została przedstawiona na rysunku \ref{zad1_output}. Obiekt 
przy stałym pobudzeniu $U_{\mathrm{pp}} = \num{0.5}$, stabilizuje sygnał wyjściowy na wartości $Y_{\mathrm{pp}} = \num{4}$.
Jednoznacznie potwierdza to poprawność zadanych wartości.

\begin{figure}[b]
    \centering
    \begin{tikzpicture}
    \begin{axis}[
    width=\textwidth,
    xmin=0,xmax=200,ymin=0,ymax=8,
    xlabel={$k$},
    ylabel={$y[k]$},
    xtick={0, 50, 100, 150, 200},
    ytick={0, 2, 4, 6, 8},
    legend pos=south east,
    y tick label style={/pgf/number format/1000 sep=},
    ]
    \addplot[red, semithick] file{../data/zad1_output.csv};
    \legend{$y[k]$}
    \end{axis}
    \end{tikzpicture}
    \caption{Odpowiedź symulowanego procesu na stałe wejście o wartości $U_{\mathrm{pp}}$}
    \label{zad1_output}
\end{figure}
