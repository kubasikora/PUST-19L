\chapter{Modelowanie obiektu rzeczywistego}
\label{lab3}

\section{Normalizacja odpowiedzi skokowej na rzecz algorytmu DMC}
\label{lab3_norm}
Wyjaśnienie czym jest odpowiedź skokowa oraz proces jej wyznaczania został 
szeroko opisany w rozdziałach \ref{zad3} oraz \ref{lab1}, dlatego też w tej części 
skupimy się tylko na prezentacji wyniku.

Za pomocą skryptu \verb+lab/zad3.m+ dokonaliśmy automatycznej normalizacji odpowiedzi skokowej.
Z czterech zebranych wybraliśmy tą uzyskaną dla skoku sterowania $\Delta u = \num{-10}$, ponieważ 
w naszej opinii została ona najdokładniej zebrana. Na podstawie tej odpowiedzi ustaliliśmy wartość
horyzontu dynamiki procesu na $D = \num{500}$ próbek.

\begin{figure}[b]
    \label{zad3_norm_odp}
    \centering
    \begin{tikzpicture}
    \begin{axis}[
    width=\textwidth,
    xmin=1,xmax=500,ymin=0.0,ymax=0.5,
    xlabel={$k$},
    ylabel={$s_{k}$},
    xtick={0, 100, 200, 300, 400, 500},
    ytick={0, 0.1, 0.2, 0.3, 0.4, 0.5},
    legend pos=south east,
    y tick label style={/pgf/number format/1000 sep=},
    ]
    \addplot[red, semithick] file{../data/lab/zad3_cut_odp.csv};
    \legend{$y[k]$}
    \end{axis}
    \end{tikzpicture}
    \caption{Postać znormalizowanej odpowiedzi skokowej rzeczywistego procesu}
\end{figure}
\FloatBarrier

\section{Aproksymacja odpowiedzi skokowej inercją drugiego rzędu z opóźnieniem}
\label{lab3_modelowanie}

