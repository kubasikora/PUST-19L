\chapter{Modelowanie obiektu rzeczywistego}
\label{lab3}

\section{Normalizacja odpowiedzi skokowej na rzecz algorytmu DMC}
\label{lab3_norm}
Wyjaśnienie czym jest odpowiedź skokowa oraz proces jej wyznaczania został 
szeroko opisany w rozdziałach \ref{zad3} oraz \ref{lab1}, dlatego też w tej części 
skupimy się tylko na prezentacji wyniku.

Za pomocą skryptu \verb+lab/zad3.m+ dokonaliśmy automatycznej normalizacji odpowiedzi skokowej.
Z czterech zebranych wybraliśmy tą uzyskaną dla skoku sterowania $\Delta u = \num{-10}$, ponieważ 
w naszej opinii została ona najdokładniej zebrana. Na podstawie tej odpowiedzi ustaliliśmy wartość
horyzontu dynamiki procesu na $D = \num{500}$ próbek.

\begin{figure}[b]
    \label{zad3_norm_odp}
    \centering
    \begin{tikzpicture}
    \begin{axis}[
    width=\textwidth,
    xmin=1,xmax=500,ymin=0.0,ymax=0.5,
    xlabel={$k$},
    ylabel={$s_{k}$},
    xtick={0, 100, 200, 300, 400, 500},
    ytick={0, 0.1, 0.2, 0.3, 0.4, 0.5},
    legend pos=south east,
    y tick label style={/pgf/number format/1000 sep=},
    ]
    \addplot[red, semithick] file{../data/lab/zad3_cut_odp.csv};
    \legend{$y[k]$}
    \end{axis}
    \end{tikzpicture}
    \caption{Postać znormalizowanej odpowiedzi skokowej rzeczywistego procesu}
\end{figure}
\FloatBarrier

\section{Aproksymacja odpowiedzi skokowej inercją drugiego rzędu z opóźnieniem}
\label{lab3_modelowanie}

\subsection{Postać modelu}
\label{lab3_postac_modelu}
Poszukiwana przez nas transmitancja dyskretna jest postaci:

\begin{equation}
G(z) = \frac{b_{\mathrm{1}} z^{-1} + b_{\mathrm{2}} z^{-2}}{1 + a_{\mathrm{1}} z^{-1} + a_{\mathrm{2}} z^{-2}} z^{-T_{\mathrm{d}}}
\end{equation}

\subsection{Optymalizacja}
Przy pomocy skryptu \verb+zad3_modelowanie.m+ znaleźliśmy parametry transmitancji 
$b_{i}$, $a_{i}$ oraz $T_{\mathrm{d}}$. Wykorzystaliśmy w nim funkcję \emph{fmincon}, 
która optymalizuje zadaną funkcję przy danych ograniczeniach. Optymalizowaną funkcją 
był bład modelu w odniesieniu do znormalizowanej odpowiedzi skokowej. Ograniczenia na 
parametry $b_{i}$, $a_{i}$ zostały tak dobrane aby algorytm mógł swobodnie przeszukać 
sporą przestrzeń.

$$ \num{-1000} \leq b_{i}, a_{i} \leq \num{1000} $$

Inaczej wyglądała sytuacja w przypadku parametru $T_{\mathrm{d}}$, który opisywał opóźnienie a 
dokładnie liczbę próbek opóźnienia modelu. W naszym przypadku dobraliśmy ograniczenie:

$$ \num{1} \leq T_{\mathrm{d}} \leq \num{20} $$

Ostatecznie otrzymaliśmy model postaci:
\begin{equation}
    G(z) =  z^{-8} \frac{\num{0.00133} z^{-1} + \num{0.00133} z^{-2}}{1 + \num{-1.1963} z^{-1} - \num{0.20402} z^{-2}}
\end{equation}

Wartość funkcji błędu wynosiła $E = \num{0.0099}$. W przypadku doboru parametrów ciągłych, 
funkcja \emph{fmincon} dobrze radzi sobie z ich doborem. W przypadku dyskretnego parametru 
$T_{\mathrm{d}}$ sprawa wygląda zupełnie inaczej. Małe zmiany tego parametru w żaden sposób
nie wpływają na wynik symulacji, ponieważ funkcja ta zaokrągla jego wartość do liczby całkowitej.

\subsection{Wykorzystanie zaawansowanych narzędzi pakietu MATLAB}
W celu dokładniejszego zamodelowania opóźnienia badanego obietku posłużyliśmy 
się zaawansowanym toolboxem pakietu MATLAB System Identification. Pakiet ten 
pozwala na znalezienie parametrów różnych modeli. Po załadowaniu odpowiedzi skokowej,
wybraliśmy \emph{Transfer Function Model} jako rodzaj żądanego modelu a następnie po chwili mieliśmy gotowy obiekt typu \verb+idtf+. 

Otrzymany model:
\begin{equation}
    G(z) =  z^{-12} \frac{\num{0.003551} z^{-1} + \num{0.003395} z^{-2}}{1 + \num{0.01013} z^{-1} - \num{0.9899} z^{-2}}
\end{equation}

Dokładniejsza identyfikacja opóźnienia pozwoliła na zmniejszenie wartości funkcji błędu do $E = \num{0.0088}$.

\begin{figure}[t]
    \label{zad3_norm_odp}
    \centering
    \begin{tikzpicture}
    \begin{axis}[
    width=\textwidth,
    xmin=1,xmax=500,ymin=0.0,ymax=0.5,
    xlabel={$k$},
    ylabel={$s_{k}$},
    xtick={0, 100, 200, 300, 400, 500},
    ytick={0, 0.1, 0.2, 0.3, 0.4, 0.5},
    legend pos=south east,
    y tick label style={/pgf/number format/1000 sep=},
    ]
    \addplot[red, semithick] file{../data/lab/model_response_fmincon.csv};
    \addplot[blue, semithick] file{../data/lab/zad3_cut_odp.csv};
    \addlegendentry{Model},
    \addlegendentry{Obiekt},
    \addlegendimage{no markers,red}
	\addlegendimage{no markers,blue}
    \end{axis}
    \end{tikzpicture}
    \caption{Porównanie modelu obiektu otrzymanego optymalizując funkcji błędu przy pomocy \emph{fmincon} z rzeczywistą odpowiedzią skokową}
\end{figure}

\begin{figure}[b]
    \label{zad3_norm_odp}
    \centering
    \begin{tikzpicture}
    \begin{axis}[
    width=\textwidth,
    xmin=1,xmax=500,ymin=0.0,ymax=0.5,
    xlabel={$k$},
    ylabel={$s_{k}$},
    xtick={0, 100, 200, 300, 400, 500},
    ytick={0, 0.1, 0.2, 0.3, 0.4, 0.5},
    legend pos=south east,
    y tick label style={/pgf/number format/1000 sep=},
    ]
    \addplot[red, semithick] file{../data/lab/model_response_MAT.csv};
    \addplot[blue, semithick] file{../data/lab/zad3_cut_odp.csv};
    \addlegendentry{Model},
    \addlegendentry{Obiekt},
    \addlegendimage{no markers,red}
	\addlegendimage{no markers,blue}
    \end{axis}
    \end{tikzpicture}
    \caption{Porównanie modelu obiektu otrzymanego przy pomocy pakietu System Identifiaction z rzeczywistą odpowiedzią skokową}
\end{figure}
\FloatBarrier

