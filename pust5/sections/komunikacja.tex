\section{Komunikacja w zespole}
\label{komunikacja}


\begin{frame}
    \frametitle{Komunikacja przez internet}
    \begin{columns}[t]
        \column{.5\textwidth}
        \begin{block}{Zalety:}
            \begin{itemize}
                \item pozostaje po niej ślad
                \item łatwo dogadać się z dużym zespołem
                \item dostępna cały czas
            \end{itemize}
        \end{block}
        \column{.5\textwidth}
        \begin{block}{Wady:}
            \begin{itemize}
                \item ciężko przekazać emocje
                \item wygodniej mówić niż pisać
                \item można nie zauważyć 
            \end{itemize}
        \end{block}
    \end{columns}
\end{frame}

\begin{frame}
    \frametitle{Komunikacja twarzą w twarz}
    \begin{columns}[t]
        \column{.5\textwidth}
        \begin{block}{Zalety:}
            \begin{itemize}
                \item najwygodniejszy sposób
                \item łatwo dogadać szczegóły
                \item można wyrażać emocje
            \end{itemize}
        \end{block}
        \column{.5\textwidth}
        \begin{block}{Wady:}
            \begin{itemize}
                \item nie pozostaje po niej ślad
                \item ciężko zebrać zespół
                \item możliwość pominięcia pewnych faktów
            \end{itemize}
        \end{block}
    \end{columns}
\end{frame}

\begin{frame}
    \frametitle{Porównanie}
    \metroset{block=fill}
    \begin{block}{Wniosek}
        Komunikacja twarzą w twarz jest zawsze wydajniejsza, jednak nie zawsze działa
        w warunkach studenckich.
    \end{block}
\end{frame}
