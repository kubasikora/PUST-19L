\section{Sposoby podziału zadań}
\label{podzial_zadan}

\begin{frame}
    \frametitle{Co to jest projekt?}
    \begin{block}{}
        Kilka faktów o projektach:
    \end{block}
    \pause
    \begin{itemize}[<+->]
        \item Projekt ma swój początek,
        \item Projekt ma jasno określony, osiągalny cel,
        \item Można w nim wydzielić zadania,
        \item Projekt informatyczny to zjawisko społeczne a nie techniczne.
    \end{itemize} 
\end{frame}

\begin{frame}
    \frametitle{Podział projektu na zadania}
    \begin{block}{}
    Projekty oraz laboratoria były naturalnie podzielone na zadania, 
    mimo to należało niektóre zadania podzielić na mniejsze.
    \end{block}
\end{frame}

\begin{frame}
    \frametitle{Przykład podziału zadania na mniejsze}
    \metroset{block=fill}
    \begin{exampleblock}{Zadanie 5, projekt 4, zespół 1}
    5. Dla zaproponowanej trajektorii zmian sygnału zadanego dobrać nastawy regulatora
    PID i parametry algorytmu DMC $\mu_{\mathrm{1}}$, $\mu_{\mathrm{2}}$, $\mu_{\mathrm{3}}$, 
    $\lambda_{\mathrm{1}}$, $\lambda_{\mathrm{2}}$, $\lambda_{\mathrm{3}}$, $\lambda_{\mathrm{4}}$natomiast horyzonty $D$,
    $N$ , $N_{\mathrm{u}}$ przyjąć stałe) w wyniku optymalizacji wskaźnika jakości regulacji $E$. Zamieścić
    wyniki symulacji.   
    \end{exampleblock}
\end{frame}

\begin{frame}
    \frametitle{Przykład podziału zadania na mniejsze}
    \begin{itemize}
        \item Przygotować program do optymalizacji parametrów regulatora PID
        \item Przygotować program do optymalizacji parametrów regulatora DMC
        \item Dokonać optymalizacji parametrów regulatora PID i zapisać dane symulacji
        \item Dokonać optymalizacji parametrów regulatora DMC i zapisać dane symulacji
        \item Zamieścić w sprawozdaniu wykres z optymalizacji parametrów regulatora PID
        \item Zamieścić w sprawozdaniu wykres z optymalizacji parametrów regulatora DMC      
    \end{itemize}
\end{frame}

\begin{frame}
    \frametitle{Sposoby podziału zadań}
    Podziału zadań można dokonać na następujące sposoby:
    \pause
    \begin{itemize}[<+->]
        \item Jedna osoba robi wszystko,
        \item Wybór lidera, rodzielającego zadania,
        \item Zwinny podział zadań.
    \end{itemize}
\end{frame}

\begin{frame}
    \frametitle{Zwinny podział zadań}
    Ważne założenia:
    \pause
    \begin{itemize}[<+->]
        \item każdy z członków zespołu jest jednakowo zmotywowany do pracy,
        \item członkowie zespołu sami wybierają co chcą robić najbardziej,
        \item informujemy innych co robimy, tak aby nie dublować pracy,
        \item planujemy własną pracę w przód, wiedząc o innych zobowiązaniach. 
    \end{itemize}
\end{frame}
